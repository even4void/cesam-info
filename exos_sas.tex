\chapter{Élements du langage et statistiques descriptives}

\begin{exo}\label{exo:8.1}
Reprendre l'énoncé 1.1 (p.~\pageref{exo:1.1}) et répondre aux questions a–c.
\end{exo}
\vskip1em

\begin{exo}\label{exo:8.2}
Reprendre l'énoncé 1.2 (p.~\pageref{exo:1.2}) et répondre aux questions a et
b.
\end{exo}
\vskip1em

\begin{exo}\label{exo:8.3}
Reprendre l'énoncé 1.5 (p.~\pageref{exo:1.5}) et répondre aux questions a–d.
\end{exo}
\vskip1em

\begin{exo}\label{exo:8.4}
Reprendre l'énoncé 2.1 (p.~\pageref{exo:2.1}) et répondre aux questions a–d.
\end{exo}
\vskip1em

\begin{exo}\label{exo:8.5}
Reprendre l'énoncé 2.3 (p.~\pageref{exo:2.3}) et répondre aux questions a–c.
\end{exo}
\vskip1em

\begin{exo}\label{exo:8.6}
Reprendre l'énoncé 2.4 (p.~\pageref{exo:2.4}) et répondre aux questions a–f.
\end{exo}

%--------------------------------------------------------------- Devoir 07 ---
\chapter*{Devoir \no 7}
\addcontentsline{toc}{chapter}{Devoir \no 7}

Les exercices sont indépendants. Une seule réponse est correcte pour chaque
question. Lorsque vous ne savez pas répondre, cochez la case correspondante.

\section*{Exercice 1}
\section*{Exercice 2}

\section*{Exercice 3}



\chapter[Mesures d'association, comparaison de moyennes et de
proportions]{Mesures d'association, comparaison de moyennes et de
  proportions pour deux échantillons ou plus}   

\begin{exo}\label{exo:9.1}
Reprendre l'énoncé 3.1 (p.~\pageref{exo:3.1}) et répondre aux questions a–c.
\end{exo}
\vskip1em

\begin{exo}\label{exo:9.2}
Reprendre l'énoncé 3.2 (p.~\pageref{exo:3.2}) et répondre aux questions a–c.
\end{exo}
\vskip1em

\begin{exo}\label{exo:9.3}
Reprendre l'énoncé 3.4 (p.~\pageref{exo:3.4}) et répondre aux questions a–d.
\end{exo}
\vskip1em

\begin{exo}\label{exo:9.4}
Reprendre l'énoncé 3.5 (p.~\pageref{exo:3.5}) et répondre aux questions a–d.
\end{exo}
\vskip1em

\begin{exo}\label{exo:9.5}
Reprendre l'énoncé 4.1 (p.~\pageref{exo:4.1}) et répondre aux questions a–d.
\end{exo}
\vskip1em

\begin{exo}\label{exo:9.6}
Reprendre l'énoncé 4.2 (p.~\pageref{exo:4.2}) et répondre aux questions a–c.
\end{exo}
\vskip1em

\begin{exo}\label{exo:9.7}
Reprendre l'énoncé 4.3 (p.~\pageref{exo:4.3}) et répondre aux questions a–e.
\end{exo}

%--------------------------------------------------------------- Devoir 08 ---
\chapter*{Devoir \no 8}
\addcontentsline{toc}{chapter}{Devoir \no 8}

Les exercices sont indépendants. Une seule réponse est correcte pour chaque
question. Lorsque vous ne savez pas répondre, cochez la case correspondante.

\section*{Exercice 1}

\section*{Exercice 2}

\section*{Exercice 3}

%--------------------------------------------------------------- Chapter 10 ---
\chapter{Régression linéaire et logistique}

\begin{exo}\label{exo:10.1}
Reprendre l'énoncé 5.1 (p.~\pageref{exo:5.1}) et répondre aux questions a–e.
\end{exo}
\vskip1em

\begin{exo}\label{exo:10.2}
Reprendre l'énoncé 5.2 (p.~\pageref{exo:5.2}) et répondre aux questions a–e.
\end{exo}
\vskip1em

\begin{exo}\label{exo:10.3}
Reprendre l'énoncé 5.3 (p.~\pageref{exo:5.3}) et répondre aux questions a–d.
\end{exo}
\vskip1em

\begin{exo}\label{exo:10.4}
Reprendre l'énoncé 6.1 (p.~\pageref{exo:6.1}) et répondre aux questions a–d.
\end{exo}
\vskip1em

\begin{exo}\label{exo:10.5}
Reprendre l'énoncé 6.3 (p.~\pageref{exo:6.3}) et répondre aux questions a–f.
\end{exo}
\vskip1em

\begin{exo}\label{exo:10.6}
Reprendre l'énoncé 6.5 (p.~\pageref{exo:6.5}) et répondre aux questions a–e.
\end{exo}

%--------------------------------------------------------------- Devoir 09 ---
\chapter*{Devoir \no 9}
\addcontentsline{toc}{chapter}{Devoir \no 9}

Les exercices sont indépendants. Une seule réponse est correcte pour chaque
question. Lorsque vous ne savez pas répondre, cochez la case correspondante.

\section*{Exercice 1}

\section*{Exercice 2}

\section*{Exercice 3}

%--------------------------------------------------------------- Chapter 11 ---
\chapter{Analyse de données de survie}

\begin{exo}\label{exo:11.1}
Reprendre l'énoncé 7.1 (p.~\pageref{exo:7.1}) et répondre aux questions a–d.
\end{exo}
\vskip1em

\begin{exo}\label{exo:11.2}
Reprendre l'énoncé 7.3 (p.~\pageref{exo:7.3}) et répondre aux questions a–i.
\end{exo}

%--------------------------------------------------------------- Devoir 10 ---
\chapter*{Devoir \no 10}
\addcontentsline{toc}{chapter}{Devoir \no 10}

Les exercices sont indépendants. Une seule réponse est correcte pour chaque
question. Lorsque vous ne savez pas répondre, cochez la case correspondante.

\section*{Exercice 1}

\section*{Exercice 2}

\section*{Exercice 3}
