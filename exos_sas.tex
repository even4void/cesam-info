\chapter{Élements du langage et statistiques descriptives}

\begin{exo}\label{exo:8.1}
{\footnotesize Identique à l'énoncé 1.1 (p.~\pageref{exo:1.1}), questions
  a–c.}

Un chercheur a recueilli les mesures biologiques suivantes (unités
arbitraires) :
\begin{verbatim}
3.68  2.21  2.45  8.64  4.32  3.43  5.11  3.87
\end{verbatim}
\begin{description}
\item[(a)] Stocker la séquence de mesures dans une variable appelée
  \texttt{x}.  
\item[(b)] Indiquer le nombre d'observations (à l'aide de \R), les valeurs
  minimale et maximale, ainsi que l'étendue.  
\item[(c)] En fait, le chercheur réalise que la valeur 8.64 correspond à une
  erreur de saisie et doit être changée en 3.64. De même, il a un doute sur
  la 7\ieme mesure et décide de la considérer comme une valeur manquante :
  effectuer les transformations correspondantes. 
\end{description}
\end{exo}
\vskip1em

\begin{exo}\label{exo:8.2}
{\footnotesize Identique à l'énoncé 1.2 (p.~\pageref{exo:1.2}), questions a
  et b.}

La charge virale plasmatique permet de décrire la quantité de virus (p.~ex.,
VIH) dans un échantillon de sang. Ce marqueur virologique qui permet de
suivre la progression de l’infection et de mesurer l’efficacité des
traitements est rapporté en nombre de copies par millilitre, et la plupart
des instruments de mesure ont un seuil de détectabilité de 50
copies/ml. Voici une série de mesures, $X$, exprimées en logarithmes (base 10)
collectées sur 20 patients :
\begin{verbatim}
3.64 2.27 1.43 1.77 4.62 3.04 1.01 2.14 3.02 5.62 5.51 5.51 1.01 1.05 4.19
2.63 4.34 4.85 4.02 5.92
\end{verbatim}
Pour rappel, une charge virale de 100 000 copies/ml équivaut à 5 log.
\begin{description}
\item[(a)] Indiquer combien de patients ont une charge virale considérée
  comme non-détectable. 
\item[(b)] Quelle est le niveau de charge virale médian, en copies/ml, pour
  les données considérées comme valides ?
\end{description}
\end{exo}
\vskip1em

\begin{exo}\label{exo:8.3}
{\footnotesize Identique à l'énoncé 1.5 (p.~\pageref{exo:1.5}), questions
  a–d.}  

Le fichier \texttt{anorexia.dat} contient les données d'une étude clinique
chez des patientes anorexiques ayant reçu l'une des trois thérapies
suivantes : thérapie comportementale, thérapie familiale, thérapie
contrôle.\autocite{hand93} 
\begin{description}
\item[(a)] Combien y'a-t-il de patientes au total ? Combien y'a-t-il de
  patientes par groupe de traitement ?
\item[(b)] Les mesures de poids sont en livres. Les convertir en
  kilogrammes.    
\item[(c)] Créer une nouvelle variable contenant les scores de différences
  (\texttt{After} - \texttt{Before}).
\item[(d)] Indiquer la moyenne et l'étendue (min/max) des scores de
  différences par groupe de traitement.
\end{description}
\end{exo}
\vskip1em

\begin{exo}\label{exo:8.4}
{\footnotesize Identique à l'énoncé 2.1 (p.~\pageref{exo:2.1}), questions
  a–d.} 

Une variable quantitative $X$ prend les valeurs suivantes sur un échantillon
de 26 sujets :
\begin{verbatim}
24.9,25.0,25.0,25.1,25.2,25.2,25.3,25.3,25.3,25.4,25.4,25.4,25.4,
25.5,25.5,25.5,25.5,25.6,25.6,25.6,25.7,25.7,25.8,25.8,25.9,26.0
\end{verbatim}
\begin{description}
\item[(a)] Calculer la moyenne, la médiane ainsi que le mode de $X$. 
\item[(b)] Quelle est la valeur de la variance estimée à partir de ces données ? 
\item[(c)] En supposant que les données sont regroupées en 4 classes dont les
  bornes sont : 24.9–25.1, 25.2–25.4, 25.5–25.7, 25.8–26.0, afficher la
  distribution des effectifs par classe sous forme d'un tableau d'effectifs. 
\item[(d)] Représenter la distribution de $X$ sous forme d'histogramme, sans
  considération d'intervalles de classe \emph{a priori}.
\end{description}
\end{exo}
\vskip1em

\begin{exo}\label{exo:8.5}
{\footnotesize Identique à l'énoncé 2.3 (p.~\pageref{exo:2.3}), questions
  a–c.}

Le fichier \texttt{elderly.dat} contient la taille mesurée en cm de 351
personnes âgées de sexe féminin, sélectionnées aléatoirement dans la
population lors d'une étude sur l'ostéoporose. Quelques observations sont
cependant manquantes.
\begin{description}
\item[(a)] Combien y'a t-il d'observations manquantes au total ?
\item[(b)] Donner un intervalle de confiance à 95~\% pour la taille moyenne
  dans cet échantillon, en utilisant une approximation normale.
\item[(c)] Représenter la distribution des tailles observées sous forme
  d'une courbe de densité.  
\end{description}
\end{exo}
\vskip1em

\begin{exo}\label{exo:8.6}
{\footnotesize Identique à l'énoncé 2.4 (p.~\pageref{exo:2.4}), questions
  a–f.} 

Le fichier \texttt{birthwt} est un des jeux de données fournis avec \R. Il
comprend les résultats d'une étude prospective visant à identifier les
facteurs de risque associés à la naissance de bébés dont le poids est
inférieur à la norme (2,5 kg). Les données proviennent de 189 femmes, dont
59 ont accouché d'un enfant en sous-poids. Parmi les variables d'intérêt
figurent l'âge de la mère, le poids de la mère lors des dernières
menstruations, l'ethnicité de la mère et le nombre de visites médicales
durant le premier trimestre de grossesse.\autocite{hosmer89}
Les variables disponibles sont décrites comme suit : \texttt{low} (= 1 si
poids $<2.5$ kg, 0 sinon), \texttt{age} (années), \texttt{lwt} (poids de la
mère en livres), \texttt{race} (ethnicité codée en trois classes, 1 = white,
2 = black, 3 = other), \texttt{smoke} (= 1 si consommation de tabac durant
la grossesse, 0 sinon), \texttt{ptl} (nombre d'accouchements pré-terme
antérieurs), \texttt{ht} (= 1 si antécédent d'hypertension, 0 sinon),
\texttt{ui} (= 1 si manifestation d'irritabilité utérine, 0 sinon),
\texttt{ftv} (nombre de consultations chez le gynécologue durant le premier
trimestre de grossesse), \texttt{bwt} (poids des bébés à la naissance, en
\emph{g}).
\begin{description}
\item[(a)] Recoder les variables \texttt{low}, \texttt{race},
  \texttt{smoke}, \texttt{ui} et \texttt{ht} en variables
  qualitatives, avec des étiquettes ("labels") plus informatives.
\item[(b)] Convertir le poids des mères en \emph{kg}. Indiquer la moyenne, la
  médiane et l'intervalle inter-quartile. Représenter la distribution des
  poids sous forme d'histogramme.
\item[(c)] Indiquer la proportion de mères consommant du tabac durant la
  grossesse, avec un intervalle de confiance à 95~\%. Représenter les
  proportions (en \%) fumeur/non-fumeur sous forme d'un diagramme en
  barres.
\item[(d)] Recoder l'âge des mères en trois classes équilibrées (tercilage)
  et indiquer la proportion d'enfants dont le poids est $<2500$ \emph{g}
  pour chacune des trois classes.
\item[(e)] Construire un tableau d'effectifs ($n$ et \%) pour la variable
  ethnicité (\texttt{race}).  
\item[(f)] Décrire la distribution des variables \texttt{race},
  \texttt{smoke}, \texttt{ui}, \texttt{ht} et \texttt{age} après
  stratification sur la variable \texttt{low}.  
\end{description}
\end{exo}

%--------------------------------------------------------------- Devoir 07 ---
% \chapter*{Devoir \no 7}
% \addcontentsline{toc}{chapter}{Devoir \no 7}

% Les exercices sont indépendants. Une seule réponse est correcte pour chaque
% question. Lorsque vous ne savez pas répondre, cochez la case correspondante.

% \section*{Exercice 1}
% \section*{Exercice 2}
% \section*{Exercice 3}\label{dev7:exo3}

\includepdf[pages={1-4}, fitpaper=true]{./devoir_sas/sas_devoir07f.pdf}

%--------------------------------------------------------------- Chapter 09 --
\chapter[Mesures d'association, comparaison de moyennes et de
proportions]{Mesures d'association, comparaison de moyennes et de
  proportions pour deux échantillons ou plus}   

\begin{exo}\label{exo:9.1}
{\footnotesize Identique à l'énoncé 3.1 (p.~\pageref{exo:3.1}), questions
  a–c.} 

On dispose des poids à la naissance d'un échantillon de 50 enfants
présentant un syndrôme de détresse respiratoire idiopathique aïgue. Ce type
de maladie peut entraîner la mort et on a observé 27 décès chez ces
enfants. Les données sont résumées dans le tableau ci-dessous et sont
disponibles dans le fichier \texttt{sirds.dat}, où les 27 premières
observations correspondent au groupe des enfants décédés au moment de
l'étude. \autocite[p.~64]{everitt01}
\vskip1em

\begin{tabular}{ll}
\toprule  
Enfants décédés &
1.050\; 1.175\; 1.230\; 1.310\; 1.500\; 1.600\; 1.720\; 1.750\; 1.770\; 2.275\; 2.500\; 1.030\; 1.100\; 1.185 \\
& 1.225\; 1.262\; 1.295\; 1.300\; 1.550\; 1.820\; 1.890\; 1.940\; 2.200\; 2.270\; 2.440\; 2.560\; 2.730 \\
Enfants vivants &
1.130\; 1.575\; 1.680\; 1.760\; 1.930\; 2.015\; 2.090\; 2.600\; 2.700\; 2.950\; 3.160\; 3.400\; 3.640\; 2.830 \\
& 1.410\; 1.715\; 1.720\; 2.040\; 2.200\; 2.400\; 2.550\; 2.570\; 3.005 \\
\bottomrule
\end{tabular}
\vskip1em

Un chercheur s'intéresse à l'existence éventuelle d'une différence entre le
poids moyen des enfants ayant survécu et celui des enfants décédés des
suites de la maladie. 
\begin{description}
\item[(a)] Réaliser un test $t$ de Student. Peut-on rejeter l'hypothèse nulle
  d'absence de différences entre les deux groupes d'enfants ? 
\item[(b)] Vérifier graphiquement que les conditions d'applications du test
(normalité et homogénéité des variances) sont vérifiées. 
\item[(c)] Quel est l'intervalle de confiance à 95~\% pour la différence de
  moyenne observée ?
\end{description}
\end{exo}
\vskip1em

\begin{exo}\label{exo:9.2}
{\footnotesize Identique à l'énoncé 3.2 (p.~\pageref{exo:3.2}), questions
  a–c.}

La qualité de sommeil de 10 patients a été mesurée avant (contrôle) et après
traitement par un des deux hypnotiques suivants : (1) D. hyoscyamine
hydrobromide et (2) L. hyoscyamine hydrobromide. Le critère de jugement
retenu par les chercheurs était le gain moyen de sommeil (en heures) par
rapport à la durée de sommeil de base
(contrôle). \autocite[p.~20]{student08} Les données sont reportées
ci-dessous et figurent également parmi les jeux de données de base de R
(\verb|data(sleep)|).  
\begin{verbatim}
D. hyoscyamine hydrobromide :
0.7 -1.6 -0.2 -1.2 -0.1  3.4  3.7  0.8  0.0  2.0
L. hyoscyamine hydrobromide :
1.9  0.8  1.1  0.1 -0.1  4.4  5.5  1.6  4.6  3.4
\end{verbatim}

Les chercheurs ont conclu que seule la deuxième molécule avait réellement un
effet soporifique. 
\begin{description}
\item[(a)] Estimer le temps moyen de sommeil pour chacune des deux
  molécules, ainsi que la différence entre ces deux moyennes.
\item[(b)] Afficher la distribution des scores de différence (LHH - DHH)
  sous forme d'un histogramme, en considérant des intervalles de classe
  d'une demi-heure, et indiquer la moyenne et l'écart-type de ces scores de
  différence.
\item[(c)] Vérifier l'exactitude des conclusions à l'aide d'un test de Student.
\end{description}
\end{exo}
\vskip1em

\begin{exo}\label{exo:9.3}
{\footnotesize Identique à l'énoncé 3.4 (p.~\pageref{exo:3.4}), questions
  a–d.}

Dans un essai clinique, on a cherché à évaluer un régime supposé réduire le
nombre de symptômes associé à une maladie bénigne du sein. Un groupe de 229
femmes ayant cette maladie ont été alétoirement réparties en deux
groupes. Le premier groupe a reçu les soins courants, tandis que les
patientes du second groupe suivaient un régime spécial (variable B =
traitement). Après un an, les individus ont été évalués et ont été classés
dans l'une des deux catégories : amélioration ou pas d'amélioration
(variable A = réponse). Les résultats sont résumés dans le tableau suivant,
pour une partie de l'échantillon :\autocite[p.~323]{selvin98}
\vskip1em

\begin{tabular}{l|cc|r}
& régime & pas de régime & total \\
\hline
amélioration & 26 & 21 & 47 \\
pas d'amélioration & 38 & 44 & 82 \\
\hline
total & 64 & 65 & 129
\end{tabular}
\vskip1em

\begin{description}
\item[(a)] Réaliser un test du chi-deux.  
\item[(b)] Quels sont les effectifs théoriques attendus sous une hypothèse
  d'indépendance ?
\item[(c)] Comparer les résultats obtenus en (a) avec ceux d'un test de
  Fisher.
\item[(d)] Donner un intervalle de confiance pour la différence de
  proportion d'amélioration entre les deux groupes de patientes.
\end{description}
\end{exo}
\vskip1em

\begin{exo}\label{exo:9.4}
{\footnotesize Identique à l'énoncé 3.5 (p.~\pageref{exo:3.5}), questions
  a–d.}

Dans un essai clinique, 1360 patients ayant déjà eu un infarctus dy myocarde
ont été assignés à l'un des deux groupes de traitement suivants : prise en
charge par aspirine à faible dose en une seule prise \emph{versus}
placebo. La table ci-après indique le nombre de décès par infarctus lors de
la période de suivi de trois ans :\autocite[p.~72]{agresti02} 
\vskip1em

\begin{tabular}{lccc}
\toprule
& \multicolumn{2}{c}{Infarctus} & \\
\cmidrule(r){2-3}
& Oui & Non & Total \\
\midrule
Placebo & 28 & 656 & 684 \\
Aspirine & 18 & 658 & 676 \\
\bottomrule
\end{tabular}
\vskip1em

\begin{description}
\item[(a)] Calculer la proportion d'infarctus du myocarde dans les deux
  groupes de patients.
\item[(b)] Représenter graphiquement le tableau précédent sous forme d'un
  diagramme en barres ou d'un diagramme en points ("dotplot" de Cleveland).
\item[(c)] Indiquer la valeur de l'odds-ratio ainsi que du risque
  relatif. Pour l'odds-ratio, on considérera comme catégories de référence
  les modalités représentées par la première ligne et la première colonne du
  tableau.  
\item[(d)] À partir de l'intervalle de confiance à 95~\% pour l'odds, quelle
  conclusion peut-on tirer sur l'effet de l'aspirine dans la prévention d'un
  infarctus du myocarde ?
\end{description}
\end{exo}
\vskip1em

\begin{exo}\label{exo:9.5}
{\footnotesize Identique à l'énoncé 4.1 (p.~\pageref{exo:4.1}), questions
  a–d.} 

Dans une étude sur le gène du récepteur à \oe strogènes, des généticiens se
sont intéressés à la relation entre le génotype et l'âge de diagnostic du
cancer du sein. Le génotype était déterminé à partir des deux allèles d'un
polymorphisme de restriction de séquence (1.6 et 0.7 kb), soit trois groupes
de sujets : patients homozygotes pour l'allèle 0.7 kb (0.7/0.7), patients
homozygotes pour l'allèle 1.6 kb (1.6/1.6), et patients hétérozygotes
(1.6/0.7). Les données ont été recueillies sur 59 patientes atteintes d'un
cancer du sein, et sont disponibles dans le fichier
\texttt{polymorphism.dta} (fichier \Stata). Les données moyennes sont
indiquées ci-dessous :\autocite[p.~327]{dupont09}
\vskip1em

\begin{tabular}{lrrrr}
\toprule
& \multicolumn{3}{c}{Génotype} & \\
\cmidrule(r){2-4}
& 1.6/1.6 & 1.6/0.7 & 0.7/0.7 & Total \\
\midrule
Nombre de patients & 14 & 29 & 16 & 59 \\
\emph{Âge lors du diagnostic} & & & & \\
\quad Moyenne & 64.64 & 64.38 & 50.38 & 60.64 \\
\quad Écart-type & 11.18 & 13.26 & 10.64 & 13.49 \\
\quad IC 95~\% & (58.1–71.1) & (59.9–68.9) & (44.3–56.5) & \\
\bottomrule
\end{tabular}
\vskip1em

\begin{description}
\item[(a)] Tester l'hypothèse nulle selon laquelle l'âge de diagnostic ne varie
  pas selon le génotype à l'aide d'une ANOVA. Représenter sous forme
  graphique la distribution des âges pour chaque génotype.
\item[(b)] Les intervalles de confiance présentés dans le tableau ci-dessus ont
  été estimés en supposant l'homogénéité des variances, c'est-à-dire en
  utilisant l'estimé de la variance commune ; donner la valeur de ces
  intervalles de confiance sans supposer l'homoscédasticité. 
\item[(c)] Estimer les différences de moyenne correspondant à l'ensemble des
  combinaisons possibles des trois génotypes, avec une estimation de
  l'intervalle de confiance à 95~\% associé et un test paramétrique
  permettant d'évaluer le degré de significativité de la différence
  observée.
\item[(d)] Représenter graphiquement les moyennes de groupe avec des
  intervalles de confiance à 95~\%.
\end{description}
\end{exo}
\vskip1em

\begin{exo}\label{exo:9.6}
{\footnotesize Identique à l'énoncé 4.2 (p.~\pageref{exo:4.2}), questions
  a–c.}

On a mesuré en fin de traitement chez 18 patients répartis par tirage au
sort en trois groupes de traitement A, B, et C, un paramètre biologique dont
on sait que la distribution est normale. Les résultats sont les suivants :
\vskip1em

\begin{tabular}{ccc}
\toprule
A & B & C \\
\midrule
19.8 & 15.9 & 15.4 \\
20.5 & 19.7 & 17.1 \\
23.7 & 20.8 & 18.2 \\
27.1 & 21.7 & 18.5 \\
29.6 & 22.5 & 19.3 \\
29.9 & 24.0 & 21.2 \\
\bottomrule
\end{tabular}
\vskip1em

\begin{description}
\item[(a)] Réaliser une ANOVA à un facteur.
\item[(b)] Selon le résultat du test, procéder aux comparaisons par paire de
  traitement des moyennes, en appliquant une correction simple de Bonferroni
  (c'est-à-dire où les degrés de significativité estimé sont multipliés par
  le nombre de comparaisons effectuées). Comparer avec de simples tests de
  Student non corrigés pour les comparaisons multiples. 
\item[(c)] D'après des études plus récentes, il s'avère que la normalité des
  distributions parentes peut-être remise en question. Effectuer la
  comparaison des trois groupes par une approche non-paramétrique.
\end{description}
\end{exo}
\vskip1em

\begin{exo}\label{exo:9.7}
{\footnotesize Identique à l'énoncé 4.3 (p.~\pageref{exo:4.3}), questions
  a–e.}

Un service d'obstétrique s'intéresse au poids de nouveaux-nés nés à terme et
âgés de 1 mois. Pour cet échantillon de 550 bébés, on dispose également
d'une information concernant la parité (nombre de frères et soeurs), mais on
sait qu'il n'y aucune relation de gemellité parmi les enfants ayant des
frères et soeurs. L'objet de l'étude est de déterminer si la parité (4
classes) influence le poids des nouveaux-nés à 1 mois. Les données sont
résumées dans le tableau suivant, et elles sont disponibles dans un fichier
SPSS, \texttt{weights.sav}.\autocite[p.~113]{peat05}
\vskip1em

\begin{tabular}{lrrrrr}
\toprule
& \multicolumn{4}{c}{Nombre de frères et soeurs} & Total \\
& 0 & 1 & 2 & $\ge 3$ & \\
\midrule
\emph{Échantillon} & & & & \\ 
Effectif & 180 & 192 & 116 & 62 & 550 \\
Fréquence & 32.7 & 34.9 & 21.1 & 11.3 & 100.0 \\
\emph{Poids (kg)} & & & & \\
Moyenne & 4.26 & 4.39 & 4.46 & 4.43 & \\
Écart-type & 0.62 & 0.59 & 0.61 & 0.54 & \\
(Min–Max) & (2.92–5.75) & (3.17–6.33) & (3.09–6.49) & (3.20–5.48) & \\
\bottomrule
\end{tabular}
\vskip1em

\begin{description}
\item[(a)] Vérifier les données reportées dans le tableau précédent.
\item[(b)] Procéder à une analyse de variance à un facteur. Conclure sur la
  significativité globale et indiquer la part de variance expliquée par le
  modèle.
\item[(c)] Afficher la distribution des poids selon la parité. Procéder à un
  test d'homogénéité des variances (rechercher dans l'aide en ligne le test
  de Levenne). 
\item[(d)] On décide de regrouper les deux dernières catégories (2 et $\ge
  3$). Refaire l'analyse et comparer aux résultats obtenus en (b).
\item[(e)] Réaliser un test de tendance linéaire (par ANOVA) sur les données
  recodées en trois niveaux pour la parité.
\end{description}
\end{exo}

%--------------------------------------------------------------- Devoir 08 ---
% \chapter*{Devoir \no 8}
% \addcontentsline{toc}{chapter}{Devoir \no 8}

% Les exercices sont indépendants. Une seule réponse est correcte pour chaque
% question. Lorsque vous ne savez pas répondre, cochez la case correspondante.

% \section*{Exercice 1}

% \section*{Exercice 2}

% \section*{Exercice 3}


\includepdf[pages={1-5}, fitpaper=true]{./devoir_sas/sas_devoir08f.pdf}

%--------------------------------------------------------------- Chapter 10 ---
\chapter{Régression linéaire et logistique}

\begin{exo}\label{exo:10.1}
{\footnotesize Identique à l'énoncé 5.1 (p.~\pageref{exo:5.1}), questions
  a–e.}

Une étude a porté sur une mesure de malnutrition chez 25 patients âgés de 7
à 23 ans et souffrant de fibrose kystique. On disposait pour ces patients de
différentes informations relatives aux caractéristiques antropométriques
(taille, poids, etc.) et à la fonction pulmonaire. \autocite[p.~180]{everitt01}
Les données sont disponibles dans le fichier \texttt{cystic.dat}.
\begin{description}
\item[(a)] Calculer le coefficient de corrélation linéaire entre les
  variables \texttt{PEmax} et \texttt{Weight}, ainsi que son intervalle de
  confiance à 95~\%.
\item[(b)] Tester si le coefficient de corrélation calculé en (a) peut être
  considéré comme significativement différent de 0.3 au seuil 5~\%.
\item[(c)] Afficher l'ensemble des données numériques sous forme de
  diagrammes de dispersion, soit 45 graphiques arrangés sous forme d'une
  "matrice de dispersion".
\item[(d)] Calculer l'ensemble des corrélations de Pearson et de Spearman
  entre les variables numériques. 
  %Reporter les coefficients de
  %Bravais-Pearson supérieurs à 0.7 en valeur absolue.
\item[(e)] Calculer la corrélation entre \texttt{PEmax} et \texttt{Weight},
  en contrôlant l'âge (\texttt{Age}) (corrélation partielle). Représenter
  graphiquement la covariation entre \texttt{PEmax} et \texttt{Weight} en
  mettant en évidence les deux terciles les plus extrêmes pour la variable
  \texttt{Age}. 
\end{description}
\end{exo}
\vskip1em

\begin{exo}\label{exo:10.2}
{\footnotesize Identique à l'énoncé 5.2 (p.~\pageref{exo:5.2}), questions
  a–e.}

Les données disponibles dans le fichier \texttt{quetelet.csv} renseignent
sur la pression artérielle systolique (\texttt{PAS}), l'indice de Quetelet
(\texttt{QTT}), l'âge (\texttt{AGE}) et la consommation de tabac
(\texttt{TAB}=1 si fumeur, 0 sinon) pour un échantillon de 32 hommes de plus
de 40 ans. 
\begin{description}
\item[(a)] Indiquer la valeur du coefficient de corrélation linéaire entre
  la pression artérielle systolique et l'indice de Quetelet, avec un
  intervalle de confiance à 90~\%.
\item[(b)] Donner les estimations des paramètres de la droite de régression
  linéaire de la pression artérielle sur l'indice de Quetelet.
\item[(c)] Tester si la pente de la droite de régression est différente de 0
  (au seuil 5~\%).
\item[(d)] Représenter graphiquement les variations de pression artérielle
  en fonction de l'indice de Quetelet, en faisant apparaître distinctement
  les fumeurs et les non-fumeurs avec des symboles ou des couleurs
  différentes, et tracer la droite de régression dont les paramètres ont été
  estimés en (b). 
\item[(e)] Refaire l'analyse (b-c) en restreignant l'échantillon aux
  fumeurs.
\end{description}
\end{exo}
\vskip1em

\begin{exo}\label{exo:10.3}
{\footnotesize Identique à l'énoncé 5.3 (p.~\pageref{exo:5.3}), questions
  a–d.}

Dans l'étude Framingham, on dispose de donnée sur la pression artérielle
systolique (\texttt{sbp}) et l'indice de masse corporelle (\texttt{bmi}) de
2047 hommes et 2643 femmes.\autocite[p.~63]{dupont09} On s'intéresse à la
relation entre ces deux variables (après transformation logarithmique) chez
les hommes et chez les femmes séparément.
Les données sont disponibles dans le fichier \texttt{Framingham.csv}.
\begin{description}
\item[(a)] Représenter graphiquement les variations entre pression
  artérielle et IMC (\texttt{bmi}) chez les hommes et chez les femmes.
\item[(b)] Les coefficients de corrélation linéaire estimés chez les hommes
  et chez les femmes sont-ils significativement différents à 5~\% ?
\item[(c)] Estimer les paramètres du modèle de régression linéaire
  considérant la pression artérielle comme variable réponse et l'IMC comme
  variable explicative, pour ces deux sous-échantillons. Donner un
  intervalle de confiance à 95~\% pour l'estimé des pentes respectives.
\item[(d)] Tester l'égalité des deux coefficients de régression associés à
  la pente (au seuil 5~\%).
\end{description}
\end{exo}
\vskip1em

\begin{exo}\label{exo:10.4}
{\footnotesize Identique à l'énoncé 6.1 (p.~\pageref{exo:6.1}) et répondre
  aux questions a–d.} 

On étudie l'effet d'un traitement prophylactique d'un macrolide à faibles
doses (Traitement A) sur les épisodes infectieux chez des patients atteints
de mucoviscidose dans un essai randomisé multicentrique contre placebo
(B). Les résultats sont les suivants :
\vskip1em

\begin{tabular}{lccc}
\toprule
& \multicolumn{2}{c}{Infection} & \\
\cmidrule(r){2-3}
& Non & Oui & Total \\
\midrule
Traitement (A) & 157 & 52 & 209 \\
Placebo (B) & 119 & 103 & 222 \\
Total & 276 & 155 & 431 \\
\bottomrule
\end{tabular}
\vskip1em

\begin{description}
\item[(a)] À partir d'un test du $\chi^2$, que peut-on répondre à la
  question : le traitement permet-il de prévenir la survenue d'épisodes
  infectieux (au seuil $\alpha=0.05$) ? Vérifier que les effectifs
  théoriques sont bien tous supérieurs à 5.
\item[(b)] Conclut-on de la même manière à partir de l'intervalle de
  confiance de l'odds-ratio associé à l'effet traitement ?
\item[(c)] On souhaite vérifier s'il existe une disparité du point de vue
  des pourcentages d'épisodes infectieux en fonction du centre. Les données
  par centre sont indiquées dans le tableau ci-après. Conclure à partir d'un
  test du $\chi^2$.

  \begin{table}[!htb] \hskip40pt
  \begin{minipage}[b]{0.33\linewidth}
  \scalebox{0.65}{\begin{tabular}{|l|r|r|r|}
    \multicolumn{1}{c}{} & \multicolumn{2}{c}{Infection} &  \multicolumn{1}{c}{} \\
    \cline{2-4}
    \multicolumn{1}{c|}{} & Non & Oui & Total \\
    \hline
    Traitement (A) & 51 & 8 & 59 \\
    \hline
    Placebo (B) & 47 & 19 & 66 \\
    \hline
    Total & 98 & 27 & 125 \\
    \hline
    \multicolumn{4}{c}{Centre 1}
  \end{tabular}} 
  \end{minipage} \hspace{0.1cm}
  \begin{minipage}[b]{0.3\linewidth}
  \scalebox{0.65}{\begin{tabular}{|l|r|r|r|}
    \multicolumn{1}{c}{} & \multicolumn{2}{c}{Infection} &  \multicolumn{1}{c}{} \\
    \cline{2-4}
    \multicolumn{1}{c|}{} & Non & Oui & Total \\
    \hline
    Traitement (A) & 91 & 35 & 126 \\
    \hline
    Placebo (B) & 61 & 71 & 132 \\
    \hline
    Total & 152 & 106 & 258 \\
    \hline
    \multicolumn{4}{c}{Centre 2}
  \end{tabular}} 
  \end{minipage} \hspace{0.1cm}
  \begin{minipage}[b]{0.3\linewidth}
  \scalebox{0.65}{\begin{tabular}{|l|r|r|r|}
    \multicolumn{1}{c}{} & \multicolumn{2}{c}{Infection} &  \multicolumn{1}{c}{} \\
    \cline{2-4}
    \multicolumn{1}{c|}{} & Non & Oui & Total \\
    \hline
    Traitement (A) & 15 & 9 & 24 \\
    \hline
    Placebo (B) & 11 & 13 & 24 \\
    \hline
    Total & 26 & 22 & 48 \\
    \hline
    \multicolumn{4}{c}{Centre 3}
  \end{tabular}}
  \end{minipage}
  \end{table}
\item[(d)] À partir du tableau précédent, on cherche à vérifier si l'effet
  traitement est indépendent du centre ou non. On se propose de réaliser un
  test de comparaison entre les deux traitements ajustés sur le centre (test
  de Mantel-Haenszel). Indiquer le résultat du test ainsi que la valeur de
  l'odds-ratio ajusté.
\end{description}
\end{exo}
\vskip1em

\begin{exo}\label{exo:10.5}
{\footnotesize Identique à l'énoncé 6.3 (p.~\pageref{exo:6.3}), questions
  a–e.}

On dispose de données issues d'une étude cherchant à établir la validité
pronostique de la concentration en créatine kinase dans l'organisme sur la
prévention de la survenue d'un infarctus du myocarde.\autocite[p.~115]{rabe-hesketh04}

Les données sont disponibles dans le fichier \texttt{sck.dat} : la première
colonne correspond à la variable créatine kinase (\texttt{ck}), la deuxième
à la variable présence de la maladie (\texttt{pres}) et la dernière à la
variable absence de maladie (\texttt{abs}).
\begin{description}
\item[(a)] Quel est le nombre total de sujets ?
\item[(b)] Calculer les fréquences relatives malades/non-malades, et
  représenter leur évolution en fonction des valeurs de créatine kinase à
  l'aide d'un diagramme de dispersion (points + segments reliant les points).
\item[(c)] À partir d'un modèle de régression logistique dans lequel on
  cherche à prédire la probabilité d'être malade, calculer la valeur de
  \texttt{ck} à partir de laquelle ce modèle prédit que les personnes
  présentent la maladie en considérant une valeur seuil de 0.5
  (si $P(\text{malade})\ge 0.5$ alors \texttt{malade=1}).
\item[(d)] Représenter graphiquement les probabilités d'être malade prédites
  par ce modèle ainsi que les proportions empiriques en fonction des valeurs
  \texttt{ck}. 
\item[(e)] Établir la courbe ROC correspondante, et reporter la
  valeur de l'aire sous la courbe. 
\end{description}
\end{exo}
\vskip1em

\begin{exo}\label{exo:10.6}
{\footnotesize Identique à l'énoncé 6.5 (p.~\pageref{exo:6.5}), questions
  a–d.} 

Une enquête cas-témoin a porté sur la relation entre la consommation
d'alcool et de tabac et le cancer de l'oesophage chez l'homme (étude "Ille
et Villaine"). Le groupe des cas était composé de 200 patients atteints d'un
cancer de l'oesophage et diagnostiqué entre janvier 1972 et avril 1974. Au
total, 775 témoins de sexe masculin ont été sélectionnés à partir des listes
électorales. Le tableau suivant indique la répartition de l'ensemble des
sujets selon leur consommation journalière d'alcool, en considérant qu'une
consommation supérieure à 80 g est considérée comme un facteur de
risque.\autocite{breslow80} 
\vskip1em

\begin{tabular}{lccc}
\toprule
& \multicolumn{2}{c}{Consommation d'alcool (g/jour)} & \\
\cmidrule(r){2-3}
& $\ge 80$ & $<80$ & Total \\
\midrule
Cas & 96 & 104 & 200 \\
Témoins & 109 & 666 & 775 \\
Total & 205 & 770 & 975 \\
\bottomrule
\end{tabular}
\vskip1em

\begin{description}
\item[(a)] Quelle est la valeur de l'odds-ratio et son intervalle de
  confiance à 95~\% (méthode de Woolf) ? Est-ce une bonne estimation du
  risque relatif ? 
\item[(b)] Est-ce que la proportion de consommateurs à risque est la même
  chez les cas et chez les témoins (considérer $\alpha=0.05$) ?
\item[(c)] Construire le modèle de régression logistique permettant de
  tester l'association entre la consommation d'alcool et le statut des
  sujets. Le coefficient de régression est-il significatif ?
\item[(d)] Retrouvez la valeur de l'odds-ratio observé, calculé en (b), et
  son intervalle de confiance à partir des résultats de l'analyse de
  régression.
\end{description}
\end{exo}

%--------------------------------------------------------------- Devoir 09 ---
% \chapter*{Devoir \no 9}
% \addcontentsline{toc}{chapter}{Devoir \no 9}

% Les exercices sont indépendants. Une seule réponse est correcte pour chaque
% question. Lorsque vous ne savez pas répondre, cochez la case correspondante.

% \section*{Exercice 1}

% \section*{Exercice 2}

% \section*{Exercice 3}


\includepdf[pages={1-5}, fitpaper=true]{./devoir_sas/sas_devoir09f.pdf}

%--------------------------------------------------------------- Chapter 11 ---
\chapter{Analyse de données de survie}

\begin{exo}\label{exo:11.1}
{\footnotesize Identique à l'énoncé 7.1 (p.~\pageref{exo:7.1}), questions
  a–i.}

Dans un essai contre placebo sur la cirrhose biliaire, la D-penicillamine
(DPCA) a été introduite dans le bras actif sur une cohorte de 312
patients. Au total, 154 patients ont été randomisés dans le bras actif
(variable traitement, \texttt{rx}, 1=Placebo, 2=DPCA). Un ensemble de
données telles que l'âge, des données biologiques et signes cliniques variés
incluant le niveau de bilirubine sérique (\texttt{bilirub}) sont disponibles
dans le fichier \texttt{pbc.txt}.\autocite{vittinghoff05} Le status du
patient est enregistré dans la variable \texttt{status} (0=vivant, 1=décédé)
et la durée de suivi (\texttt{years}) représente le temps écoulé en années
depuis la date de diagnostic.
\begin{description}
\item[(a)] Combien dénombre-t-on d'individus décédés? Quelle proportion de
  ces décès retrouve-t-on dans le bras actif ?  
\item[(b)] Afficher la distribution des durées de suivi des 312 patients, en
  faisant apparaître distinctement les individus décédés. Calculer le temps
  médian (en années) de suivi pour chacun des deux groupes de
  traitement. Combien y'a-t-il d'événements positifs au-delà de 10.5 années
  et quel est le sexe de ces patients ?
\item[(c)] Les 19 patients dont le numéro (\texttt{number}) figure parmi la
  liste suivante ont subi une transplantation durant la période de suivi.
\begin{verbatim}  
5 105 111 120 125 158 183 241 246 247 254 263 264 265 274 288 291
295 297 345 361 362 375 380 383
\end{verbatim}   
  Indiquer leur âge moyen, la distribution selon le sexe et la durée médiane
  de suivi en jours jusqu'à la transplantation.
\item[(d)] Afficher un tableau résumant la distribution des événements à
  risque en fonction du temps, avec la valeur de survie associée.
\item[(e)] Afficher la courbe de Kaplan-Meier avec un intervalle de
  confiance à 95~\%, sans considérer le type de traitement.
\item[(f)] Calculer la médiane de survie et son intervalle de confiance à
  95~\% pour chaque groupe de sujets et afficher les courbes de survie
  correspondantes.
\item[(g)] Effectuer un test du log-rank en considérant comme prédicteur le
  facteur \texttt{rx}. Comparer avec un test de Wilcoxon.
\item[(h)] Effectuer un test du log-rank sur le facteur d'intérêt
  (\texttt{rx}) en stratifiant sur l'âge. On considèrera trois groupe
  d'âge : 40 ans ou moins, entre 40 et 55 ans inclus, plus de 55 ans.
\item[(i)] Retrouver les résultats de l'exercice 1.g avec une régression de
  Cox. 
\end{description}
\end{exo}
\vskip1em

\begin{exo}\label{exo:11.2}
{\footnotesize Identique à l'énoncé 7.3 (p.~\pageref{exo:7.3}), questions
  a–d}.

Dans un essai randomisé, on a cherché à comparer deux traitements pour le
cancer de la prostate. Les patients prenaient chaque jour par voie orale
soit 1 mg de diethylstilbestrol (DES, bras actif) soit un placebo, et le
temps de survie est mesuré en mois.\autocite{collett94} La question
d'intérêt est de savoir si la survie diffère entre les deux groupes de
patients, et on négligera les autres variables présentes dans le fichier de
données \texttt{prostate.dat}. 
\begin{description}
\item[(a)] Calculer la médiane de survie pour l'ensemble des patients, et
  par groupe de traitement.
\item[(b)] Quelle est la différence entre les proportions de survie dans les
  deux groupes à 50 mois ?
\item[(c)] Afficher les courbes de survie pour les deux groupes de patients.
\item[(d)] Effectuer un test du log-rank pour tester l'hypothèse selon
  laquelle le traitement par DES a un effet positif sur la survie des
  patients. 
\end{description}
\end{exo}

%--------------------------------------------------------------- Devoir 10 ---
% \chapter*{Devoir \no 10}
% \addcontentsline{toc}{chapter}{Devoir \no 10}

% Les exercices sont indépendants. Une seule réponse est correcte pour chaque
% question. Lorsque vous ne savez pas répondre, cochez la case correspondante.

% \section*{Exercice 1}

% \section*{Exercice 2}


\includepdf[pages={1-3}, fitpaper=true]{./devoir_sas/sas_devoir10f.pdf}
