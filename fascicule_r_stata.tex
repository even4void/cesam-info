\documentclass[11pt]{report}
\usepackage[francais]{babel}
\usepackage{amsmath,amsfonts,amsthm}
\usepackage{fancyhdr}
\usepackage[framemethod=TikZ]{mdframed}
\usepackage{url}
\usepackage{calc}
\usepackage{titlesec}
\usepackage{booktabs}
\usepackage{parskip}
\usepackage{hyperref,wasysym}
\usepackage{multirow}
\usepackage{dcolumn}
\newcolumntype{.}{D{.}{.}{-1}}
\newcolumntype{/}{D{/}{/}{-1}}
\usepackage{xspace}
\usepackage{wasysym}
\usepackage{enumitem}  % iterable item list
\usepackage{fancyvrb}
\usepackage{pdfpages}

\usepackage[noae,nogin]{Sweave}


% bibliography stuff
\usepackage[style=verbose-trad2,natbib=true,backend=bibtex]{biblatex}
\bibliography{refs}

% Xelatex setup
\usepackage{fontspec,xltxtra,xunicode}
%\usepackage[math-style=ISO,bold-style=ISO]{unicode-math}
\usepackage[vargreek-shape=unicode]{unicode-math} %controls which shape of
                                        %epsilon and phi are considered default
\unimathsetup{math-style=french} % uses so called french style italic lc
                                 % latin, upright everything else
\setmathfont{Latin Modern Math} % this is obligatory or no math font will be
                                % set...
\defaultfontfeatures{Scale=MatchLowercase}
\setmainfont{Myriad Pro}
%\setsansfont{Myriad Pro}
%\setmathfont[Numbers=OldStyle]{Asana Math}
%\setmonofont[Scale=0.86]{Inconsolata}
\setmonofont[Scale=0.71]{Menlo}

% page layout
\usepackage[left=2cm,bottom=2cm,right=2cm]{geometry}

\usepackage[french=guillemets*]{csquotes} 
\MakeOuterQuote{"} 

% we now need the following two import statements with TeXLive 2012 (updated
% Sep. 2012) -- don't know why.
\usepackage{xcolor}
\usepackage{textcomp}
% custom color for R code chunks
\definecolor{Sinput}{rgb}{0.3,0.3,0.3}
\definecolor{Soutput}{rgb}{0.5,0.5,0.5}
\definecolor{Scode}{rgb}{0,0.42,0.33}

% add line break when using paragraph sectioning
\makeatletter
\renewcommand\paragraph{\@startsection{paragraph}{4}{\z@}%
  {-3.25ex\@plus -1ex \@minus -.2ex}%
  {1.5ex \@plus .2ex}%
  {\normalfont\normalsize\bfseries}}

\renewcommand\part{%
  \if@openright
  \cleardoublepage
  \else
  \clearpage
  \fi
  \thispagestyle{empty}%   % Original »plain« replaced by »emptyx
  \if@twocolumn
  \onecolumn
  \@tempswatrue
  \else
  \@tempswafalse
  \fi
  \null\vfil
  \secdef\@part\@spart}
\makeatother

\setlength{\headheight}{0.3in}
\setlength{\headwidth}{\textwidth}

\addto\captionsfrancais{%
  \renewcommand\chaptername{Semaine}}
\titleformat{\chapter}
  {\normalfont\huge\bfseries}{\chaptertitlename\ \thechapter.}{20pt}{\huge}

\setlist[enumerate]{label=\fbox{\textbf{\arabic{chapter}.\arabic*}}}

% Manage solutions
\usepackage{answers}   %[nosolutionfiles]
\Newassociation{sol}{Solution}{solutions}
\renewcommand{\Solutionlabel}[1]{\fbox{#1}}
\newcommand{\soln}[1]{\vskip1ex\par\noindent\fbox{#1}\quad}

\theoremstyle{definition}
\newtheorem{exo}{Exercice}[chapter]
\newcommand{\R}{\textsf{R}\xspace}
\newcommand{\Stata}{\textsf{Stata}\xspace}
\newcommand{\SAS}{\textsf{SAS}\xspace}
\SetLabelAlign{parright}{\parbox[t]{\labelwidth}{\raggedleft#1}}

\setlist[description]{style=multiline,topsep=10pt,leftmargin=35pt,font=\normalfont,%
    align=parright}


% index
\usepackage{multicol}
\usepackage{makeidx}
\addto\captionsfrancais{%
\renewcommand*\indexname{Liste des commandes \R}}
\newcommand{\foo}[1]{\texttt{#1}}
\newcommand{\cmd}[1]{\index{#1@\foo{#1}}}
\newcommand{\blankpage}{
  \newpage
  \thispagestyle{empty}
  \mbox{}
  \newpage
  }

\mdfdefinestyle{titlep}{innerleftmargin=20,innerrightmargin=20,innertopmargin=20,innerbottommargin=20,roundcorner=10pt}
  
\begin{document}
\thispagestyle{empty}
\centerline{\small Centre d'Enseignement de la Statistique Appliquée, à la Médecine et à la Biologie Médicale}
\vspace*{2cm}
\begin{center}
\centerline{\includegraphics[scale=.55]{cesam}}
\vspace*{2cm}
\begin{minipage}{.75\textwidth}
\begin{mdframed}[style=titlep]
\centerline{\Huge Programme de travail}
\vskip1em
\centerline{\Huge du cours d'informatique du CESAM}
\end{mdframed}
\end{minipage}
\end{center}
\vskip3em
\centerline{\Huge\bf Introduction au logiciel R}
\vskip5em
\begin{center}
  \begin{tabular}{ll}
    \textbf{Responsables :} & \\
    Christophe LALANNE & \url{christophe.lalanne@inserm.fr} \\
    Yassin MAZROUI     & \url{yassin.mazroui@upmc.fr} \\
    Pr Mounir MESBAH   & \url{mounir.mesbah@upmc.fr}
  \end{tabular}
\end{center}
\vskip3em
\centerline{\Large \url{http://www.cesam.upmc.fr}}
\vskip3em
\centerline{\LARGE Année Universitaire 2015–2016}
\vfill
\begin{center}
\begin{minipage}{.6\textwidth}
\centering
Adresser toute correspondance à :\\
Université Pierre et Marie Curie – Paris 6
Secrétariat du CESAM – Les Cordeliers
Service Formation Continue, esc. B, 4ème étage,
15 rue de l’école de médecine,
75006 PARIS\\
ou par Courriel à : \url{cesam@upmc.fr}
\end{minipage}
\end{center}  

\blankpage

% Sweave custom outputs
% could use R CMD pgfsweave --pgfsweave-only $< but it's too much color
\DefineVerbatimEnvironment{Sinput}{Verbatim}
{formatcom = {\color{Sinput}}} 
\DefineVerbatimEnvironment{Soutput}{Verbatim}
{formatcom = {\color{Soutput}},fontsize=\small}
\DefineVerbatimEnvironment{Scode}{Verbatim}
{formatcom = {\color{Sinput}}} 
\fvset{listparameters={\setlength{\topsep}{0pt}}}
\renewenvironment{Schunk}{\vspace{\topsep}}{\vspace{\topsep}}
%\renewcommand{\texttt}[1]{\textcolor{Sinput}{#1}}
\setkeys{Gin}{width=.4\textwidth}



\chapter*{Calendrier}
\thispagestyle{empty}
\vskip3em


\begin{center}
\resizebox{\linewidth}{!}{\begin{tabular}{|l|p{10cm}|l|l|l|}
\hline
  \multicolumn{5}{|c|}{Introduction au logiciel R} \\
\hline
  Sem. 08/02 & Éléments du langage, gestion de données & Cours 1 & Corrigés
  pp.~\pageref{start:sol1}–\pageref{stop:sol1} & \\
  Sem. 15/02 & Statistiques descriptives et estimation & Cours 2 & Corrigés
  pp.~\pageref{start:sol2}–\pageref{stop:sol2} & Devoir \no 1\\
  Sem. 07/03 & Comparaisons de deux variables & Cours 3 & Corrigés
  pp.~\pageref{start:sol3}–\pageref{stop:sol3} & Devoir \no 2\\
  Sem. 14/03 & Analyse de variance et plans d'expérience & Cours 4 &
  Corrigés pp.~\pageref{start:sol4}–\pageref{stop:sol4} & Devoir \no 3\\
  Sem. 21/03 & Corrélation et régression linéaire & Cours 5 & Corrigés
  pp.~\pageref{start:sol5}–\pageref{stop:sol5} & Devoir \no 4\\
  Sem. 28/03 & Mesures d'association en épidémiologie et régression
  logistique & Cours 6 & Corrigés
  pp.~\pageref{start:sol6}–\pageref{stop:sol6} & Devoir \no 5\\
  Sem. 04/04 & Données de survie & Cours 7 & Corrigés
  pp.~\pageref{start:sol7}–\pageref{stop:sol7} & Devoir \no 6\\
\hline
\end{tabular}}
\end{center}

\chapter*{Organisation}
\setcounter{page}{1}

Une bonne partie des actions effectuées à partir d'un logiciel statistique
revient à manipuler, voire transformer des données numériques représentant
des données statistiques au sens propre. Il est donc primordial de bien
comprendre comment sont représentées de telles données dans un logiciel tel
que \R ou \Stata. La description des variables d'intérêt et le résumé de leur
distribution sous forme numérique et graphique constitue une étape préalable
et fondamentale à toute modélisation statistique, d'où l'importance de ces
premières étapes dans le déroulement d'un projet d'analyse statistique. Dans
un second temps, il est essentiel de bien maîtriser les commandes permettant
de calculer les principales mesures d'association en recherche médicale et
de construire les modèles explicatifs et prédictifs classiques : analyse de
variance, régression linéaire et logistique et régression de Cox. À quelques
exceptions près, on préfèrera recourir aux commandes \R et \Stata disponibles
lors de l'installation du logiciel (commandes de base), plutôt qu'à des
packages ou librairies spécialisées de commandes. Des éléments de
comparaison entre les commandes de base et d'autres commandes plus
spécifiques, ainsi que des pointeurs vers des ressources bibliographiques ou
en ligne, sont fournis dans les chapitres de cours.

Dans la \textbf{première séance}, on introduira les commandes de base pour
la gestion de données sous \R. Il s'agit principalement de la création et de
la manipulation de variables quantitatives et qualitatives (recodage de
valeurs individuelles, comptage des observations manquantes), de
l'importation de bases de données stockées sous forme de fichiers texte,
ainsi que d'opérations arithmétiques élémentaires (minimum, maximum, moyenne
arithmétique, différence, fréquence, etc.). On verra également comment
sauvegarder des bases de données pré-traitées au format texte ou
\R. L'objectif est de comprendre comment les données sont représentées sous
\R, et comment travailler à partir de celles-ci.

La \textbf{deuxième séance} porte sur les commandes utiles pour la
description d'un tableau de données constitué de variables quantitatives ou
qualitatives. L'approche descriptive est strictement univariée, ce qui
constitue le préalable à toute démarche statistique. Les commandes
graphiques de base (histogramme, courbe de densité, diagramme en barres)
seront présentées en complément des résumés descriptifs numériques usuels de
tendance centrale (moyenne, médiane) et de dispersion (variance,
quartiles). On abordera également l'estimation ponctuelle et par intervalle
à l'aide d'une moyenne arithmétique et d'une proportion empirique.
L'objectif est de se familiariser avec l'emploi de commandes \R simples
opérant sur une variable, éventuellement en précisant certaines options pour
le calcul, et à la sélection d'unités statistiques parmi l'ensemble des
observations disponibles.

La \textbf{troisième séance} est consacrée à la comparaison de deux
échantillons, pour des mesures quantitatives ou qualitatives. On aborde les
tests d'hypothèse suivants : test de Student pour échantillons indépendants
ou appariés, test non-paramétrique de Wilcoxon, test du $\chi^2$ et test
exact de Fisher, test de NcNemar, à partir des principales mesures
d'association pour deux variables (différence de moyennes, odds-ratio et
risque relatif). À partir de cette séance, on insistera moins sur la
description univariée de chaque variable, mais il est conseillé de toujours
procéder aux étapes de description des données vue lors de la 2\ieme\
séance. L'objectif est de maîtriser les principaux tests statistiques dans
le cas où l'on s'intéresse à la relation entre une variable quantitative et
une variable qualitative, ou pour deux variables qualitatives.

La \textbf{quatrième séance} est une introduction à l'analyse de variance
dans laquelle on cherche à expliquer la variabilité observée au niveau d'une
variable réponse numérique par la prise en compte d'un facteur de groupe ou
de classification et à l'estimation par intervalle de différences de
moyennes. On mettra l'accent sur la construction d'un tableau d'ANOVA
résumant les différentes sources de variabilité et sur les méthodes
graphiques permettant de résumer la distribution des données individuelles
ou aggrégées. On discutera également le test de tendance linéaire lorsque le
facteur de classification peut être considéré comme naturellement
ordonné. L'objectif est de comprendre comment construire un modèle
explicatif dans le cas où l'on a un, voire deux, facteurs explicatifs, et
comment présenter numériquement et graphiquement les résultats d'un tel
modèle à l'aide de \R.

La \textbf{cinquième séance} porte sur l'analyse de la relation linéaire
entre deux variables quantitatives continues. Dans l'approche de corrélation
linéaire, qui suppose une relation symétrique entre les deux variables, on
s'intéressera à quantifier la force et la direction de l'association de
manière paramétrique (corrélation de Bravais-Pearson) ou non-paramétrique
(corrélation de Spearman basée sur les rangs) et à la représentation
graphique de cette relation. La régression linéaire simple sera utilisée
dans le cas où l'une des deux variables numériques joue le rôle d'une
variable réponse et l'autre celui d'une variable explicative. On présentera
les commandes utiles pour l'estimation des coefficients de la droite de
régression, la construction du tableau d'ANOVA associé à la régression, et
la prédiction. L'objectif de la séance reste identique à celui de la 4\ieme\
séance, à savoir présenter les commandes \R nécessaires à la construction
d'un modèle statistique simple entre deux variables, dans une optique
explicative ou prédictive.

Lors de la \textbf{sixième séance} seront abordées les principales mesures
d'association rencontrées dans les études épidémiologiques : odds-ratio,
risque relatif, prévalence, etc. Les commandes \R permettant l'estimation
(ponctuelle et par intervalle) et les tests d'hypothèse associés seront
illustrées sur des données de cohorte ou d'études cas-témoins. La mise en
\oe uvre d'un modèle de régression logistique simple permet de compléter
l'éventail des méthodes statistiques permettant d'expliquer la variabilité
observée au niveau d'une variable réponse binaire. L'objectif est de
comprendre les commandes \R à utiliser dans le cas où les variables sont
binaires, soit pour résumer un tableau de contingence sous forme
d'indicateurs d'association soit pour modéliser la relation entre une
réponse binaire (malade/non-malade) et une variable explicative qualitative
à partir de données dites groupées.

La \textbf{septième séance} est une introduction à l'analyse de données
censurées et au principaux tests (log-rank, Wilcoxon) et modèles de survie
(modèle de Cox). La spécificité des données censurées impose un soin
particulier dans le codage des données sous \R, et l'objectif est de
présenter les commandes \R essentielles à la bonne représentation des
données de survie sous forme numérique, à leur résumé numérique (médiane de
survie) et graphique (courbe de Kaplan-Meier) et à la mise en \oe uvre des
tests courants. 

%% Les \textbf{séances 8 à 11} reprennent essentiellement les mêmes idées, en
%% utilisant \Stata au lieu de \R : importation et recodage de données,
%% description numérique et graphique des variables quantitatives et
%% qualitatives, mesures d'association entre une variable quantitative ou
%% qualitative et une variable qualitative, analyse de la variance, association
%% entre deux variables quantitatives (corrélation linéaire et de rangs) et
%% régression linéaire, régression logistique et analyse de données de survie
%% par le modèle de Cox et le test du log-rank. L'illustration des commandes
%% \Stata permettant de répondre à ces questions est basée sur une partie des
%% exercices des 7 premières séances sert de base, ce qui permet d'éviter
%% d'avoir à se familiariser avec de nouveaux exemples ou cas d'utilisation.

Durant les séances, le travail réalisé sur machine pour répondre aux
questions posées dans les exercices doit être sauvegardé dans un fichier
script R avec un nom permettant de facilement retrouver le travail lors des
séances ultérieures (par exemple, \texttt{semaine01.R}). Il est conseillé
d'enregistrer ce fichier dans un répertoire spécifique, par exemple
\texttt{script}, sur une clé USB. De même, on supposera que toutes les
données utilisées dans les séances sont disponibles dans un répertoire
appelé \texttt{data}, également sur une clé USB.

Après chaque séance (à l'exception de la première), il est proposé un
\textbf{devoir} composé d'une quinzaine de questions et permettant à
l'étudiant de s'auto-évaluer. Les questions se limitent strictement aux
éléments vus en cours et durant les exercices dirigés. Les réponses aux
questions doivent être fournies sur le site web du cours :
\url{http://cesam-informatique.net}. 

\cleardoublepage

%% \part*{Module R}

%% \pagestyle{fancy}
%% \fancyhf{}
%% \fancyhead[L]{\bf\nouppercase{\leftmark}}
%% \fancyhead[R]{\raisebox{-.5ex}{\includegraphics[height=0.2in,width=0.38in]{cesam}}}
%% \fancyfoot[C]{\thepage}

\chapter{Éléments du langage}\label{chap:langage}
\Opensolutionfile{solutions}[solutions1]

\section*{Énoncés}
%
%
%
\begin{exo}\label{exo:1.1}
Un chercheur a recueilli les mesures biologiques suivantes (unités
arbitraires) :
\begin{verbatim}
3.68  2.21  2.45  8.64  4.32  3.43  5.11  3.87
\end{verbatim}
\begin{description}
\item[(a)] Stocker la séquence de mesures dans une variable appelée
  \texttt{x}.  
\item[(b)] Indiquer le nombre d'observations (à l'aide de \R), les valeurs
  minimale et maximale, ainsi que l'étendue.  
\item[(c)] En fait, le chercheur réalise que la valeur 8.64 correspond à une
  erreur de saisie et doit être changée en 3.64. De même, il a un doute sur
  la 7\ieme mesure et décide de la considérer comme une valeur manquante :
  effectuer les transformations correspondantes. 
\end{description}
\begin{sol}
On commence par saisir les données :
\begin{Schunk}
\begin{Sinput}
> x <- c(3.68,2.21,2.45,8.64,4.32,3.43,5.11,3.87)
> x
\end{Sinput}
\begin{Soutput}
[1] 3.68 2.21 2.45 8.64 4.32 3.43 5.11 3.87
\end{Soutput}
\end{Schunk}
Le nombre d'éléments de \texttt{x}, c'est-à-dire le nombre d'observations ou
unités statistiques, s'obtient ainsi :
\begin{Schunk}
\begin{Sinput}
> length(x)
\end{Sinput}
\begin{Soutput}
[1] 8
\end{Soutput}
\end{Schunk}
En ce qui concerne l'étendue de variation des mesures collectées, on peut
utiliser une commande générique, telle que \verb|summary(x)|, ou des
commandes plus spécifiques, comme indiqué ci-dessous :
\begin{Schunk}
\begin{Sinput}
> min(x)
\end{Sinput}
\begin{Soutput}
[1] 2.21
\end{Soutput}
\begin{Sinput}
> max(x)
\end{Sinput}
\begin{Soutput}
[1] 8.64
\end{Soutput}
\begin{Sinput}
> range(x)
\end{Sinput}
\begin{Soutput}
[1] 2.21 8.64
\end{Soutput}
\begin{Sinput}
> range(x)[2] - range(x)[1]  # ou diff(range(x))
\end{Sinput}
\begin{Soutput}
[1] 6.43
\end{Soutput}
\end{Schunk}
\cmd{min}\cmd{max}\cmd{range}
On notera que \verb|range(x)| renvoit deux valeurs, correspondant dans
l'ordre de lecture au minimum et au maximum, de la variable \texttt{x}, ce
qui nous permet de calculer l'étendue comme la différence entre ces deux
valeurs. 

Concernant les transformations suggérées, on peut remplacer la 4\ieme
mesure, \verb|x[4]|, comme ceci : \verb|x[4] <- 3.64|. En fait cela
nécessite de connaître le numéro ou la position de l'observation, ce qui
se révèle peu pratique dans le cas où le nombre d'observations est
grand. Une solution alternative consiste donc à "isoler" l'observation en
question à l'aide de sa valeur (on parlera d'un test logique sur les valeurs
de \texttt{x}) :
\begin{Schunk}
\begin{Sinput}
> x[x == 8.64] <- 3.64
> x
\end{Sinput}
\begin{Soutput}
[1] 3.68 2.21 2.45 3.64 4.32 3.43 5.11 3.87
\end{Soutput}
\end{Schunk}

Enfin, pour recoder la 7\ieme observation en valeur manquante, on procèdera
comme suit :
\begin{Schunk}
\begin{Sinput}
> x[7] <- NA
> x
\end{Sinput}
\begin{Soutput}
[1] 3.68 2.21 2.45 3.64 4.32 3.43   NA 3.87
\end{Soutput}
\end{Schunk}
\cmd{NA}
\end{sol}
\end{exo}
%
%
%
\begin{exo}\label{exo:1.2}
La charge virale plasmatique permet de décrire la quantité de virus (p.~ex.,
VIH) dans un échantillon de sang. Ce marqueur virologique qui permet de
suivre la progression de l’infection et de mesurer l’efficacité des
traitements est rapporté en nombre de copies par millilitre, et la plupart
des instruments de mesure ont un seuil de détectabilité de 50
copies/ml. Voici une série de mesures, $X$, exprimées en logarithmes (base 10)
collectées sur 20 patients :
\begin{verbatim}
3.64 2.27 1.43 1.77 4.62 3.04 1.01 2.14 3.02 5.62 5.51 5.51 1.01 1.05 4.19
2.63 4.34 4.85 4.02 5.92
\end{verbatim}
Pour rappel, une charge virale de 100 000 copies/ml équivaut à 5 log.
\begin{description}
\item[(a)] Indiquer combien de patients ont une charge virale considérée
  comme non-détectable. 
\item[(b)] Quelle est le niveau de charge virale médian, en copies/ml, pour
  les données considérées comme valides ?
\end{description}
\begin{sol}
Avant toute chose, il est nécessaire d'exprimer la limite de détection (50
copies/ml) en logarithmes ; celle-ci vaut en fait
\begin{Schunk}
\begin{Sinput}
> log10(50)
\end{Sinput}
\begin{Soutput}
[1] 1.69897
\end{Soutput}
\end{Schunk}
\cmd{log10}
Ensuite, il suffit de filtrer les observations qui ne remplissent pas la
condition $X>1.70$ (on utilisera le résultat numérique exact, pas la valeur
approchée) :
\begin{Schunk}
\begin{Sinput}
> X <- c(3.64,2.27,1.43,1.77,4.62,3.04,1.01,2.14,3.02,5.62,5.51,5.51,1.01,1.05,4.19,
+        2.63,4.34,4.85,4.02,5.92)
> length(X[X <= log10(50)])
\end{Sinput}
\begin{Soutput}
[1] 4
\end{Soutput}
\end{Schunk}
\cmd{length}

Finalement, la charge virale médiane pour les 16 patients ayant une mesure
considérée comme valide peut se calculer comme suit :
\begin{Schunk}
\begin{Sinput}
> Xc <- X[X > log10(50)]
> round(median(10^Xc), 0)
\end{Sinput}
\begin{Soutput}
[1] 12980
\end{Soutput}
\end{Schunk}
\cmd{median}\cmd{round}
Notons que le résultat ci-dessus est présenté sans décimales.
\end{sol}
\end{exo}
%
%
%
\begin{exo}\label{exo:1.3}
Le fichier \texttt{dosage.txt} contient une série de 15 dosages biologiques,
stockés au format numérique avec 3 décimales, comme suit
\begin{verbatim}
6.379 6.683 5.120 ...
\end{verbatim}
\begin{description}
\item[(a)] Utiliser \verb|scan| pour lire ces données (bien lire l'aide en
  ligne concernant l'usage de cette commande, en particulier l'option
  \texttt{what=}). 
\item[(b)] Corriger la série de mesures afin de pouvoir calculer la moyenne
  arithmétique.   
\item[(c)] Enregistrer les données corrigées dans fichier texte appelé
  \texttt{data.txt}. 
\end{description}
\begin{sol}
Lorsque les données stockées dans un fichier texte ne sont pas trop
volumineuses, il est recommendé d'y jeter un oeil à l'aide d'un simple
éditeur de texte avant de les importer dans \R. Dans le cas présent, il
s'avère qu'il y a eu un problème de codage du séparateur décimal : pour une
des mesures, une virgule a été utilisée à la place d'un point pour séparer
la partie entière de la partie décimale des nombres.

Si l'on se contente de l'instruction suivante
\begin{Schunk}
\begin{Sinput}
> x <- scan("data/dosage.txt")
\end{Sinput}
\end{Schunk}
\cmd{scan}
\R retournera un message d'erreur du type
\begin{verbatim}
Error in scan(file, what, nmax, sep, dec, quote, skip, nlines, na.strings,  : 
  scan() attendait 'a real' et a reçu '2,914'
\end{verbatim}
puisque par défaut \R tente de lire des nombres (au format
anglo-saxon). Pour corriger cela, il faut donc demander à \R 
\begin{Schunk}
\begin{Sinput}
> x <- scan("data/dosage.txt", what="character")
> str(x)
\end{Sinput}
\begin{Soutput}
 chr [1:15] "6.379" "6.683" "5.120" "6.707" "6.149" "5.060" "2,914" "4.261" ...
\end{Soutput}
\begin{Sinput}
> head(x)
\end{Sinput}
\begin{Soutput}
[1] "6.379" "6.683" "5.120" "6.707" "6.149" "5.060"
\end{Soutput}
\end{Schunk}
\cmd{scan}\cmd{str}\cmd{head}

Toutefois, il n'est pas possible de calculer directement la moyenne sur les
données importées de cette manière puisqu'il s'agit de caractères
(\texttt{character}). On corrigera donc l'observation mal enregistrée, avant
de convertir les mesures en nombres exploitables par \R :
\begin{Schunk}
\begin{Sinput}
> x[x == "2,914"] <- "2.914"
> x <- as.numeric(x)
> head(x)
\end{Sinput}
\begin{Soutput}
[1] 6.379 6.683 5.120 6.707 6.149 5.060
\end{Soutput}
\begin{Sinput}
> round(mean(x), 3)
\end{Sinput}
\begin{Soutput}
[1] 4.339
\end{Soutput}
\end{Schunk}
\cmd{as.numeric}\cmd{head}\cmd{round}\cmd{mean}

Enfin, pour sauvegarder les données corrigées dans un nouveau fichier, on
peut utiliser la commande générale \texttt{write.table}, qui possède
beaucoup d'options, mais qui en générale fonctionne de la manière suivante :
\begin{Schunk}
\begin{Sinput}
> write.table(x, file="data/data.txt", row.names=FALSE)
\end{Sinput}
\end{Schunk}
Un aperçu du fichier texte ainsi sauvegardé est fourni ci-dessous :
\begin{verbatim}
"x"
6.379
6.683
5.12
6.707
6.149
5.06
\end{verbatim}
Si l'on ne souhaite pas faire apparaître le nom de la variable sur la
première ligne, on rajoutera l'option \verb|col.names=FALSE|.
\end{sol}
\end{exo}
%
%
%
\begin{exo}\label{exo:1.4}
Lors d'une enquête épidémiologique, les données suivantes ont été collectées
(à partir d'un questionnaire retourné par voie postale) : l'âge (en années),
le sexe (M, masculin, F, féminin, ou T, transgenre), le niveau de QI (score
numérique, positif), le statut socio-économique (variable qualitative à
trois modalités, A, B et C), et un score de qualité de vie (exprimé sur une
échelle allant de 0 à 100 points). Un aperçu des données pour les 10
premiers individus est fourni ci-dessous :
\vskip1em

\begin{tabular}{cccccc}
\toprule
id & age & sexe & qi & sse & qdv \\
\midrule
1 &  26  &  M & 126 &  B & 72 \\
2 &  31  &  F & 123 &  A & 73 \\
3 &  28  &  M & 114 &  B & 72 \\
4 &  28  &  M & 125 &  B & 72 \\
5 &  29  &  F & 134 &  A & 76 \\
6 &  33  &  F & 141 &  B & 74 \\
7 &  32  &  F & 123 &  B & 72 \\
8 &  21  &  M & 114 &  A & 71 \\
9 &  36  &  M & 122 &  C & 71 \\
10 & 30  &  M & 127 &  A & 66 \\
\bottomrule
\end{tabular}
\vskip1em

\begin{description}
  \item[(a)] Créer un \verb|data.frame|, nommé \texttt{dfrm}, pour stocker les données de
ces 10 individus.
  \item[(b)] L'ensemble de la base de données a été enregistré dans
un fichier Excel, puis exportée au format \textsf{CSV} (c'est-à-dire un
fichier texte dans lequel les données de chaque individu sont écrites sur
une même ligne, avec des virgules séparant les valeurs de chaque variable,
encore appelé "champs") sous le nom \texttt{enquete.csv} : charger le
fichier sous \R, et vérifier la concordance des données avec celles créées
au préalable.
\item[(c)] Indiquer la proportion d'hommes et de femmes dans cet échantillon.
\item[(d)] La variable qdv contient-elle des valeurs manquantes ? Si oui,
  combien et pour quels numéros d'observations (lignes du tableau de
  données) ?
\end{description}
\begin{sol}
Commençons par saisir les mesures correspondant à chacune des 6 variables
une par une :
\begin{Schunk}
\begin{Sinput}
> id <- 1:10
> age <- c(26,31,28,28,29,33,32,21,36,30)
> sexe <- c("M","F","M","M","F","F","F","M","M","M")
> qi <- c(126,123,114,125,134,141,123,114,122,127)
> sse <- c("B","A","B","B","A","B","B","A","C","A")
> qdv <- c(72,73,72,72,76,74,72,71,71,66)
> dfrm <- data.frame(id, age, sexe, qi, sse, qdv)
> dfrm
\end{Sinput}
\begin{Soutput}
   id age sexe  qi sse qdv
1   1  26    M 126   B  72
2   2  31    F 123   A  73
3   3  28    M 114   B  72
4   4  28    M 125   B  72
5   5  29    F 134   A  76
6   6  33    F 141   B  74
7   7  32    F 123   B  72
8   8  21    M 114   A  71
9   9  36    M 122   C  71
10 10  30    M 127   A  66
\end{Soutput}
\end{Schunk}
\cmd{data.frame}
Les variables intermédiaires (\texttt{id}, \texttt{age}, etc.) sont toujours
présentes dans l'espace de travail de R (ce que l'on peut vérifier en
tapant la commande \verb|ls()|), et on peut vouloir les supprimer pour
éviter toute confusion lors de leur manipulation puisqu'elles sont à présent
dispnibles directement depuis le \texttt{data.frame} nommé \texttt{dfrm}. On
utilisera la commande \texttt{rm} comme ceci :
\begin{Schunk}
\begin{Sinput}
> rm(id, age, sexe, qi, sse, qdv)
> ls()
\end{Sinput}
\begin{Soutput}
[1] "dfrm"   "ltheme" "x"      "X"      "Xc"    
\end{Soutput}
\end{Schunk}
\cmd{rm}\cmd{ls}
Pour exporter le fichier au format CSV, on utiliserait la commande
\texttt{write.csv} (séparateur de champ = virgule) ou \texttt{write.csv2}
(séparateur de champ = point-virgule).
\begin{Schunk}
\begin{Sinput}
> write.csv(dfrm, file="data/dfrm.csv")
\end{Sinput}
\end{Schunk}
\cmd{write.csv}
Chargeons à présent les données réelles du fichier \texttt{enquete.csv}. 
\begin{Schunk}
\begin{Sinput}
> enquete <- read.csv("data/enquete.csv")
> str(enquete)
\end{Sinput}
\begin{Soutput}
'data.frame':	100 obs. of  6 variables:
 $ id  : int  1 2 3 4 5 6 7 8 9 10 ...
 $ age : int  26 31 28 28 29 33 32 21 36 30 ...
 $ sexe: Factor w/ 3 levels "F","M","T": 2 1 2 2 1 1 1 2 2 2 ...
 $ qi  : int  126 123 114 125 134 141 123 114 122 127 ...
 $ sse : Factor w/ 3 levels "A","B","C": 2 1 2 2 1 2 2 1 3 1 ...
 $ qdv : int  72 73 72 72 76 74 72 71 71 66 ...
\end{Soutput}
\end{Schunk}
\cmd{read.csv}\cmd{str}
La commande \texttt{str} fournit un aperçu du type de chaque variable
(variable numérique ou qualitative) et des premières observations. On voit
ici que les variables \texttt{sexe} et \texttt{sse} sont bien considérées
comme des variables qualitatives (\texttt{factor} sous R). Pour afficher les
10 premières observations, on peut utiliser la commande \texttt{head}, par exemple
\begin{Schunk}
\begin{Sinput}
> head(enquete, n=10)
\end{Sinput}
\begin{Soutput}
   id age sexe  qi sse qdv
1   1  26    M 126   B  72
2   2  31    F 123   A  73
3   3  28    M 114   B  72
4   4  28    M 125   B  72
5   5  29    F 134   A  76
6   6  33    F 141   B  74
7   7  32    F 123   B  72
8   8  21    M 114   A  71
9   9  36    M 122   C  71
10 10  30    M 127   A  66
\end{Soutput}
\end{Schunk}
\cmd{head}
ou alors directement indexer le tableau de données avec les 10 premiers
éléments : 
\begin{Schunk}
\begin{Sinput}
> enquete[1:10,]
\end{Sinput}
\begin{Soutput}
   id age sexe  qi sse qdv
1   1  26    M 126   B  72
2   2  31    F 123   A  73
3   3  28    M 114   B  72
4   4  28    M 125   B  72
5   5  29    F 134   A  76
6   6  33    F 141   B  74
7   7  32    F 123   B  72
8   8  21    M 114   A  71
9   9  36    M 122   C  71
10 10  30    M 127   A  66
\end{Soutput}
\end{Schunk}
Ici, on indique entre crochets les observations qui nous intéressent (1 à
10, le \og à\fg\ se traduisant par \texttt{:} sous R) puis les colonnes
d'intérêt (ici, on ne met rien pour avoir l'ensemble des variables). On peut
ainsi vérifier la bonne concordance des valeurs.

La proportion d'hommes et de femmes peut être obtenue à l'aide de la
commande \texttt{table}, en tenant compte du nombre d'observations que l'on
peut peut calculer comme le nombre de lignes du tableau de données :
\begin{Schunk}
\begin{Sinput}
> table(enquete$sexe) / nrow(enquete)
\end{Sinput}
\begin{Soutput}
   F    M    T 
0.41 0.58 0.01 
\end{Soutput}
\end{Schunk}
\cmd{table}
Notons que l'on aurait très bien pu compter chacune des modalités prises par
la variable \texttt{sexe}, soit \texttt{F} ou \texttt{H}, en se rappelant
que les classes ou modalités d'une variable qualitatives doivent être
entourées d'apostrophes anglo-saxonnes (\og quote\fg).
Par exemple, pour les femmes on pourrait procéder ainsi :
\begin{Schunk}
\begin{Sinput}
> sum(enquete$sexe == "F") / nrow(enquete) 
\end{Sinput}
\begin{Soutput}
[1] 0.41
\end{Soutput}
\end{Schunk}
\cmd{sum}\cmd{nrow}
Cette approche ne fonctionnera que s'il n'y a aucune donnée manquante car la
commande \texttt{nrow} renvoit la taille du tableau de données (nombre de
lignes) indépendemment de son contenu. Si l'on veut contrôler la présence de
valeurs manquantes, il est donc préférable d'utiliser la commande
\texttt{complete.cases}, soit
\begin{Schunk}
\begin{Sinput}
> nrow(enquete)
\end{Sinput}
\begin{Soutput}
[1] 100
\end{Soutput}
\begin{Sinput}
> sum(complete.cases(enquete$sexe))
\end{Sinput}
\begin{Soutput}
[1] 100
\end{Soutput}
\begin{Sinput}
> sum(complete.cases(enquete$qdv))
\end{Sinput}
\begin{Soutput}
[1] 97
\end{Soutput}
\end{Schunk}
\cmd{nrow}\cmd{sum}\cmd{complete.cases}
ce qui montre que pour la variable \texttt{qdv} il y a en réalité trois
données manquantes. La même information peut être obtenue à partir de la
commande plus générale \texttt{summary} :
\begin{Schunk}
\begin{Sinput}
> summary(enquete)
\end{Sinput}
\begin{Soutput}
       id              age        sexe         qi        sse         qdv       
 Min.   :  1.00   Min.   :21.00   F:41   Min.   :111.0   A:35   Min.   :64.00  
 1st Qu.: 25.75   1st Qu.:26.00   M:58   1st Qu.:120.0   B:36   1st Qu.:68.00  
 Median : 50.50   Median :27.00   T: 1   Median :125.0   C:29   Median :70.00  
 Mean   : 50.50   Mean   :27.64          Mean   :125.2          Mean   :70.18  
 3rd Qu.: 75.25   3rd Qu.:30.00          3rd Qu.:130.0          3rd Qu.:72.00  
 Max.   :100.00   Max.   :37.00          Max.   :141.0          Max.   :77.00  
                                                                NA's   :3      
\end{Soutput}
\end{Schunk}
\cmd{summary}
Pour identifier les observations manquantes pour la variable \texttt{qdv},
le plus simple est d'utiliser \texttt{which} en combinaison avec la commande
\texttt{is.na} qui renvoit vrai ou faux selon que la donnée est considérée
comme manquante ou non, soit :
\begin{Schunk}
\begin{Sinput}
> which(is.na(enquete$qdv))
\end{Sinput}
\begin{Soutput}
[1] 53 69 97
\end{Soutput}
\end{Schunk}
\cmd{which}\cmd{is.na}
\end{sol}
\end{exo}
\begin{exo}\label{exo:1.5}
Le fichier \texttt{anorexia.dat} contient les données d'une étude clinique
chez des patientes anorexiques ayant reçu l'une des trois thérapies
suivantes : thérapie comportementale, thérapie familiale, thérapie
contrôle.\autocite{hand93} 
\begin{description}
\item[(a)] Combien y'a-t-il de patientes au total ? Combien y'a-t-il de
  patientes par groupe de traitement ?
\item[(b)] Les mesures de poids sont en livres. Les convertir en
  kilogrammes.    
\item[(c)] Créer une nouvelle variable contenant les scores de différences
  (\texttt{After} - \texttt{Before}).
\item[(d)] Indiquer la moyenne et l'étendue (min/max) des scores de
  différences par groupe de traitement.
\end{description}
\begin{sol}
Un aperçu des données contenues dans le fichier \texttt{anorexia.dat} est
fourni ci-dessous (5 premières lignes du fichier) :
\begin{verbatim}
Group Before After
g1 80.5  82.2
g1 84.9  85.6
g1 81.5  81.4
g1 82.6  81.9
\end{verbatim}
On voit donc que la première ligne est une ligne d'en-tête indiquant le nom
des variables, et chacune des lignes suivantes représente une unité
statistique pour laquelle on a le groupe d'appartenance, la mesure de poids
avant le début de la prise en charge et la mesure de poids à la fin de la thérapie.
Pour importer ce type de données, on utilisera la commande
\texttt{read.table} en précisant l'option \verb|header=TRUE| pour bien
prendre en compte la ligne d'en-tête.
\begin{Schunk}
\begin{Sinput}
> anorex <- read.table("data/anorexia.dat", header=TRUE)
> names(anorex)
\end{Sinput}
\begin{Soutput}
[1] "Group"  "Before" "After" 
\end{Soutput}
\begin{Sinput}
> head(anorex)
\end{Sinput}
\begin{Soutput}
  Group Before After
1    g1   80.5  82.2
2    g1   84.9  85.6
3    g1   81.5  81.4
4    g1   82.6  81.9
5    g1   79.9  76.4
6    g1   88.7 103.6
\end{Soutput}
\end{Schunk}
Le nombre total de patientes correspond au nombre de lignes du tableau de
données :
\begin{Schunk}
\begin{Sinput}
> nrow(anorex)
\end{Sinput}
\begin{Soutput}
[1] 72
\end{Soutput}
\end{Schunk}
Il y a donc au total 72 patientes. Pour trouver la répartition des effectifs
par groupe, le plus simple est de faire un tri à plat de la variable
qualitative \texttt{Group} :
\begin{Schunk}
\begin{Sinput}
> table(anorex$Group)
\end{Sinput}
\begin{Soutput}
g1 g2 g3 
29 26 17 
\end{Soutput}
\end{Schunk}
Pour convertir les poids exprimés en livres en kilogrammes, il suffit de
diviser chaque mesure par 2.2 (approximativement). On réalise cette
opération pour chacune des deux variables \texttt{Before} et \texttt{After}.
\begin{Schunk}
\begin{Sinput}
> anorex$Before <- anorex$Before/2.2
> anorex$After <- anorex$After/2.2
\end{Sinput}
\end{Schunk}
On notera que dans ce cas-là, les valeurs d'origine de ces deux variables
seront simplement remplacées. Si l'on souhaite y avoir accès de nouveau, il
faudra recharger le fichier. Une autre solution aurait consisté à créer deux
nouvelles variables, comme ceci :
\begin{Schunk}
\begin{Sinput}
> anorex$Before.kg <- anorex$Before/2.2
> anorex$After.kg <- anorex$After/2.2
\end{Sinput}
\end{Schunk}
Pour calculer les scores de différences \texttt{After} - \texttt{Before}, il
suffit de soustraire les valeurs des deux variables, en se rappelant que ce
genre d'opérations opère élément par élément (c'est-à-dire pour chaque unité
statistique). Ici, on ajoutera la variable nouvellement créée au tableau de
données \texttt{anorex}.
\begin{Schunk}
\begin{Sinput}
> anorex$poids.diff <- anorex$After - anorex$Before
> head(anorex)
\end{Sinput}
\begin{Soutput}
  Group   Before    After  poids.diff
1    g1 36.59091 37.36364  0.77272727
2    g1 38.59091 38.90909  0.31818182
3    g1 37.04545 37.00000 -0.04545455
4    g1 37.54545 37.22727 -0.31818182
5    g1 36.31818 34.72727 -1.59090909
6    g1 40.31818 47.09091  6.77272727
\end{Soutput}
\end{Schunk}
Enfin, pour calculer la moyenne et l'étendue des scores de différence par
groupe de traitement, on peut procéder de deux manières. Soit on isole
chaque groupe et on calcule les statistiques demandées, soit on \og
factorise\fg\ l'opération de calcul en opérant par modalité de la variable
qualitative. La première solution s'obtiendrait ainsi, pour le 1\ier\ groupe
par exemple :
\begin{Schunk}
\begin{Sinput}
> mean(anorex$poids.diff[anorex$Group == "g1"])
\end{Sinput}
\begin{Soutput}
[1] 1.366771
\end{Soutput}
\begin{Sinput}
> range(anorex$poids.diff[anorex$Group == "g1"])
\end{Sinput}
\begin{Soutput}
[1] -4.136364  9.500000
\end{Soutput}
\end{Schunk}
L'autre solution plus économique consiste à utiliser la commande
\texttt{tapply} de la manière suivante :
\begin{Schunk}
\begin{Sinput}
> tapply(anorex$poids.diff, anorex$Group, mean)
\end{Sinput}
\begin{Soutput}
        g1         g2         g3 
 1.3667712 -0.2045455  3.3021390 
\end{Soutput}
\begin{Sinput}
> tapply(anorex$poids.diff, anorex$Group, range)
\end{Sinput}
\begin{Soutput}
$g1
[1] -4.136364  9.500000
$g2
[1] -5.545455  7.227273

$g3
[1] -2.409091  9.772727
\end{Soutput}
\end{Schunk}
Comme on le voit, pour utiliser la commande \texttt{tapply} on spécifie la
variable réponse (\texttt{poids.diff}), le facteur de classification
(\texttt{Group}) et la commande à appliquer (\texttt{mean}). On notera que
plutôt que de préfixer chaque nom de variable par le nom du tableau de
données, on peut simplifier la syntaxe en utilisant \texttt{with} :
\begin{Schunk}
\begin{Sinput}
> with(anorex, tapply(poids.diff, Group, mean))
\end{Sinput}
\begin{Soutput}
        g1         g2         g3 
 1.3667712 -0.2045455  3.3021390 
\end{Soutput}
\end{Schunk}
\end{sol}
\end{exo}
%
% STAB TD 3
%
\begin{exo}\label{exo:1.6}
Soient les mesures de bioluminescence recueillies dans une étude sur
l'antagonisme Na/An. Les données sont présentées dans le tableau suivant et
elles ont été enregistrées dans le fichier \texttt{bioluminescence.dat} où
les mesures recueillies pour chaque traitement sont arrangées en colonnes
(comme dans le tableau).
\vskip1em

\begin{tabular}{cccc}
\toprule
Sans Na & Avec Na & Sans Na & Avec Na \\
Sans An & Sans An & Avec An & Avec An \\
\midrule
6.5 & 26.5 & 12.0 & 12.6 \\
6.2 & 22.7 & 8.2 & 13.4 \\
6.8 & 17.0 & 12.3 & 12.0 \\
5.7 & 18.0 & 11.3 & 6.0 \\
7.2 & 14.9 & 7.0 & – \\
5.9 & 24.0 & 10.5 & – \\
8.2 & 19.9 & – & – \\
8.0 & 24.0 & – & – \\
10.5 & – & – & – \\
7.4 & – & – & – \\
8.9 & – & – & – \\
11.5 & – & – & – \\
\bottomrule
\end{tabular}
\vskip1em

\begin{description}
\item[(a)] Importer les données sous R et reformater le tableau de données
  de manière à ce que la variable réponse soit dans une seule colonne, et
  les facteurs d'étude (\texttt{Na} et \texttt{An}) dans deux colonnes
  séparées (\texttt{data.frame}).
\item[(b)] Vérifier le nombre d'observations disponible pour chacun des
  quatre traitements, un traitement correspondant au croisement des niveaux
  de chacun des facteurs d'étude.
\item[(c)] Enregistrer le tableau de données ainsi constitué au format
  \texttt{RData}. Il sera exploité lors de la séance~4 (p.~\pageref{chap:anova}).
\end{description}
\begin{sol}
Pour importer les données, on prendra garde au fait que le fichier texte
\texttt{bioluminescence.dat} ne comprend pas de ligne d'en-tête.  
\begin{Schunk}
\begin{Sinput}
> biolum <- read.table("data/bioluminescence.dat", header=FALSE)
> biolum
\end{Sinput}
\begin{Soutput}
     V1   V2   V3   V4
1   6.5 26.5 12.0 12.6
2   6.2 22.7  8.2 13.4
3   6.8 17.0 12.3 12.0
4   5.7 18.0 11.3  6.0
5   7.2 14.9  7.0   NA
6   5.9 24.0 10.5   NA
7   8.2 19.9   NA   NA
8   8.0 24.0   NA   NA
9  10.5   NA   NA   NA
10  7.4   NA   NA   NA
11  8.9   NA   NA   NA
12 11.5   NA   NA   NA
\end{Soutput}
\end{Schunk}

On crée ensuite les deux facteurs d'étude, \texttt{An} et \texttt{Na}, en
indiquant le nombre de répétitions des niveaux à partir du nombre
d'observations non manquantes dans chaque colonne du tableau d'origine. Pour
le premier traitement (\texttt{An-/Na-}) on sait qu'il y a 12 observations,
sachant qu'au total \texttt{An-} totalise 20 observations (1\iere\ et
2\ieme\ colonnes).
\begin{Schunk}
\begin{Sinput}
> An <- rep(c("-","+"), c(20,10))
> Na <- rep(c("-","+","-","+"), c(12,8,6,4))
> biolum <- with(biolum, data.frame(y=c(V1,V2[1:8],V3[1:6],V4[1:4]), An, Na))
> head(biolum)
\end{Sinput}
\begin{Soutput}
    y An Na
1 6.5  -  -
2 6.2  -  -
3 6.8  -  -
4 5.7  -  -
5 7.2  -  -
6 5.9  -  -
\end{Soutput}
\begin{Sinput}
> summary(biolum)
\end{Sinput}
\begin{Soutput}
       y         An     Na    
 Min.   : 5.70   -:20   -:18  
 1st Qu.: 7.25   +:10   +:12  
 Median :10.90                
 Mean   :12.17                
 3rd Qu.:14.53                
 Max.   :26.50                
\end{Soutput}
\end{Schunk}

Pour obtenir le nombre d'observations par croisement des niveaux des deux
facteurs (soit 4 combinaisons ou traitements), on utilisera la commande
\texttt{length} pour compter le nombre d'observations avec la commande
\texttt{tapply} qui permet d'isoler les combinaisons de niveaux de facteurs.
\begin{Schunk}
\begin{Sinput}
> with(biolum, tapply(y, list(An, Na), length))
\end{Sinput}
\begin{Soutput}
   - +
- 12 8
+  6 4
\end{Soutput}
\end{Schunk}

Pour sauvegarder le nouveau tableau de données, on utilisera la commande
\texttt{save} en donnant le nom du fichier dans lequel stocker les données,
par exemple \texttt{bioluminescence.RData}.
\begin{Schunk}
\begin{Sinput}
> save(biolum, file="data/bioluminescence.RData")
\end{Sinput}
\end{Schunk}
\end{sol}
\end{exo}

\Closesolutionfile{solutions}

\chapter{Statistiques descriptives et estimation}\label{chap:descriptive}
\Opensolutionfile{solutions}[solutions2]


\section*{Énoncés}
%
%
%
\begin{exo}\label{exo:2.1}
Une variable quantitative $X$ prend les valeurs suivantes sur un échantillon
de 26 sujets :
\begin{verbatim}
24.9,25.0,25.0,25.1,25.2,25.2,25.3,25.3,25.3,25.4,25.4,25.4,25.4,
25.5,25.5,25.5,25.5,25.6,25.6,25.6,25.7,25.7,25.8,25.8,25.9,26.0
\end{verbatim}
\begin{description}
\item[(a)] Calculer la moyenne, la médiane ainsi que le mode de $X$. 
\item[(b)]  Quelle est la valeur de la variance estimée à partir de ces données ? 
\item[(c)] En supposant que les données sont regroupées en 4 classes dont les
  bornes sont : 24.9–25.1, 25.2–25.4, 25.5–25.7, 25.8–26.0, afficher la
  distribution des effectifs par classe sous forme d'un tableau d'effectifs. 
\item[(d)] Représenter la distribution de $X$ sous forme d'histogramme, sans
  considération d'intervalles de classe \emph{a priori}.
\end{description}
\begin{sol}
Une manière de représenter ce type de données consiste à utiliser saisir les
observations comme ceci :
\begin{Schunk}
\begin{Sinput}
> x <- c(24.9,25.0,25.0,25.1,25.2,25.2,25.3,25.3,25.3,25.4,25.4,25.4,25.4,
+        25.5,25.5,25.5,25.5,25.6,25.6,25.6,25.7,25.7,25.8,25.8,25.9,26.0)
\end{Sinput}
\end{Schunk}

La médiane et la moyenne sont obtenues à l'aide des commandes
suivantes :
\begin{Schunk}
\begin{Sinput}
> median(x)
\end{Sinput}
\begin{Soutput}
[1] 25.45
\end{Soutput}
\begin{Sinput}
> mean(x)
\end{Sinput}
\begin{Soutput}
[1] 25.44615
\end{Soutput}
\end{Schunk}
\cmd{median}\cmd{mean}
Concernant la médiane, on peut vérifier qu'il s'agit bien de la valeur
correspondant au deuxième quartile ou au 50\ieme\ percentile, c'est-à-dire
la valeur de $X$ telle que 50~\% des observations lui sont inférieures :
\begin{Schunk}
\begin{Sinput}
> quantile(x)
\end{Sinput}
\begin{Soutput}
   0%   25%   50%   75%  100% 
24.90 25.30 25.45 25.60 26.00 
\end{Soutput}
\end{Schunk}
\cmd{quantile}

Pour le mode, il est nécessaire d'afficher la distribution des effectifs
selon les valeurs de $X$, puis vérifier à quelle valeur de $X$ est associé
l'effectif le plus grand:
\begin{Schunk}
\begin{Sinput}
> table(x)
\end{Sinput}
\begin{Soutput}
x
24.9   25 25.1 25.2 25.3 25.4 25.5 25.6 25.7 25.8 25.9   26 
   1    2    1    2    3    4    4    3    2    2    1    1 
\end{Soutput}
\end{Schunk}
\cmd{table}
Ici, on voit qu'il y a deux modes : 25.4 et 25.5.

L'estimé de la variance est donné par :
\begin{Schunk}
\begin{Sinput}
> var(x)
\end{Sinput}
\begin{Soutput}
[1] 0.07938462
\end{Soutput}
\end{Schunk}
\cmd{var}

En supposant que les données sont regroupées en 4 classes dont les bornes
sont : 24.9–25.1, 25.2–25.4, 25.5–25.7, 25.8–26.0, on peut recalculer la
distribution des effectifs par classe.
\begin{Schunk}
\begin{Sinput}
> xc <- cut(x, breaks=c(24.9,25.2,25.5,25.8,26.0), include.lowest=TRUE, right=FALSE)
> table(xc)
\end{Sinput}
\begin{Soutput}
xc
[24.9,25.2) [25.2,25.5) [25.5,25.8)   [25.8,26] 
          4           9           9           4 
\end{Soutput}
\end{Schunk}
\cmd{cut}\cmd{table}
On remarquera que R utilise la notation anglo-saxonne pour représenter les
bornes des intervalles de classe : le symbole \texttt{)} à droite d'un
nombre signifie que ce nombre est exclu de l'intervalle, alors que
\texttt{]} signifie que l'intervalle contient ce nombre.

On peut visualiser la distribution des effectifs à l'aide d'un histogramme
comme suit :
\begin{Schunk}
\begin{Sinput}
> histogram(~ x, type="count")
\end{Sinput}
\end{Schunk}
\includegraphics{figs/fig-ex2-1g}
\cmd{histogram}

Par défaut, R détermine automatiquement les intervalles de classe. Si l'on
souhaite spécifier soi-même les bornes des intervalles de classe, on
utilisera l'option \texttt{breaks=}. Par exemple, pour afficher la
distribution des effectifs selon les 4 classes définies plus haut, on écrirait :
\begin{verbatim}
> histogram(~ x, type="count", breaks=c(24.9,25.2,25.5,25.8,26.0))
\end{verbatim}
\end{sol}
\end{exo}
%
% Everitt 2011 p. 38
%
\begin{exo}\label{exo:2.2}
On dispose des temps de survie de 43 patients souffrant de leucémie
granulocytaire chronique, mesurés en jours depuis le diagnostique :
\autocite[p.~38]{everitt01} 
\begin{verbatim}
7,47,58,74,177,232,273,285,317,429,440,445,455,468,495,497,532,571,
579,581,650,702,715,779,881,900,930,968,1077,1109,1314,1334,1367,
1534,1712,1784,1877,1886,2045,2056,2260,2429,2509
\end{verbatim}
\begin{description}
\item[(a)] Calculer le temps de survie médian. 
\item[(b)] Combien de patients ont une survie inférieure (strictement) à 900
  jours au moment de l'étude ? 
\item[(c)] Quelle est la durée de survie associée au 90\ieme\ percentile ?
\end{description}
\begin{sol}
Les données sont saisies comme à l'exercice précédent.  
\begin{Schunk}
\begin{Sinput}
> s <- c(7,47,58,74,177,232,273,285,317,429,440,445,455,468,495,497,532,571,
+        579,581,650,702,715,779,881,900,930,968,1077,1109,1314,1334,1367,
+        1534,1712,1784,1877,1886,2045,2056,2260,2429,2509)
\end{Sinput}
\end{Schunk}
Le temps de survie médian est obtenu facilement :
\begin{Schunk}
\begin{Sinput}
> median(s)
\end{Sinput}
\begin{Soutput}
[1] 702
\end{Soutput}
\end{Schunk}
\cmd{median}

Pour déterminer le nombre de patients avec une survie $\le 900$ jours, il
suffit d'effectuer un test :
\begin{Schunk}
\begin{Sinput}
> table(s < 900)
\end{Sinput}
\begin{Soutput}
FALSE  TRUE 
   18    25 
\end{Soutput}
\end{Schunk}
\cmd{table}
La colonne libellée \texttt{TRUE} indique le nombre d'observations vérifiant
la condition ci-dessus. 
% Une autre solution possible est \verb|length(which(s <= 900))| (expliquer).

La survie associée au 90\ieme\ percentile peut s'obtenir à l'aide de la commande
\texttt{quantile}, par exemple :
\begin{Schunk}
\begin{Sinput}
> quantile(s, 0.9)
\end{Sinput}
\begin{Soutput}
   90% 
2013.2 
\end{Soutput}
\end{Schunk}
\cmd{quantile}
On peut vérifier que le résultat correspond bien à la valeur de survie telle
que 90~\% des observations ne dépasse pas cette valeur.
\begin{Schunk}
\begin{Sinput}
> table(s <= quantile(s, 0.9))
\end{Sinput}
\begin{Soutput}
FALSE  TRUE 
    5    38 
\end{Soutput}
\end{Schunk}
\cmd{table}
Ici, $38/43$ est bien inférieur à 0.90 tandis que $39/43=0.91$.
\end{sol}
\end{exo}
%
% Everitt 2011 p. 38
%
\begin{exo}\label{exo:2.3}
Le fichier \texttt{elderly.dat} contient la taille mesurée en cm de 351
personnes âgées de sexe féminin, sélectionnées aléatoirement dans la
population lors d'une étude sur l'ostéoporose. Quelques observations sont
cependant manquantes.
\begin{description}
\item[(a)] Combien y'a t-il d'observations manquantes au total ?
\item[(b)] Donner un intervalle de confiance à 95~\% pour la taille moyenne
  dans cet échantillon, en utilisant une approximation normale.
\item[(c)] Représenter la distribution des tailles observées sous forme
  d'une courbe de densité.  
\end{description}
\begin{sol}
Pour lire le fichier qui ne se compose que d'une série de valeurs
numériques, on utilise la commande \texttt{scan}. Attention, comme il y a
des valeurs manquantes, codées dans le fichier texte par des ".", il est
nécessaire de s'assurer que ces observations sont bien identifiées comme
telles par \R. 
\begin{Schunk}
\begin{Sinput}
> tailles <- scan("data/elderly.dat", na.strings=".")
\end{Sinput}
\end{Schunk}
\cmd{scan}
On pourrait afficher la distribution des observations à l'aide d'un
simple histogramme (\verb|histogram(~ tailles)|). Cela permet en particulier
de vérifier la forme générale de la distribution et la présence
d'eventuelles valeurs "extrêmes".

Le nombre d'observations non complètes (données manquantes) est déterminé en
comptant le nombre de valeurs manquantes (\texttt{NA}), ce qui est
équivalent à établir un tri à plat des valeurs considérées comme manquantes
par \R à l'aide de la commande \texttt{table}.
\begin{Schunk}
\begin{Sinput}
> sum(is.na(tailles))
\end{Sinput}
\begin{Soutput}
[1] 5
\end{Soutput}
\begin{Sinput}
> table(is.na(tailles))
\end{Sinput}
\begin{Soutput}
FALSE  TRUE 
  346     5 
\end{Soutput}
\end{Schunk}
\cmd{sum}\cmd{is.na}\cmd{table}
On a donc 5 observations manquantes au total.

La moyenne est obtenue à l'aide de la commande \texttt{mean}. Cependant,
comme il existe des valeurs manquantes il est nécessaire de préciser
l'option \verb|na.rm=TRUE| pour indiquer à \R de calculer la moyenne
arithmétique sur les cas complets. La même remarque vaut pour l'usage de
\texttt{sd}. On a également besoin d'avoir une estimation de
l'écart-type. Le quantile de référence (97.5~\%) pour la loi normale 97.5~\%
est généralement pris à 1.96 dans les tables statistiques mais on peut
obtenir sa valeur avec R grâce à \texttt{qnorm}.  Soit au final :
\begin{Schunk}
\begin{Sinput}
> m <- mean(tailles, na.rm=TRUE)    # moyenne
> s <- sd(tailles, na.rm=TRUE)      # écart-type
> n <- sum(!is.na(tailles))         # nombre d'observations
> m - qnorm(0.975) * s/sqrt(n)      # borne inf. IC 95 %
\end{Sinput}
\begin{Soutput}
[1] 159.1988
\end{Soutput}
\begin{Sinput}
> m + qnorm(0.975) * s/sqrt(n)      # borne sup. IC 95 %
\end{Sinput}
\begin{Soutput}
[1] 160.4659
\end{Soutput}
\end{Schunk}
\cmd{mean}\cmd{sd}\cmd{sum}\cmd{is.na}\cmd{qnorm}\cmd{sqrt}
Attention, utiliser la commande \texttt{length} pour compter le nombre
d'observations serait erroné dans ce contexte : en présence de valeurs
manquantes, il faut explicitement s'assurer que l'on travaille bien sur les
cas complets. La commande \verb|sum(!is.na(tailles))| est équivalente à la
commande, peut-être plus explicite, \verb|sum(complete.cases(tailles))|.
Les deux dernières instructions peuvent se simplifier en exploitant la
capacité de \R à répéter le même calcul sur des séries de données stockées
dans une variable. Ici, on se contente de reproduire la formule $m\pm
z_{0.975}\frac{s}{\sqrt{n}}$ :
\begin{Schunk}
\begin{Sinput}
> m + c(-1,1) * qnorm(0.975) * s/sqrt(n)
\end{Sinput}
\end{Schunk}
\cmd{qnorm}\cmd{sqrt}
% FIXME:
% sans doute à supprimer car trop compliqué.

Pour représenter la distribution des tailles sous forme d'une courbe de
densité, qui ne pose pas le problème du choix de classes \emph{a priori}, la
commande à utiliser est \texttt{densityplot}.
\begin{Schunk}
\begin{Sinput}
> densityplot(~ tailles)
\end{Sinput}
\end{Schunk}
\includegraphics{figs/fig-ex2-3e}
\cmd{densityplot}

\noindent Le degré de lissage de la courbe de densité estimée à partir des
données peut être contrôlé à l'aide de l'option \texttt{bw}. L'exemple
suivant produirait une courbe présentant beaucoup moins de variations
locales, par exemple.
\begin{verbatim}
> densityplot(~ tailles, bw=4)
\end{verbatim}
\cmd{densityplot}
\end{sol}
\end{exo}
% 
% Hosmer & Lemeshow 1989
%
\begin{exo}\label{exo:2.4}
Le fichier \texttt{birthwt} est un des jeux de données fournis avec \R. Il
comprend les résultats d'une étude prospective visant à identifier les
facteurs de risque associés à la naissance de bébés dont le poids est
inférieur à la norme (2,5 kg). Les données proviennent de 189 femmes, dont
59 ont accouché d'un enfant en sous-poids. Parmi les variables d'intérêt
figurent l'âge de la mère, le poids de la mère lors des dernières
menstruations, l'ethnicité de la mère et le nombre de visites médicales
durant le premier trimestre de grossesse.\autocite{hosmer89}
Les variables disponibles sont décrites comme suit : \texttt{low} (= 1 si
poids $<2.5$ kg, 0 sinon), \texttt{age} (années), \texttt{lwt} (poids de la
mère en livres), \texttt{race} (ethnicité codée en trois classes, 1 = white,
2 = black, 3 = other), \texttt{smoke} (= 1 si consommation de tabac durant
la grossesse, 0 sinon), \texttt{ptl} (nombre d'accouchements pré-terme
antérieurs), \texttt{ht} (= 1 si antécédent d'hypertension, 0 sinon),
\texttt{ui} (= 1 si manifestation d'irritabilité utérine, 0 sinon),
\texttt{ftv} (nombre de consultations chez le gynécologue durant le premier
trimestre de grossesse), \texttt{bwt} (poids des bébés à la naissance, en
\emph{g}).
\begin{description}
\item[(a)] Recoder les variables \texttt{low}, \texttt{race},
  \texttt{smoke}, \texttt{ui} et \texttt{ht} en variables
  qualitatives, avec des étiquettes ("labels") plus informatives.
\item[(b)] Convertir le poids des mères en \emph{kg}. Indiquer la moyenne, la
  médiane et l'intervalle inter-quartile. Représenter la distribution des
  poids sous forme d'histogramme.
\item[(c)] Indiquer la proportion de mères consommant du tabac durant la
  grossesse, avec un intervalle de confiance à 95~\%. Représenter les
  proportions (en \%) fumeur/non-fumeur sous forme d'un diagramme en
  barres.
\item[(d)] Recoder l'âge des mères en trois classes équilibrées (tercilage)
  et indiquer la proportion d'enfants dont le poids est $<2500$ \emph{g}
  pour chacune des trois classes.
\item[(e)] Construire un tableau d'effectifs ($n$ et \%) pour la variable
  ethnicité (\texttt{race}).  
\item[(f)] Décrire la distribution des variables \texttt{race},
  \texttt{smoke}, \texttt{ui}, \texttt{ht} et \texttt{age} après
  stratification sur la variable \texttt{low}.  
\end{description}
\begin{sol}
Les données fournies avec \R sont dans un format accessible à partir de la
commande \texttt{data}. Il est parfois nécessaire d'indiquer à \R dans quel
package trouver les données. Pour charger les données \texttt{birthwt}, on
utilisera donc
\begin{Schunk}
\begin{Sinput}
> data(birthwt, package="MASS")
> str(birthwt)
\end{Sinput}
\begin{Soutput}
'data.frame':	189 obs. of  10 variables:
 $ low  : int  0 0 0 0 0 0 0 0 0 0 ...
 $ age  : int  19 33 20 21 18 21 22 17 29 26 ...
 $ lwt  : int  182 155 105 108 107 124 118 103 123 113 ...
 $ race : int  2 3 1 1 1 3 1 3 1 1 ...
 $ smoke: int  0 0 1 1 1 0 0 0 1 1 ...
 $ ptl  : int  0 0 0 0 0 0 0 0 0 0 ...
 $ ht   : int  0 0 0 0 0 0 0 0 0 0 ...
 $ ui   : int  1 0 0 1 1 0 0 0 0 0 ...
 $ ftv  : int  0 3 1 2 0 0 1 1 1 0 ...
 $ bwt  : int  2523 2551 2557 2594 2600 2622 2637 2637 2663 2665 ...
\end{Soutput}
\begin{Sinput}
> head(birthwt, 5)
\end{Sinput}
\begin{Soutput}
   low age lwt race smoke ptl ht ui ftv  bwt
85   0  19 182    2     0   0  0  1   0 2523
86   0  33 155    3     0   0  0  0   3 2551
87   0  20 105    1     1   0  0  0   1 2557
88   0  21 108    1     1   0  0  1   2 2594
89   0  18 107    1     1   0  0  1   0 2600
\end{Soutput}
\end{Schunk}
\cmd{data}\cmd{str}\cmd{head}
Comme on peut le vérifier dans les sorties précédentes, l'ensemble des
variables est au format numérique. Pour recoder certaines des variables en
variables qualitatives, on utilisera la commande \texttt{factor}. 
\begin{Schunk}
\begin{Sinput}
> yesno <- c("No","Yes")
> ethn <- c("White","Black","Other")
> birthwt <- within(birthwt, {
+   low <- factor(low, labels=yesno)
+   race <- factor(race, labels=ethn)
+   smoke <- factor(smoke, labels=yesno)
+   ui <- factor(ui, labels=yesno)
+   ht <- factor(ht, labels=yesno)
+ })
\end{Sinput}
\end{Schunk}
\cmd{within}\cmd{factor}
La commande \texttt{within} utilisée permet simplement d'éviter de devoir
indiquer à chaque fois le nom du tableau de données (\texttt{data.frame}
dans la terminologie \R) suivi du nom de la variable, comme par exemple dans
les instructions suivantes :
\begin{Schunk}
\begin{Sinput}
> birthwt$low <- factor(birthwt$low, labels=yesno)
> birthwt$race <- factor(birthwt$race, labels=ethn)
\end{Sinput}
\end{Schunk}
\cmd{factor}
On verra plus tard une autre construction (\texttt{with}) permettant
d'éviter de saisir le nom du tableau de données à plusieurs reprises dans
une même commande. 

La conversion des poids des mères en \emph{kg} ne pose pas vraiment de
problème puisqu'il suffit de diviser les poids exprimés en livres par 2.2 :
\begin{Schunk}
\begin{Sinput}
> birthwt$lwt <- birthwt$lwt/2.2
\end{Sinput}
\end{Schunk}
Notons que la transformation effectuée est définitive : il n'y a plus moyen
de retrouver les valeurs d'origine de la variable \texttt{lwt} dans le
\texttt{data.frame} \texttt{birthwt} (à moins de réimporter les
données). Dans le cas présent, ce n'est pas très important, mais en règle
générale il est préférable de créer de nouvelles variables.

On peut ensuite utiliser une commande telle que \texttt{summary} pour
obtenir les principaux indicateurs de tendance centrale et de forme de la
distribution (à partir de l'étendue et des quartiles).
\begin{Schunk}
\begin{Sinput}
> summary(birthwt$lwt)
\end{Sinput}
\begin{Soutput}
   Min. 1st Qu.  Median    Mean 3rd Qu.    Max. 
  36.36   50.00   55.00   59.01   63.64  113.60 
\end{Soutput}
\end{Schunk}
\cmd{summary}
La médiane et la moyenne sont de 59 et 55 \emph{kg},
respectivement. L'intervalle inter-quartile se calcule aisément comme la
différence entre le 3\ieme (63.64) et le 1\ier quartile (50.00), mais on
peut également utiliser la commande \texttt{IQR} :
\begin{Schunk}
\begin{Sinput}
> IQR(birthwt$lwt)
\end{Sinput}
\begin{Soutput}
[1] 13.63636
\end{Soutput}
\end{Schunk}
\cmd{IQR}
De manière générale, les quartiles peuvent être retrouvés ainsi :
\begin{Schunk}
\begin{Sinput}
> quantile(birthwt$lwt)
\end{Sinput}
\begin{Soutput}
       0%       25%       50%       75%      100% 
 36.36364  50.00000  55.00000  63.63636 113.63636 
\end{Soutput}
\end{Schunk}
\cmd{quantile}
d'où une autre manière de calculer l'intervalle inter-quartile,
\begin{Schunk}
\begin{Sinput}
> lwt.quartiles <- quantile(birthwt$lwt)
> lwt.quartiles
\end{Sinput}
\begin{Soutput}
       0%       25%       50%       75%      100% 
 36.36364  50.00000  55.00000  63.63636 113.63636 
\end{Soutput}
\begin{Sinput}
> lwt.quartiles[4] - lwt.quartiles[2]
\end{Sinput}
\begin{Soutput}
     75% 
13.63636 
\end{Soutput}
\end{Schunk}
\cmd{quantile}

La distribution des poids est indiquée dans l'histogramme suivant, en
utilisant la commande suivante :
\begin{Schunk}
\begin{Sinput}
> histogram(~ lwt, data=birthwt, type="count", xlab="Poids de la mère (kg)", ylab="Effectifs")
\end{Sinput}
\end{Schunk}
\includegraphics{figs/fig-ex2-4i}
\cmd{histogram}

La proportion de mères ayant fumé durant leur grossesse est obtenue à partir
d'un simple tri à plat de la variable \texttt{smoke}.
\begin{Schunk}
\begin{Sinput}
> table(birthwt$smoke)
\end{Sinput}
\begin{Soutput}
 No Yes 
115  74 
\end{Soutput}
\begin{Sinput}
> round(prop.table(table(birthwt$smoke))*100, 1)
\end{Sinput}
\begin{Soutput}
  No  Yes 
60.8 39.2 
\end{Soutput}
\end{Schunk}
\cmd{table}\cmd{round}\cmd{prop.table}
Pour obtenir une estimation de l'intervalle de confiance à 95~\% associé à
cette proportion empirique, on peut utiliser la commande \texttt{prop.test}
qui effectue également un test d'hypothèse sur l'égalité des proportions
estimées dans différents groupes. Cependant, la commande suivante ne renvoit
pas le résultat correct car, comme on peut le vérifier dans la proportion
qui est estimée dans ce cas, \R a travaillé sur la proportion de non-fumeurs.
\begin{Schunk}
\begin{Sinput}
> prop.test(table(birthwt$smoke), correct=FALSE)
\end{Sinput}
\begin{Soutput}
	1-sample proportions test without continuity correction
data:  table(birthwt$smoke), null probability 0.5 
X-squared = 8.8942, df = 1, p-value = 0.002861
alternative hypothesis: true p is not equal to 0.5 
95 percent confidence interval:
 0.5373819 0.6752280 
sample estimates:
        p 
0.6084656 
\end{Soutput}
\end{Schunk}
\cmd{prop.test}
On peut donc utiliser directement la forme suivante :
\begin{Schunk}
\begin{Sinput}
> prop.test(74, 189, correct=FALSE)
\end{Sinput}
\begin{Soutput}
	1-sample proportions test without continuity correction
data:  74 out of 189, null probability 0.5 
X-squared = 8.8942, df = 1, p-value = 0.002861
alternative hypothesis: true p is not equal to 0.5 
95 percent confidence interval:
 0.3247720 0.4626181 
sample estimates:
        p 
0.3915344 
\end{Soutput}
\end{Schunk}
On retrouve bien la proportion estimée à l'étape précédente (\texttt{sample
  estimates}) (39.1~\%), ainsi que l'intervalle de confiance qui, ici, vaut
$[0.325;0.463]$. 

Un diagramme en barres représentant la proportion de fumeurs/non-fumeurs
dans cet échantillon est construit à partir de la commande \texttt{barchart}
comme proposé ci-après.
\begin{Schunk}
\begin{Sinput}
> barchart(prop.table(table(birthwt$smoke))*100, horizontal=FALSE, 
+          ylab="Proportion (%)", ylim=c(-5,105))
\end{Sinput}
\end{Schunk}
\includegraphics{figs/fig-ex2-4m}
\cmd{barchart}\cmd{prop.table}

Pour recoder l'âge en variable qualitative à partir des terciles, on
utilisera la commande \texttt{cut} :
\begin{Schunk}
\begin{Sinput}
> birthwt$age.dec <- cut(birthwt$age, breaks=quantile(birthwt$age, c(0, 0.33, 0.66, 1)), 
+                        include.lowest=FALSE)
> table(birthwt$age.dec)
\end{Sinput}
\begin{Soutput}
(14,20] (20,25] (25,45] 
     66      66      54 
\end{Soutput}
\end{Schunk}
\cmd{cmd}\cmd{table}\cmd{quantile}
Le nombre d'enfants ayant un poids $<2.5$ \emph{kg} dans chacune des
classes est obtenue à partir d'un simple croisement des deux variables
\texttt{age.dec} et \texttt{low} :
\begin{Schunk}
\begin{Sinput}
> with(birthwt, table(age.dec, low))
\end{Sinput}
\begin{Soutput}
         low
age.dec   No Yes
  (14,20] 45  21
  (20,25] 43  23
  (25,45] 41  13
\end{Soutput}
\end{Schunk}
\cmd{with}\cmd{table}
Pour obtenir la fréquence relative des enfants de faible poids, il suffit de
diviser les effectifs précédents par les totaux colonnes, ce que l'on peut
réaliser ainsi avec \R :
\begin{Schunk}
\begin{Sinput}
> tab <- with(birthwt, table(age.dec, low))
> prop.table(tab, 2)[,"Yes"]
\end{Sinput}
\begin{Soutput}
  (14,20]   (20,25]   (25,45] 
0.3684211 0.4035088 0.2280702 
\end{Soutput}
\end{Schunk}
\cmd{prop.table}
La dernière commande permet d'isoler la colonne \texttt{Yes} du tableau
croisé.

Concernant l'ethnicité, on utilisera également la commande \texttt{table}
pour produire un tableau des effectifs, associée à \texttt{prop.table} pour
calculer les fréquences relatives, que l'on exprimera en \%.
\begin{Schunk}
\begin{Sinput}
> res <- rbind(table(birthwt$race), prop.table(table(birthwt$race))*100)
> rownames(res) <- c("n","%")
> round(res, 2)
\end{Sinput}
\begin{Soutput}
  White Black Other
n 96.00 26.00 67.00
% 50.79 13.76 35.45
\end{Soutput}
\end{Schunk}
\cmd{rbind}\cmd{table}\cmd{prop.table}\cmd{rownames}\cmd{round}

Pour résumer la distribution des variables selon le facteur de
stratification, on pourrait très bien utiliser la commande \texttt{summary}
sur chacune des variables (\texttt{race}, \texttt{smoke}, \texttt{ui},
\texttt{ht} et \texttt{age}). Toutefois, il existe une manière plus
"économique" d'effectuer ce type de tâche avec la commande
\texttt{summary.formula} disponible dans le package \texttt{Hmisc}.
\begin{Schunk}
\begin{Sinput}
> library(Hmisc)
> summary(low ~ race + smoke + ui + ht + age, data=birthwt, method="reverse")
\end{Sinput}
\begin{Soutput}
Descriptive Statistics by low
+------------+--------------+--------------+
|            |No            |Yes           |
|            |(N=130)       |(N=59)        |
+------------+--------------+--------------+
|race : White|   56% (73)   |   39% (23)   |
+------------+--------------+--------------+
|    Black   |   12% (15)   |   19% (11)   |
+------------+--------------+--------------+
|    Other   |   32% (42)   |   42% (25)   |
+------------+--------------+--------------+
|smoke : Yes |   34% (44)   |   51% (30)   |
+------------+--------------+--------------+
|ui : Yes    |   11% ( 14)  |   24% ( 14)  |
+------------+--------------+--------------+
|ht : Yes    |    4% (  5)  |   12% (  7)  |
+------------+--------------+--------------+
|age         |19.0/23.0/28.0|19.5/22.0/25.0|
+------------+--------------+--------------+
\end{Soutput}
\end{Schunk}
\cmd{library}\cmd{summary.formula}
Le principe général de cette commande est le suivant : on exprime à l'aide
d'une formule, selon la terminologie \R, la relation entre les variables ;
ici, décrire selon les modalités de la variable \texttt{low} les
distributions des autres variables indiquées dans la commande
(\texttt{table} pour les variables qualitatives et \texttt{quantile} pour
les variables numériques). Les commandes \texttt{tapply} (pour les variables
numériques) et \texttt{table} (pour les variables qualitatives) restent
toutefois applicables.
\end{sol}
\end{exo}
% 
% Tri à plat, graphique en barres
%
%% \begin{exo}\label{exo:2.5}
  
%% \end{exo}

\Closesolutionfile{solutions}


\chapter*{Devoir \no 1}
\addcontentsline{toc}{chapter}{Devoir \no 1}
  
Les exercices sont indépendants. Une seule réponse est correcte pour chaque
question. Lorsque vous ne savez pas répondre, cochez la case correspondante.

\section*{Exercice 1}
\noindent À partir des données sur les poids à la naissance, décrits à
l'exercice~\ref{exo:2.4} et que vous pouvez charger à l'aide de la commande
\verb|data(birthwt, package="MASS")|, veuillez indiquer les commandes
permettant de répondre aux questions suivantes :
\begin{description}
\item[\bf 1.1] \marginpar{\phantom{text}1.1 $\square$} Quel est le poids moyen des bébés
  dont la mère est âgée de 20 ans ou moins ?
  \begin{description}
  \item[A.] \verb|mean(birthwt$bwt & birthwt$age < 20)|
  \item[B.] \verb|mean(birthwt$bwt[birthwt$age <= 20])|
  \item[C.] \verb|mean(birthwt$bwt[age <= 20])|
  \item[D.] \verb|mean(birthwt$bwt[birthwt$age < 20])|
  \item[E.] Je ne sais pas.
  \end{description}
\item[\bf 1.2] \marginpar{\phantom{text}1.2 $\square$} Combien de bébés ont une mère ayant
  eu des antécédents d'hypertension ?
  \begin{description}
  \item[A.] \verb|table(birthwt$ht)[1]|
  \item[B.] \verb|nrow(birthwt$ht == 1)|
  \item[C.] \verb|sum(birtwht$ht == 1)|
  \item[D.] \verb|birthwt$ht[birthwt$low == 1]|
  \item[E.] Je ne sais pas.
  \end{description}
\item[\bf 1.3] \marginpar{\phantom{text}1.3 $\square$} On souhaite afficher la
  distribution des valeurs prises par la variable \texttt{ftv}, sous forme
  de pourcentage. Quelle commande est la plus appropriée ?
  \begin{description}
  \item[A.] \verb|barchart(ftv, data=birthwt, type="percent")|
  \item[B.] \verb|barchart(ftv, data=birthwt, type="%")|
  \item[C.] \verb|histogram(prop.table(table(birthwt$ftv))*100)|
  \item[D.] \verb|barchart(prop.table(table(birthwt$ftv))*100)|
  \item[E.] Je ne sais pas.    
  \end{description}  
\end{description}

Les sorties suivantes indiquent les résultats de différentes
procédures inférentielles (estimation ou test). Quelles commandes ont été
utilisées dans chacun des cas ? 
\begin{description}
\item[\bf 1.4] \marginpar{\phantom{text} 1.4 $\square$}
\begin{Verbatim}[frame=single]
	1-sample proportions test without continuity correction

data:  table(birthwt$ht), null probability 0.5 
X-squared = 144.0476, df = 1, p-value < 2.2e-16
alternative hypothesis: true p is not equal to 0.5 
95 percent confidence interval:
 0.8923149 0.9633103 
sample estimates:
        p 
0.9365079 
\end{Verbatim}
  \begin{description}
  \item[A.] \verb|bin.test(birthwt$ht)|
  \item[B.] \verb|prop.test(birthwt$ht)|
  \item[C.] \verb|prop.test(table(birthwt$ht))|
  \item[D.] \verb|prop.test(table(birthwt$ht), correct=FALSE)|
  \item[E.] Je ne sais pas.
  \end{description}  
\item[\bf 1.5] \marginpar{\phantom{text} 1.5 $\square$}
\begin{Verbatim}[frame=single]
  	One Sample t-test

data:  birthwt$bwt 
t = 55.5137, df = 188, p-value < 2.2e-16
alternative hypothesis: true mean is not equal to 0 
95 percent confidence interval:
 2839.952 3049.222 
sample estimates:
mean of x 
 2944.587 
\end{Verbatim}
  \begin{description}
  \item[A.] \verb|t.test(birthwt$bwt)|
  \item[B.] \verb|t.test(birthwt$bwt, alternative=0)|
  \item[C.] \verb|t.test(bwt, data=birthwt)|
  \item[D.] \verb|t.test(birthwt$bwt, correct=FALSE)|
  \item[E.] Je ne sais pas.
  \end{description}  
\end{description}

\begin{description}
\item[\bf 1.6] \marginpar{\phantom{text}1.6 $\square$} Soit l'histogramme de fréquences
  suivant :
\includegraphics{figs/fig-exo1-1-6}

Quelle commande permet de reproduire cette figure (indépendemment du rapport
largeur/hauteur ou de la couleur) ?
\begin{description}
\item[A.] \verb|histogram(~ bwt/nrow(birthwt), data=birthwt, xlab="Poids des bébés (kg)")|
\item[B.] \verb|histogram(~ bwt, data=birthwt, type="density", xlab="Poids des bébés (kg)")|
\item[C.] \verb|histogram(~ bwt, data=birthwt, type="percent", xlab="Poids des bébés (kg)")|
\item[D.] \verb|histogram(~ bwt/1000, data=birthwt, type="percent", xlab="Poids des bébés (kg)")|
\item[E.] Je ne sais pas.
\end{description}
\end{description}

\section*{Exercice 2}
Soit les deux séries de mesures, $X_1$ et $X_2$, que l'on
supposera indépendantes :
\vskip1em

\begin{tabular}{l|rrrrrrrrrrrrrrr}
\texttt{x1} & 17.5 & 7.7 &16.5 &18.4 & 3.2 & 5.8 &13.5 &13.2 &12.0 & 7.8 &11.2 &13.2 & 4.1 & 2.4 & 8.0\\
\hline
\texttt{x2} & 18.6 &10.5 &15.0 & 3.0 &18.9& 13.1&  8.6& 16.9& 14.7 & 1.5 &17.6& 15.4& 17.9& 19.1&  9.8
\end{tabular}
\vskip1em

\begin{description}
\item[\bf 2.1] \marginpar{\phantom{text}2.1 $\square$} Quelles sont les moyennes de $X_1$
  et $X_2$ ?  
  \begin{description}
  \item[A.] \verb|mean(x1,x2)|
  \item[B.] \verb|c(mean(x1),mean(x2))|
  \item[C.] \verb|mean(c(x1,x2))|
  \item[D.] Je ne sais pas.
  \end{description}
\item[\bf 2.2] \marginpar{\phantom{text}2.2 $\square$} Quelle est l'intervalle
  inter-quartile de $X_1-X_2$ ?
  \begin{description}
  \item[A.] \verb|iqr(x1-x2)|
  \item[B.] \verb|sum(x1-x2)/14|
  \item[C.] \verb|diff(median(c(x1,x2)))|
  \item[D.] \verb|diff(quantile(x1-x2, c(.25,.75)))| 
  \item[E.] Je ne sais pas.
  \end{description}
\item[\bf 2.3] \marginpar{\phantom{text}2.3 $\square$} Donner un intervalle de confiance à
  95~\% pour la différence $X_1-X_2$ (en utilisant une loi de Student).
  \begin{description}
  \item[A.] \verb|t.test(x1 - x2)|
  \item[B.] \verb|(x1-x2) + c(-1,1) * qt(0.975, 28)*sd(x1-x2)/sqrt(length(c(x1,x2)))|
  \item[C.] \verb|(x1-x2) + c(-1,1) * qt(0.975, 28)*sd(x1-x2)/sqrt(length(c(x1,x2)))/2-2|
  \item[D.] Je ne sais pas.
  \end{description}
\item[\bf 2.4] \marginpar{\phantom{text}2.4 $\square$} On suppose que ces deux variables
  numériques ont été converties en variables qualitatives à 4 classes
  équilibrées de la manière suivante :
\begin{verbatim}
> x1b <- cut(x1, breaks=quantile(x1), include.lowest=TRUE)
> x2b <- cut(x2, breaks=quantile(x2), include.lowest=TRUE) 
\end{verbatim}  
  Indiquer la ou les classes modales dans chacun des deux cas. 
  \begin{description}
  \item[A.] \verb|which.max(table(x1b)); which.max(table(x2b))|
  \item[B.] \verb|max(table(x1b, x2b))|
  \item[C.] \verb|which.max(table(x1b, x2b))|
  \item[D.] Je ne sais pas.
  \end{description}
\item[\bf 2.5] \marginpar{\phantom{text}2.5 $\square$} Soit la série de commandes
  suivantes :  
\begin{verbatim}
> x3 <- factor((x1-x2)>0, labels=c("-","+"))
> table(x3)["+"]
\end{verbatim}
  Quel est le résultat renvoyé ?
\begin{description}
\item[A.] La somme des différences $X_1-X_2$ positives.
\item[B.] Le nombre de différences $X_1-X_2$ positives.
\item[C.] La proportion de valeurs de $X_1$ supérieures à celles de $X_2$.
\item[D.] Je ne sais pas.
\end{description}
\end{description}

\section*{Exercice 3}
Soit les données arrangées sous forme de tableau comme indiqué ci-dessous :
\vskip1em

\begin{tabular}{cccc}
\toprule  
id & age & sexe & taille \\  
\midrule
01 & 23 &  G  & 172 \\
02 & 22.5  & G & 178 \\
03 & 22.5 & F & 159 \\
04 & . & G & 182 \\
05 & 22 & F & 154 \\
06 & 21.5 & F & 161 \\
07 & 22 & G & 176 \\
08 & 21.5 & G & 191 \\
09 & 22.5 & F & 1.60 \\
10 & 23 & G & 173\\
\bottomrule
\end{tabular}
\vskip1em

Les données manquantes sont indiquées à l'aide du symbole \verb|.| (point),
le sexe des étudiants est codé \texttt{G} (garçons) ou \texttt{F} (filles),
et les tailles sont exprimées en \emph{cm}. Le tableau a été enregistré sous
\R sous le nom \texttt{dfrm}, et le résultat de la commande \verb|str(dfrm)|
est fourni ci-dessous :
\begin{Schunk}
\begin{Sinput}
> str(dfrm)
\end{Sinput}
\begin{Soutput}
'data.frame':	10 obs. of  4 variables:
 $ id    : Factor w/ 10 levels "01","02","03",..: 1 2 3 4 5 6 7 8 9 10
 $ age   : num  23 22.5 22.5 NA 22 21.5 22 21.5 22.5 23
 $ sexe  : Factor w/ 2 levels "F","G": 2 2 1 2 1 1 2 2 1 2
 $ taille: num  172 178 159 182 154 161 176 191 1.6 173
\end{Soutput}
\end{Schunk}
\begin{description}
\item[\bf 3.1] \marginpar{\phantom{text}3.1 $\square$} La commande \verb|mean(dfrm$age)|
  fournit l'âge moyen des étudiants.
  \begin{description}
  \item[A.] Vrai.
  \item[B.] Faux.
  \item[C.] Je ne sais pas.
  \end{description}
\item[\bf 3.2] \marginpar{\phantom{text}3.2 $\square$} On veut remplacer l'âge manquant
  par l'âge moyen. Laquelle parmi ces commandes est correcte ?
  \begin{description}
  \item[A.] \verb|dfrm$age[4] <- mean(dfrm$age)|
  \item[B.] \verb|dfrm$age[is.na(dfrm$age)] <- mean(dfrm$age, na.rm=TRUE)|
  \item[C.] \verb|dfrm$age[dfrm$id == "04"] <- mean(dfrm$age)|
  \item[D.] Je ne sais pas.
  \end{description}
\item[\bf 3.3] \marginpar{\phantom{text}3.3 $\square$} On souhaite afficher la proportion
  de garçons et de filles sous forme d'un diagramme en barres. La commande
  \verb|barchart(sexe, data=dfrm)| est-elle correcte ?
  \begin{description}
  \item[A.] Vrai.
  \item[B.] Faux.
  \item[C.] Je ne sais pas.
  \end{description}
\item[\bf 3.4] \marginpar{\phantom{text}3.4 $\square$} Pour afficher un histogramme des
  tailles des garçons, quelle commande faut-il utiliser ?
  \begin{description}
  \item[A.] \verb|histogram(taille, subset=sexe == "G")|
  \item[B.] \verb|barchart(~ taille, data=subset(dfrm, sexe=="G"))|
  \item[C.] \verb|histogram(taille, data=dfrm, subset=sexe == "G")|
  \item[D.] Je ne sais pas.
  \end{description}
\item[\bf 3.5] \marginpar{\phantom{text}3.5 $\square$} Que fait la commande suivante ? 
\begin{verbatim}
with(subset(dfrm, sexe=="F"), mean(age) + c(-1,1) * qt(0.95, 3) * sd(age)/sqrt(4))
\end{verbatim}
  \begin{description}
  \item[A.] Elle fournit un intervalle de confiance à 95~\% pour l'âge moyen
    chez les filles.
  \item[B.] Elle fournit un intervalle de confiance à 90~\% pour l'âge moyen
    chez les filles.
  \item[C.] Je ne sais pas.
  \end{description}
\end{description}
  
  


%--------------------------------------------------------------- Chapter 03 --
\chapter{Comparaisons de deux variables}\label{chap:comparaisons}
\Opensolutionfile{solutions}[solutions3]


\section*{Énoncés}
%
% Everitt 2001 p. 64
%
\begin{exo}\label{exo:3.1}
On dispose des poids à la naissance d'un échantillon de 50 enfants
présentant un syndrôme de détresse respiratoire idiopathique aïgue. Ce type
de maladie peut entraîner la mort et on a observé 27 décès chez ces
enfants. Les données sont résumées dans le tableau ci-dessous et sont
disponibles dans le fichier \texttt{sirds.dat}, où les 27 premières
observations correspondent au groupe des enfants décédés au moment de
l'étude. \autocite[p.~64]{everitt01}
\vskip1em

\begin{tabular}{ll}
\toprule  
Enfants décédés &
1.050\; 1.175\; 1.230\; 1.310\; 1.500\; 1.600\; 1.720\; 1.750\; 1.770\; 2.275\; 2.500\; 1.030\; 1.100\; 1.185 \\
& 1.225\; 1.262\; 1.295\; 1.300\; 1.550\; 1.820\; 1.890\; 1.940\; 2.200\; 2.270\; 2.440\; 2.560\; 2.730 \\
Enfants vivants &
1.130\; 1.575\; 1.680\; 1.760\; 1.930\; 2.015\; 2.090\; 2.600\; 2.700\; 2.950\; 3.160\; 3.400\; 3.640\; 2.830 \\
& 1.410\; 1.715\; 1.720\; 2.040\; 2.200\; 2.400\; 2.550\; 2.570\; 3.005 \\
\bottomrule
\end{tabular}
\vskip1em

Un chercheur s'intéresse à l'existence éventuelle d'une différence entre le
poids moyen des enfants ayant survécu et celui des enfants décédés des
suites de la maladie. 
\begin{description}
\item[(a)] Réaliser un test $t$ de Student. Peut-on rejeter l'hypothèse nulle
  d'absence de différences entre les deux groupes d'enfants ? 
\item[(b)] Vérifier graphiquement que les conditions d'applications du test
(normalité et homogénéité des variances) sont vérifiées. 
\item[(c)] Quel est l'intervalle de confiance à 95~\% pour la différence de
  moyenne observée ?
\end{description}
\begin{sol}
Pour le chargement des données, on lit séparément les données numériques
contenues dans le fichier que l'on stocke dans une variable appelée
\texttt{poids}, et on crée une seconde variable codant pour le statut des
enfants au moment de l'étude.
\begin{Schunk}
\begin{Sinput}
> poids <- scan("data/sirds.dat")
> status <- factor(rep(c("décédé", "vivant"), c(27,23)))
\end{Sinput}
\end{Schunk}
\cmd{scan}\cmd{factor}
On réalise ensuite un test $t$ pour échantillons indépendants, en supposant
l'homogénéité des variances :
\begin{Schunk}
\begin{Sinput}
> t.test(poids ~ status, var.equal=TRUE)
\end{Sinput}
\begin{Soutput}
	Two Sample t-test
data:  poids by status 
t = -3.6797, df = 48, p-value = 0.0005902
alternative hypothesis: true difference in means is not equal to 0 
95 percent confidence interval:
 -0.9520466 -0.2792545 
sample estimates:
mean in group décédé mean in group vivant 
            1.691741             2.307391 
\end{Soutput}
\end{Schunk}
Le résultat du test, bilatéral par défaut (\verb|alternative="two.sided"|),
indique que l'on peut effectivement rejeter l'hypothèse nulle.

Pour vérifier l'hypothèse de normalité des populations parentes, on peut
représenter graphiquement la distribution des poids à la naissance pour
chacun des groupes à l'aide d'un histogramme. Par exemple,
\begin{Schunk}
\begin{Sinput}
> histogram(~ poids | status, type="count")
\end{Sinput}
\end{Schunk}
\includegraphics{figs/fig-ex3-1c}
\cmd{histogram}

Une alternative consiste à utiliser des graphiques de type quantile-quantile.

L'homogénéité des variances peut être évaluée à l'aide de diagrammes de type
boîtes à moustache :
\begin{Schunk}
\begin{Sinput}
> bwplot(poids ~ status, xlab="status")
\end{Sinput}
\end{Schunk}
\includegraphics{figs/fig-ex3-1d}
\cmd{bwplot}
\end{sol}
\end{exo}
%
% data(sleep)
%
\begin{exo}\label{exo:3.2}
La qualité de sommeil de 10 patients a été mesurée avant (contrôle) et après
traitement par un des deux hypnotiques suivants : (1) D. hyoscyamine
hydrobromide et (2) L. hyoscyamine hydrobromide. Le critère de jugement
retenu par les chercheurs était le gain moyen de sommeil (en heures) par
rapport à la durée de sommeil de base
(contrôle). \autocite[p.~20]{student08} Les données sont reportées
ci-dessous et figurent également parmi les jeux de données de base de R
(\verb|data(sleep)|).  
\begin{verbatim}
D. hyoscyamine hydrobromide :
0.7 -1.6 -0.2 -1.2 -0.1  3.4  3.7  0.8  0.0  2.0
L. hyoscyamine hydrobromide :
1.9  0.8  1.1  0.1 -0.1  4.4  5.5  1.6  4.6  3.4
\end{verbatim}

Les chercheurs ont conclu que seule la deuxième molécule avait réellement un
effet soporifique. 
\begin{description}
\item[(a)] Estimer le temps moyen de sommeil pour chacune des deux
  molécules, ainsi que la différence entre ces deux moyennes.
\item[(b)] Afficher la distribution des scores de différence (LHH - DHH)
  sous forme d'un histogramme, en considérant des intervalles de classe
  d'une demi-heure, et indiquer la moyenne et l'écart-type de ces scores de
  différence.
\item[(c)] Vérifier l'exactitude des conclusions à l'aide d'un test de Student.
\end{description}
\begin{sol}
On notera qu'il s'agit des mêmes patients qui sont testés dans les deux
conditions (le sujet est pris comme son propre témoin). L'aide en ligne pour
le jeu de données \texttt{sleep} indique bien que les deux premières
variables, \texttt{extra} (différentiel de temps de sommeil) et
\texttt{group} (type de médicament), sont les variables d'intérêt.
\begin{Schunk}
\begin{Sinput}
> data(sleep)
> mean(sleep$extra[sleep$group == 1])  # D. hyoscyamine hydrobromide
\end{Sinput}
\begin{Soutput}
[1] 0.75
\end{Soutput}
\begin{Sinput}
> mean(sleep$extra[sleep$group == 2])  # L. hyoscyamine hydrobromide
\end{Sinput}
\begin{Soutput}
[1] 2.33
\end{Soutput}
\begin{Sinput}
> m <- with(sleep, tapply(extra, group, mean))
> m[2] - m[1]
\end{Sinput}
\begin{Soutput}
   2 
1.58 
\end{Soutput}
\end{Schunk}
\cmd{data}\cmd{mean}\cmd{with}\cmd{tapply}
Les commandes ci-dessus permettent de vérifier les moyennes de groupe et la
différence de gain de sommeil entre les molécules L. hyoscyamine
hydrobromide et D. hyoscyamine hydrobromide, en gardant à l'esprit que la
commande 
\begin{verbatim}
with(sleep, tapply(extra, group, mean)) 
\end{verbatim}
renvoit deux valeurs (les moyennes par type de molécule) et que la
différence deux valeurs stockées dans la variable auxiliaire \texttt{m}
renvoit bien la différence de moyennes entre les deux traitements.

Pour calculer les scores de différences, on procèdera comme dans
l'exercice~1.5 (p.~\pageref{exo:1.5}), soit
\begin{Schunk}
\begin{Sinput}
> sdif <- sleep$extra[sleep$group == 2] - sleep$extra[sleep$group == 1]
> c(mean=mean(sdif), sd=sd(sdif))
\end{Sinput}
\begin{Soutput}
    mean       sd 
1.580000 1.229995 
\end{Soutput}
\end{Schunk}
\cmd{mean}
Le calcul de la moyenne et de l'écart-type de ces scores de différences ne
pose pas de difficulté particulière, pas plus que l'affichage de leur
distribution sous forme d'histogramme.
\begin{Schunk}
\begin{Sinput}
> histogram(~ sdif, breaks=seq(0, 5, by=0.5), xlab="LHH - DHH")
\end{Sinput}
\end{Schunk}
\includegraphics{figs/fig-ex3-2c}
\cmd{histogram}

Le résultat du test $t$ est obtenu à partir de la commande \texttt{t.test}
en fournissant la variable réponse et le facteur de classification sous
forme d'une formule (variable réponse à gauche, variable explicative à
droite). Le résultat du test est reporté ci-dessous :
\begin{Schunk}
\begin{Sinput}
> t.test(extra ~ group, data=sleep, paired=TRUE)
\end{Sinput}
\begin{Soutput}
	Paired t-test
data:  extra by group 
t = -4.0621, df = 9, p-value = 0.002833
alternative hypothesis: true difference in means is not equal to 0 
95 percent confidence interval:
 -2.4598858 -0.7001142 
sample estimates:
mean of the differences 
                  -1.58 
\end{Soutput}
\end{Schunk}
\cmd{t.test}
On notera qu'il est nécessaire d'indiquer à R où trouver les variables
d'intérêt, d'où l'usage de \verb|data=sleep|. L'option \verb|paired=TRUE|
est nécessaire pour rendre compte de l'appariement des observations. Notons
également que, par défaut, R présente la différence de moyennes entre le
premier et le second traitement, et non le second moins le premier comme
calculé précédemment.

Le résultat significatif du test et le sens de la différence de moyenne
observée (au niveau du gain horaire de sommeil) est bien en accord avec les
conclusions des chercheurs. On peut visualiser les résultats sous forme d'un
diagramme en barres :
\begin{Schunk}
\begin{Sinput}
> barchart(with(sleep, tapply(extra, group, mean)))
\end{Sinput}
\end{Schunk}
\includegraphics{figs/fig-3-2e}
\cmd{barchart}
\end{sol}
\end{exo}
%
% Hollander 1999 p. 29
%
\begin{exo}\label{exo:3.3}
Dans une étude clinique, des chercheurs se sont intéressés à l'effet d'une
certaine forme de thérapie (par administration de tranquilisants) sur
l'évolution de 9 patients souffrant d'un trouble mixte combinant anxiété et
dépression. Le niveau de dépression était mesuré à partir de l'échelle de
dépression de Hamilton à l'inclusion (première visite) et lors d'une seconde
visite (après traitement). Il s'agit d'une échelle composé de 21 items à
plusieurs modalités de réponse ordonnées à partir de laquelle on peut
calculer un score total ou moyen. Les données recueillies sont indiquées
ci-dessous :\autocite[p.~29]{hollander99}
\vskip1em

\begin{verbatim}
T0 :
1.83 0.50 1.62 2.48 1.68 1.88 1.55 3.06 1.30
T1 : 
0.878 0.647 0.598 2.050 1.060 1.290 1.060 3.140 1.290
\end{verbatim}
\vskip1em

On cherche à démontrer que le traitement a bien un effet se traduisant par
une diminution des scores moyens individuels. Pour cela, on se propose de
réaliser un test non-paramétrique.
\begin{description}
\item[(a)] Représenter la distribution des scores sous forme
  d'un diagramme de dispersion (en abscisses, données à la 1\iere\ visite ; en
  ordonnées, données à la 2\ieme\ visite).
\item[(b)] Effectuer un test de Wilcoxon.
\end{description}
\begin{sol}
Pour la saisie de données, on crée deux variables quantitatives représentant
les mesures individuelles collectées à la 1\iere\ et à la 2\ieme\ visite :
\begin{Schunk}
\begin{Sinput}
> occ1 <- c(1.83,0.50,1.62,2.48,1.68,1.88,1.55,3.06,1.30)
> occ2 <- c(0.878,0.647,0.598,2.05,1.06,1.29,1.06,3.14,1.29)
\end{Sinput}
\end{Schunk}

L'affichage graphique des deux séries de mesure ne pose pas de difficulté :
\begin{Schunk}
\begin{Sinput}
> xyplot(occ2 ~ occ1, type=c("p","g"), xlab="Première visite", ylab="Seconde visite")
\end{Sinput}
\end{Schunk}
\includegraphics{figs/fig-ex3-3b}
\cmd{xyplot}

Pour faciliter la lisibilité, on pourrait rajouter une droite de pente 1. On
se contentera de superposer une grille pour faciliter la comparaison des
mesures avant-après (\verb|type="g"|).

Les données étant appariées (les mêmes patients sont mesurés à deux
occasions), il faut le préciser à R à l'aide de l'option \verb|paired=TRUE|.   
\begin{Schunk}
\begin{Sinput}
> wilcox.test(occ1, occ2, paired=TRUE)
\end{Sinput}
\begin{Soutput}
	Wilcoxon signed rank test
data:  occ1 and occ2 
V = 40, p-value = 0.03906
alternative hypothesis: true location shift is not equal to 0 
\end{Soutput}
\end{Schunk}
\cmd{wilcox.test}
Le résultat du test suggère une amélioration de l'état (diminution du
score Hamilton de -0.43 points en moyenne).
\end{sol}
\end{exo}
%
% Selvin 1998 p. 323
%
\begin{exo}\label{exo:3.4}
Dans un essai clinique, on a cherché à évaluer un régime supposé réduire le
nombre de symptômes associé à une maladie bénigne du sein. Un groupe de 229
femmes ayant cette maladie ont été alétoirement réparties en deux
groupes. Le premier groupe a reçu les soins courants, tandis que les
patientes du second groupe suivaient un régime spécial (variable B =
traitement). Après un an, les individus ont été évalués et ont été classés
dans l'une des deux catégories : amélioration ou pas d'amélioration
(variable A = réponse). Les résultats sont résumés dans le tableau suivant,
pour une partie de l'échantillon :\autocite[p.~323]{selvin98}
\vskip1em

\begin{tabular}{l|cc|r}
& régime & pas de régime & total \\
\hline
amélioration & 26 & 21 & 47 \\
pas d'amélioration & 38 & 44 & 82 \\
\hline
total & 64 & 65 & 129
\end{tabular}
\vskip1em

\begin{description}
\item[(a)] Réaliser un test du chi-deux.  
\item[(b)] Quels sont les effectifs théoriques attendus sous une hypothèse
  d'indépendance ?
\item[(c)] Comparer les résultats obtenus en (a) avec ceux d'un test de
  Fisher.
\item[(d)] Donner un intervalle de confiance pour la différence de
  proportion d'amélioration entre les deux groupes de patientes.
\end{description}
\begin{sol}
On ne dispose que des données concernant les effectifs tels que reportés
dans le tableau, mais il n'est pas nécessaire d'avoir les données brutes
pour réaliser le test du $\chi^2$. Celui-ci est obtenu à partir de la
commande \texttt{chisq.test} et inclut par défaut une correction de
continuité (Yates).
\begin{Schunk}
\begin{Sinput}
> regime <- matrix(c(26,38,21,44), nrow=2)
> dimnames(regime) <- list(c("amélioration","pas d'amélioration"), c("régime","pas de régime"))
> regime
\end{Sinput}
\begin{Soutput}
                   régime pas de régime
amélioration           26            21
pas d'amélioration     38            44
\end{Soutput}
\begin{Sinput}
> chisq.test(regime)
\end{Sinput}
\begin{Soutput}
	Pearson's Chi-squared test with Yates' continuity correction
data:  regime 
X-squared = 0.6376, df = 1, p-value = 0.4246
\end{Soutput}
\end{Schunk}
\cmd{matrix}\cmd{dimnames}\cmd{list}\cmd{chisq.test}
Si l'on ne souhaite pas appliquer de correction de continuité, il faut
ajouter l'option \verb|correct=FALSE|.

Les effectifs théoriques ne sont pas affichés avec le résultat du test, mais
on peut les obtenir comme suit :
\begin{Schunk}
\begin{Sinput}
> chisq.test(regime)$expected
\end{Sinput}
\begin{Soutput}
                     régime pas de régime
amélioration       23.31783      23.68217
pas d'amélioration 40.68217      41.31783
\end{Soutput}
\end{Schunk}

Concernant le test de Fisher, on procède de la même manière en utilisant la
comamnde \texttt{fisher.test} à partir du tableau de contingence.
\begin{Schunk}
\begin{Sinput}
> fisher.test(regime)
\end{Sinput}
\begin{Soutput}
	Fisher's Exact Test for Count Data
data:  regime 
p-value = 0.3636
alternative hypothesis: true odds ratio is not equal to 1 
95 percent confidence interval:
 0.6562985 3.1429451 
sample estimates:
odds ratio 
  1.429543 
\end{Soutput}
\end{Schunk}
% FIXME:
% finir commentaires

Les proportions d'intérêt sont 26/64=0.41 (régime) et 21/65=0.32 (pas de
régime), d'où le test de proportion qui donne l'intervalle de confiance à
95~\% pour la différence recherchée. 
\begin{Schunk}
\begin{Sinput}
> prop.test(c(26,21), c(64,65))
\end{Sinput}
\begin{Soutput}
	2-sample test for equality of proportions with continuity correction
data:  c(26, 21) out of c(64, 65) 
X-squared = 0.6376, df = 1, p-value = 0.4246
alternative hypothesis: two.sided 
95 percent confidence interval:
 -0.09787049  0.26421664 
sample estimates:
   prop 1    prop 2 
0.4062500 0.3230769 
\end{Soutput}
\end{Schunk}

\end{sol}
\end{exo}
%
% Agresti 2002 p. 72
%
\begin{exo}\label{exo:3.5}
Dans un essai clinique, 1360 patients ayant déjà eu un infarctus dy myocarde
ont été assignés à l'un des deux groupes de traitement suivants : prise en
charge par aspirine à faible dose en une seule prise \emph{versus}
placebo. La table ci-après indique le nombre de décès par infarctus lors de
la période de suivi de trois ans :\autocite[p.~72]{agresti02} 
\vskip1em

\begin{tabular}{lccc}
\toprule
& \multicolumn{2}{c}{Infarctus} & \\
\cmidrule(r){2-3}
& Oui & Non & Total \\
\midrule
Placebo & 28 & 656 & 684 \\
Aspirine & 18 & 658 & 676 \\
\bottomrule
\end{tabular}
\vskip1em

\begin{description}
\item[(a)] Calculer la proportion d'infarctus du myocarde dans les deux
  groupes de patients.
\item[(b)] Représenter graphiquement le tableau précédent sous forme d'un
  diagramme en barres ou d'un diagramme en points ("dotplot" de Cleveland).
\item[(c)] Indiquer la valeur de l'odds-ratio ainsi que du risque
  relatif. Pour l'odds-ratio, on considérera comme catégories de référence
  les modalités représentées par la première ligne et la première colonne
  du tableau. 
\item[(d)] À partir de l'intervalle de confiance à 95~\% pour l'odds, quelle
  conclusion peut-on tirer sur l'effet de l'aspirine dans la prévention d'un
  infarctus du myocarde ?
\end{description}
\begin{sol}
On travaillera à partir du tableau de contingence directement (aucune des
questions posées ne nécessite d'avoir accès aux données individuelles).
\begin{Schunk}
\begin{Sinput}
> aspirine <- matrix(c(28,18,656,658), nrow=2) 
> dimnames(aspirine) <- list(c("Placebo","Aspirine"), c("Oui","Non"))
> aspirine
\end{Sinput}
\begin{Soutput}
         Oui Non
Placebo   28 656
Aspirine  18 658
\end{Soutput}
\end{Schunk}
\cmd{matrix}\cmd{dimnames}\cmd{list}
On retiendra que par défaut R organise les données en colonnes (sauf si l'on
précise l'option \verb|byrow=TRUE|), d'où la saisie des effectifs selon la
présence ou non d'un infarctus (cf. exercice~\ref{exo:3.4}). Les effectifs
marginaux (totaux lignes et colonnes) s'obtiennent simplement en calculant
les sommes des cellules correspondantes. Par exemple, pour reproduire le
premier total ligne du tableau donné, on pourrait procéder ainsi :
\begin{Schunk}
\begin{Sinput}
> sum(aspirine["Placebo",])
\end{Sinput}
\begin{Soutput}
[1] 684
\end{Soutput}
\end{Schunk}
\cmd{sum}
ou, de manière équivalente, \verb|sum(aspirine[1,])|. Pour obtenir
simultanément les deux totaux lignes, sans répéter la commande précédente,
on peut utiliser la commande \texttt{apply}. Par exemple, pour les totaux
lignes 
\begin{Schunk}
\begin{Sinput}
> apply(aspirine, 1, sum)
\end{Sinput}
\begin{Soutput}
 Placebo Aspirine 
     684      676 
\end{Soutput}
\end{Schunk}
\cmd{apply}
Dans cette expression, l'option \texttt{1} signifie que l'on souhaite
travailler "par lignes". Pour obtenir les totaux colonnes, on utilisera
\begin{Schunk}
\begin{Sinput}
> apply(aspirine, 2, sum)
\end{Sinput}
\begin{Soutput}
 Oui  Non 
  46 1314 
\end{Soutput}
\end{Schunk}
\cmd{apply}
D'où les proportions demandées :
\begin{Schunk}
\begin{Sinput}
> round(aspirine[,"Oui"]/sum(aspirine[,"Oui"]) * 100, 2)
\end{Sinput}
\begin{Soutput}
 Placebo Aspirine 
   60.87    39.13 
\end{Soutput}
\end{Schunk}
\cmd{round}

La solution proposée donne la répartition des infarctus (en~\%) entre les deux traitements. Si l'on conditionne sur le traitement, les proportions d'intérêt sont 28/684=0.04 (placebo) et 18/676=0.03 (aspirine) et la commande R à utiliser est alors : 
\begin{Schunk}
\begin{Sinput}
> round(aspirine["Placebo","Oui"]/sum(aspirine["Placebo",])*100, 2)
\end{Sinput}
\end{Schunk}
ou plus simplement \texttt{prop.table(aspirine, margin=1)[,"Oui"]}.

La distribution des effectifs peut être visualisée sous forme de diagramme
en barres comme ceci :
\begin{Schunk}
\begin{Sinput}
> barchart(aspirine, horizontal=FALSE, stack=FALSE, ylab="Effectifs")
\end{Sinput}
\end{Schunk}
\includegraphics{figs/fig-ex3-5f}
\cmd{barchart}

En fait, cette représentation graphique n'est pas très informative car la
prévalence observée est très faible dans les deux groupes de traitement. Une
représentation des données plus appropriée consisterait à afficher les
événements positifs (survenue d'un infarctus pendant la période de suivi)
rapportés aux effectifs marginaux (nombre total de patients randomisés dans
chaque groupe). Une solution possible est indiquée ci-après :
\begin{Schunk}
\begin{Sinput}
> prop.infarctus <- aspirine[,"Oui"]/sum(aspirine[,"Oui"])
> barchart(prop.infarctus, ylab="Fréquence relative d'infarctus")
\end{Sinput}
\end{Schunk}
\includegraphics{figs/fig-ex3-5g}
\cmd{barchart}\cmd{sum}

On peut calculer l'odds-ratio et le risque relatif comme on l'a fait plus
haut, c'est-à-dire en travaillant directement avec les valeurs des cellules
extraites du tableau \texttt{aspirine}. Par exemple, pour le risque relatif
\begin{Schunk}
\begin{Sinput}
> (aspirine["Placebo","Oui"]/sum(aspirine["Placebo",])) / (aspirine["Aspirine","Oui"]/sum(aspirine["Aspirine",]))
\end{Sinput}
\begin{Soutput}
[1] 1.537362
\end{Soutput}
\end{Schunk}
Toutefois, il existe des librairies spécialisées pour effectuer ce genre de
calcul, et fournir une estimation de la précision de ces estimateurs sous
forme d'intervalles de confiance. Pour calculer l'odds-ratio, nous
utiliserons donc la commande disponible dans le package \texttt{vcd}, qu'il
est nécessaire de charger au préalable à l'aide de la commande
\texttt{library} : 
\begin{Schunk}
\begin{Sinput}
> library(vcd)
> asp.or <- oddsratio(aspirine, log=FALSE) 
> print(list(or=asp.or , conf.int=confint(asp.or))) 
\end{Sinput}
\begin{Soutput}
$or
[1] 1.560298
$conf.int
           lwr      upr
[1,] 0.8605337 2.829093
\end{Soutput}
\begin{Sinput}
> summary(oddsratio(aspirine))
\end{Sinput}
\begin{Soutput}
     Log Odds Ratio Std. Error z value Pr(>|z|)
[1,]        0.44488    0.30362  1.4653  0.07143
\end{Soutput}
\end{Schunk}
\cmd{oddsratio}\cmd{vcd}\cmd{confint}
L'estimé du "vrai" OR est plutôt imprécis (l'IC à 95~\% est large et
contient la valeur 1), et le test sur le $\log(\text{OR})$ n'est pas
significatif ($p = 0.071$). On concluera que rien ne permet d'affirmer que
la probabilité de décès dûe à un infarctus diffère selon le facteur
d'exposition.
\end{sol}
\end{exo}
%
% Peat 2005 p. 235
%
\begin{exo}\label{exo:3.6}
Une étude a porté sur 86 enfants suivis dans un centre pour apprendre à
gérer leur maladie. Lors de leur arrivée, on a demandé aux enfants s'ils
savaient gérer leur maladie dans les conditions optimales (non détaillées),
c'est-à-dire s'ils savaient à quel moment il leur fallait recourir au
traitement prescrit. La même question a été posée à ces enfants à la fin de
leur suivi dans le centre. La variable mesurée est la réponse (affirmative
ou non) à cette question à l'entrée et à la sortie de l'étude. Les données,
disponibles au format SPSS dans le fichier \texttt{health-camp.sav}, sont
résumées dans le tableau ci-après :\autocite[p.~235]{peat05}  
\vskip1em

\begin{tabular}{lll...}
\toprule
& & & \multicolumn{2}{c}{Gestion maladie (sortie)} &  \\
\cmidrule(r){4-5}
& & & \multicolumn{1}{c}{Non} & \multicolumn{1}{c}{Oui} & Total \\
\midrule
Gestion maladie (entrée) & Non & Effectif & 27 & 29 & 56 \\
& & Fréquence & 31.4~\% & 33.7~\% & 65.1~\% \\
& Oui & Effectif & 6 & 24 & 30 \\
& & Fréquence & 7.0~\% & 27.9~\% & 34.9~\% \\
Total & & Effectif & 33 & 53 & 86 \\
& & Fréquence & 38.4~\% & 61.6~\% & 100.0~\% \\
\bottomrule
\end{tabular}
\vskip1em

On se demande si le fait d'avoir suivi le programme de formation proposé
dans le centre a augmenté le nombre d'enfants ayant une bonne connaissance
de leur maladie et de sa gestion quotidienne.
\begin{description}
\item[(a)] Reproduire le tableau d'effectifs et de fréquences relatives
  précédent à partir des données brutes.
\item[(b)] Indiquer le résultat d'un test de McNemar.
\item[(c)] Comparer le résultat du test réalisé en (b) mais sans continuité
  de correction avec le résultat obtenu à partir d'un test binomial.
\end{description}
\begin{sol}
Il n'est pas vraiment nécessaire d'importer le tableau de données brutes et
l'on pourrait se contenter de travailler avec les effectifs présentés dans
le tableau, comme dans l'exercice~\ref{exo:3.5}. Toutefois, pour charger le
fichier SPSS, il est nécessaire d'utiliser une commande \R spécifique et
disponible dans le package \texttt{foreign}.
\begin{Schunk}
\begin{Sinput}
> library(foreign)
> hc <- read.spss("data/health-camp.sav", to.data.frame=TRUE)
\end{Sinput}
\end{Schunk}
\cmd{foreign}\cmd{read.spss}
L'option \verb|to.data.frame=TRUE| est importante car c'est celle qui assure
qu'à l'issue de l'importation les données seront bien stockées dans un
tableau où les lignes figurent les observations et les colonnes les
variables. Si l'on regarde comment les données sont représentées dans le
tableau importé, on constate que l'information est un peu vague :
\begin{Schunk}
\begin{Sinput}
> head(hc)
\end{Sinput}
\begin{Soutput}
  ID BEFORE AFTER BEFORE2 AFTER2
1  1    Yes   Yes     Yes    Yes
2  2     No    No      No     No
3  3    Yes   Yes     Yes    Yes
4  4    Yes    No     Yes     No
5  5     No    No      No     No
6  6    Yes   Yes     Yes    Yes
\end{Soutput}
\end{Schunk}
\cmd{head}
Ceci s'explique par la perte des labels associés aux variables dans SPSS. On
peut retrouver cette informaiton en interrogeant la base de données \R avec
la commande \texttt{str} :
\begin{Schunk}
\begin{Sinput}
> str(hc)
\end{Sinput}
\begin{Soutput}
'data.frame':	86 obs. of  5 variables:
 $ ID     : num  1 2 3 4 5 6 7 8 9 10 ...
 $ BEFORE : Factor w/ 2 levels "No","Yes": 2 1 2 2 1 2 2 1 2 1 ...
 $ AFTER  : Factor w/ 2 levels "No","Yes": 2 1 2 1 1 2 2 1 2 1 ...
 $ BEFORE2: Factor w/ 2 levels "No","Yes": 2 1 2 2 1 2 2 1 2 1 ...
 $ AFTER2 : Factor w/ 2 levels "No","Yes": 2 1 2 1 1 2 1 1 2 1 ...
 - attr(*, "variable.labels")= Named chr  "ID" "Knowledge-Time1" "Knowledge-Time2" "Medication-Time1" ...
  ..- attr(*, "names")= chr  "ID" "BEFORE" "AFTER" "BEFORE2" ...
\end{Soutput}
\end{Schunk}
\cmd{str}
À partir de la ligne \verb|attr(*, "variable.labels")|, on voit donc que les
colonnes \texttt{BEFORE} et \texttt{AFTER} correspondent à la question
portant sur la connaissance de la maladie, alors que les colonnes
\texttt{BEFORE2} et \texttt{AFTER2} correspondent à la question portant sur
le traitement.

Finalement, on peut reproduire le tableau initial (effectifs et fréquences
relatives) ainsi :
\begin{Schunk}
\begin{Sinput}
> table(hc[,c("BEFORE","AFTER")])
\end{Sinput}
\begin{Soutput}
      AFTER
BEFORE No Yes
   No  27  29
   Yes  6  24
\end{Soutput}
\begin{Sinput}
> round(prop.table(table(hc[,c("BEFORE","AFTER")])), 2)
\end{Sinput}
\begin{Soutput}
      AFTER
BEFORE   No  Yes
   No  0.31 0.34
   Yes 0.07 0.28
\end{Soutput}
\end{Schunk}
\cmd{table}\cmd{prop.table}
Pour obtenir les distributions marginales, on peut utiliser la commande
\texttt{margin.table}. 
\begin{Schunk}
\begin{Sinput}
> margin.table(table(hc[,c("BEFORE","AFTER")]), 1)
\end{Sinput}
\begin{Soutput}
BEFORE
 No Yes 
 56  30 
\end{Soutput}
\begin{Sinput}
> margin.table(table(hc[,c("BEFORE","AFTER")]), 2)
\end{Sinput}
\begin{Soutput}
AFTER
 No Yes 
 33  53 
\end{Soutput}
\end{Schunk}
\cmd{margin.table}

Pour réaliser le test de McNemar, on peut utiliser directement le tableau de
contingence construit précédemment. Par souci de commodité, on stockera ce
dernier dans une variable \R appelé \texttt{hc.tab}.
\begin{Schunk}
\begin{Sinput}
> hc.tab <- table(hc[,c("BEFORE","AFTER")])
> mcnemar.test(hc.tab)
\end{Sinput}
\begin{Soutput}
	McNemar's Chi-squared test with continuity correction
data:  hc.tab 
McNemar's chi-squared = 13.8286, df = 1, p-value = 0.0002003
\end{Soutput}
\end{Schunk}

On peut comparer les résultats du test de McNemar sans appliquer la
correction de continuité avec ceux d'un test binomial (test exact).
\begin{Schunk}
\begin{Sinput}
> binom.test(6, 6+29)
\end{Sinput}
\begin{Soutput}
	Exact binomial test
data:  6 and 6 + 29 
number of successes = 6, number of trials = 35, p-value = 0.0001168
alternative hypothesis: true probability of success is not equal to 0.5 
95 percent confidence interval:
 0.0656218 0.3364983 
sample estimates:
probability of success 
             0.1714286 
\end{Soutput}
\begin{Sinput}
> mcnemar.test(hc.tab, correct=FALSE)
\end{Sinput}
\begin{Soutput}
	McNemar's Chi-squared test
data:  hc.tab 
McNemar's chi-squared = 15.1143, df = 1, p-value = 0.0001012
\end{Soutput}
\end{Schunk}
\end{sol}
\end{exo}

\Closesolutionfile{solutions}

%--------------------------------------------------------------- Devoir 02 ---
\chapter*{Devoir \no 2}
\addcontentsline{toc}{chapter}{Devoir \no 2}

Les exercices sont indépendants. Une seule réponse est correcte pour chaque
question. Lorsque vous ne savez pas répondre, cochez la case correspondante.

\section*{Exercice 1}
Voici des données anthropométriques (poids, taille et périmètre crânien)
recueillies sur deux groupes de bébés (119 garçons, 137 filles) à leur
naissance. Un bref résumé des principaux indicateurs descriptifs pour ces
trois variables est fourni ci-après.
\vskip1em

\begin{tabular}{llrr}
\toprule
& & Garçons & Filles \\
\cmidrule(r){3-4}
& & (N=119) & (N=137) \\
\midrule
Poids (kg) & Moyenne & 3.44 & 3.53 \\
\texttt{BWEIGHT}           & Écart-type & 0.33 & 0.43 \\
           & Étendue & 2.70–4.43 & 2.71–4.72 \\
Taille (cm) & Moyenne & 50.33 & 50.28 \\
\texttt{BLENGTH}            & Écart-type & 0.78 & 0.85 \\
            & Étendue & 49.0–51.5 & 49.0–52.0 \\
PC (cm) & Moyenne & 34.94 & 34.25 \\
\texttt{BHEADCIR}        & Écart-type & 1.31 & 1.38 \\
        & Étendue & 31.5–38.0 & 29.5–38.0 \\
\bottomrule
\end{tabular}
\vskip1em

Les données ont été enregistrés dans un fichier texte nommé
\texttt{babies.dat} dont un aperçu est fourni ci-dessous :
\begin{verbatim}
ID BWEIGHT BLENGTH BHEADCIR GENDER
L002     3.09 50 33.5 1
L003     3.94 50 35 2
L006     3.2 49 36 1
L007     2.93 51 31.5 2
L017     3.16 49 34 1
L025     3.36 51 33 2
L034     4.24 50 33.5 2
\end{verbatim}

Veuillez indiquer les commandes permettant de répondre aux questions suivantes :
\begin{description}
\item[\bf 1.1] \marginpar{\phantom{text}1.1 $\square$} On souhaite importer les données
  sous R en utilisant la commande \texttt{read.table}. Quelles options
  faut-il renseigner ?
  \begin{description}
  \item[A.] \verb|babies <- read.table("babies.dat")|
  \item[B.] \verb|babies <- read.table("babies.dat", row.names=FALSE)|
  \item[C.] \verb|babies <- read.table(header=TRUE, "babies.dat")|
  \item[D.] \verb|babies <- read.table("babies.dat", header=TRUE)|
  \item[E.] Je ne sais pas
  \end{description}
\item[\bf 1.2] \marginpar{\phantom{text}1.2 $\square$} Comment recoder la variable
  \texttt{GENDER} en variable qualitative avec les modalités \texttt{G} (=1)
  et \texttt{F} (=2) ?
  \begin{description}
  \item[A.] \verb|babies$GENDER <- factor(babies$GENDER, levels=0:1, labels=c("G","F"))||
  \item[B.] \verb|babies$GENDER <- factor(babies$GENDER, levels=c("G","F"))|
  \item[C.] \verb|babies$GENDER <- factor(babies$GENDER, labels=c("G","F"))|
  \item[D.] \verb|babies$GENDER <- as.factor(babies$GENDER, labels=c("G","F"))|
  \item[E.] Je ne sais pas.
  \end{description}
\item[\bf 1.3] \marginpar{\phantom{text}1.3 $\square$} La distribution des effectifs selon
  le sexe peut être obtenue de la manière suivante
  \begin{description}
  \item[A.] \verb|table(GENDER, data=babies)|
  \item[B.] \verb|with(babies, table(~ GENDER))|
  \item[C.] \verb|margin.table(babies$GENDER)|
  \item[D.] \verb|table(babies$GENDER)|
  \item[E.] Je ne sais pas.
  \end{description}  
\item[\bf 1.4] \marginpar{\phantom{text}1.4 $\square$} Le poids moyen des bébés à la
  naissance, exprimé en kilogrammes, chez les sujets des deux sexes peut être
  obtenu de la manière suivante
  \begin{description}
  \item[A.] \verb|tapply(BWEIGHT ~ GENDER, data=babies, FUN=mean)|
  \item[B.] \verb|with(babies, tapply(BWEIGHT, GENDER, mean))|
  \item[C.] \verb|with(babies, tapply(GENDER, BWEIGHT, mean))|
  \item[D.] \verb|by(babies$GENDER, babies$BWEIGHT, mean)|
  \item[E.] Je ne sais pas.
  \end{description}  
\end{description}

\begin{description}
\item[\bf 1.5] \marginpar{\phantom{text}1.5 $\square$} Quelle commande permet de
  reproduire la figure suivante (indépendemment du rapport largeur/hauteur
  et de la couleur) ?
\begin{center}
  \includegraphics{./figs/dev2_histo}
\end{center}
\begin{description}
\item[A.] \verb|histogram(BHEADCIR ~ GENDER, data=babies)|
\item[B.] \verb|histogram(~ BHEADCIR, data=babies, groups=GENDER)|
\item[C.] \verb+histogram(~ BHEADCIR | GENDER, data=babies)+
\item[D.] \verb+histogram(~ BHEADCIR | GENDER, data=babies, type="count")+
\item[E.] Je ne sais pas.
\end{description}

\item[\bf 1.6] \marginpar{\phantom{text}1.6 $\square$} Quelle commande permet de
  reproduire la figure suivante (indépendemment du rapport largeur/hauteur
  et de la couleur) ?
\begin{center}
  \includegraphics{./figs/dev2_bwplot}
\end{center}
\begin{description}
\item[A.] \verb|bwplot(GENDER ~ BWEIGHT, data=babies)|
\item[B.] \verb|bwplot(~ BWEIGHT, data=babies, groups=GENDER)|
\item[C.] \verb+bwplot(~ BWEIGHT | GENDER, data=babies)+
\item[D.] \verb|bwplot(BWEIGHT ~ GENDER, data=babies, horizontal=TRUE)|
\item[E.] Je ne sais pas.
\end{description}
\end{description}

\begin{description}
\item[\bf 1.7] \marginpar{\phantom{text}1.7 $\square$} Voici les résultats d'un test de
  Student comparant les tailles moyennes des bébés des deux sexes à la
  naissance.
\begin{Verbatim}[frame=single]
data:  BWEIGHT by GENDER 
t = -1.9108, df = 249.659, p-value = 0.9714
alternative hypothesis: true difference in means is greater than 0 
95 percent confidence interval:
 -0.1681625        Inf 
sample estimates:
  mean in group Male mean in group Female 
            3.441429             3.531642 
\end{Verbatim}
De quel type de test s'agit-il exactement ?
\begin{description}
\item[A.] C'est un test unilatéral supposant l'homogénéité des variances.
\item[B.] C'est un test bilatéral supposant l'homogénéité des variances.
\item[C.] C'est un test unilatéral ne supposant pas l'homogénéité des variances.
\item[D.] C'est un test bilatéral ne supposant pas l'homogénéité des variances.
\item[E.] Je ne sais pas.
\end{description}
\item[\bf 1.8] \marginpar{\phantom{text}1.8 $\square$} Le statisticien décide de réaliser
  un test non-paramétrique à l'aide de la commande
\begin{verbatim}
wilcox.test(BWEIGHT ~ GENDER, data=babies, paired=TRUE)
\end{verbatim}
Ce test est-t-il correctement spécifié dans le contexte de cette étude ?
  \begin{description}
  \item[A.] Oui.
  \item[B.] Non.
  \item[C.] Je ne sais pas.
  \end{description}
\end{description}

\section*{Exercice 2}
La distribution des fumeurs (F) et non-fumeurs (NF) dans un ensemble de 177
personnes vivant en couple est indiquée dans le tableau suivant :
\vskip1em

\begin{tabular}{llrrr}
\toprule
& & \multicolumn{2}{c}{Mari} & \\
\cmidrule(r){3-4}
& & F & NF & Total \\
\midrule
Femme & F & 42 & 22 & 64 \\
& NF & 32 & 81 & 113 \\
& Total & 74 & 103 & 177 \\
\bottomrule
\end{tabular}
\vskip1em

Veuillez indiquer les commandes permettant de répondre aux questions suivantes :
\begin{description}
\item[\bf 2.1] \marginpar{\phantom{text}2.1 $\square$} Pour saisir ces données tabulaires
  sous R, on peut utiliser la commande
  \begin{description}
  \item[A.] \verb|smoke <- matrix(c(42,22,32,81))|
  \item[B.] \verb|smoke <- matrix(c(42,32,22,81), nrow=2, byrow=TRUE)|
  \item[C.] \verb|smoke <- matrix(c(42,22,32,81), nrow=2, byrow=TRUE)|
  \item[D.] Je ne sais pas.
  \end{description}
\item[\bf 2.2] \marginpar{\phantom{text}2.2 $\square$} Quelle commande permet d'obtenir la
  distribution marginale chez les femmes (totaux lignes) ?
  \begin{description}
  \item[A.] \verb|table(smoke)|
  \item[B.] \verb|margin.table(smoke, 1)|
  \item[C.] \verb|margin.table(smoke, 2)|
  \item[D.] \verb|prop.table(smoke)|
  \item[E.] Je ne sais pas.
  \end{description}
\item[\bf 2.3] \marginpar{\phantom{text}2.3 $\square$} Le nombre total d'individus peut
  être obtenu à l'aide de la commande
  \begin{description}
  \item[A.] \verb|sum(smoke)|
  \item[B.] \verb|length(smoke)|
  \item[C.] \verb|dim(smoke)|
  \item[D.] Je ne sais pas.
  \end{description}
\end{description}

\begin{description}
\item[\bf 2.4] \marginpar{\phantom{text}2.4 $\square$} La commande 
\begin{verbatim}
(smoke[1,1]/smoke[1,2]) / (smoke[2,1]/smoke[2,2])
\end{verbatim}
fournit une estimation
\begin{description}
\item[A.] du risque relatif,
\item[B.] de l'odds-ratio,
\item[C.] de la statistique $\chi^2$ ?
\item[D.] Je ne sais pas.
\end{description}  
\end{description}

\begin{description}
\item[\bf 2.5] \marginpar{\phantom{text}2.5 $\square$} Voici le résultat d'un test
  statistique :
\begin{Verbatim}[frame=single]
data:  smoke 
p-value = 1.669e-06
alternative hypothesis: true odds ratio is not equal to 1 
95 percent confidence interval:
 2.381413 9.864192 
sample estimates:
odds ratio 
  4.784393
\end{Verbatim}
De quel test s'agit-il ?
\begin{description}
\item[A.] Il s'agit d'un test du $\chi^2$.
\item[B.] Il s'agit d'un test binomial pour la proportion de maris fumeurs
  versus non-fumeurs.
\item[C.] Il s'agit d'un test exact de Fisher.
\item[D.] Je ne sais pas.
\end{description}  
\end{description}

\section*{Exercice 3}
Soit les données suivantes :
\begin{verbatim}
    M+  M-
E+  72  89
E-  28 131
\end{verbatim}
où les colonnes \texttt{M+/M-} désignent le nombre d'individus malades et
non-malades, respectivement, et les lignes \texttt{E+/E-} les individus
exposés et non-exposés dans une étude de cohorte. Le tableau est appelé
\texttt{tab} sous R.

\begin{description}
\item[\bf 3.1] \marginpar{\phantom{text}3.1 $\square$} Quelle commande permet de
  reproduire la figure suivante (indépendemment du rapport largeur/hauteur
  et de la couleur) ?
\begin{center}
  \includegraphics{./figs/dev2_barchart}
\end{center}
\begin{description}
\item[A.] \verb|barchart(prop.table(tab, 2)*100, stack=FALSE, xlim=c(-5, 105))|
\item[B.] \verb|barchart(prop.table(tab, 1)*100, stack=FALSE, xlim=c(-5, 105))|
\item[C.] \verb|barchart(prop.table(tab, 1)*100, xlim=c(-5, 105))|
\item[D.] Je ne sais pas.
\end{description}
\item[\bf 3.2] \marginpar{\phantom{text}3.2 $\square$} La prévalence observée est donnée par : 
\begin{description}
\item[A.] \verb|margin.table(tab, 2)[1]/320|
\item[B.] \verb|prop.table(tab, 2)[1]|
\item[C.] \verb|sum(tab[1,])/320|
\item[D.] Je ne sais pas.
\end{description}
\item[\bf 3.3] \marginpar{\phantom{text}3.3 $\square$} La commande suivante fournit un
  intervalle de confiance à 95~\% pour 
l'odds-ratio : 
\begin{verbatim}
confint(oddsratio(tab))
\end{verbatim}
\begin{description}
\item[A.] Vrai.
\item[B.] Faux.
\item[C.] Je ne sais pas.
\end{description}
\end{description}


% --------------------------------------------------------------- Chapter 04 --
\chapter{Analyse de variance et plans d'expérience}\label{chap:anova}
\Opensolutionfile{solutions}[solutions4]


\section*{Énoncés}
%
% Dupont 2009 p. 326
%
\begin{exo}\label{exo:4.1}
Dans une étude sur le gène du récepteur à \oe strogènes, des généticiens se
sont intéressés à la relation entre le génotype et l'âge de diagnostic du
cancer du sein. Le génotype était déterminé à partir des deux allèles d'un
polymorphisme de restriction de séquence (1.6 et 0.7 kb), soit trois groupes
de sujets : patients homozygotes pour l'allèle 0.7 kb (0.7/0.7), patients
homozygotes pour l'allèle 1.6 kb (1.6/1.6), et patients hétérozygotes
(1.6/0.7). Les données ont été recueillies sur 59 patientes atteintes d'un
cancer du sein, et sont disponibles dans le fichier
\texttt{polymorphism.dta} (fichier \Stata). Les données moyennes sont
indiquées ci-dessous :\autocite[p.~327]{dupont09}
\vskip1em

\begin{tabular}{lrrrr}
\toprule
& \multicolumn{3}{c}{Génotype} & \\
\cmidrule(r){2-4}
& 1.6/1.6 & 1.6/0.7 & 0.7/0.7 & Total \\
\midrule
Nombre de patients & 14 & 29 & 16 & 59 \\
\emph{Âge lors du diagnostic} & & & & \\
\quad Moyenne & 64.64 & 64.38 & 50.38 & 60.64 \\
\quad Écart-type & 11.18 & 13.26 & 10.64 & 13.49 \\
\quad IC 95~\% & (58.1–71.1) & (59.9–68.9) & (44.3–56.5) & \\
\bottomrule
\end{tabular}
\vskip1em

\begin{description}
\item[(a)] Tester l'hypothèse nulle selon laquelle l'âge de diagnostic ne varie
  pas selon le génotype à l'aide d'une ANOVA. Représenter sous forme
  graphique la distribution des âges pour chaque génotype.
\item[(b)] Les intervalles de confiance présentés dans le tableau ci-dessus ont
  été estimés en supposant l'homogénéité des variances, c'est-à-dire en
  utilisant l'estimé de la variance commune ; donner la valeur de ces
  intervalles de confiance sans supposer l'homoscédasticité. 
\item[(c)] Estimer les différences de moyenne correspondant à l'ensemble des
  combinaisons possibles des trois génotypes, avec une estimation de
  l'intervalle de confiance à 95~\% associé et un test paramétrique
  permettant d'évaluer le degré de significativité de la différence
  observée.
\item[(d)] Représenter graphiquement les moyennes de groupe avec des
  intervalles de confiance à 95~\%.
\end{description}
\begin{sol}
Pour charger les données, il est nécessaire d'importer la librairie
\texttt{foreign} qui permet de lire les fichiers enregistrés par \Stata.
\begin{Schunk}
\begin{Sinput}
> library(foreign)
> polymsm <- read.dta("data/polymorphism.dta")
> head(polymsm)
\end{Sinput}
\begin{Soutput}
  id age genotype
1  1  43  1.6/1.6
2  2  47  1.6/1.6
3  3  55  1.6/1.6
4  4  57  1.6/1.6
5  5  61  1.6/1.6
6  6  63  1.6/1.6
\end{Soutput}
\end{Schunk}
\cmd{read.dta}\cmd{library}\cmd{head}  
Notons que \texttt{polymsm} est un \texttt{data.frame}, ce qui sera utile
pour utiliser les commandes graphiques ou réaliser l'ANOVA. La première
colonne contient une série d'identifiants uniques pour les individus ; elle
ne sera pas utile dans le cas présent. Pour calculer les moyennes et
écart-types de chaque groupe, on peut procéder comme suit : 
\begin{Schunk}
\begin{Sinput}
> with(polymsm, tapply(age, genotype, mean))
\end{Sinput}
\begin{Soutput}
 1.6/1.6  1.6/0.7  0.7/0.7 
64.64286 64.37931 50.37500 
\end{Soutput}
\begin{Sinput}
> with(polymsm, tapply(age, genotype, sd))
\end{Sinput}
\begin{Soutput}
 1.6/1.6  1.6/0.7  0.7/0.7 
11.18108 13.25954 10.63877 
\end{Soutput}
\end{Schunk}
\cmd{tapply}\cmd{mean}\cmd{sd}
La distribution des âges selon le génotype est indiquée dans le diagramme en
boîtes à moustaches suivant.
\begin{Schunk}
\begin{Sinput}
> bwplot(age ~ genotype, data=polymsm)
\end{Sinput}
\end{Schunk}
\includegraphics{figs/fig-ex4-1c}
\cmd{bwplot}

Il est également possible d'utiliser des histogrammes, en utilisant
\texttt{histogram} au lieu de \texttt{bwplot}.
\begin{Schunk}
\begin{Sinput}
> histogram(~ age | genotype, data=polymsm)
\end{Sinput}
\end{Schunk}
\includegraphics{figs/fig-ex4-1d}
\cmd{histogram}

Le modèle d'ANOVA est réalisé à l'aide de la commande \texttt{aov} :
\begin{Schunk}
\begin{Sinput}
> aov.res <- aov(age ~ genotype, data=polymsm)
> summary(aov.res)
\end{Sinput}
\begin{Soutput}
            Df Sum Sq Mean Sq F value   Pr(>F)
genotype     2   2316  1157.9   7.863 0.000978
Residuals   56   8246   147.2                 
\end{Soutput}
\end{Schunk}
\cmd{aov}\cmd{summary.aov} 
La statistique de test $F$ est reportée dans la colonne \texttt{F value},
avec le degré de significativité associé dans la colonne suivante
(\texttt{Pr(>F)}). L'ANOVA indique qu'au moins une paire de moyennes est
significativement différente, en considérant un rique d'erreur de 5~\%.

On peut vérifier l'exactitude des intervalles de confiance reportés dans le
tableau présenté plus haut. Pour cela, il nous faut une estimation de
l'erreur résiduelle, qui est simplement la racine carrée du carré moyen
associé au terme d'erreur dans le tableau d'ANOVA ci-dessus.
\begin{Schunk}
\begin{Sinput}
> mse <- unlist(summary(aov.res))["Mean Sq2"]
> se <- sqrt(mse)
\end{Sinput}
\end{Schunk}
\cmd{unlist}\cmd{sqrt}
Soit, une estimation de la racine de la variance commune de
12.13. À partir de là, on peut construire les intervalles de
confiance à 95~\% pour chaque moyenne de groupe comme suit :
\begin{Schunk}
\begin{Sinput}
> ni <- table(polymsm$genotype)
> n <- sum(ni)
> m <- with(polymsm, tapply(age, genotype, mean))
> lci <- m - qt(0.975, n-3) * se / sqrt(ni)
> uci <- m + qt(0.975, n-3) * se / sqrt(ni)
> rbind(lci, uci) 
\end{Sinput}
\begin{Soutput}
     1.6/1.6  1.6/0.7  0.7/0.7
lci 58.14618 59.86536 44.29791
uci 71.13953 68.89326 56.45209
\end{Soutput}
\end{Schunk}
\cmd{table}\cmd{mean}\cmd{qt}\cmd{sqrt}\cmd{rbind}

Dans le cas plus général, pour l'estimation de l'intervalle de confiance
d'une moyenne (ou d'une proportion) on pourra se référer à la commande
\texttt{epi.conf} du package \texttt{epiR}, par exemple :
\begin{Schunk}
\begin{Sinput}
> library(epiR)
> epi.conf(polymsm$age[polymsm$genotype=="1.6/1.6"], ctype = "mean.single")
\end{Sinput}
\begin{Soutput}
       est       se   lower    upper
1 64.64286 2.988269 58.1871 71.09862
\end{Soutput}
\end{Schunk}

Pour représenter graphiquement les moyennes de groupe, on peut utiliser deux
approches. La première consiste à constuire le graphique manuellement, en
affichant les moyennes sous forme de points et les intervalles de confiance
sous forme de segments.
\begin{Schunk}
\begin{Sinput}
> mm <- as.data.frame(cbind(m, lci, uci))
> mm$g <- levels(polymsm$genotype)
> rownames(mm) <- NULL
> dotplot(m ~ g, data=mm, ylim=c(40,75),
+         panel=function(x, y, ...) {
+           panel.dotplot(x, y, ...)
+           panel.segments(x, mm$lci, x, mm$uci)
+         })
\end{Sinput}
\end{Schunk}
\includegraphics{figs/fig-ex4-1i}

On a donc regroupé les moyennes de groupe, ainsi que les bornes inférieures
et supérieures des intervalles de confiance et les niveaux du facteur de
classification dans un même \texttt{data.frame}. La commande graphique
consiste à enchaîner un appel à \texttt{panel.dotplot} pour afficher les
moyennes, puis \texttt{panel.segments} pour tracer les intervalles de
confiance associés.

L'autre solution consiste la commande \texttt{xYplot} (à ne pas confondre
avec \texttt{xyplot}) du package \texttt{Hmisc}. La seule subtilité est que
la variable explicative (reportée sur l'axe des abscisses) doit être traitée
en tant que variable numérique. Il faut adapter légèrement la syntaxe pour
parvenir à l'effet désiré, en particulier afficher correctement les
étiquettes correspondant aux différents génotypes.
\begin{Schunk}
\begin{Sinput}
> library(Hmisc)
> xYplot(Cbind(m,lci,uci) ~ 1:3, data=mm, scales=list(x=list(at=1:3, labels=mm$g)), 
+        xlab="", ylim=c(40,75))
\end{Sinput}
\end{Schunk}
\end{sol}
\end{exo}
%
% METHO p. 87
%
\begin{exo}\label{exo:4.2}
On a mesuré en fin de traitement chez 18 patients répartis par tirage au
sort en trois groupes de traitement A, B, et C, un paramètre biologique dont
on sait que la distribution est normale. Les résultats sont les suivants :
\vskip1em

\begin{tabular}{ccc}
\toprule
A & B & C \\
\midrule
19.8 & 15.9 & 15.4 \\
20.5 & 19.7 & 17.1 \\
23.7 & 20.8 & 18.2 \\
27.1 & 21.7 & 18.5 \\
29.6 & 22.5 & 19.3 \\
29.9 & 24.0 & 21.2 \\
\bottomrule
\end{tabular}
\vskip1em

\begin{description}
\item[(a)] Réaliser une ANOVA à un facteur.
\item[(b)] Selon le résultat du test, procéder aux comparaisons par paire de
  traitement des moyennes, en appliquant une correction simple de Bonferroni
  (c'est-à-dire où les degrés de significativité estimé sont multipliés par
  le nombre de comparaisons effectuées). Comparer avec de simples tests de
  Student non corrigés pour les comparaisons multiples. 
\item[(c)] D'après des études plus récentes, il s'avère que la normalité des
  distributions parentes peut-être remise en question. Effectuer la
  comparaison des trois groupes par une approche non-paramétrique.
\end{description}
\begin{sol}
Dans un premier temps, il s'agit de saisir les données sous un format
approprié pour leur traitement sous \R. Plutôt que de construire un tableau
à trois colonnes, il est préférable de construire un \texttt{data.frame},
contenant une colonne avec l'ensemble des mesures biologiques (soit $3\times
6=18$ observations) et une autre colonne codant pour le type de traitement
(trois niveaux, répétés 6 fois chacun).
\begin{Schunk}
\begin{Sinput}
> pb <- c(19.8,20.5,23.7,27.1,29.6,29.9,
+         15.9,19.7,20.8,21.7,22.5,24.0,
+         15.4,17.1,18.2,18.5,19.3,21.2)
> tx <- gl(3, 6, labels=c("A","B","C"))
> dfrm <- data.frame(pb, tx)
> head(dfrm, 8)
\end{Sinput}
\begin{Soutput}
    pb tx
1 19.8  A
2 20.5  A
3 23.7  A
4 27.1  A
5 29.6  A
6 29.9  A
7 15.9  B
8 19.7  B
\end{Soutput}
\end{Schunk}
\cmd{gl}\cmd{data.frame}\cmd{head}
On notera que l'on a arrangé les données différemment : dans une colonne
figure l'ensemble des mesures et dans l'autre les niveaux du facteur de
classification, qui ont été générés à l'aide de la commande
\texttt{gl}. Cette représentation des données sera beaucoup plus commode par
la suite, notamment pour réaliser l'ANOVA. Cependant, on pourrait dans un
premier temps travailler avec un tableau à 3 colonnes :
\begin{Schunk}
\begin{Sinput}
> pbm <- matrix(pb, nc=3)
> pbm
\end{Sinput}
\begin{Soutput}
     [,1] [,2] [,3]
[1,] 19.8 15.9 15.4
[2,] 20.5 19.7 17.1
[3,] 23.7 20.8 18.2
[4,] 27.1 21.7 18.5
[5,] 29.6 22.5 19.3
[6,] 29.9 24.0 21.2
\end{Soutput}
\end{Schunk}
et "opérer par colonne" pour les résumés numériques. Par exemple, les
commandes suivantes sont équivalentes et fourniront une estimation de la
moyenne par traitement :
\begin{Schunk}
\begin{Sinput}
> apply(pbm, 2, mean)
\end{Sinput}
\begin{Soutput}
[1] 25.10000 20.76667 18.28333
\end{Soutput}
\begin{Sinput}
> sapply(data.frame(pbm), mean)
\end{Sinput}
\begin{Soutput}
      X1       X2       X3 
25.10000 20.76667 18.28333 
\end{Soutput}
\end{Schunk}
\cmd{apply}\cmd{sapply}
Notons également qu'il est très facile de revenir à une structure à deux
variables/colonnes à l'aide de la commande \texttt{melt} du package
\texttt{reshape}. Cette commande a pour effet de concaténer verticalement
les séries de mesure et d'associer à chaque ligne le nom de la colonne
correspodant, en l'occurence le type de traitement.
\begin{Schunk}
\begin{Sinput}
> library(reshape)
> head(melt(data.frame(pbm)))
\end{Sinput}
\begin{Soutput}
  variable value
1       X1  19.8
2       X1  20.5
3       X1  23.7
4       X1  27.1
5       X1  29.6
6       X1  29.9
\end{Soutput}
\end{Schunk}

Si l'on revient à la représentation des données en deux variables/colonnes,
on peut produire un rapide résumé numérique (moyenne et variance) des
mesures enregistrées par groupe de traitement comme suit :
\begin{Schunk}
\begin{Sinput}
> with(dfrm, tapply(pb, tx, mean))
\end{Sinput}
\begin{Soutput}
       A        B        C 
25.10000 20.76667 18.28333 
\end{Soutput}
\begin{Sinput}
> with(dfrm, tapply(pb, tx, var))
\end{Sinput}
\begin{Soutput}
        A         B         C 
19.700000  7.830667  3.861667 
\end{Soutput}
\end{Schunk}
\cmd{tapply}\cmd{mean}\cmd{var}
Enfin, on peut visualiser la distribution des mesures individuelles par
groupe de traitement à l'aide d'un diagramme de type boîte à moustaches.
\begin{Schunk}
\begin{Sinput}
> bwplot(pb ~ tx, data=dfrm)
\end{Sinput}
\end{Schunk}
\includegraphics{figs/fig-ex4-2f}
\cmd{bwplot}

À présent, il est possible de réaliser l'ANOVA avec la commande \texttt{aov}.
\begin{Schunk}
\begin{Sinput}
> aov.res <- aov(pb ~ tx, data=dfrm)
> summary(aov.res)
\end{Sinput}
\begin{Soutput}
            Df Sum Sq Mean Sq F value  Pr(>F)
tx           2  142.8   71.41   6.824 0.00781
Residuals   15  157.0   10.46                
\end{Soutput}
\end{Schunk}
\cmd{aov}\cmd{summary.aov}
Le résultat du test significatif indique qu'au moins une paire de moyennes
peut être considérée comme significativement différente au seuil 5~\%. Pour
comparer l'ensemble des traitements deux à deux, en se protégeant de
l'inflation du risque d'erreur, il est possible d'utiliser la commande
\texttt{pairwise.t.test} pour réaliser des tests de Student, avec l'option
\verb|p.adjust="bonf"| pour appliquer une correction de Bonferroni.
\begin{Schunk}
\begin{Sinput}
> pairwise.t.test(dfrm$pb, dfrm$tx, p.adjust="bonf")
\end{Sinput}
\begin{Soutput}
	Pairwise comparisons using t tests with pooled SD 
data:  dfrm$pb and dfrm$tx 

  A      B     
B 0.1045 -     
C 0.0071 0.6105

P value adjustment method: bonferroni 
\end{Soutput}
\end{Schunk}

Si l'hypothèse de normalité n'est pas réaliste ou peut être remise en
question, on peut réaliser une ANOVA en considérant les rangs des
observations avec la commande \texttt{kruskal.test} :
\begin{Schunk}
\begin{Sinput}
> kruskal.test(pb ~ tx, data=dfrm)
\end{Sinput}
\begin{Soutput}
	Kruskal-Wallis rank sum test
data:  pb by tx 
Kruskal-Wallis chi-squared = 7.9415, df = 2, p-value = 0.01886
\end{Soutput}
\end{Schunk}
La comparaison des paires de traitements peut être effectuée à l'aide de
simples tests de Wilcoxon pour échantillons indépendants, par exemple 
\texttt{wilcox.test}. 
\begin{Schunk}
\begin{Sinput}
> wilcox.test(pb ~ tx, data=dfrm, subset=tx!="C")
\end{Sinput}
\end{Schunk}
ou bien
\begin{Schunk}
\begin{Sinput}
> with(dfrm, wilcox.test(pb[tx=="A"], pb[tx=="B"]))
\end{Sinput}
\begin{Soutput}
	Wilcoxon rank sum test
data:  pb[tx == "A"] and pb[tx == "B"] 
W = 27, p-value = 0.1797
alternative hypothesis: true location shift is not equal to 0 
\end{Soutput}
\end{Schunk}
L'option \texttt{subset} est disponible dans la plupart des tests que l'on
rencontrera (ANOVA, régression, tests sur deux échantillons), ainsi que dans
les commandes graphiques : elle permet de restreindre l'analyse à un
sous-ensemble de l'échantillon remplissant certaines conditions (ici, les
individus ne figurant pas dans le groupe C).
\end{sol}
\end{exo}
%
% Peat 2005 p. 113
%
\begin{exo}\label{exo:4.3}
Un service d'obstétrique s'intéresse au poids de nouveaux-nés nés à terme et
âgés de 1 mois. Pour cet échantillon de 550 bébés, on dispose également
d'une information concernant la parité (nombre de frères et soeurs), mais on
sait qu'il n'y aucune relation de gemellité parmi les enfants ayant des
frères et soeurs. L'objet de l'étude est de déterminer si la parité (4
classes) influence le poids des nouveaux-nés à 1 mois. Les données sont
résumées dans le tableau suivant, et elles sont disponibles dans un fichier
SPSS, \texttt{weights.sav}.\autocite[p.~113]{peat05}
\vskip1em

\begin{tabular}{lrrrrr}
\toprule
& \multicolumn{4}{c}{Nombre de frères et soeurs} & Total \\
& 0 & 1 & 2 & $\ge 3$ & \\
\midrule
\emph{Échantillon} & & & & \\ 
Effectif & 180 & 192 & 116 & 62 & 550 \\
Fréquence & 32.7 & 34.9 & 21.1 & 11.3 & 100.0 \\
\emph{Poids (kg)} & & & & \\
Moyenne & 4.26 & 4.39 & 4.46 & 4.43 & \\
Écart-type & 0.62 & 0.59 & 0.61 & 0.54 & \\
(Min–Max) & (2.92–5.75) & (3.17–6.33) & (3.09–6.49) & (3.20–5.48) & \\
\bottomrule
\end{tabular}
\vskip1em

\begin{description}
\item[(a)] Vérifier les données reportées dans le tableau précédent.
\item[(b)] Procéder à une analyse de variance à un facteur. Conclure sur la
  significativité globale et indiquer la part de variance expliquée par le
  modèle.
\item[(c)] Afficher la distribution des poids selon la parité. Procéder à un
  test d'homogénéité des variances (rechercher dans l'aide en ligne le test
  de Levenne). 
\item[(d)] On décide de regrouper les deux dernières catégories (2 et $\ge
  3$). Refaire l'analyse et comparer aux résultats obtenus en (b).
\item[(e)] Réaliser un test de tendance linéaire (par ANOVA) sur les données
  recodées en trois niveaux pour la parité.
\end{description}
\begin{sol}
Pour charger les données, il est nécessaire d'importer la librairie
\texttt{foreign} qui permet de lire les fichiers enregistrés par SPSS.
\begin{Schunk}
\begin{Sinput}
> library(foreign)
> weights <- read.spss("data/weights.sav", to.data.frame=TRUE)
> str(weights)
\end{Sinput}
\begin{Soutput}
'data.frame':	550 obs. of  7 variables:
 $ ID      : Factor w/ 550 levels "L001","L003",..: 1 2 3 4 5 6 7 8 9 10 ...
 $ WEIGHT  : num  3.95 4.63 4.75 3.92 4.56 ...
 $ LENGTH  : num  55.5 57 56 56 55 51.5 56 57 58.5 52 ...
 $ HEADC   : num  37.5 38.5 38.5 39 39.5 34.5 38 39.7 39 38 ...
 $ GENDER  : Factor w/ 2 levels "Male","Female": 2 2 1 1 1 2 2 1 1 1 ...
 $ EDUCATIO: Factor w/ 3 levels "year10","year12",..: 3 3 2 3 1 3 1 1 3 1 ...
 $ PARITY  : Factor w/ 4 levels "Singleton","One sibling",..: 4 1 3 2 3 1 4 4 3 2 ...
 - attr(*, "variable.labels")= Named chr  "ID" "Weight (kg)" "Length (cms)" "Head circumference (cms)" ...
  ..- attr(*, "names")= chr  "ID" "WEIGHT" "LENGTH" "HEADC" ...
\end{Soutput}
\end{Schunk}
\cmd{foreign}\cmd{read.spss}
Comme on l'a vu dans l'exercice~\ref{exo:3.6}, l'option
\verb|to.data.frame=TRUE| est importante car c'est elle qui permet de
stocker les données lues sous forme de \texttt{data.frame} (variable en
colonnes, individus en lignes).

Dans un premier temps, procédons au résumé numérique uni- et bivarié des
données d'intérêt (variables \texttt{WEIGHT} et \texttt{PARITY}). Pour la
variable qualitative, les tableaux d'effectifs et de fréquences relatives
sont obtenus ainsi :
\begin{Schunk}
\begin{Sinput}
> table(weights$PARITY)
\end{Sinput}
\begin{Soutput}
         Singleton        One sibling         2 siblings 3 or more siblings 
               180                192                116                 62 
\end{Soutput}
\begin{Sinput}
> round(prop.table(table(weights$PARITY))*100, 1)
\end{Sinput}
\begin{Soutput}
         Singleton        One sibling         2 siblings 3 or more siblings 
              32.7               34.9               21.1               11.3 
\end{Soutput}
\end{Schunk}
\cmd{table}\cmd{prop.table}\cmd{round}
Pour la variable quantitative, les moyennes et écart-types par type de
parité sont obtenus ainsi :
\begin{Schunk}
\begin{Sinput}
> round(with(weights, tapply(WEIGHT, PARITY, mean)), 2)
\end{Sinput}
\begin{Soutput}
         Singleton        One sibling         2 siblings 3 or more siblings 
              4.26               4.39               4.46               4.43 
\end{Soutput}
\begin{Sinput}
> round(with(weights, tapply(WEIGHT, PARITY, sd)), 2)
\end{Sinput}
\begin{Soutput}
         Singleton        One sibling         2 siblings 3 or more siblings 
              0.62               0.59               0.61               0.54 
\end{Soutput}
\end{Schunk}
\cmd{round}\cmd{mean}\cmd{sd}

Pour l'analyse de variance, on utilise le même principe qu'à
l'exercice~\ref{exo:4.2}. 
\begin{Schunk}
\begin{Sinput}
> aov.res <- aov(WEIGHT ~ PARITY, data=weights)
> summary(aov.res)
\end{Sinput}
\begin{Soutput}
             Df Sum Sq Mean Sq F value Pr(>F)
PARITY        3   3.48  1.1590   3.239 0.0219
Residuals   546 195.36  0.3578               
\end{Soutput}
\end{Schunk}
\cmd{aov}\cmd{summary.aov}
Le test $F$ est significatif, donc on peut rejeter l'hypothèse nulle
d'égalité des quatre moyennes. La part de variance expliquée est simplement
le rapport entre la somme des carrés (\texttt{Sum Sq}) associée au facteur
d'étude (\texttt{PARITY}), soit 3.48, et la somme des carrés totaux, soit
3.48+3.48+195.36 : on obtient 0.018, soit environ 2~\%.

Pour afficher la distribution des poids, on peut bien sûr utiliser des
boîtes à moustaches, comme dans les exercices précédents. Si l'on souhaite
visualiser directement les données individuelles, un diagramme de dispersion
conditionné sur les groupes est également intéressant.
\begin{Schunk}
\begin{Sinput}
> stripplot(WEIGHT ~ PARITY, data=weights, ylab="Poids (kg)", jitter.data=TRUE)
\end{Sinput}
\end{Schunk}
\includegraphics{figs/fig-ex4-3e}
\cmd{stripplot}

L'autre alternative consiste à utiliser des histogrammes pour chaque groupe.
\begin{Schunk}
\begin{Sinput}
> histogram(~ WEIGHT | PARITY, data=weights)
\end{Sinput}
\end{Schunk}
\includegraphics{figs/fig-ex4-3f}

On notera que l'on a ajouté un léger décalage aléatoire des données (sur
l'axe horizontal uniquement) en utilisant l'option \verb|jitter.data=TRUE|,
ce qui permet d'éviter le chevauchement total des points.

Le test de Levene n'est pas disponible dans les commandes de base de R, mais
on peut installer le package \texttt{car} qui fournit la commande
\texttt{leveneTest} :
\begin{Schunk}
\begin{Sinput}
> library(car)
> leveneTest(WEIGHT ~ PARITY, data=weights)
\end{Sinput}
\begin{Soutput}
Levene's Test for Homogeneity of Variance (center = median)
       Df F value Pr(>F)
group   3  0.6442 0.5869
      546               
\end{Soutput}
\end{Schunk}
Une alternative consisterait à utiliser un test de Bartlett
(\texttt{bartlett.test}) pour l'homogénéité des variances. Cette commande
s'utilise exactement de la même manière que \texttt{leveneTest} (variable
réponse décrite par le facteur d'étude, nom du \texttt{data.frame} où
chercher les variables).

Pour regrouper les deux dernières catégories, on peut créer une nouvelle
variable et générer manuellement les nouvelles modalités associés, ou plus
simplement "recoder" la variable qualitative \texttt{PARITY} en une nouvelle
variable :
\begin{Schunk}
\begin{Sinput}
> PARITY2 <- weights$PARITY
> levels(PARITY2)[3:4] <- "2 siblings or more"
\end{Sinput}
\end{Schunk}
Le modèle d'analyse de variance est construit de la même manière que
précédemment :
\begin{Schunk}
\begin{Sinput}
> aov.res2 <- aov(WEIGHT ~ PARITY2, data=weights)
> summary(aov.res2)
\end{Sinput}
\begin{Soutput}
             Df Sum Sq Mean Sq F value  Pr(>F)
PARITY2       2   3.45  1.7249   4.829 0.00834
Residuals   547 195.39  0.3572                
\end{Soutput}
\end{Schunk}

Enfin, pour réaliser un test de tendance centrale, il y a deux solutions :
soit par la méthode des contrastes, soit par une approche de régression
linéaire. Ces deux approches fournissent des résultats identiques, et
suppose généralement que les niveaux du facteur de classification sont
équi-espacés (variation du même nombre d'unités entre chaque niveau du
facteur). Dans le cas présent, on utlisera la commande \texttt{lm} pour
réaliser une régression linéaire simple (voir \ref{chap:reg},
p.~\pageref{chap:reg}, pour plus de détails). 
\begin{Schunk}
\begin{Sinput}
> levels(PARITY2)
\end{Sinput}
\begin{Soutput}
[1] "Singleton"          "One sibling"        "2 siblings or more"
\end{Soutput}
\begin{Sinput}
> lm.res <- lm(WEIGHT ~ as.numeric(PARITY2), data=weights)
> summary(lm.res)
\end{Sinput}
\begin{Soutput}
Call:
lm(formula = WEIGHT ~ as.numeric(PARITY2), data = weights)
Residuals:
     Min       1Q   Median       3Q      Max 
-1.37289 -0.40967 -0.02483  0.41170  2.02710 

Coefficients:
                    Estimate Std. Error t value Pr(>|t|)
(Intercept)          4.17451    0.06798  61.409  < 2e-16
as.numeric(PARITY2)  0.09613    0.03157   3.045  0.00244

Residual standard error: 0.5973 on 548 degrees of freedom
Multiple R-squared: 0.01664,	Adjusted R-squared: 0.01484 
F-statistic: 9.271 on 1 and 548 DF,  p-value: 0.00244 
\end{Soutput}
\end{Schunk}
Le test de la tendance linéaire pour l'ANOVA correspond au test de la pente
de la droite de régression, qui ici est significatif suggérant une
augmentation du poids moyen avec la taille de la fratrie.
De manière équivalente, on peut utiliser l'approche consistant à recoder le
facteur de classification en facteur à modalités ordonnées (ou niveaux) à
l'aide la commande \texttt{as.ordered}. Le test pour la tendance linéaire
correspondant au contraste nommé \texttt{as.ordered(PARITY2).L} dans la
sortie suivante.
\begin{Schunk}
\begin{Sinput}
> levels(as.ordered(PARITY2))
\end{Sinput}
\begin{Soutput}
[1] "Singleton"          "One sibling"        "2 siblings or more"
\end{Soutput}
\begin{Sinput}
> summary(lm(WEIGHT ~ as.ordered(PARITY2), data=weights))
\end{Sinput}
\begin{Soutput}
Call:
lm(formula = WEIGHT ~ as.ordered(PARITY2), data = weights)
Residuals:
     Min       1Q   Median       3Q      Max 
-1.36107 -0.40888 -0.02382  0.41124  2.03893 

Coefficients:
                      Estimate Std. Error t value Pr(>|t|)
(Intercept)            4.36624    0.02550 171.233  < 2e-16
as.ordered(PARITY2).L  0.13585    0.04467   3.041  0.00247
as.ordered(PARITY2).Q -0.02751    0.04365  -0.630  0.52883

Residual standard error: 0.5977 on 547 degrees of freedom
Multiple R-squared: 0.01735,	Adjusted R-squared: 0.01376 
F-statistic: 4.829 on 2 and 547 DF,  p-value: 0.008339 
\end{Soutput}
\end{Schunk}
\end{sol}
\end{exo}
%
% STAB TD 3
%
\begin{exo}\label{exo:4.4}
On souhaite analyser les données traitées à l'exercice~\ref{exo:1.6}
(p.~\pageref{exo:1.6}) par un modèle d'analyse de variance.

\begin{description}
\item[(a)] Calculer moyenne et variance des mesures pour chaque traitement.
\item[(b)] Représenter graphiquement les moyennes par traitement dans un
  graphique d'interaction.
\item[(c)] Effectuer une ANOVA à deux facteurs, sans interaction.
\item[(d)] Refaire une ANOVA en incluant l'interaction entre les facteurs
  \texttt{Na} et \texttt{An}.
\end{description}
\begin{sol}
Les données ont été normalement été sauvegardées au format \R à
l'exercice~\ref{exo:1.6}. On supposera que le fichier a été sauvegardé sous
le nom \texttt{bioluminescence.RData}. Pour l'importer sous \R, on utilise
la commande \texttt{load} directement :
\begin{Schunk}
\begin{Sinput}
> load("data/bioluminescence.RData")
> ls()
\end{Sinput}
\begin{Soutput}
 [1] "An"             "anorex"         "aov.res"        "aov.res2"       "asp.or"        
 [6] "aspirine"       "biolum"         "birthwt"        "dfrm"           "enquete"       
[11] "ethn"           "hc"             "hc.tab"         "lci"            "lm.res"        
[16] "ltheme"         "lwt.quartiles"  "m"              "mm"             "mse"           
[21] "n"              "Na"             "ni"             "occ1"           "occ2"          
[26] "PARITY2"        "pb"             "pbm"            "poids"          "polymsm"       
[31] "prop.infarctus" "regime"         "res"            "s"              "sdif"          
[36] "se"             "sleep"          "status"         "tab"            "tailles"       
[41] "tx"             "uci"            "weights"        "x"              "X"             
[46] "xc"             "Xc"             "yesno"         
\end{Soutput}
\end{Schunk}

Pour calculer la moyenne et la variance pour chaque traitement, on peut
utiliser le même principe qu'à l'exercice~\ref{exo:1.6}, c'est-à-dire une
combinaison des commandes \texttt{tapply} et \texttt{mean} ou \texttt{var}. 
\begin{Schunk}
\begin{Sinput}
> with(biolum, tapply(y, list(An=An, Na=Na), mean))
\end{Sinput}
\begin{Soutput}
   Na
An          -      +
  -  7.733333 20.875
  + 10.216667 11.000
\end{Soutput}
\begin{Sinput}
> with(biolum, tapply(y, list(An=An, Na=Na), var))
\end{Sinput}
\begin{Soutput}
   Na
An         -        +
  - 3.284242 16.34786
  + 4.637667 11.44000
\end{Soutput}
\end{Schunk}
L'ajout des noms de variables dans \texttt{list} permet de mieux interpréter
les résultats afficher par \R : le premier facteur de la liste, \texttt{An},
est affiché en ligne, alors que les niveaux du second sont affichés en
colonnes. 

Pour représenter les données moyennes sous forme graphique (diagramme ou
graphique d'intercation), on utilisera la commande \texttt{xyplot} avec
l'option \verb|type="a"| qui permet de calculer automatiquement les moyennes
de groupe à partir des données brutes. Le reste des options
(\texttt{auto.key}) permet de générer la légende pour faciliter la lecture
du graphique.
\begin{Schunk}
\begin{Sinput}
> xyplot(y ~ An, data=biolum, groups=Na, type=c("a","g"), 
+        auto.key=list(corner=c(0,1), lines=TRUE, points=FALSE, title="Na", cex.title=.8))
\end{Sinput}
\end{Schunk}
\includegraphics{figs/fig-ex4-4c}

\begin{Schunk}
\begin{Sinput}
> summary(aov(y ~ An + Na, data=biolum))
\end{Sinput}
\begin{Soutput}
            Df Sum Sq Mean Sq F value   Pr(>F)
An           1   40.3    40.3   2.408    0.132
Na           1  586.1   586.1  34.976 2.65e-06
Residuals   27  452.4    16.8                 
\end{Soutput}
\end{Schunk}

\begin{Schunk}
\begin{Sinput}
> summary(aov(y ~ An + Na + An:Na, data=biolum))
\end{Sinput}
\begin{Soutput}
            Df Sum Sq Mean Sq F value   Pr(>F)
An           1   40.3    40.3   5.041   0.0335
Na           1  586.1   586.1  73.236 4.88e-09
An:Na        1  244.4   244.4  30.535 8.44e-06
Residuals   26  208.1     8.0                 
\end{Soutput}
\end{Schunk}

\end{sol}
\end{exo}

% ANOVA two-way cross-over
%
%% \begin{exo}
%% Les données inclues dans le fichier \texttt{headache.txt} ont été collectées
%% dans un plan randomisé en essais croisés (cross-over) avec trois bras de
%% traitement visant à comparer deux analgésiques (A et B) et un placebo (P)
%% pour le traitement des céphalées.\autocite[p.~617]{fitzmaurice04} La
%% comparaison principale porte sur les deux traitements actifs (l'un des deux
%% incluant en plus de la caffeine). On notera qu'il n'y a que deux périodes,
%% donc les patients n'ont reçu que deux des trois traitements, par
%% randomisation. La réponse mesurée est la diminution moyenne de douleur. Le
%% fichier de données comporte un descriptif détaillé de l'étude ainsi que le
%% nom des variables.

%% Par souci de simplicité, on ne va s'intéresser qu'aux deux bras actifs, A et
%% B, d'où les deux facteurs d'intérêt suivants : période (1 et 2) et
%% traitements séquentiels (AB ou BA). Les résultats moyens sont résumés dans
%% le tableau suivant (Tableau 21.1, p.~618) :
%% \vskip1em

%% \begin{tabular}{lrrrrr}
%% \toprule
%% & & \multicolumn{2}{c}{Période 1} & \multicolumn{2}{c}{Période 2} \\
%% \cmidrule(r){3-6}
%% Séquence & N & Moyenne & DS & Moyenne & DS \\
%% \midrule
%% AB & 126 & 10.196 & 3.347 & 9.153 & 3.429 \\
%% BA & 127 & 9.581 & 3.881 & 10.791 & 3.530 \\
%% \bottomrule
%% \end{tabular}
%% \vskip1em

%% \begin{description}
%% \item[(a)] Après avoir importé les données et restreint le tableau de
%%   données aux seuls comparaisons AB et BA, reonstruire le tableau de
%%   synthèse ci-dessus.
%% \item[(b)] Résumer l'effet traitement dans un graphique, en prenant en
%%   considération les deux périodes et la séquence.
%% \item[(c)] Tester l'effet séquence (traitement) et l'éffet période à l'aide
%%   de tests t de Student. Conclure au risque $\alpha$ de 5~\%.
%% \item[(d)] Réaliser une ANOVA en considérant ces deux mêmes facteurs, et
%%   leur interaction. Existe-t-il un effet rémanent (carry-over) ? Conclure
%%   sur l'effet global du traitement. 
%% \end{description}
%% \begin{sol}
%% Le fichier de données \texttt{headache.txt} contient un en-tête de 35 lignes
%% qui comprend la description de l'étude et le nom des variables. Les données
%% réelles sont ensuite arrangées sous forme de 7 colonnes (lignes
%% 36--881). Pour importer ces données, il conviendra donc de lire le fichier à
%% partir de la 36\ieme\ ligne. La 6\ieme\ colonne est identique à la 3\ieme\ qui
%% peut donc être supprimée.

%% <<exo4-5a>>=
%% headache <- read.table("headache.txt", header=FALSE, skip=35)
%% headache <- headache[,-3]
%% varnames <- c("ID", "Center", "Sequence", "Period", "Treatment", "Response")
%% names(headache) <- varnames
%% head(headache)
%% @ 
%% Pour restreindre les résultats aux seuls comparaisons AB et BA (séquences 1
%% et 2), on utilise \texttt{subset} de la manière suivante :
%% <<exo4-5b>>=
%% headache <- subset(headache, Sequence == 1 | Sequence == 2)
%% headache$Sequence <- factor(headache$Sequence, levels=1:2, labels=c("BA","AB"))
%% headache$Period <- factor(headache$Period, levels=0:1, labels=1:2)
%% @
%% Pour reconstruire le tableau de synthèse (moyenne et écart-type par
%% séquence et période), on peut utiliser la commande \texttt{aggregate}. La
%% subtilité consiste à correctement identifier les séquences AB et BA, ce qui
%% a été fait à l'étape précédente en renommant les étiquettes associées à
%% cette variable qualitative.
%% <<exo4-5c>>=
%% resultats <- aggregate(Response ~ Sequence + Period, data=headache, mean)
%% xtabs(Response ~ Sequence + Period, resultats)
%% @ 
%% On procèdera de même pour les écart-types en remplaçant la commande
%% \texttt{mean} par \texttt{sd}.

%% Ces résultats peuvent être résumés dans un graphique d'interaction où l'on
%% représente les moyennes calculées ci-dessus.
%% <<exo4-5d, fig=TRUE>>=
%% xyplot(Response ~ Period, data=resultats, groups=Sequence, 
%%        type=c("b","g"), auto.key=list(corner=c(0,1)))
%% @ 

%% On peut tester séparément les effets séquence (traitement) et période à
%% l'aide de simples tests $t$ pour échantillons indépendants.
%% <<exo4-5e>>=
%% t.test(Response ~ Period, data=headache, var.equal=TRUE)
%% t.test(Response ~ Sequence, data=headache, var.equal=TRUE)
%% @ 
%% Ces résultats suggère que l'ordre d'administration des traitements
%% n'influence pas le niveau moyen de la réponse, mais en même temps on ne met
%% pas en évidence d'effet global du traitement (indépendemment de la
%% période). 

%% Le modèle d'ANOVA considérant les deux facteurs, \texttt{Sequence} (AB ou
%% BA) et \texttt{Period} (1 et 2), se construit sur la base de la formule
%% \verb|Response ~ Period + Sequence + Period:Sequence|, le dernier terme
%% correspondant à l'effet d'interaction. Ce modèle peut s'abbréger sous la
%% forme \verb|Response ~ Period * Sequence|.
%% <<exo4-5f>>=
%% aov.res <- aov(Response ~ Period * Sequence, data=headache)
%% summary(aov.res)
%% @ 
%% À l'évidence, on retrouve un effet d'interaction suggérant que les deux
%% effets de séquence et période sont inter-dépendants, sans qu'aucun des deux ne
%% soit significatifs à 5~\%. Ceci suggère 
%% \end{sol}
%% \end{exo}
%
% ANCOVA
%
%% \begin{exo}
%%   ANCOVA
%% \end{exo}
\Closesolutionfile{solutions}

\chapter*{Devoir \no 3}
\addcontentsline{toc}{chapter}{Devoir \no 3}

Les exercices sont indépendants. Une seule réponse est correcte pour chaque
question. Lorsque vous ne savez pas répondre, cochez la case correspondante.

\section*{Exercice 1}
Dans une étude d'obstétrique, on s'est intéressé à la relation entre le
poids d'un bébé à la naissance et le gain de poids de la mère durant sa
grossesse. Les mères ont été regroupées en quatre classes selon leur gain de
poids : faible ($\leq$ 11 kg), normal (11-16 kg), modéré (16-22 kg) et extrême (>
22 kg). Les bornes inférieures des intervalles sont exclues, les bornes
supérieures sont inclues. Le poids du bébé à la naissance est mesuré en
kilogrammes. Au total, on dispose des données de 59 dyades. Voici un aperçu
des données du \texttt{data.frame} appelé \texttt{bt} :
\begin{verbatim}
      bwt              wt              gain      
 Min.   :1.400   Min.   : 45.00   Min.   : 1.20  
 1st Qu.:3.095   1st Qu.: 52.15   1st Qu.:12.30  
 Median :3.480   Median : 54.50   Median :14.80  
 Mean   :3.434   Mean   : 57.93   Mean   :15.36  
 3rd Qu.:3.860   3rd Qu.: 59.00   3rd Qu.:18.05  
 Max.   :4.780   Max.   :104.00   Max.   :32.60
\end{verbatim}
On cherche à démontrer qu'il existe une relation entre le poids du bébé
(\texttt{bwt}) et le gain de poids (\texttt{gain}) des mères durant
leur grossesse, en considérant un risque de première espèce de 5~\%. La
variable \texttt{wt} représente le poids de la mère avant le début de la
grossesse.  

\begin{description}
\item[\bf 1.1] \marginpar{\phantom{text}\phantom{text} 1.1 $\square$} Quelle commande a permis de
  produire le résultat présenté ci-dessus ?
  \begin{description}
  \item[A.] \verb|describe|
  \item[B.] \verb|summary|
  \item[C.] \verb|str|
  \item[D.] Je ne sais pas.
  \end{description}
\item[\bf 1.2] \marginpar{\phantom{text}1.2 $\square$} Les données comportent-elles des
  valeurs manquantes ?
  \begin{description}
  \item[A.] Oui.
  \item[B.] Non.
  \item[C.] Je ne sais pas.
  \end{description}
\item[\bf 1.3] \marginpar{\phantom{text}1.3 $\square$} On désire recoder la variable
  \texttt{gain} en variable qualitative à quatre classes telles que décrites
  dans l'énoncé. Quelle commande est la plus appropriée ?
  \begin{description}
  \item[A.] \verb|cut(bt$gain, c(1,11,16,22,33), levels=c("faible","normal","modéré","extrême"))|
  \item[B.] \verb|cut(bt$gain, c(1,11,16,22,33), labels=c("faible","normal","modéré","extrême"))|
  \item[C.] \verb|cut(bt$gain, c(11,16,22), labels=c("faible","normal","modéré","extrême"))|
  \item[D.] \verb|factor(bt$gain, levels=c(11,16,22), labels=c("faible","normal","modéré","extrême"))|
  \item[E.] Je ne sais pas.
  \end{description}  
\item[\bf 1.4] \marginpar{\phantom{text}1.4 $\square$} Quelle commande a permis de fournir
  les résultats suivants :
\begin{verbatim}
  faible   normal   modéré  extrême 
3.126667 3.402400 3.641250 3.320000
\end{verbatim}
  \begin{description}
  \item[A.] \verb|with(bt, tapply(bwt, gain, mean))|
  \item[B.] \verb|with(bt, tapply(wt, gain, mean))|
  \item[C.] \verb|with(bt, tapply(bwt, gain, var))|
  \item[D.] \verb|with(bt, tapply(wt, gain, var))|
  \item[E.] Je ne sais pas.
  \end{description}  
\item[\bf 1.5] \marginpar{\phantom{text}1.5 $\square$} Voici le tableau d'analyse de
  variance où l'on considère \texttt{bwt} comme variable réponse et
  \texttt{gain} (4 classes) comme facteur d'étude :
\begin{verbatim}
            Df Sum Sq Mean Sq F value Pr(>F)
gain         3  1.799  0.5997   1.572  0.207
Residuals   55 20.980  0.3815
\end{verbatim}
Quelle est la part de variance expliquée par ce modèle ? 
\begin{description}
\item[A.] 7.9~\%.  % SSG/SSE
\item[B.] 21.8~\%. % MSG-MSE
\item[C.] 61.1~\%.  % (SSG*3)/(SSE*55)
\item[D.] Je ne sais pas.
\end{description}  
\item[\bf 1.6] \marginpar{\phantom{text}1.6 $\square$} Les données individuelles sont
  représentées dans le graphique suivant.
\begin{center}
  \includegraphics{./figs/dev3_bt}
\end{center}
On décide d'exclure toutes les observations pour lesquelles le poids du
bébé à la naissance est inférieur à 2.5 kg (représenté par le trait
horizontal), toujours en considérant la variable \texttt{gain} comme
facteur de classification à 4 niveaux. Comment peut-on procéder pour refaire
l'ANOVA dans ces conditions ? 
\begin{description}
\item[A.] \verb|aov(bwt, gain, data=bt, select = bwt >= 2.5)|
\item[B.] \verb|aov(bwt ~ gain, data=bt, select = bwt >= 2.5)|
\item[C.] \verb|aov(bwt ~ gain, data=bt, subset = bwt >= 2.5)|
\item[D.] Je ne sais pas.
\end{description}  
\end{description}

\section*{Exercice 2}\label{dev3:exo2}
Considérons certains des résultats d'un essai clinique sur la thérapie
hormonale \citep{hulley98}. Les variables d'intérêt sont l'ethnicité
(\texttt{raceth}) des participants et la pression systolique
(\texttt{SBP}). Au total, il y a 2763 sujets et, une fois importées sous R,
les données (appelée \texttt{d}) apparaissent comme reporté ci-dessous :
\begin{verbatim}
     raceth           SBP       
 Min.   :1.000   Min.   : 83.0  
 1st Qu.:1.000   1st Qu.:122.0  
 Median :1.000   Median :134.0  
 Mean   :1.147   Mean   :135.1  
 3rd Qu.:1.000   3rd Qu.:147.0  
 Max.   :3.000   Max.   :224.0  
\end{verbatim}
La variable \texttt{raceth} code pour l'ethnicité des participants, avec les
conventions suivantes : 1 = \texttt{White}, 2 = \texttt{Afr Amer} et 3 =
\texttt{Other}. La variable \texttt{SBP} représente la pression
systolique. On souhaite vérifier si celle-ci varie selon le facteur de
groupe. Voici un résumé numérique descriptif (effectif, moyenne, écart-type,
minimum et maximum) de la variable réponse selon les niveaux de
\texttt{raceth} : 
\begin{verbatim}
  raceth    n     mean       sd min max
1      1 2451 134.7838 18.83169  83 224
2      2  218 138.2339 19.99252  98 194
3      3   94 135.1809 21.25977  95 187
\end{verbatim}
\begin{description}
\item[\bf 2.1] \marginpar{\phantom{text}2.1 $\square$} Pour calculer le rapport entre la
  variance du groupe 3 et celle du groupe 1, on peut utiliser la commande
\begin{description}
\item[A.] \verb|var(d$SBP[d$raceth=3]/var(d$SBP[d$raceth=1]|
\item[B.] \verb|var(d$SBP[d$raceth==3]/var(d$SBP[d$raceth==1]|
\item[C.] \verb|var(d$SBP[d$raceth="3"]/var(d$SBP[d$raceth="1"]|
\item[D.] Je ne sais pas.
\end{description}  
\item[\bf 2.2] \marginpar{\phantom{text}2.2 $\square$} On souhaite représenter la
  distribution de la pression systolique à l'aide d'un histogramme pour
  chacun des trois groupes. Quelle commande est la plus appropriée ?
  \begin{description}
  \item[A.] \verb|histogram(SBP, raceth, data=d)|
  \item[B.] \verb+histogram(~ SBP | raceth, data=d)+
  \item[C.] \verb|histogram(SBP ~ raceth, data=d)|
  \item[D.] Je ne sais pas.
  \end{description}  
\item[\bf 2.3] \marginpar{\phantom{text}2.3 $\square$} On réalise une ANOVA à l'aide de la
  commande suivante :
\begin{verbatim}
> summary(aov(SBP ~ raceth, data=d))
              Df Sum Sq Mean Sq F value Pr(>F)
raceth         1    945   945.5   2.613  0.106
Residuals   2761 999057   361.8
\end{verbatim}
  Est-ce la commande appropriée ?  
  \begin{description}
  \item[A.] Oui.
  \item[B.] Non.
  \item[C.] Je ne sais pas.
  \end{description}  
\item[\bf 2.4] \marginpar{\phantom{text}2.4 $\square$} Que réalise la commande suivante ? 
\begin{verbatim}
bartlett.test(SBP ~ as.factor(raceth), data=d)
\end{verbatim}
  \begin{description}
  \item[A.] Elle permet de tester l'hypothèse d'égalité des variances (par
    rapport à la variable réponse \texttt{SBP}) dans les groupes définis par le
    facteur \texttt{raceth}.
  \item[B.] Elle permet de tester l'hypothèse de normalité de la distribution
    de \texttt{SBP} dans les groupes définis par le facteur \texttt{raceth}.
  \item[C.] Elle permet de vérifier que les effectifs par niveaux de
    \texttt{raceth} sont suffisants pour réaliser une ANOVA à un facteur.
  \item[D.] Je ne sais pas.
  \end{description}  
\end{description}


\section*{Exercice 3}
On s'intéresse à l'effet de différentes solutions à base de sucre sur la
croissance de pois de culture en présence d'auxine (phytohormone de
croissance végétale). Le diamètre des pois est mesuré en unité oculaire
($\times 0.114$ = mm). Il était prévu de disposer de 10 répliques par
traitement, mais certaines cultures ont dû être retirées en cours
d'expérimentation \citep[p.~218, données modifiées du tableau
9.4]{sokal95}. Les données, disponibles dans le fichier \texttt{peas.txt},
sont résumées dans le tableau suivant. Le fichier texte contient exactement
les mêmes données, présentées sous la même forme mais sans le nom des
colonnes. 
\vskip1em

\begin{tabular}{ccccc}
  \toprule
  Contrôle & 2~\% G & 2~\% F & 1~\% G + 1~\% F & 2~\% S \\
  \midrule
  75 & 57 & 58 & 58 & 62 \\
  67 & 58 & 61 & 59 & 66 \\
  70 & 60 & -  & 58 & 65 \\
  75 & 59 & 58 & 61 & 63 \\
  65 & 62 & 57 & 57 & 64 \\
  71 & 60 & 56 & -  & 62 \\
  67 & 60 & 61 & 58 & -  \\
  67 & 57 & 60 & 57 & -  \\
  76 & -  & 57 & 57 & 62 \\
  68 & 61 & 58 & 59 & 67 \\
  \bottomrule
  \multicolumn{5}{l}{\small G = glucose, F = fructose, S = sucrose} \\
\end{tabular}
\vskip1em

\begin{description}
\item[\bf 3.1] \marginpar{\phantom{text}3.1 $\square$} En supposant que le tableau de
  données (10 lignes x 5 colonnes) aît été stocké dans un
  \texttt{data.frame} appelé \texttt{peas} sous \R, de la manière suivante :
\begin{verbatim}
> tx <- c("C","2G","2F","1G1F","2S")
> peas <- read.table("peas.txt", header=FALSE, na.strings="-", col.names=tx)
> head(peas)
   C X2G X2F X1G1F X2S
1 75  57  58    58  62
2 67  58  61    59  66
3 70  60  NA    58  65
4 75  59  58    61  63
5 65  62  57    57  64
6 71  60  56    NA  62
\end{verbatim}
  quelle commande doit-on utiliser pour compter le nombre
  d'observations manquantes par traitement ? 
  \begin{description}
  \item[A.] \verb|summary(is.na(peas))|
  \item[B.] \verb|table(is.na(peas))|
  \item[C.] \verb|apply(peas, 1, function(x) sum(is.na(x))|
  \item[D.] \verb|sapply(peas, function(x) sum(is.na(x)))|
  \item[E.] Je ne sais pas.
  \end{description}  
\item[\bf 3.2] \marginpar{\phantom{text}3.2 $\square$} Plutôt que de travailler avec un
  tableau à 5 colonnes, on décide de convertir les données en une structure
  plus facile à manipuler, c'est-à-dire un tableau dans lequel on indique
  dans la première colonne le type de traitement auquel les poids ont été
  soumis, et dans la seconde le diamètre des poids. Les données devraient
  donc ressembler à l'aperçu ci-dessous (11 premières observations
  représentées uniquement) :
\begin{verbatim}
Using  as id variables
      tx value
1      C    75
2      C    67
3      C    70
4      C    75
5      C    65
6      C    71
7      C    67
8      C    67
9      C    76
10     C    68
11   X2G    57
\end{verbatim}
  Voici la commande que l'on se propose d'utiliser afin de réaliser cette
  opération :
\begin{verbatim}
> library(reshape)
> melt(peas, variable_name="tx")
\end{verbatim}
  Cette commande est-elle correcte ? 
  \begin{description}
  \item[A.] Oui. 
  \item[B.] Non.
  \item[C.] Je ne sais pas.
  \end{description}  
\item[\bf 3.3] \marginpar{\phantom{text}3.3 $\square$} On souhaite représenter sous forme
  d'un diagramme en points l'ensemble des mesures individuelles (diamètre,
  en abscisses) pour chacune des conditions expérimentales (traitement, en
  ordonnées). On supposera que le tableau est à présent correctement
  spécifié sous forme de deux colonnes, comme indiqué à
  l'exercice~3.2. Quelle commande est la plus appropriée ?
  \begin{description}
  \item[A.] \verb|stripplot(value ~ tx, data=peas, jitter=TRUE)|
  \item[B.] \verb|stripplot(tx ~ value, data=peas, jitter=TRUE)|
  \item[C.] \verb|dotplot(value ~ tx, data=peas, jitter=TRUE)|
  \item[D.] Je ne sais pas.
  \end{description}
\item[\bf 3.4] \marginpar{\phantom{text}3.4 $\square$} Voici le résultat d'une ANOVA à un
  facteur réalisé sur la structure de données décrite à l'exercice~3.2 :
\begin{verbatim}
            Df Sum Sq Mean Sq F value   Pr(>F)    
tx           4  989.6  247.41   42.37 7.13e-14 ***
Residuals   40  233.6    5.84                     
---
Signif. codes:  0 ‘***’ 0.001 ‘**’ 0.01 ‘*’ 0.05 ‘.’ 0.1 ‘ ’ 1 
5 observations deleted due to missingness
\end{verbatim}
  Quelle commande permet de retrouver les degrés de liberté associés à la
  résiduelle ? 
  \begin{description}
  \item[A.] \verb|nrow(peas)-nlevels(peas$tx)|
  \item[B.] \verb|length(peas$value)-nlevels(peas$tx)|
  \item[C.] \verb|sum(!is.na(peas$value))-nlevels(peas$tx)|
  \item[D.] Je ne sais pas.
  \end{description}  
\item[\bf 3.5] \marginpar{\phantom{text}3.5 $\square$} Quelle commande fournit les
  moyennes de groupe ?
  \begin{description}
  \item[A.] \verb|with(peas, tapply(value, tx, mean))|
  \item[B.] \verb|with(peas, tapply(tx, value, mean))|
  \item[C.] \verb|tapply(peas$value, peas$tx, mean, na.rm=TRUE)|
  \item[D.] Je ne sais pas.
  \end{description}  
\item[\bf 3.6] \marginpar{\phantom{text}3.6 $\square$} On souhaite à présent comparer les
  traitements 2~\%~G et 2~\%~S. Quelle procédure peut-on utiliser ?
  \begin{description}
  \item[A.] \verb+summary(aov(value ~ tx, data=peas, subset=tx == "X2G" | tx == "X2S"))+
  \item[B.] \verb|summary(aov(value ~ tx, data=peas, subset=tx == "X2G" & tx == "X2S"))|
  \item[C.] Je ne sais pas.
  \end{description}
\item[\bf 3.7] \marginpar{\phantom{text}3.7 $\square$} Le résultat obtenu en 3.6 peut être
  retrouvé à partir de la commande :
  \begin{description}
  \item[A.] \verb|with(peas, t.test(value[tx=="X2G"], value[tx=="X2S"]))|
  \item[B.] \verb|with(peas, t.test(value[tx=="X2G"], value[tx=="X2S"]), var.equal=TRUE)|
  \item[C.] \verb|with(peas, t.test(value[tx=="X2G"], value[tx=="X2S"], var.equal=TRUE))|
  \item[D.] Je ne sais pas.
  \end{description}
\item[\bf 3.8] \marginpar{\phantom{text}3.8 $\square$} Supposons que le
  modèle testé à l'exercice~3.4 ait été stocké dans une variable appelée
  \texttt{m}. Que renverrait la commande suivante ?
\begin{verbatim}
fitted(m)-peas$value[!is.na(peas$value)]
\end{verbatim}
  \begin{description}
  \item[A.] Les valeurs prédites par le modèle pour chaque traitement,
    c'est-à-dire les moyennes de groupe, après exclusion des valeurs
    manquantes. 
  \item[B.] Les valeurs des résidus du modèle (écarts entre valeurs prédites
    et valeurs observées).
  \item[C.] Je ne sais pas.
  \end{description}
\end{description}  

%--------------------------------------------------------------- Chapter 05 --
\chapter{Corrélation et régression linéaire}\label{chap:reg}
\Opensolutionfile{solutions}[solutions5]

\section*{Énoncés}
%
% Everitt 2011 p. 184
%
\begin{exo}\label{exo:5.1}
Une étude a porté sur une mesure de malnutrition chez 25 patients âgés de 7
à 23 ans et souffrant de fibrose kystique. On disposait pour ces patients de
différentes informations relatives aux caractéristiques antropométriques
(taille, poids, etc.) et à la fonction pulmonaire. \autocite[p.~180]{everitt01}
Les données sont disponibles dans le fichier \texttt{cystic.dat}.
\begin{description}
\item[(a)] Calculer le coefficient de corrélation linéaire entre les
  variables \texttt{PEmax} et \texttt{Weight}, ainsi que son intervalle de
  confiance à 95~\%.
\item[(b)] Tester si le coefficient de corrélation calculé en (a) peut être
  considéré comme significativement différent de 0.3 au seuil 5~\%.
\item[(c)] Afficher l'ensemble des données numériques sous forme de
  diagrammes de dispersion, soit 45 graphiques arrangés sous forme d'une
  "matrice de dispersion".
\item[(d)] Calculer l'ensemble des corrélations de Pearson et de Spearman
  entre les variables numériques. 
  %Reporter les coefficients de
  %Bravais-Pearson supérieurs à 0.7 en valeur absolue.
\item[(e)] Calculer la corrélation entre \texttt{PEmax} et \texttt{Weight},
  en contrôlant l'âge (\texttt{Age}) (corrélation partielle). Représenter
  graphiquement la covariation entre \texttt{PEmax} et \texttt{Weight} en
  mettant en évidence les deux terciles les plus extrêmes pour la variable
  \texttt{Age}. 
\end{description}
\begin{sol}
Les données étant disponibles dans un format texte où les valeurs ("champs")
sont séparées par des tabulations, on utilisera la commande
\texttt{read.table} pour les importer dans \R. Comme la première ligne dans
le fichier permet d'identifier les variables, on ajoutera l'option
\verb|header=TRUE|. 
\begin{Schunk}
\begin{Sinput}
> cystic <- read.table("data/cystic.dat", header=TRUE)
> str(cystic)
\end{Sinput}
\begin{Soutput}
'data.frame':	25 obs. of  11 variables:
 $ Sub   : int  1 2 3 4 5 6 7 8 9 10 ...
 $ Age   : int  7 7 8 8 8 9 11 12 12 13 ...
 $ Sex   : int  0 1 0 1 0 0 1 1 0 1 ...
 $ Height: int  109 112 124 125 127 130 139 150 146 155 ...
 $ Weight: num  13.1 12.9 14.1 16.2 21.5 17.5 30.7 28.4 25.1 31.5 ...
 $ BMP   : int  68 65 64 67 93 68 89 69 67 68 ...
 $ FEV   : int  32 19 22 41 52 44 28 18 24 23 ...
 $ RV    : int  258 449 441 234 202 308 305 369 312 413 ...
 $ FRC   : int  183 245 268 146 131 155 179 198 194 225 ...
 $ TLC   : int  137 134 147 124 104 118 119 103 128 136 ...
 $ PEmax : int  95 85 100 85 95 80 65 110 70 95 ...
\end{Soutput}
\begin{Sinput}
> summary(cystic)
\end{Sinput}
\begin{Soutput}
      Sub          Age             Sex           Height          Weight           BMP       
 Min.   : 1   Min.   : 7.00   Min.   :0.00   Min.   :109.0   Min.   :12.90   Min.   :64.00  
 1st Qu.: 7   1st Qu.:11.00   1st Qu.:0.00   1st Qu.:139.0   1st Qu.:25.10   1st Qu.:68.00  
 Median :13   Median :14.00   Median :0.00   Median :156.0   Median :37.20   Median :71.00  
 Mean   :13   Mean   :14.48   Mean   :0.44   Mean   :152.8   Mean   :38.44   Mean   :78.28  
 3rd Qu.:19   3rd Qu.:17.00   3rd Qu.:1.00   3rd Qu.:174.0   3rd Qu.:51.10   3rd Qu.:90.00  
 Max.   :25   Max.   :23.00   Max.   :1.00   Max.   :180.0   Max.   :73.80   Max.   :97.00  
      FEV              RV             FRC             TLC          PEmax      
 Min.   :18.00   Min.   :158.0   Min.   :104.0   Min.   : 81   Min.   : 65.0  
 1st Qu.:26.00   1st Qu.:188.0   1st Qu.:127.0   1st Qu.:101   1st Qu.: 85.0  
 Median :33.00   Median :225.0   Median :139.0   Median :112   Median : 95.0  
 Mean   :34.72   Mean   :255.2   Mean   :155.4   Mean   :114   Mean   :109.1  
 3rd Qu.:44.00   3rd Qu.:305.0   3rd Qu.:183.0   3rd Qu.:128   3rd Qu.:130.0  
 Max.   :57.00   Max.   :449.0   Max.   :268.0   Max.   :147   Max.   :195.0  
\end{Soutput}
\end{Schunk}
\cmd{read.table}\cmd{str}\cmd{summary}
On voit d'emblée que le sexe des patients n'est pas codé sous la forme d'une
variable qualitative mais d'un nombre (0/1). Bien que cela ne soit pas
fondamentalement nécessaire dans cet exercice, il est toujours préférable de
convertir les variables dans le bon format.
\begin{Schunk}
\begin{Sinput}
> cystic$Sex <- factor(cystic$Sex, labels=c("M","F"))
> table(cystic$Sex)
\end{Sinput}
\begin{Soutput}
 M  F 
14 11 
\end{Soutput}
\end{Schunk}
\cmd{factor}\cmd{table}

La corrélation linéaire entre \texttt{PEmax} et \texttt{Weight} est obtenue
à l'aide de la commande \texttt{cor}, qui par défaut calcule un coefficient
de corrélation de Bravais-Pearson :
\begin{Schunk}
\begin{Sinput}
> with(cystic, cor(PEmax, Weight))
\end{Sinput}
\begin{Soutput}
[1] 0.6362889
\end{Soutput}
\end{Schunk}
\cmd{with}\cmd{cor}
La construction un peu étrange \texttt{with(cystic, ...} permet de ne pas avoir
à répéter le nom du \texttt{data.frame}, \texttt{cystic}, pour désigner les
variables d'intérêt. Autrement, on aurait écrit : \verb|cor(cystic$PEmax, cystic$Weight)|. 
La commande \texttt{cor} ne permet que l'estimation ponctuelle du
paramètre. Pour obtenir l'intervalle de confiance associé, on utilisera
directement la commande \texttt{cor.test} qui répond en même temps au test
d'hypothèse sur la nullité du coefficient dans la population.
\begin{Schunk}
\begin{Sinput}
> with(cystic, cor.test(PEmax, Weight))
\end{Sinput}
\begin{Soutput}
	Pearson's product-moment correlation
data:  PEmax and Weight 
t = 3.9556, df = 23, p-value = 0.0006281
alternative hypothesis: true correlation is not equal to 0 
95 percent confidence interval:
 0.3221528 0.8242013 
sample estimates:
      cor 
0.6362889 
\end{Soutput}
\end{Schunk}
\cmd{with}\cmd{cor.test}

Pour tester si ce coefficient de corrélation est significativement différent
d'une autre valeur que 0, il est nécessaire d'utiliser la commande
\texttt{r.test} du package \texttt{psych}.
Pour tester l'hypothèse $H_0:\, \rho=0.300$ au seuil 5~\%, on indiquera la
taille de l'échantillon, la corrélation dans la population et la corrélation
observée, soit :
\begin{Schunk}
\begin{Sinput}
> library(psych)
> r.test(25, 0.300, 0.6363)
\end{Sinput}
\begin{Soutput}
Correlation tests 
Call:r.test(n = 25, r12 = 0.3, r34 = 0.6363)
Test of difference between two independent correlations 
 z value 1.47    with probability  0.14
\end{Soutput}
\end{Schunk}
\cmd{r.test}

Pour afficher l'ensemble des diagrammes de dispersion, on peut utiliser la
commande de base \texttt{pairs} ou bien la solution suivante :
\begin{Schunk}
\begin{Sinput}
> splom(cystic[,-c(1,3)], varname.cex=0.7, cex=.8)
\end{Sinput}
\end{Schunk}
\includegraphics{figs/fig-ex5-1f}
\cmd{splom}

Notons que nous avons exclus la variable \texttt{Sex} qui figure dans la
3\ieme colonne du tableau de données. Les deux autres options
(\texttt{varname.cex} et \texttt{cex}) permettent de contrôler la taille des
polices et des points.

Pour calculer les corrélations de toutes les paires de variables, on peut
toujours utiliser la commande \texttt{cor}. Par contre, il ne sera pas
possible de tester ces corrélations avec la commande \texttt{cor.test} ; il
faudrait pour cela utiliser par exemple la commande \texttt{corr.test} du
package \texttt{psych}.
\begin{Schunk}
\begin{Sinput}
> round(cor(cystic[,-3]), 3)
\end{Sinput}
\begin{Soutput}
          Sub    Age Height Weight    BMP    FEV     RV    FRC    TLC  PEmax
Sub     1.000  0.989  0.940  0.909  0.393  0.322 -0.604 -0.682 -0.492  0.604
Age     0.989  1.000  0.926  0.906  0.378  0.294 -0.552 -0.639 -0.473  0.613
Height  0.940  0.926  1.000  0.922  0.441  0.317 -0.570 -0.624 -0.460  0.599
Weight  0.909  0.906  0.922  1.000  0.670  0.449 -0.623 -0.618 -0.421  0.636
BMP     0.393  0.378  0.441  0.670  1.000  0.546 -0.582 -0.434 -0.363  0.230
FEV     0.322  0.294  0.317  0.449  0.546  1.000 -0.666 -0.665 -0.443  0.453
RV     -0.604 -0.552 -0.570 -0.623 -0.582 -0.666  1.000  0.911  0.590 -0.316
FRC    -0.682 -0.639 -0.624 -0.618 -0.434 -0.665  0.911  1.000  0.706 -0.417
TLC    -0.492 -0.473 -0.460 -0.421 -0.363 -0.443  0.590  0.706  1.000 -0.181
PEmax   0.604  0.613  0.599  0.636  0.230  0.453 -0.316 -0.417 -0.181  1.000
\end{Soutput}
\end{Schunk}
\cmd{round}\cmd{cor}
Pour les corrélations de Spearman, il suffit d'ajouter l'option
\verb|method="spearman"| comme ci-dessous :
\begin{Schunk}
\begin{Sinput}
> round(cor(cystic[,-3], method="spearman"), 3)
\end{Sinput}
\begin{Soutput}
          Sub    Age Height Weight    BMP    FEV     RV    FRC    TLC  PEmax
Sub     1.000  0.996  0.945  0.912  0.525  0.325 -0.598 -0.719 -0.508  0.509
Age     0.996  1.000  0.934  0.901  0.509  0.297 -0.582 -0.718 -0.493  0.520
Height  0.945  0.934  1.000  0.962  0.573  0.426 -0.622 -0.664 -0.473  0.592
Weight  0.912  0.901  0.962  1.000  0.726  0.464 -0.701 -0.671 -0.485  0.488
BMP     0.525  0.509  0.573  0.726  1.000  0.562 -0.692 -0.555 -0.494  0.222
FEV     0.325  0.297  0.426  0.464  0.562  1.000 -0.683 -0.604 -0.440  0.314
RV     -0.598 -0.582 -0.622 -0.701 -0.692 -0.683  1.000  0.855  0.589 -0.309
FRC    -0.719 -0.718 -0.664 -0.671 -0.555 -0.604  0.855  1.000  0.672 -0.384
TLC    -0.508 -0.493 -0.473 -0.485 -0.494 -0.440  0.589  0.672  1.000 -0.148
PEmax   0.509  0.520  0.592  0.488  0.222  0.314 -0.309 -0.384 -0.148  1.000
\end{Soutput}
\end{Schunk}
\cmd{round}\cmd{cor}

%% La question du filtrage des corrélations les plus élevées en valeur absolue
%% est plus délicate car la matrice de corrélation contient deux fois la même
%% information puisqu'elle est symétrique. Il est relativement aisé d'isoler
%% les corrélations situées dans la diagonale supérieure ; par exemple, les
%% commandes suivantes nous donnent l'ensemble de ces corrélations (sans
%% considération de la paire de variables en question) :
%% <<ex5-1i>>=
%% cor.mat <- cor(cystic[,-3])
%% cor.mat[upper.tri(cor.mat)]
%% @ 
%% \cmd{cor}\cmd{upper.tri}
%% En revanche, il est plus élégant de transformer la matrice de corrélation en
%% un tableau à 3 colonnes indiquant pour chaque paire de variables la
%% corrélation de Pearson associée :
%% <<ex5-1j>>=
%% cor.mat[upper.tri(cor.mat, diag=TRUE)] <- NA
%% library(reshape)
%% subset(melt(cor.mat), abs(value) > 0.7)
%% @ 
%% \cmd{upper.tri}\cmd{library}\cmd{subset}\cmd{melt}\cmd{abs}
%% Pour éviter les doublons, on a recodé en valeurs manquantes les éléments
%% diagonaux et sous-diagonaux de la matrice de corrélation. Ensuite, on a
%% filtré les valeurs supérieures à 0.7 en valeur absolue.

Enfin, pour estimer la corrélation partielle entre \texttt{PEmax},
\texttt{Weight} et \texttt{Age}, on utilisera la commande \texttt{pcor.test}
du package \texttt{ppcor} :
\begin{Schunk}
\begin{Sinput}
> library(ppcor)
> with(cystic, pcor.test(PEmax, Weight, Age))
\end{Sinput}
\begin{Soutput}
   estimate   p.value statistic  n gp  Method
1 0.2404745 0.2452255  1.162024 25  1 pearson
\end{Soutput}
\end{Schunk}
\cmd{library}\cmd{with}\cmd{pcor.test}
On voit que la corrélation entre \texttt{PEmax} et \texttt{Weight} diminue
fortement lorsque l'on tient compte de la corrélation entre ces deux
variables et l'âge.

\begin{Schunk}
\begin{Sinput}
> cystic$Age.ter <- cut(cystic$Age, breaks=quantile(cystic$Age, c(0,0.33,0.66,1)), 
+                       include.lowest=TRUE)
> cystic2 <- subset(cystic, as.numeric(Age.ter) %in% c(1,3))
> cystic2$Age.ter <- factor(cystic2$Age.ter)
> xyplot(PEmax ~ Weight, data=cystic2, groups=Age.ter, auto.key=list(corner=c(0,1)))
\end{Sinput}
\end{Schunk}
\includegraphics{figs/fig-ex5-1l}
\cmd{cut}\cmd{subset}\cmd{factor}\cmd{xyplot}\cmd{quantile}

\noindent Les commandes utilisées ont réalisé, dans l'ordre : la création d'une
variable \texttt{Age.ter} représentant trois classes, déterminées à partir
des trois déciles ([7,12], (12,17] et (17,23]) ; la restriction du jeu de
données initiale aux seules observations comprises dans les premier et
troisième déciles de la variable \texttt{Age} ; la suppression de la
deuxième classe d'âge devenue inutile au niveau des labels associés à la
variable \texttt{Age.ter} ; et enfin, un diagramme de dispersion de
\texttt{PEmax}, en fonction de \texttt{Weight}, en soulignant à l'aide de
symboles différents les observations situées dans chacun des deux terciles
retenus. 
\end{sol}
\end{exo}
%
% METHO TD8 Exo 1
%
\begin{exo}\label{exo:5.2}
Les données disponibles dans le fichier \texttt{quetelet.csv} renseignent
sur la pression artérielle systolique (\texttt{PAS}), l'indice de Quetelet
(\texttt{QTT}), l'âge (\texttt{AGE}) et la consommation de tabac
(\texttt{TAB}=1 si fumeur, 0 sinon) pour un échantillon de 32 hommes de plus
de 40 ans. 
\begin{description}
\item[(a)] Indiquer la valeur du coefficient de corrélation linéaire entre
  la pression artérielle systolique et l'indice de Quetelet, avec un
  intervalle de confiance à 90~\%.
\item[(b)] Donner les estimations des paramètres de la droite de régression
  linéaire de la pression artérielle sur l'indice de Quetelet.
\item[(c)] Tester si la pente de la droite de régression est différente de 0
  (au seuil 5~\%).
\item[(d)] Représenter graphiquement les variations de pression artérielle
  en fonction de l'indice de Quetelet, en faisant apparaître distinctement
  les fumeurs et les non-fumeurs avec des symboles ou des couleurs
  différentes, et tracer la droite de régression dont les paramètres ont été
  estimés en (b). 
\item[(e)] Refaire l'analyse (b-c) en restreignant l'échantillon aux
  fumeurs.
\end{description}
\begin{sol}
Les données ont été exportées à partir d'un tableur de type Excel, avec
comme séparateur de champ le ";". On pourrait utiliser la commande
\texttt{read.table} et spécifier les options adéquates (\texttt{sep} et
\texttt{header}). Heureusement, \R offre des commandes qui simplifient cette
tâche : \texttt{read.csv} lit des fichiers dans lesquels le séparateur de
champ est "," tandis que \texttt{read.csv2} traite les fichiers avec un
séparateur de type ";".
\begin{Schunk}
\begin{Sinput}
> dfrm <- read.csv2("data/quetelet.csv")
> head(dfrm)
\end{Sinput}
\begin{Soutput}
  ID PAS   QTT AGE TAB
1  1 135 2.876  45   0
2  2 122 3.251  41   0
3  3 130 3.100  49   0
4  4 148 3.768  52   0
5  5 146 2.979  54   1
6  6 129 2.790  47   1
\end{Soutput}
\begin{Sinput}
> dfrm$TAB <- factor(dfrm$TAB, labels=c("NF","F"))
> summary(dfrm[,-1])
\end{Sinput}
\begin{Soutput}
      PAS             QTT             AGE        TAB    
 Min.   :120.0   Min.   :2.368   Min.   :41.00   NF:15  
 1st Qu.:134.8   1st Qu.:3.022   1st Qu.:48.00   F :17  
 Median :143.0   Median :3.381   Median :53.50          
 Mean   :144.5   Mean   :3.441   Mean   :53.25          
 3rd Qu.:152.0   3rd Qu.:3.776   3rd Qu.:58.25          
 Max.   :180.0   Max.   :4.637   Max.   :65.00          
\end{Soutput}
\end{Schunk}
\cmd{read.csv2}\cmd{head}\cmd{factor}\cmd{summary}
On a profité de l'inspection rapide des premières observations indiquant que
la variable "consommation de tabac" était représentée sous forme de nombres
(0/1) pour la recoder en variable qualitative avec labels plus informatifs
(\texttt{NF}=non-fumeur, \texttt{F}=fumeur).
  
L'estimation du coefficient de corrélation liénaire et son intervalle de
confiance à 90~\% ne pose pas de problème particulier : on utilisera la même
commande qu'à l'exercice~\ref{exo:5.1}, \texttt{cor.test}, en négligeant le
test d'hypothèse mais en modifiant l'option \texttt{conf.level} qui, par
défault, est fixée à 0.95.
\begin{Schunk}
\begin{Sinput}
> with(dfrm, cor.test(PAS, QTT, conf.level=0.90))
\end{Sinput}
\begin{Soutput}
	Pearson's product-moment correlation
data:  PAS and QTT 
t = 6.0623, df = 30, p-value = 1.172e-06
alternative hypothesis: true correlation is not equal to 0 
90 percent confidence interval:
 0.5713213 0.8511648 
sample estimates:
      cor 
0.7420041 
\end{Soutput}
\end{Schunk}
\cmd{with}\cmd{cor.test}

La commande pour effectuer une régression linéaire, simple ou multiple, est
\texttt{lm} (pour \underline{l}inear \underline{m}odel). La forme générale
du modèle est symbolisée comme suit : variable réponse ~ variable
explivative. Les commandes \texttt{summary} et \texttt{coef} permettent
d'obtenir, respectivement, un tableau résumant les coefficients de
régression estimés à partir des données et leur degré de significativité (à
partir d'un test $t$ de Student), ainsi que d'autres informations que nous
discuterons plus loin, ou simplement les coefficients de régression
(ordonnée à l'origine et pente de la droite de régression).
\begin{Schunk}
\begin{Sinput}
> reg.res <- lm(PAS ~ QTT, data=dfrm)
> summary(reg.res)
\end{Sinput}
\begin{Soutput}
Call:
lm(formula = PAS ~ QTT, data = dfrm)
Residuals:
    Min      1Q  Median      3Q     Max 
-19.231  -7.145  -1.604   7.798  22.531 

Coefficients:
            Estimate Std. Error t value Pr(>|t|)
(Intercept)   70.576     12.322   5.728 2.99e-06
QTT           21.492      3.545   6.062 1.17e-06

Residual standard error: 9.812 on 30 degrees of freedom
Multiple R-squared: 0.5506,	Adjusted R-squared: 0.5356 
F-statistic: 36.75 on 1 and 30 DF,  p-value: 1.172e-06 
\end{Soutput}
\begin{Sinput}
> coef(reg.res)
\end{Sinput}
\begin{Soutput}
(Intercept)         QTT 
   70.57640    21.49167 
\end{Soutput}
\end{Schunk}
\cmd{lm}\cmd{summary.lm}\cmd{coef}
Le test sur la pente est directement accessible à partir du tableau de
régression ; ici $t=6.062$ et $p<0.001$. Le coefficient associé à
\texttt{QTT} reflète l'augmentation de \texttt{PAS} (21.5 points) lorsque
\texttt{QTT} varie d'une unité.

Pour représenter les données sous forme graphique, on utilisera
\texttt{xyplot} qui permet de représenter les variations d'une variable en
fonction des valeurs prises par une autre variable. Les options utilisées
ci-après permettent d'afficher les observations avec des symboles ou des
couleurs différentes selon le statut realtif à la consommation de tabac
(\texttt{group=TAB}) ainsi que les droites de régression estimées sur ces
deux sous-échantillons (option \verb|type="r"|).
\begin{Schunk}
\begin{Sinput}
> xyplot(PAS ~ QTT, data=dfrm, groups=TAB, type=c("p", "g", "r"),
+        auto.key=list(points=FALSE, lines=TRUE))
\end{Sinput}
\end{Schunk}
\includegraphics{figs/fig-ex5-2d}
\cmd{xyplot}

\noindent Il existe une autre manière de représenter la droite de régression
associée estimée sur l'ensemble de l'échantillon ou un ou plusieurs
sous-groupes. Ici, l'option \verb|type="r"| simplifie beaucoup les choses
puisqu'elle tient compte de l'option de groupement, \texttt{groups=}.

Pour restreindre l'analyse aux seules observations pour lesquelles
\verb|TAB="F"| (les fumeurs), on peut utiliser la commande \texttt{subset}
pour filtrer les lignes du tableau de données. Toutefois, cette commande
peut également être utilisée sous forme d'une option lorsque l'on utilise la
commande \texttt{lm} pour le modèle de régression linéaire.
\begin{Schunk}
\begin{Sinput}
> reg.res2 <- lm(PAS ~ QTT, data=dfrm, subset=TAB=="F")
> summary(reg.res2)
\end{Sinput}
\begin{Soutput}
Call:
lm(formula = PAS ~ QTT, data = dfrm, subset = TAB == "F")
Residuals:
    Min      1Q  Median      3Q     Max 
-22.371  -6.385   1.908   7.617  17.105 

Coefficients:
            Estimate Std. Error t value Pr(>|t|)
(Intercept)   79.255     15.768   5.026 0.000151
QTT           20.118      4.567   4.405 0.000512

Residual standard error: 10.37 on 15 degrees of freedom
Multiple R-squared: 0.564,	Adjusted R-squared: 0.5349 
F-statistic:  19.4 on 1 and 15 DF,  p-value: 0.0005119 
\end{Soutput}
\begin{Sinput}
> coef(reg.res2)
\end{Sinput}
\begin{Soutput}
(Intercept)         QTT 
   79.25533    20.11804 
\end{Soutput}
\end{Schunk}
\cmd{lm}\cmd{summary.lm}\cmd{coef}
\end{sol}
\end{exo}
%
% Dupont 2009 p. 63
%
\begin{exo}\label{exo:5.3}
Dans l'étude Framingham, on dispose de donnée sur la pression artérielle
systolique (\texttt{sbp}) et l'indice de masse corporelle (\texttt{bmi}) de
2047 hommes et 2643 femmes.\autocite[p.~63]{dupont09} On s'intéresse à la
relation entre ces deux variables (après transformation logarithmique) chez
les hommes et chez les femmes séparément.
Les données sont disponibles dans le fichier \texttt{Framingham.csv}.
\begin{description}
\item[(a)] Représenter graphiquement les variations entre pression
  artérielle et IMC (\texttt{bmi}) chez les hommes et chez les femmes.
\item[(b)] Les coefficients de corrélation linéaire estimés chez les hommes
  et chez les femmes sont-ils significativement différents à 5~\% ?
\item[(c)] Estimer les paramètres du modèle de régression linéaire
  considérant la pression artérielle comme variable réponse et l'IMC comme
  variable explicative, pour ces deux sous-échantillons. Donner un
  intervalle de confiance à 95~\% pour l'estimé des pentes respectives.
\item[(d)] Tester l'égalité des deux coefficients de régression associés à
  la pente (au seuil 5~\%).
\end{description}
\begin{sol}
Cette fois, les données ont été générées à partir d'un tableur (Excel ou
autre) mais le séparateur de champ est la ",", d'où l'usage de
\texttt{read.csv} au lieu de \texttt{read.csv2} comme dans
l'exercice~\ref{exo:5.2}. 
\begin{Schunk}
\begin{Sinput}
> fram <- read.csv("data/Framingham.csv")
> head(fram)
\end{Sinput}
\begin{Soutput}
  sex sbp dbp scl chdfate followup age  bmi month   id
1   1 120  80 267       1       18  55 25.0     8 2642
2   1 130  78 192       1       35  53 28.4    12 4627
3   1 144  90 207       1      109  61 25.1     8 2568
4   1  92  66 231       1      147  48 26.2    11 4192
5   1 162  98 271       1      169  39 28.4    11 3977
6   2 212 118 182       1      199  61 33.3     2  659
\end{Soutput}
\begin{Sinput}
> str(fram)
\end{Sinput}
\begin{Soutput}
'data.frame':	4699 obs. of  10 variables:
 $ sex     : int  1 1 1 1 1 2 1 1 1 1 ...
 $ sbp     : int  120 130 144 92 162 212 140 174 142 115 ...
 $ dbp     : int  80 78 90 66 98 118 85 102 94 70 ...
 $ scl     : int  267 192 207 231 271 182 276 259 242 242 ...
 $ chdfate : int  1 1 1 1 1 1 1 1 1 1 ...
 $ followup: int  18 35 109 147 169 199 201 209 265 278 ...
 $ age     : int  55 53 61 48 39 61 44 39 47 60 ...
 $ bmi     : num  25 28.4 25.1 26.2 28.4 ...
 $ month   : int  8 12 8 11 11 2 6 11 5 10 ...
 $ id      : int  2642 4627 2568 4192 3977 659 2290 4267 2035 3587 ...
\end{Soutput}
\end{Schunk}
\cmd{read.csv}\cmd{head}\cmd{str}
La variable \texttt{sex} est traitée comme une variable quantitative (1/2)
et pour faciliter l'interprétation, nous la recodons d'emblée en variable
qualitative. 
\begin{Schunk}
\begin{Sinput}
> table(fram$sex)
\end{Sinput}
\begin{Soutput}
   1    2 
2049 2650 
\end{Soutput}
\begin{Sinput}
> fram$sex <- factor(fram$sex, labels=c("M","F"))
\end{Sinput}
\end{Schunk}
\cmd{table}\cmd{factor}

Avant de répondre à la question concernant les variations entre pression
artérielle et IMC, on peut vouloir vérifier la présence de valeurs
manquantes. Cela ne gêne en rien l'estimation des moyennes et écart-types,
des paramètres de la droite de régression ou la représentation graphique des
données, mais cela permet de connaître le nombre de "cas complets" sur les
données d'intérêt.
\begin{Schunk}
\begin{Sinput}
> apply(fram, 2, function(x) sum(is.na(x)))
\end{Sinput}
\begin{Soutput}
     sex      sbp      dbp      scl  chdfate followup      age      bmi    month       id 
       0        0        0       33        0        0        0        9        0        0 
\end{Soutput}
\end{Schunk}
\cmd{apply}\cmd{sum}\cmd{is.na}\cmd{function}
On utilise une commande \texttt{apply} pour répéter une même opération pour
chaque variable, cette opération consistant à compter (\texttt{sum}) le
nombre de valeurs manquantes (\texttt{is.na(x)}). On constate que pour l'une
des variables de notre modèles, l'IMC (\texttt{bmi}), 9 observations sont
manquantes. D'où la distribution par sexe suivante :
\begin{Schunk}
\begin{Sinput}
> with(fram, table(sex[!is.na(bmi)]))
\end{Sinput}
\begin{Soutput}
   M    F 
2047 2643 
\end{Soutput}
\end{Schunk}
\cmd{with}\cmd{table}

Pour représenter les données dans un diagramme de dispersion, on prendra
garde au fait qu'en raison des "grands" effectifs par sous-groupe (plus de
2000 observations), il risque d'y avoir beaucoup de points qui se
chevauchent. Une possibilité est d'utiliser la semi-transparence et de
réduire la taille des points.
\begin{Schunk}
\begin{Sinput}
> xyplot(sbp ~ bmi | sex, data=fram, type=c("p","g"), alpha=0.5, cex=0.7, pch=19)
\end{Sinput}
\end{Schunk}
\includegraphics{figs/fig-ex5-3e}
\cmd{xyplot}

Les corrélations entre les variables \texttt{sbp} et \texttt{bmi} chez les
hommes et chez les femmes peuvent être obtenues à partir de la commande
\texttt{cor}, en restreignant bien sûr l'analyse à chacun des deux
sous-échantillons :
\begin{Schunk}
\begin{Sinput}
> with(subset(fram, sex=="M"), cor(sbp, bmi, use="pair"))
\end{Sinput}
\begin{Soutput}
[1] 0.2364408
\end{Soutput}
\begin{Sinput}
> with(subset(fram, sex=="F"), cor(sbp, bmi, use="pair"))
\end{Sinput}
\begin{Soutput}
[1] 0.3736216
\end{Soutput}
\end{Schunk}
\cmd{with}\cmd{subset}\cmd{cor}
On notera que l'on a rajouté l'option \verb|use="pair"| (l'option complète
se lit \verb|"pairwise.complete.obs"| mais il est possible d'abréger les
options lorsque cela ne pose pas de problème d'ambiguïté) pour calculer les
corrélations sur l'ensemble des données observées. Pour tester si ces deux
coefficients estimés à partir des données peuvent être considérés comme
significativement différents au seuil 5~\%, il faut utiliser la commande
\texttt{r.test} du package \texttt{psych}.
\begin{Schunk}
\begin{Sinput}
> library(psych)
> r.test(n=2047, r12=0.23644, n2=2643, r34=0.37362)
\end{Sinput}
\begin{Soutput}
Correlation tests 
Call:r.test(n = 2047, r12 = 0.23644, r34 = 0.37362, n2 = 2643)
Test of difference between two independent correlations 
 z value 5.15    with probability  0
\end{Soutput}
\end{Schunk}
\cmd{library}\cmd{r.test}

Pour vérifier l'effet de la transformation logarithmique sur la distribution
des variables \texttt{sbp} et \texttt{bmi}, on peut procéder de deux
manières : soit créer un \texttt{data.frame} contenant les valeurs de
densité de chaque histogramme, associées au type de variable (\texttt{bmi},
\texttt{log(bmi)}, \texttt{sbp} et \texttt{log(sbp)}), soit construire
séparément les quatre histogrammes et les combiner en une seule figure à
l'aide de la commande \texttt{grid.arrange} du package
\texttt{gridExtra}. Voici ce qui est obtenu avec cette deuxième solution :
\begin{Schunk}
\begin{Sinput}
> library(gridExtra)
> p1 <- histogram(~ bmi, data=fram)
> p2 <- histogram(~ log(bmi), data=fram)
> p3 <- histogram(~ sbp, data=fram)
> p4 <- histogram(~ log(sbp), data=fram)
> grid.arrange(p1, p2, p3, p4)
\end{Sinput}
\end{Schunk}
\includegraphics{figs/fig-ex5-3h}
\cmd{library}\cmd{grid.arrange}

Le modèle de régression dans chaque sous groupe est établi comme suit :
\begin{Schunk}
\begin{Sinput}
> reg.resM <- lm(log(sbp) ~ log(bmi), data=fram, subset=sex=="M")
> reg.resF <- lm(log(sbp) ~ log(bmi), data=fram, subset=sex=="F")
> summary(reg.resM)   # Hommes
\end{Sinput}
\begin{Soutput}
Call:
lm(formula = log(sbp) ~ log(bmi), data = fram, subset = sex == 
    "M")
Residuals:
     Min       1Q   Median       3Q      Max 
-0.40881 -0.09414 -0.01401  0.07435  0.56783 

Coefficients:
            Estimate Std. Error t value Pr(>|t|)
(Intercept)  3.98804    0.07546   52.85   <2e-16
log(bmi)     0.27265    0.02322   11.74   <2e-16

Residual standard error: 0.137 on 2045 degrees of freedom
  (2 observations deleted due to missingness)
Multiple R-squared: 0.06319,	Adjusted R-squared: 0.06273 
F-statistic: 137.9 on 1 and 2045 DF,  p-value: < 2.2e-16 
\end{Soutput}
\begin{Sinput}
> confint(reg.resM)
\end{Sinput}
\begin{Soutput}
                2.5 %    97.5 %
(Intercept) 3.8400599 4.1360266
log(bmi)    0.2271181 0.3181739
\end{Soutput}
\begin{Sinput}
> summary(reg.resF)   # Femmes
\end{Sinput}
\begin{Soutput}
Call:
lm(formula = log(sbp) ~ log(bmi), data = fram, subset = sex == 
    "F")
Residuals:
     Min       1Q   Median       3Q      Max 
-0.42452 -0.11675 -0.01952  0.09771  0.78367 

Coefficients:
            Estimate Std. Error t value Pr(>|t|)
(Intercept)  3.59302    0.05979   60.09   <2e-16
log(bmi)     0.39859    0.01855   21.49   <2e-16

Residual standard error: 0.1616 on 2641 degrees of freedom
  (7 observations deleted due to missingness)
Multiple R-squared: 0.1489,	Adjusted R-squared: 0.1485 
F-statistic: 461.9 on 1 and 2641 DF,  p-value: < 2.2e-16 
\end{Soutput}
\begin{Sinput}
> confint(reg.resF)
\end{Sinput}
\begin{Soutput}
                2.5 %    97.5 %
(Intercept) 3.4757793 3.7102541
log(bmi)    0.3622278 0.4349616
\end{Soutput}
\end{Schunk}
\cmd{lm}\cmd{summary.lm}\cmd{confint}
Pour les intervalles de confiance, on utilise la commande
\texttt{confint}. On peut regrouper tous les résultats qui nous intéressent
dans un même tableau. Par exemple,
\begin{Schunk}
\begin{Sinput}
> res <- data.frame(pente=c(coef(reg.resM)[2], coef(reg.resF)[2]), 
+                   rbind(confint(reg.resM)[2,], confint(reg.resF)[2,]))
> rownames(res) <- c("M","F")
> colnames(res)[2:3] <- c("2.5 %", "97.5 %")
> round(res, 3)
\end{Sinput}
\begin{Soutput}
  pente 2.5 % 97.5 %
M 0.273 0.227  0.318
F 0.399 0.362  0.435
\end{Soutput}
\end{Schunk}
\cmd{data.frame}\cmd{rownames}\cmd{colnames}\cmd{round}\cmd{rbind}
\cmd{confint}\cmd{coef}
% FIXME: test de l'égalité des pentes
\end{sol}
\end{exo}
%
% Hosmer & Lemeshow 1989
%
\begin{exo}
À partir des données sur les poids à la naissance décrites dans
l'exercice~\ref{exo:2.4}, on cherche à étudier la relation entre le poids des
bébés (traité en tant que variable numérique, \texttt{bwt}) et deux
caractéristiques de la mère : son poids (\texttt{lwt}) et son origine
ethnique (\texttt{race}).
\begin{description}
\item[(a)] Représenter graphiquement la relation entre poids des bébés et
  poids des mères, en fonction de l'ethnicité des mères.
\item[(b)] Estimer les paramètres de la régression linéaire en considérant
  les poids des bébés comme variable réponse et les poids des mères centrés
  sur leur moyenne comme variable explicative. La pente estimée est-elle
  significative au seuil usuel de 5~\% ?
\item[(c)] Estimer les paramètres de la régression linéaire où cette fois la
  variable explicative est l'ethnicité des mères, la variable réponse
  restant le poids des bébés. Comparer la significativité du modèle dans son
  ensemble avec les résultats obtenus à partir d'une ANOVA à un facteur
  (ethnicité). 
\item[(d)] Quelle est le poids prédit pour un bébé dont la mère pèse 60 kg ?
  Donner un intervalle de confiance à 95~\% pour une prédiction ponctuelle
  en moyenne.
%% \item[(d)] Refaire l'analyse de régression décrite en (c) après avoir
%%   modifié la manière dont \R génère les contrastes pour les variables
%%   qualitatives en tapant ceci à l'invite de commande \R :
%% \begin{verbatim}
%% > options(contrasts=c("contr.sum", "contr.poly"))
%% \end{verbatim}
%% Comparer avec les résultats précédents.
\end{description}
\begin{sol}
Les données peuvent être importées et recodées exactement comme dans les
solutions proposées pour l'exercice~\ref{exo:2.4}, soit les séries de
commandes \R suivantes :
\begin{Schunk}
\begin{Sinput}
> data(birthwt, package="MASS")
> str(birthwt)
> head(birthwt, 5)
> yesno <- c("No","Yes")
> ethn <- c("White","Black","Other")
> birthwt <- within(birthwt, {
+   low <- factor(low, labels=yesno)
+   race <- factor(race, labels=ethn)
+   smoke <- factor(smoke, labels=yesno)
+   ui <- factor(ui, labels=yesno)
+   ht <- factor(ht, labels=yesno)
+ })
\end{Sinput}
\end{Schunk}

On peut représenter la relation entre poids des bébés et poids des mères à
l'aide d'un simple diagramme de dispersion. Ici, on a choisi de stratifier
sur l'ethnicité des mères et d'afficher trois graphiques séparés.
\begin{Schunk}
\begin{Sinput}
> xyplot(bwt ~ lwt | race, data=birthwt, layout=c(3,1), type=c("p","g"), aspect=0.8)
\end{Sinput}
\end{Schunk}
\includegraphics{figs/fig-ex5-4c}
\cmd{xyplot}

\noindent L'option \texttt{aspect=0.8} permet de modifier le rapport
largeur/hauteur des graphiques. L'option \verb|layout=c(3,1)| permet
d'afficher les trois graphiques côte à côte (1 ligne, 3 colonnes). En
modifiant légèrement la commande précédente, il serait tout à fait possible
d'afficher toutes les observations dans le même graphique mais en utilisant
des symboles ou couleurs différents selon l'origine ethnique :
\begin{verbatim}
> xyplot(bwt ~ lwt, data=birthwt, groups=race)
\end{verbatim}
\cmd{xyplot}

Les paramètres du modèle de régression linéaire sont estimés à partir de la
commande \texttt{lm}, sachant que la commande \texttt{scale} permet de
centrer et/ou réduire une série de mesures. Ici, on souhaite uniquement
centrer les poids des mères sur leur moyenne, pas les standardiser par
unités d'écart-type.
\begin{Schunk}
\begin{Sinput}
> reg.res <- lm(bwt ~ scale(lwt, scale=FALSE), data=birthwt)
> summary(reg.res)
\end{Sinput}
\begin{Soutput}
Call:
lm(formula = bwt ~ scale(lwt, scale = FALSE), data = birthwt)
Residuals:
     Min       1Q   Median       3Q      Max 
-2192.12  -497.97    -3.84   508.32  2075.60 

Coefficients:
                          Estimate Std. Error t value Pr(>|t|)
(Intercept)               2944.587     52.259  56.346   <2e-16
scale(lwt, scale = FALSE)    9.744      3.770   2.585   0.0105

Residual standard error: 718.4 on 187 degrees of freedom
Multiple R-squared: 0.0345,	Adjusted R-squared: 0.02933 
F-statistic: 6.681 on 1 and 187 DF,  p-value: 0.0105 
\end{Soutput}
\end{Schunk}
\cmd{lm}\cmd{summary.lm}
Le test $t$ évaluant la nullité de l'hypothèse nulle pour la pente est
significatif si l'on considère un risque de première espèce de 5~\%
($p=0.011$).

Si l'on considère l'ethnicité comme variable explicative, le modèle de
régression est évalué de la même manière :
\begin{Schunk}
\begin{Sinput}
> reg.res2 <- lm(bwt ~ race, data=birthwt)
> summary(reg.res2)
\end{Sinput}
\begin{Soutput}
Call:
lm(formula = bwt ~ race, data = birthwt)
Residuals:
     Min       1Q   Median       3Q      Max 
-2096.28  -502.72   -12.72   526.28  1887.28 

Coefficients:
            Estimate Std. Error t value Pr(>|t|)
(Intercept)  3102.72      72.92  42.548  < 2e-16
raceBlack    -383.03     157.96  -2.425  0.01627
raceOther    -297.44     113.74  -2.615  0.00965

Residual standard error: 714.5 on 186 degrees of freedom
Multiple R-squared: 0.05017,	Adjusted R-squared: 0.03996 
F-statistic: 4.913 on 2 and 186 DF,  p-value: 0.008336 
\end{Soutput}
\end{Schunk}
\cmd{lm}\cmd{summary.lm}
L'ANOVA nous donne :
\begin{Schunk}
\begin{Sinput}
> aov.res <- aov(bwt ~ race, data=birthwt)
> summary(aov.res)
\end{Sinput}
\begin{Soutput}
             Df   Sum Sq Mean Sq F value  Pr(>F)
race          2  5015725 2507863   4.913 0.00834
Residuals   186 94953931  510505                
\end{Soutput}
\end{Schunk}
\cmd{aov}\cmd{summary.aov}
En fait, il est tout à fait possible d'obtenir le tableau de régression
précédent en utilisant \texttt{summary.lm} au lieu de \texttt{summary},
comme on peut le vérifier dans la sortie suivante.
\begin{Schunk}
\begin{Sinput}
> summary.lm(aov.res)
\end{Sinput}
\begin{Soutput}
Call:
aov(formula = bwt ~ race, data = birthwt)
Residuals:
     Min       1Q   Median       3Q      Max 
-2096.28  -502.72   -12.72   526.28  1887.28 

Coefficients:
            Estimate Std. Error t value Pr(>|t|)
(Intercept)  3102.72      72.92  42.548  < 2e-16
raceBlack    -383.03     157.96  -2.425  0.01627
raceOther    -297.44     113.74  -2.615  0.00965

Residual standard error: 714.5 on 186 degrees of freedom
Multiple R-squared: 0.05017,	Adjusted R-squared: 0.03996 
F-statistic: 4.913 on 2 and 186 DF,  p-value: 0.008336 
\end{Soutput}
\end{Schunk}
\cmd{summary.lm}
Réciproquement, on peut tout à fait afficher un tableau d'ANOVA
correspondant au modèle de régression précédent :
\begin{Schunk}
\begin{Sinput}
> anova(reg.res2)
\end{Sinput}
\begin{Soutput}
Analysis of Variance Table
Response: bwt
           Df   Sum Sq Mean Sq F value   Pr(>F)
race        2  5015725 2507863  4.9125 0.008336
Residuals 186 94953931  510505                 
\end{Soutput}
\end{Schunk}
\cmd{anova}
On peut vérifier que la statistique F est identique dans les deux cas (elle
vaut 4.913, pour 2 et 186 degrés de liberté).

Les contrastes utilisés dans le modèle de régression (\texttt{reg.res2})
sont appelés contrastes de traitement et ils permettent de tester la
différence entre les scores moyens (ici, l'âge) de deux catégories d'une
variable qualitative (ici, l'ethnicité), l'une des deux catégories servant
de catégorie dite \og de référence\fg. Avec R, la catégorie de référence est
toujours le premier niveau du facteur, en suivant l'ordre lexicographique,
soit dans le cas présent le niveau \texttt{White}. Les coefficients de
régression représentent alors la différence de poids moyen entre les
bébés des mères de la catégorie \texttt{Black} \emph{versus} \texttt{White}
(-383.03), et \texttt{Other} \emph{versus} \texttt{White} (-297.44). On peut
le vérifier en calculant manuellement ces différences de moyennes :
\begin{Schunk}
\begin{Sinput}
> grp.means <- with(birthwt, tapply(bwt, race, mean))
> grp.means[2:3] - grp.means[1]     # 1er modèle de régression (reg.res2)
\end{Sinput}
\begin{Soutput}
    Black     Other 
-383.0264 -297.4352 
\end{Soutput}
\end{Schunk}
%% Si l'on change la manière de coder les contrastes, par exemple en utilisant
%% des contrastes de 
%% <<ex5-4h>>=
%% op <- options(contrasts=c("contr.sum", "contr.poly"))
%% reg.res3 <- lm(bwt ~ race, data=birthwt)
%% summary(reg.res3)
%% options(op)
%% @ 
%% \cmd{options}\cmd{lm}\cmd{sumamry.lm}
%% On peut le vérifier à partir des moyennes de groupe :
%% <<ex5-4i>>=
%% grp.means <- with(birthwt, tapply(bwt, race, mean))
%% grp.means[2:3] - grp.means[1]     # 1er modèle de régression (reg.res2)
%% grp.means[1:2] - mean(grp.means)  # 2ème modèle de régression (reg.res3)
%% @ 
%% \cmd{with}\cmd{tapply}
%% Dans le premier modèle, l'intercept vaut \texttt{grp.means[1]}, tandis
%% que dans le second modèle il s'agit de la moyenne des moyennes de
%% groupe (\texttt{mean(grp.means)}). Les deux coefficients associés aux
%% pentes représentent dans le premier cas les déviations entre
%% \texttt{Black} et \texttt{Other} par rapport à \texttt{White}, et dans
%% le second cas entre \texttt{White} et \texttt{Black} et la moyenne des
%% trois groupes.
% \url{http://bit.ly/LFkFBg}.

Pour prédire le poids d'un individu dont la mère pèse 60 kg, on utilisera la
commande \texttt{predict}, sachant que cette même commande permet également
de faire une estimation par intervalle (en moyenne ou pour une observation
future) grâce à l'option \texttt{interval}.
\begin{Schunk}
\begin{Sinput}
> m <- lm(bwt ~ lwt, data=birthwt)
> d <- data.frame(lwt=60)
> predict(m, newdata=d, interval="confidence") 
\end{Sinput}
\begin{Soutput}
       fit      lwr      upr
1 2954.266 2850.909 3057.623
\end{Soutput}
\end{Schunk}
Les colonnes \texttt{lwr} et \texttt{upr} correspondent respectivement aux
bornes inférieure et supérieure de l'IC à 95~\%.
On notera qu'il est nécessaire de fournir la valeur des cofacteurs
d'intérêt dans un \texttt{data.frame} séparé (ici, \texttt{d}). Si l'on
souhaitait obtenir plusieurs prédictions, ou utuliser plusieurs cofacteurs,
on procèderait excatement de la même manière en construisant un tableau des
valeurs d'intérêt. À titre d'illustration, voici comment on procèderait si
l'on s'intéressait à la prédiction du poids d'un bébé dont la mère pèse 55
ou 60 kg et dont la classe d'appartenance pour l'ethnicité (\texttt{race})
est \texttt{Other} :
\begin{Schunk}
\begin{Sinput}
> m <- lm(bwt ~ lwt + race, data=birthwt)
> d <- data.frame(lwt=c(55, 60), race=rep("Other", 2))
> predict(m, d, interval="confidence")
\end{Sinput}
\begin{Soutput}
       fit      lwr      upr
1 2809.877 2640.389 2979.366
2 2861.175 2686.741 3035.610
\end{Soutput}
\end{Schunk}
Si l'on ne précise aucune valeur pour l'option \texttt{newdata}, R produira
une estimation de la valeur prédite pour chaque unité statistique avec les
valeurs observées pour le ou les cofacteurs présents dans le modèle.
\end{sol}
\end{exo}
\Closesolutionfile{solutions}

\chapter*{Devoir \no 4}
\addcontentsline{toc}{chapter}{Devoir \no 4}

Les exercices sont indépendants. Une seule réponse est correcte pour chaque
question. Lorsque vous ne savez pas répondre, cochez la case correspondante.

\section*{Exercice 1}
Dans la même étude sur la thérapie hormonale (cf. exercice~2 du devoir \no
3, p.~\pageref{dev3:exo2}), on disposait de l'âge des participants dont la
distribution est reportée ci-dessous :
\begin{verbatim}
      age            raceth           SBP       
 Min.   :44.00   Min.   :1.000   Min.   : 83.0  
 1st Qu.:62.00   1st Qu.:1.000   1st Qu.:122.0  
 Median :67.00   Median :1.000   Median :134.0  
 Mean   :66.65   Mean   :1.147   Mean   :135.1  
 3rd Qu.:72.00   3rd Qu.:1.000   3rd Qu.:147.0  
 Max.   :79.00   Max.   :3.000   Max.   :224.0  
\end{verbatim}
Les données sont toujours contenue dans un \texttt{data.frame} nommé
\texttt{d} sous R. La question d'intérêt porte sur la relation entre les
variables \texttt{age} et \texttt{SBP}.
\begin{description}
\item[\bf 1.1] \marginpar{\phantom{text}1.1 $\square$} On souhaite afficher un diagramme
  de dispersion où l'âge figure sur l'axe horizontal (abscisses) et la
  pression systolique sur l'axe vertical (ordonnées). Quelle commande
  peut-on utiliser ?
  \begin{description}
  \item[A.] \verb|xyplot(age, SBP, data=d)|
  \item[B.] \verb|xyplot(age ~ SBP, data=d)|
  \item[C.] \verb|xyplot(SBP ~ age, data=d)|
  \item[D.] \verb|xyplot(SBP ~ age + raceth, data=d)|
  \item[E.] Je ne sais pas.
  \end{description}
\item[\bf 1.2] \marginpar{\phantom{text}1.2 $\square$} On souhaite vérifier l'étendue des
  âges (minimum et maximum) par groupe d'ethnicité (variable
  \texttt{raceth}). Quelle commande permet de répondre à cette question ?
  (On pourra considérer que R agira bien comme si la variable
  \texttt{raceth} était un facteur.)
  \begin{description}
  \item[A.] \verb|with(d, tapply(age, raceth, range(x)))|
  \item[B.] \verb|with(d, tapply(age, raceth, function(x) min(x), max(x)))|
  \item[C.] \verb|with(d, tapply(age, raceth, range))|
  \item[D.] \verb|with(d, tapply(age, raceth, c(max(age),min(age))))|
  \item[E.] Je ne sais pas.
  \end{description}
\item[\bf 1.3] \marginpar{\phantom{text}1.3 $\square$} On souhaite calculer les déciles
  pour la variable \texttt{age}. Quelle commande faut-il utiliser ?
  \begin{description}
  \item[A.] \verb|cut(d$age, 1:10)|
  \item[B.] \verb|factor(d$age, 1:10)|
  \item[C.] \verb|quantile(d$age, breaks=seq(0,1,.1))|
  \item[D.] \verb|quantile(d$age, probs=seq(0,1,.1))|
  \item[E.] Je ne sais pas.
  \end{description}  
\item[\bf 1.4] \marginpar{\phantom{text}1.4 $\square$} On souhaite à présent recoder la
  variable numérique \texttt{age} en une variable qualitative \texttt{aged}
  définie de sorte que chaque valeur numérique observée est remplacée par la
  valeur du décile dans laquelle elle se situe. Supposons que le résultat
  calculé à l'exercice~1.3 ait été stocké dans une variable appelée
  \texttt{ageq}. La commande suivante fournit-elle le résultat escompté ?
\begin{verbatim}
d$aged <- cut(d$age, ageq)
\end{verbatim}
  \begin{description}
  \item[A.] Oui.
  \item[B.] Non.
  \item[C.] Je ne sais pas.
  \end{description}
\item[\bf 1.5] \marginpar{\phantom{text}1.5 $\square$} On souhaite calculer la différence
  moyenne de pression systolique entre les deux déciles les plus extrêmes
  (10~\% d'âges les plus bas et 10~\% d'âge les plus hauts). En supposant
  que la variable \texttt{aged} (âges individuels recodés sous forme de
  déciles) ait été correctement créée (voir aperçu ci-après), quelle
  solution peut-on utiliser ?
\begin{verbatim}
> summary(d$aged)
[44,58] (58,61] (61,63] (63,65] (65,67] (67,69] (69,71] (71,73] (73,75] (75,79] 
    346     304     231     264     276     323     290     289     202     238
\end{verbatim}
  \begin{description}
  \item[A.] \verb|with(d, mean(SBP[aged==10]) - mean(SBP[aged==1]))|
  \item[B.] \verb|with(d, mean(SBP[aged=="10"]) - mean(SBP[aged=="1"]))|
  \item[C.] \verb|diff(with(d, tapply(SBP, aged, mean))[c(1,10)])|
  \item[D.] \verb|diff(mean(d$SBP[d$aged %in% c("[44,58]","(75,79]")]))|
  \item[E.] Je ne sais pas.
  \end{description}  
\item[\bf 1.6] \marginpar{\phantom{text}1.6 $\square$} Quel résultat produit cette commande : 
\begin{verbatim}
xyplot(SBP ~ age | factor(raceth), data=d, layout=c(3,1), type="r")
\end{verbatim}
  \begin{description}
  \item[A.] Trois diagrammes de dispersion de la pression systolique en
    fonction de l'âge pour chaque sous-groupe de participants défini selon
    le facteur ethnicité (\texttt{raceth}).
  \item[B.] Un diagramme de dispersion de la pression systolique en
    fonction de l'âge de l'ensemble des participants avec des symboles
    graphiques variables selon le facteur ethnicité (\texttt{raceth}).
  \item[C.] Trois graphiques montrant la droite de régression de la
    variable \texttt{age} sur la variable réponse \texttt{SBP} pour
    chaque sous-groupe de participants défini selon le facteur ethnicité
    (\texttt{raceth}).
  \item[D.] Je ne sais pas.
  \end{description}  
\item[\bf 1.7] \marginpar{\phantom{text}1.7 $\square$} Voici un diagramme de dispersion
  résumant la distribution conjointe des mesures de pression systolique et
  d'âge. Les déciles d'âge apparaissent sous forme de disques noirs.
\begin{center}
\includegraphics{./figs/dev4_sbp}  
\end{center}
Quelle commande doit-on utiliser pour estimer le coefficient de corrélation
linéaire de Bravais-Pearson et un intervalle de confiance à 90~\% ?

\begin{description}
\item[A.] \verb|confint(cor(d$age, d$SBP))|
\item[B.] \verb|cor.test(d$age, d$SBP)|
\item[C.] \verb|with(d, cor.test(age, SBP, conf.level=0.90))|
\item[D.] \verb|with(d, cor.test(age, SBP, conf.level=90))|
\item[E.] Je ne sais pas.
\end{description}  
\end{description}

\section*{Exercice 2}
Cet exercice est basé sur les mêmes données que celles présentées à
l'exercice précédent.

\begin{description}
\item[\bf 2.1] \marginpar{\phantom{text}2.1 $\square$} On souhaite à présent réaliser une
  régression linéaire en considérant la variable \texttt{SBP} comme variable
  réponse et l'âge comme variable prédictrice. On souhaite restreindre
  l'analyse à la classe ethnique majoritaire (qui vaut 1). Quelle commande
  faut-il utiliser ?
    \begin{description}
    \item[A.] \verb|lm(SBP ~ age, data=d[raceth==1,])|
    \item[B.] \verb|lm(SBP ~ age, data=subset(d, raceth=1))|
    \item[C.] \verb|lm(SBP ~ age, data=d, subset=raceth==1)|
    \item[D.] Je ne sais pas.
    \end{description}  
  \item[\bf 2.2] \marginpar{\phantom{text}2.2 $\square$} En considérant que le modèle
    décrit en 2.1 a été stocké dans une variable appelée \texttt{mod}, que
    renverrait la commande \verb|anova(mod)| ?
    \begin{description}
    \item[A.] Le tableau des coefficients de régression avec leur
      intervalles de confiance.
    \item[B.] Le tableau des coefficients de régression sans leur
      intervalles de confiance.
    \item[C.] Le tableau d'analyse de variance correspondant au modèle de
      régression.
    \item[D.] Je ne sais pas.
    \end{description}  
  \item[\bf 2.3] \marginpar{\phantom{text}2.3 $\square$} On souhaite extraire la valeur
    estimée pour la pente de la droite de régression à partir des
    coefficients du modèle. On se propose d'utiliser
    \verb|coef(mod)[j]|. Quelle valeur doit-on spécifier pour \texttt{j} ?
  \begin{description}
  \item[A.] 0
  \item[B.] 1
  \item[C.] 2
  \item[D.] Je ne sais pas.
  \end{description}
\item[\bf 2.4] \marginpar{\phantom{text}2.4 $\square$} En supposant \texttt{j}
  correctement défini en 2.3, pour obtenir un intervalle de confiance à
  95~\% pour l'estimé de la pente de régression, on utilise la commande :
  \begin{description}
  \item[A.] \verb|confint(coef(mod)[j])|
  \item[B.] \verb|confint(mod)[j]|
  \item[C.] \verb|confint(mod)[j,]|
  \item[D.] Je ne sais pas.
  \end{description}  
\item[\bf 2.5] \marginpar{\phantom{text}2.5 $\square$} Quelle hypothèse de la régression
  linéaire la commande graphique suivante permet-elle de vérifier ? (Le
  choix des variables portées sur les horizontal et vertical n'a pas
  d'importance majeure.)
\begin{verbatim}
xyplot(fitted(mod) ~ resid(mod))    
\end{verbatim}
  \begin{description}
  \item[A.] L'indépendance des observations.
  \item[B.] La constance de la variance.
  \item[C.] La normalité des résidus.
  \item[D.] Je ne sais pas.
  \end{description}
\end{description}

\section*{Exercice 3}
Considérons l'étude d'obstétrique décrite à l'exercice~4.3
(p.~\pageref{exo:4.3}). En plus du poids des bébés à la naissance, on
dispose de mesures de leur taille et de leur périmètre crânien. On cherche à
établir un lien entre ces trois variables. Pour rappel, voici un aperçu des
données (6 premières observations) après avoir importé les données sous \R :
\begin{verbatim}
> head(weights)
    ID WEIGHT LENGTH HEADC GENDER EDUCATIO             PARITY
1 L001   3.95   55.5  37.5 Female tertiary 3 or more siblings
2 L003   4.63   57.0  38.5 Female tertiary          Singleton
3 L004   4.75   56.0  38.5   Male   year12         2 siblings
4 L005   3.92   56.0  39.0   Male tertiary        One sibling
5 L006   4.56   55.0  39.5   Male   year10         2 siblings
6 L007   3.64   51.5  34.5 Female tertiary          Singleton
\end{verbatim}
Les variables d'intérêt sont : \texttt{WEIGHT} (poids, en kg),
\texttt{LENGTH} (taille, en cm) et \texttt{HEADC} (périmètre crânien, en
cm). 
\begin{description}
\item[\bf 3.1] \marginpar{\phantom{text}3.1 $\square$} On souhaite afficher les
  covariations entre ces trois mesures sous forme d'une matrice de
  diagrammes de dispersion. Quelle commande doit-on utiliser ?
  \begin{description}
  \item[A.] \verb|xyplot|
  \item[B.] \verb|stripplot|
  \item[C.] \verb|splom|
  \item[D.] \verb|splot|
  \item[E.] Je ne sais pas.
  \end{description}  
\item[\bf 3.2] \marginpar{\phantom{text}3.2 $\square$} Quelle commande permettrait
  d'afficher les identifiants des sujets dont le poids est inférieur
  (strictement) à 3500 g et la taille égale ou supérieure au premier
  quartile des tailles enregistrées chez ces 550 bébés ?
  \begin{description}
  \item[A.] \verb|subset(weights, WEIGHT < 3500 & LENGTH >= quantile(LENGTH, 0.25), ID)|
  \item[B.] \verb|subset(weights, WEIGHT < 3.500 & LENGTH >= quantile(LENGTH, 0.25), ID)|
  \item[C.] \verb|weights$ID[WEIGHT < 3.500 & LENGTH >= quantile(LENGTH, 0.25)]|
  \item[D.] Je ne sais pas.
  \end{description}  
\item[\bf 3.3] \marginpar{\phantom{text}3.3 $\square$} En supposant le résultat précédent
  (liste des identifiants des sujets remplissant les conditions mentionnées
  en 3.2) sauvegardé dans une variable appelée \texttt{idx}, comment
  calculerait-on le périmètre crânien moyen de ces individus ?
\begin{verbatim}
> idx
 [1] L077 L104 L236 W054 W093 W123 W172 W189 W214 W249 W303
550 Levels: L001 L003 L004 L005 L006 L007 L008 L009 L010 L011 L013 L014 ... W323
\end{verbatim}
  \begin{description}
  \item[A.] \verb|mean(weights$HEADC)[idx]|
  \item[B.] \verb|mean(weights$HEADC[weights$ID == idx])|
  \item[C.] \verb|mean(weights$HEADC[weights$ID %in% idx])|
  \item[E.] Je ne sais pas.
  \end{description}  
\item[\bf 3.4] \marginpar{\phantom{text}3.4 $\square$} Voici le résultat des deux
  régressions des variables \texttt{LENGTH} et \texttt{HEADC} sur le poids
  (\texttt{WEIGHT}) des bébés. Quel modèle explique la plus grande part de
  variance ?  
\begin{verbatim}
> summary(lm(WEIGHT ~ HEADC, data=weights))
Coefficients:
            Estimate Std. Error t value Pr(>|t|)    
(Intercept) -6.05987    0.56036  -10.81   <2e-16 ***
HEADC        0.27513    0.01478   18.62   <2e-16 ***
---
Signif. codes:  0 ‘***’ 0.001 ‘**’ 0.01 ‘*’ 0.05 ‘.’ 0.1 ‘ ’ 1 

Residual standard error: 0.4714 on 548 degrees of freedom
Multiple R-squared: 0.3875,Adjusted R-squared: 0.3863 
F-statistic: 346.6 on 1 and 548 DF,  p-value: < 2.2e-16 

> summary(lm(WEIGHT ~ LENGTH, data=weights))
Coefficients:
             Estimate Std. Error t value Pr(>|t|)    
(Intercept) -5.412145   0.411040  -13.17   <2e-16 ***
LENGTH       0.178308   0.007488   23.81   <2e-16 ***
---
Signif. codes:  0 ‘***’ 0.001 ‘**’ 0.01 ‘*’ 0.05 ‘.’ 0.1 ‘ ’ 1 

Residual standard error: 0.4223 on 548 degrees of freedom
Multiple R-squared: 0.5085,Adjusted R-squared: 0.5076 
F-statistic:   567 on 1 and 548 DF,  p-value: < 2.2e-16 
\end{verbatim}
  \begin{description}
  \item[A.] \verb|lm(WEIGHT ~ HEADC, data=weights)|
  \item[B.] \verb|lm(WEIGHT ~ LENGTH, data=weights)|
  \item[C.] Je ne sais pas.
  \end{description}  
\item[\bf 3.5] \marginpar{\phantom{text}3.5 $\square$} Pour calculer le poids attendu d'un
  bébé ayant un périmètre crânien de 37 cm (1\ier quartile des périmètres
  crâniens observés dans l'échantillon), quelle commande doit-on utiliser ?
  \begin{description}
  \item[A.] \verb|predict(lm(WEIGHT ~ HEADC, data=weights), 37)|
  \item[B.] \verb|predict(lm(WEIGHT ~ HEADC, data=weights), HEADC=37)|
  \item[C.] \verb|predict(lm(WEIGHT ~ HEADC, data=weights), data.frame(HEADC=37))| 
  \item[D.] Je ne sais pas.
  \end{description}  
\end{description}

%--------------------------------------------------------------- Chapter 06 --
\chapter{Mesures d'association en épidémiologie et régression logistique}\label{chap:logistic}
% FIXME:
% exo kappa ici ou dans chapitre comparaison deux groupes ?
\Opensolutionfile{solutions}[solutions6]



\section*{Énoncés}
%
% RC TD3 exo 2
%
\begin{exo}\label{exo:6.1}
On étudie l'effet d'un traitement prophylactique d'un macrolide à faibles
doses (Traitement A) sur les épisodes infectieux chez des patients atteints
de mucoviscidose dans un essai randomisé multicentrique contre placebo
(B). Les résultats sont les suivants :
\vskip1em

\begin{tabular}{lccc}
\toprule
& \multicolumn{2}{c}{Infection} & \\
\cmidrule(r){2-3}
& Non & Oui & Total \\
\midrule
Traitement (A) & 157 & 52 & 209 \\
Placebo (B) & 119 & 103 & 222 \\
Total & 276 & 155 & 431 \\
\bottomrule
\end{tabular}
\vskip1em

\begin{description}
\item[(a)] À partir d'un test du $\chi^2$, que peut-on répondre à la
  question : le traitement permet-il de prévenir la survenue d'épisodes
  infectieux (au seuil $\alpha=0.05$) ? Vérifier que les effectifs
  théoriques sont bien tous supérieurs à 5.
\item[(b)] Conclut-on de la même manière à partir de l'intervalle de
  confiance de l'odds-ratio associé à l'effet traitement ?
\item[(c)] On souhaite vérifier s'il existe une disparité du point de vue
  des pourcentages d'épisodes infectieux en fonction du centre. Les données
  par centre sont indiquées dans le tableau ci-après. Conclure à partir d'un
  test du $\chi^2$.

  \begin{table}[!htb] \hskip40pt
  \begin{minipage}[b]{0.33\linewidth}
  \scalebox{0.65}{\begin{tabular}{|l|r|r|r|}
    \multicolumn{1}{c}{} & \multicolumn{2}{c}{Infection} &  \multicolumn{1}{c}{} \\
    \cline{2-4}
    \multicolumn{1}{c|}{} & Non & Oui & Total \\
    \hline
    Traitement (A) & 51 & 8 & 59 \\
    \hline
    Placebo (B) & 47 & 19 & 66 \\
    \hline
    Total & 98 & 27 & 125 \\
    \hline
    \multicolumn{4}{c}{Centre 1}
  \end{tabular}} 
  \end{minipage} \hspace{0.1cm}
  \begin{minipage}[b]{0.3\linewidth}
  \scalebox{0.65}{\begin{tabular}{|l|r|r|r|}
    \multicolumn{1}{c}{} & \multicolumn{2}{c}{Infection} &  \multicolumn{1}{c}{} \\
    \cline{2-4}
    \multicolumn{1}{c|}{} & Non & Oui & Total \\
    \hline
    Traitement (A) & 91 & 35 & 126 \\
    \hline
    Placebo (B) & 61 & 71 & 132 \\
    \hline
    Total & 152 & 106 & 258 \\
    \hline
    \multicolumn{4}{c}{Centre 2}
  \end{tabular}} 
  \end{minipage} \hspace{0.1cm}
  \begin{minipage}[b]{0.3\linewidth}
  \scalebox{0.65}{\begin{tabular}{|l|r|r|r|}
    \multicolumn{1}{c}{} & \multicolumn{2}{c}{Infection} &  \multicolumn{1}{c}{} \\
    \cline{2-4}
    \multicolumn{1}{c|}{} & Non & Oui & Total \\
    \hline
    Traitement (A) & 15 & 9 & 24 \\
    \hline
    Placebo (B) & 11 & 13 & 24 \\
    \hline
    Total & 26 & 22 & 48 \\
    \hline
    \multicolumn{4}{c}{Centre 3}
  \end{tabular}}
  \end{minipage}
  \end{table}
\item[(d)] À partir du tableau précédent, on cherche à vérifier si l'effet
  traitement est indépendent du centre ou non. On se propose de réaliser un
  test de comparaison entre les deux traitements ajustés sur le centre (test
  de Mantel-Haenszel). Indiquer le résultat du test ainsi que la valeur de
  l'odds-ratio ajusté.
\end{description}
\begin{sol}
Dans un premier temps, il faut créer le tableau de données sous \R.
\begin{Schunk}
\begin{Sinput}
> macrolid <- matrix(c(157,119,52,103), nr=2, 
+                    dimnames=list(traitement=c("A","B"), 
+                      infection=c("Non","Oui")))
> macrolid
\end{Sinput}
\begin{Soutput}
          infection
traitement Non Oui
         A 157  52
         B 119 103
\end{Soutput}
\end{Schunk}
\cmd{matrix}

On peut représenter graphiquement les données en considérant les proportions
relatives d'épisodes infectieux selon le type de traitement (52/209 contre
103/222) sous forme de diagrammes en barres :
\begin{Schunk}
\begin{Sinput}
> barchart(prop.table(macrolid, 1), xlab="Fréquence relative", 
+          auto.key=list(space="top", column=2, title="Infection"))
\end{Sinput}
\end{Schunk}
\includegraphics{figs/fig-ex6-1b}
\cmd{barchart}\cmd{prop.table}

Le test du $\chi^2$ est simple à mettre en \oe uvre à l'aide de la commande
\texttt{chisq.test} :
\begin{Schunk}
\begin{Sinput}
> chisq.test(macrolid, correct=FALSE)
\end{Sinput}
\begin{Soutput}
	Pearson's Chi-squared test
data:  macrolid 
X-squared = 21.6401, df = 1, p-value = 3.289e-06
\end{Soutput}
\end{Schunk}
\cmd{chisq.test}
On a désactivé la correction de continuité proposée par défaut par \R avec
l'option \verb|correct=FALSE|. Le résultat du test suggère l'existence d'une
association entre le traitement par macrolides et la survenue d'un épisode
infectieux au seuil usuel de 5~\%. On peut obtenir les effectifs théoriques
à partir de la même commande :
\begin{Schunk}
\begin{Sinput}
> chisq.test(macrolid, correct=FALSE)$expected
\end{Sinput}
\begin{Soutput}
          infection
traitement      Non      Oui
         A 133.8376 75.16241
         B 142.1624 79.83759
\end{Soutput}
\end{Schunk}
\cmd{chisq.test}
En règle générale, \R signalera que l'approximation par la distribution du
$\chi^2$ est incorrecte dans le cas où les effectifs théoriques sont trop
petits. 

L'odds-ratio peut se calculer manuellement, en tenant compte de l'ordre de
présentation des modalités des variables dans le tableau (il faut
intervertir l'ordre des colonnes) :
\begin{Schunk}
\begin{Sinput}
> (52*119)/(103*157)
\end{Sinput}
\begin{Soutput}
[1] 0.3826603
\end{Soutput}
\begin{Sinput}
> macrolid.bis <- matrix(nc=2, nr=2)
> macrolid.bis[,1:2] <- macrolid[,2:1]  # échange colonne 1 et 2
> or <- (macrolid.bis[1,1]*macrolid.bis[2,2]) / (macrolid.bis[2,1]*macrolid.bis[1,2])
> or
\end{Sinput}
\begin{Soutput}
[1] 0.3826603
\end{Soutput}
\end{Schunk}
\cmd{matrix}
de même que son intervalle de confiance
\begin{Schunk}
\begin{Sinput}
> se <- sqrt(sum(1/macrolid.bis))  # erreur standard
> or * exp(qnorm(0.025)*se)        # borne inf. 95 %
\end{Sinput}
\begin{Soutput}
[1] 0.2540087
\end{Soutput}
\begin{Sinput}
> or * exp(qnorm(0.975)*se)        # borne sup. 95 %
\end{Sinput}
\begin{Soutput}
[1] 0.5764721
\end{Soutput}
\end{Schunk}
\cmd{sqrt}\cmd{sum}\cmd{exp}\cmd{qnorm}
Mais en règle générale on préfèrera utiliser la commande \texttt{oddsratio}
du package \texttt{vcd}.
\begin{Schunk}
\begin{Sinput}
> library(vcd)
> oddsratio(macrolid.bis, log=FALSE)
\end{Sinput}
\begin{Soutput}
[1] 0.3826603
\end{Soutput}
\begin{Sinput}
> confint(oddsratio(macrolid.bis, log=FALSE))
\end{Sinput}
\begin{Soutput}
           lwr       upr
[1,] 0.2543493 0.5757001
\end{Soutput}
\end{Schunk}
\cmd{library}\cmd{oddsratio}\cmd{confint}
On concluerait donc de la même manière qu'avec le test $\chi^2$ puisque
l'intervalle de confiance à 95~\% pour l'odds-ratio ne contient pas la
valeur 1 (signifiant l'absence d'association entre les deux variables
étudiées). On notera qu'il existe également une commande relativement
pratique pour les analyses épidémiologiques dans le package \texttt{epiR} ;
en ce qui concerne le calcul de l'odds-ratio, il faudra tout de même prendre
garde à la manière dont le tableau de contingence est arrangé. Pour utiliser
la commande \texttt{epi.2by2} de ce package, il faudra échanger les colonnes
du tableau \texttt{macrolid}, soit
\begin{Schunk}
\begin{Sinput}
> library(epiR)
> epi.2by2(macrolid[,c(2,1)])
\end{Sinput}
\begin{Soutput}
             Disease +    Disease -      Total        Inc risk *        Odds
Exposed +           52          157        209              24.9       0.331
Exposed -          103          119        222              46.4       0.866
Total              155          276        431              36.0       0.562
Point estimates and 95 % CIs:
---------------------------------------------------------
Inc risk ratio                         0.54 (0.41, 0.71)
Odds ratio                             0.38 (0.25, 0.59)
Attrib risk *                          -21.52 (-30.31, -12.72)
Attrib risk in population *            -10.43 (-18.41, -2.46)
Attrib fraction in exposed (%)         -86.48 (-145.44, -41.68)
Attrib fraction in population (%)      -29.01 (-42.07, -17.16)
---------------------------------------------------------
 * Cases per 100 population units 
\end{Soutput}
\end{Schunk}

Pour vérifier une éventuelle disparité entre les centres, il nous faut dans
un premier temps calculer les pourcentages d'épisodes infectieux dans chaque
centre. On peut le faire manuellement à partir des tableaux données dans
l'énoncé (par exemple, pour le centre 1 il s'agit de 27/125), auquel cas il
nous suffirait de créer un tableau avec les totaux colonnes :
\begin{Schunk}
\begin{Sinput}
> macrolid.centre <- matrix(c(98,27,152,106,26,22), nr=3, byrow=TRUE)
> rownames(macrolid.centre) <- paste("Centre", 1:3, sep=":")
> colnames(macrolid.centre) <- c("Non","Oui")
> macrolid.centre
\end{Sinput}
\begin{Soutput}
         Non Oui
Centre:1  98  27
Centre:2 152 106
Centre:3  26  22
\end{Soutput}
\end{Schunk}
\cmd{matrix}\cmd{rownames}\cmd{colnames}\cmd{paste}
Le test du $\chi^2$ est ensuite facile à réaliser :
\begin{Schunk}
\begin{Sinput}
> chisq.test(macrolid.centre)
\end{Sinput}
\begin{Soutput}
	Pearson's Chi-squared test
data:  macrolid.centre 
X-squared = 16.1673, df = 2, p-value = 0.0003085
\end{Soutput}
\end{Schunk}
\cmd{chisq.test}
Mais comme à la question suivante on aura besoin de l'ensemble des données,
on peut également décider de travailler avec les trois tableaux. Dans ce
cas, pour répondre à la question de l'hétérogénéité entre centre du point de
vue du critère principal, il nous faudra recalculer les effectifs marginaux.
\begin{Schunk}
\begin{Sinput}
> tab1 <- matrix(c(8,19,51,47), nr=2)
> tab2 <- matrix(c(35,71,91,61), nr=2)
> tab3 <- matrix(c(9,13,15,11), nr=2)
> colnames(tab1) <- colnames(tab2) <- colnames(tab3) <- c("Oui","Non")
> rownames(tab1) <-rownames(tab2) <- rownames(tab3) <- c("A","B")
\end{Sinput}
\end{Schunk}
\cmd{matrix}\cmd{colnames}\cmd{rownames}
Une fois les trois tableaux saisis, on calcule et on assemble les marges ainsi :
\begin{Schunk}
\begin{Sinput}
> macrolid.centre <- rbind(centre1=apply(tab1, 2, sum), 
+                          centre2=apply(tab2, 2, sum), 
+                          centre3=apply(tab3, 2, sum))
> chisq.test(macrolid.centre)
\end{Sinput}
\end{Schunk}
\cmd{rbind}\cmd{chisq.test}
La dernière commande produit bien évidemment le même résultat que celui
affiché précédemment.

Enfin, pour réaliser un test de Mantel-Haenszel, il faut réarranger les
données de sorte que chacun des trois tableaux construits à l'étape
précédent soit bien interprété par \R comme désignant une strate. Pour cela,
on utilise la commande \texttt{array} qui permet de généraliser la commande
\texttt{matrix} à des tableaux à plus de deux dimensions.
\begin{Schunk}
\begin{Sinput}
> macrolid2 <- array(c(tab1, tab2, tab3), dim=c(2,2,3))
> dimnames(macrolid2) <- list(c("A","B"), c("Oui","Non"), paste("Centre", 1:3, sep=":"))
> macrolid2
\end{Sinput}
\begin{Soutput}
, , Centre:1
  Oui Non
A   8  51
B  19  47

, , Centre:2

  Oui Non
A  35  91
B  71  61

, , Centre:3

  Oui Non
A   9  15
B  13  11
\end{Soutput}
\begin{Sinput}
> mantelhaen.test(macrolid2)
\end{Sinput}
\begin{Soutput}
	Mantel-Haenszel chi-squared test with continuity correction
data:  macrolid2 
Mantel-Haenszel X-squared = 22.0405, df = 1, p-value = 2.67e-06
alternative hypothesis: true common odds ratio is not equal to 1 
95 percent confidence interval:
 0.2379264 0.5510569 
sample estimates:
common odds ratio 
        0.3620925 
\end{Soutput}
\end{Schunk}
\cmd{array}\cmd{dimnames}\cmd{mantelhaen.test}
La valeur de l'odds-ratio ajusté figure à la fin du résultat produit par \R
: dans ce cas, il est estimé à 0.362, avec un intervalle de confiance à
95~\% de [0.238–0.551]. Le résultat du test indique bien entendu que
l'odds-ratio ajusté peut être considéré comme significativement différent de
1 (absence d'association).

Une représentation graphique des trois odds-ratio avec leurs intervalles de
confiance à 95~\% s'obtient à l'aide de la commande 
\begin{Schunk}
\begin{Sinput}
> library(vcd)
> plot(oddsratio(macrolid2, log=FALSE))
\end{Sinput}
\end{Schunk}
\includegraphics{figs/fig-ex1-6mbis}
%% Le graphique suivant montre les trois odds-ratio et leurs intervalles de
%% confiance à 95~\%.
%% <<ex1-6m, fig=TRUE>>=
%% tab1.or <- oddsratio(tab1, log=FALSE)
%% tab2.or <- oddsratio(tab2, log=FALSE)
%% tab3.or <- oddsratio(tab3, log=FALSE)
%% dfrm <- data.frame(centre=factor(1:3), or=c(tab1.or, tab2.or, tab3.or), 
%%                    rbind(confint(tab1.or), confint(tab2.or), confint(tab3.or)))
%% dfrm
%% # -- %< -----
%% prepanel.ci <- function(x, y, lx, ux, subscripts, ...) {
%%   x <- as.numeric(x)
%%   lx <- as.numeric(lx[subscripts])
%%   ux <- as.numeric(ux[subscripts])
%%   list(ylim = range(x, ux, lx, finite=TRUE))
%% }
%% panel.ci <- function(x, y, lx, ux, subscripts, pch=16, ...) {
%%   x <- as.numeric(x)
%%   y <- as.numeric(y)
%%   lx <- as.numeric(lx[subscripts])
%%   ux <- as.numeric(ux[subscripts])
%%   panel.abline(h = 1, col = "black", lty = 2)
%%   panel.abline(v = unique(x), col = "grey")
%%   panel.arrows(x, lx, x, ux, col = 'black',
%%                length = 0.05, unit = "native",
%%                angle = 90, code = 3)
%%   panel.xyplot(x, y, pch = pch, ...)
%% }   
%% # -- %< -----
%% xyplot(or ~ centre, data=dfrm, pch=20, ylab="Odds-ratio (IC 95 %)", 
%%        lx=dfrm$lwr, ux=dfrm$upr, prepanel=prepanel.ci, panel=panel.ci)
%% @
%% \cmd{xyplot}\cmd{oddsratio}\cmd{data.frame}
\end{sol}
\end{exo}
%
% Pepe 2004 p. 22
%
\begin{exo}\label{exo:6.2}
Voici les résultats d'une étude de cohorte visant à déterminer, entre
autres, l'intérêt d'utliser comme outil de screening une mesure de test
d'effort physique (EST), pour lequel un résultat de type réussite/échec peut
être dérivé, lors du diagnostic d'une maladie coronarienne
(CAD).\autocite{pepe04}  
\vskip1em

\begin{tabular}{l|l|c|c|c}
\multicolumn{2}{c}{}&\multicolumn{2}{c}{CAD}&\\
\cline{3-4}
\multicolumn{2}{c|}{}&Non-malade&Malade&\multicolumn{1}{c}{Total}\\
\cline{2-4}
\multirow{2}{*}{EST}& Négatif & 327 & 208 & 535\\
\cline{2-4}
& Positif & 115 & 815 & 930\\
\cline{2-4}
\multicolumn{1}{c}{} & \multicolumn{1}{c}{Total} & \multicolumn{1}{c}{442} & 
\multicolumn{1}{c}{1023} & \multicolumn{1}{c}{1465}\\
\end{tabular}
\vskip1em
On fera l'hypothèse qu'il n'y a pas de biais de vérification.

\begin{description}
\item[(a)] À partir de cette matrice de confusion, indiquer les valeurs
  suivantes (avec intervalles de confiance à 95~\%) : sensibilité et
  spécificité, valeur prédictive positive et négative. 
\item[(b)] Quelle est la valeur de l'aire sous la courbe pour les données
  reportées ?
\end{description}
\begin{sol}
La création du tableau de données peut se faire comme indiqué ci-après, à
partir d'un tableau de type \texttt{matrix} :
\begin{Schunk}
\begin{Sinput}
> tab <- as.table(matrix(c(815,115,208,327), nrow=2, byrow=TRUE, 
+                        dimnames=list(EST=c("+","-"), CAD=c("+","-"))))
> tab
\end{Sinput}
\begin{Soutput}
   CAD
EST   +   -
  + 815 115
  - 208 327
\end{Soutput}
\end{Schunk}
\cmd{matrix}
On notera que le tableau a été légèrement réorganisé de manière à
correspondre aux notations habituelles, avec les événements \og positifs\fg\
sur la première ligne (typiquement, l'exposition) et la première colonne
(typiquement, la maladie) du tableau.

Il serait tout à fait possible de calculer en très peu d'opérations
arithmétiques avec \R toutes les quantités demandées. Cependant, le package
\texttt{epiR} contient toutes les commandes nécessaires pour répondre aux
questions épidémiologiques sur les tableaux $2\times 2$. Ainsi, la commande
\texttt{epi.tests} fournit les valeurs de prévalence,
sensibilité/spécificité, et valeurs prédictive positive et négative.
\begin{Schunk}
\begin{Sinput}
> library(epiR)
> epi.tests(tab)
\end{Sinput}
\begin{Soutput}
          Disease +    Disease -      Total
Test +          815          115        930
Test -          208          327        535
Total          1023          442       1465
Point estimates and 95 % CIs:
---------------------------------------------------------
Apparent prevalence                    0.63 (0.61, 0.66)
True prevalence                        0.7 (0.67, 0.72)
Sensitivity                            0.8 (0.77, 0.82)
Specificity                            0.74 (0.7, 0.78)
Positive predictive value              0.88 (0.85, 0.9)
Negative predictive value              0.61 (0.57, 0.65)
---------------------------------------------------------
\end{Soutput}
\end{Schunk}
Concernant l'aire sous la courbe, on peut également utiliser une commande
externe, \texttt{roc.from.table}, dans le package \texttt{epicalc}.
\begin{Schunk}
\begin{Sinput}
> library(epicalc)
> roc.from.table(tab, graph=FALSE)
\end{Sinput}
\begin{Soutput}
$auc
[1] 0.7682477
$original.table
   CAD
EST Non-diseased Diseased
  +          815      115
  -          208      327

$diagnostic.table
    1-Specificity Sensitivity
        1.0000000    1.000000
> +     0.2033236    0.739819
> -     0.0000000    0.000000
\end{Soutput}
\end{Schunk}
\cmd{library}\cmd{roc.from.table}
\end{sol}
\end{exo}
%
% RC TD8 diagnostic cutoff
%
\begin{exo}\label{exo:6.3}
On dispose de données issues d'une étude cherchant à établir la validité
pronostique de la concentration en créatine kinase dans l'organisme sur la
prévention de la survenue d'un infarctus du myocarde.\autocite[p.~115]{rabe-hesketh04}

Les données sont disponibles dans le fichier \texttt{sck.dat} : la première
colonne correspond à la variable créatine kinase (\texttt{ck}), la deuxième
à la variable présence de la maladie (\texttt{pres}) et la dernière à la
variable absence de maladie (\texttt{abs}).
\begin{description}
\item[(a)] Quel est le nombre total de sujets ?
\item[(b)] Calculer les fréquences relatives malades/non-malades, et
  représenter leur évolution en fonction des valeurs de créatine kinase à
  l'aide d'un diagramme de dispersion (points + segments reliant les points).
\item[(c)] À partir d'un modèle de régression logistique dans lequel on
  cherche à prédire la probabilité d'être malade, calculer la valeur de
  \texttt{ck} à partir de laquelle ce modèle prédit que les personnes
  présentent la maladie en considérant une valeur seuil de 0.5
  (si $P(\text{malade})\ge 0.5$ alors \texttt{malade=1}).
\item[(d)] Représenter graphiquement les probabilités d'être malade prédites
  par ce modèle ainsi que les proportions empiriques en fonction des valeurs
  \texttt{ck}. 
\item[(e)] Établir la courbe ROC correspondante, et reporter la
  valeur de l'aire sous la courbe. 
%\item[(f)] Quelle est la valeur de seuil optimisant le compromis
%  sensibilité/spécificité ?
\end{description}
\begin{sol}
Comme le nom des variables ne figure pas dans le fichier de données, il
faudra les attribuer aussitôt après avoir importé les données.
\begin{Schunk}
\begin{Sinput}
> sck <- read.table("data/sck.dat", header=FALSE)
> names(sck) <- c("ck", "pres", "abs")
> summary(sck)
\end{Sinput}
\begin{Soutput}
       ck           pres            abs    
 Min.   :  0   Min.   : 2.00   Min.   : 0  
 1st Qu.:120   1st Qu.:13.00   1st Qu.: 0  
 Median :240   Median :18.00   Median : 1  
 Mean   :240   Mean   :17.69   Mean   :10  
 3rd Qu.:360   3rd Qu.:21.00   3rd Qu.: 5  
 Max.   :480   Max.   :35.00   Max.   :88  
\end{Soutput}
\end{Schunk}
\cmd{read.table}\cmd{names}\cmd{summary}

Le nombre total de sujets correspond à la somme des effectifs pour les deux
variables \texttt{pres} et \texttt{abs}, soit
\begin{Schunk}
\begin{Sinput}
> sum(sck[,c("pres","abs")])
\end{Sinput}
\begin{Soutput}
[1] 360
\end{Soutput}
\end{Schunk}
\cmd{sum}
ou de manière équivalente : \verb|sum(sck$pres) + sum(sck$abs)| (mais pas
\verb|sck$pres + sck$abs| !).

Pour calculer les fréquences relatives de ces deux variables, il est
nécessaire de connaître les effectifs totaux par variable. Ceux-ci peuvent
être obtenu en utilisant la commande \texttt{apply} et en opérant par
colonnes :
\begin{Schunk}
\begin{Sinput}
> ni <- apply(sck[,c("pres","abs")], 2, sum)
\end{Sinput}
\end{Schunk}
\cmd{apply}
À partir de là, il suffit de diviser chaque valeur des variables
\texttt{pres} et \texttt{abs} par les nombres calculés ci-dessus. On
stockera les valeurs obtenues dans deux nouvelles variables, dans le même
tableau de données.
\begin{Schunk}
\begin{Sinput}
> sck$pres.prop <- sck$pres/ni[1]
> sck$abs.prop <- sck$abs/ni[2]
\end{Sinput}
\end{Schunk}
On peut vérifier que les calculs sont corrects : la somme des valeurs pour
chaque variable doit maintenant valoir 1 :
\begin{Schunk}
\begin{Sinput}
> apply(sck[,c("pres.prop","abs.prop")], 2, sum)
\end{Sinput}
\begin{Soutput}
pres.prop  abs.prop 
        1         1 
\end{Soutput}
\end{Schunk}
\cmd{apply}
Il suffit ensuite de représenter les proportions obtenues dans un même
graphique, en considérant les valeurs de la variable \texttt{ck} comme
abscisses.
\begin{Schunk}
\begin{Sinput}
> xyplot(pres.prop + abs.prop ~ ck, data=sck, type=c("b", "g"),
+        auto.key=TRUE, ylab="Fréquence")
\end{Sinput}
\end{Schunk}
\includegraphics{figs/fig-ex6-3f}
\cmd{xyplot}

L'instruction \verb|type=c("b", "g")| signifie que l'on souhaite afficher
des points reliés par des lignes (\verb|"b"|=\verb|"o"|+\verb|"l"|)
ainsi qu'un quadrillage (\verb|"g"|).

Le modèle de régression pour données groupées
\begin{Schunk}
\begin{Sinput}
> glm.res <- glm(cbind(pres, abs) ~ ck, data=sck, family=binomial)
> summary(glm.res)
\end{Sinput}
\begin{Soutput}
Call:
glm(formula = cbind(pres, abs) ~ ck, family = binomial, data = sck)
Deviance Residuals: 
     Min        1Q    Median        3Q       Max  
-2.79579  -1.34637   0.00587   0.07173   2.26860  

Coefficients:
             Estimate Std. Error z value Pr(>|z|)
(Intercept) -2.326272   0.299354  -7.771 7.79e-15
ck           0.035104   0.004081   8.602  < 2e-16

(Dispersion parameter for binomial family taken to be 1)

    Null deviance: 311.29  on 12  degrees of freedom
Residual deviance:  28.14  on 11  degrees of freedom
AIC: 51.596

Number of Fisher Scoring iterations: 6
\end{Soutput}
\end{Schunk}
\cmd{glm}\cmd{summary.glm}

Les prédictions, exprimées sous forme de probabilités et non sur l'échelle
log-odds, sont obtenues à l'aide de la commande \texttt{predict} en
précisant l'option \verb|type="response"| comme ci-dessous :
\begin{Schunk}
\begin{Sinput}
> glm.pred <- predict(glm.res, type="response")
> names(glm.pred) <- sck$ck
\end{Sinput}
\end{Schunk}
\cmd{predict}\cmd{names}
En considérant que des probabilités $\ge 0.5$ signifie que les individus
sont malades, on obtient donc la répartition suivante :
\begin{Schunk}
\begin{Sinput}
> glm.pred[glm.pred >= 0.5]
\end{Sinput}
\begin{Soutput}
       80       120       160       200       240       280       320       360       400       440 
0.6182388 0.8683280 0.9640991 0.9909384 0.9977594 0.9994489 0.9998646 0.9999667 0.9999918 0.9999980 
      480 
0.9999995 
\end{Soutput}
\end{Schunk}
On en conclut que les personnes seront considérées malades, selon ce modèle,
pour des valeurs \texttt{ck} de 80 ou plus.

Cela se vérifie aisément sur un graphique dans lequel on reporte les
probabilités prédites en fonction des valeurs de la variable \texttt{ck}.
\begin{Schunk}
\begin{Sinput}
> sck$malade <- sck$pres/(sck$pres+sck$abs)
> xyplot(glm.pred ~ sck$ck, type="l", 
+        ylab="Probabilité", xlab="ck", 
+        panel=function(...) {
+          panel.xyplot(...)
+          panel.xyplot(sck$ck, sck$malade, pch=19, col="grey")
+        })
\end{Sinput}
\end{Schunk}
\includegraphics{figs/fig-ex6-3j}
\cmd{xyplot}

Dans un premier temps, il faut "décompacter" les données groupées et crée un
tableau avec deux colonnes : la première représentant la variable
\texttt{ck} et la seconde représentant la présence ou absence de maladie. On
utilisera les effectifs par sous-groupe calculé précédemment et disponible
dans la variable \texttt{ni}.
% FIXME:
% voir epitools::expand.table
\begin{Schunk}
\begin{Sinput}
> sck.expand <- data.frame(ck=c(rep(sck$ck, sck$pres), rep(sck$ck, sck$abs)), 
+                          malade=c(rep(1, ni[1]), rep(0, ni[2])))
> table(sck.expand$malade)
\end{Sinput}
\begin{Soutput}
  0   1 
130 230 
\end{Soutput}
\begin{Sinput}
> with(sck.expand, tapply(malade, ck, sum))
\end{Sinput}
\begin{Soutput}
  0  40  80 120 160 200 240 280 320 360 400 440 480 
  2  13  30  30  21  19  18  13  19  15   7   8  35 
\end{Soutput}
\end{Schunk}
\cmd{data.frame}\cmd{table}\cmd{tapply}
Les deux dernières commandes visent à s'assurer que l'on retombe bien sur
les mêmes effectifs et que la distribution des personnes malades par valeur
de \texttt{ck} est correcte. On peut également vérifier que l'on obtient
bien les mêmes résultats concernant la régression logistique, et en profiter
pour ajouter les valeurs prédites au tableau de données précédent (on
pourrait procéder comme précédemment et répliquer les prédictions, mais cela
est plus simple ainsi)
\begin{Schunk}
\begin{Sinput}
> glm.res2 <- glm(malade ~ ck, data=sck.expand, family=binomial)
> sck.expand$prediction <- ifelse(predict(glm.res2, type="response") >= 0.5, 1, 0)
> with(sck.expand, table(malade, prediction))
\end{Sinput}
\begin{Soutput}
      prediction
malade   0   1
     0 114  16
     1  15 215
\end{Soutput}
\end{Schunk}
\cmd{glm}\cmd{ifelse}\cmd{predict}\cmd{with}\cmd{table}
La dernière commande permet d'afficher une matrice de confusion dans
laquelle on croise les diagnostiques réels et ceux prédits par le modèle de
régression. On peut ainsi comparer les taux de classification correcte
lorsque l'on varie le seuil de référence :
\begin{Schunk}
\begin{Sinput}
> classif.tab <- with(sck.expand, table(malade, prediction))
> sum(diag(classif.tab))/sum(classif.tab)
\end{Sinput}
\begin{Soutput}
[1] 0.9138889
\end{Soutput}
\end{Schunk}
\cmd{with}\cmd{sum}\cmd{diag}\cmd{table}

Pour afficher la courbe ROC, on utilisera le package \texttt{ROCR}.
\begin{Schunk}
\begin{Sinput}
> library(ROCR)
> pred <- prediction(predict(glm.res2, type="response"), sck.expand$malade)
> perf <- performance(pred, "tpr","fpr")
> plot(perf, ylab="Sensibilité", xlab="1-Spécificité")
> grid()
> abline(0, 1, lty=2)
\end{Sinput}
\end{Schunk}
\includegraphics{figs/fig-ex6-3n}
\cmd{library}\cmd{prediction}\cmd{performance}\cmd{plot}\cmd{grid}\cmd{abline}

La valeur de l'aire sous la courbe est obtenue comme suit :
\begin{Schunk}
\begin{Sinput}
> performance(pred, "auc")@"y.values"
\end{Sinput}
\begin{Soutput}
[[1]]
[1] 0.9592642
\end{Soutput}
\end{Schunk}

% FIXME:
% for previous exercice
% Une autre solution consiste à utiliser la commande \texttt{lroc} dans le
% package \texttt{epicalc}, mais celle-ci offre moins de
% facilités. L'avantage, en revanche, est que cette commande permet de
% travailler directement avec des matrices de confusion avec des modèles de
% régression logistique estimés à partir de données groupées.
% <<ex6-3p, eval=FALSE, fig=FALSE>>=
% library(epicalc)
% lroc(glm.res, line.col="black", grid.col="grey70", auc.coords=c(0.4,0.2))
% @
\end{sol}
\end{exo}
% 
% Everitt 2001 p. 208
%
\begin{exo}\label{exo:6.4}
Le tableau suivant résume la proportion d'infarctus du myocarde observée
chez des hommes âgés de 40 à 59 ans et pour lesquels on a relevé le niveau
de tension artérielle et le taux de cholesterol, considérées sous forme de
classes ordonnées.
\vskip1em

\begin{tabular}{l///////}
\toprule
& \multicolumn{7}{c}{Cholesterol (mg/100 ml)} \\
\cmidrule(r){2-8}
TA & \multicolumn{1}{c}{$<200$} & \multicolumn{1}{c}{$200-209$} & \multicolumn{1}{c}{$210-219$} & \multicolumn{1}{c}{$220-244$} & \multicolumn{1}{c}{$245-259$} & \multicolumn{1}{c}{$260-284$} & \multicolumn{1}{c}{$>284$} \\
$<117$ & 2/53 & 0/21 & 0/15 & 0/20 & 0/14 & 1/22 & 0/11 \\
$117-126$ & 0/66 & 2/27 & 1/25 & 8/69 & 0/24 & 5/22 & 1/19 \\
$127-136$ & 2/59 & 0/34 & 2/21 & 2/83 & 0/33 & 2/26 & 4/28 \\
$137-146$ & 1/65 & 0/19 & 0/26 & 6/81 & 3/23 & 2/34 & 4/23 \\
$147-156$ & 2/37 & 0/16 & 0/6 & 3/29 & 2/19 & 4/16 & 1/16 \\
$157-166$ & 1/13 & 0/10 & 0/11 & 1/15 & 0/11 & 2/13 & 4/12 \\
$167-186$ & 3/21 & 0/5 & 0/11 & 2/27 & 2/5 & 6/16 & 3/14 \\
$>186$ & 1/5 & 0/1 & 3/6 & 1/10 & 1/7 & 1/7 & 1/7 \\
\bottomrule
\end{tabular}
\vskip1em

Les données sont disponibles dans le fichier \texttt{hdis.dat} sous forme
d'un tableau comprenant 4 colonnes indiquant, respectivement, la pression
artérielle (8 catégories, notées 1 à 8), le taux de cholesterol (7
catégories, notées 1 à 7), le nombre d'infarctus et le nombre total
d'individus. On s'intéresse à l'association entre la pression artérielle et
la probabilité d'avoir un infarctus du myocarde.
\begin{description}
\item[(a)] Calculer les proportions d'infarctus pour chaque niveau de
  pression artérielle et les représenter dans un tableau et sous forme
  graphique.
\item[(b)] Exprimer les proportions calculées en (a) sous forme de
  \emph{logit}. 
\item[(c)] À partir d'un modèle de régression logistique, déterminer s'il
  existe une association significative au seuil $\alpha=0.05$ entre la
  pression artérielle, traitée en tant que variable quantitative en
  considérant les centres de classe, et la probabilité d'avoir un
  infarctus.
\item[(d)] Exprimer en unités \emph{logit} les probabilités d'infarctus
  prédites par le modèle pour chacun des niveaux de pression artérielle.
\item[(e)] Afficher sur un même graphique les proportions empiriques et la
  courbe de régression logistique en fonction des valeurs de pression
  artérielle (centres de classe).
\end{description}
% FIXME:
% Ajouter question sur prédiction pour une certaine valeur
% (niveau observé, et niveau non observé)
\begin{sol}
L'importation des données \texttt{hdis.dat} se fait comme suit :
\begin{Schunk}
\begin{Sinput}
> bp <- read.table("data/hdis.dat", header=TRUE)
> str(bp)
\end{Sinput}
\begin{Soutput}
'data.frame':	56 obs. of  4 variables:
 $ bpress: int  1 1 1 1 1 1 1 2 2 2 ...
 $ chol  : int  1 2 3 4 5 6 7 1 2 3 ...
 $ hdis  : int  2 0 0 0 0 1 0 0 2 1 ...
 $ total : int  53 21 15 20 14 22 11 66 27 25 ...
\end{Soutput}
\end{Schunk}
\cmd{read.table}\cmd{str}
Comme on peut le constater, aucun label n'est associé aux modalités de
variables d'intérêt (\texttt{bpress} pour la pression artérielle,
\texttt{chol} pour le taux de cholesterol). Pour générer et associer les
labels, on peut utiliser les commandes suivantes :
\begin{Schunk}
\begin{Sinput}
> blab <- c("<117","117-126","127-136","137-146",
+           "147-156","157-166","167-186",">186")
> clab <- c("<200","200-209","210-219","220-244",
+           "245-259","260-284",">284")
> bp$bpress <- factor(bp$bpress, labels=blab)
> bp$chol <- factor(bp$chol, labels=clab)
\end{Sinput}
\end{Schunk}
\cmd{factor}
La dernière commande convertit les variables d'origine en variables
qualitatives et associe à leurs modalités les labels définis par
\texttt{blab} et \texttt{clab}. Pour vérifier que la base de données est
bien sous la forme désirée, les commandes suivantes sont toujours utiles :
\begin{Schunk}
\begin{Sinput}
> str(bp)
\end{Sinput}
\begin{Soutput}
'data.frame':	56 obs. of  4 variables:
 $ bpress: Factor w/ 8 levels "<117","117-126",..: 1 1 1 1 1 1 1 2 2 2 ...
 $ chol  : Factor w/ 7 levels "<200","200-209",..: 1 2 3 4 5 6 7 1 2 3 ...
 $ hdis  : int  2 0 0 0 0 1 0 0 2 1 ...
 $ total : int  53 21 15 20 14 22 11 66 27 25 ...
\end{Soutput}
\begin{Sinput}
> summary(bp)
\end{Sinput}
\begin{Soutput}
     bpress        chol        hdis           total      
 <117   : 7   <200   :8   Min.   :0.000   Min.   : 1.00  
 117-126: 7   200-209:8   1st Qu.:0.000   1st Qu.:11.00  
 127-136: 7   210-219:8   Median :1.000   Median :19.00  
 137-146: 7   220-244:8   Mean   :1.643   Mean   :23.73  
 147-156: 7   245-259:8   3rd Qu.:2.000   3rd Qu.:27.00  
 157-166: 7   260-284:8   Max.   :8.000   Max.   :83.00  
 (Other):14   >284   :8                                  
\end{Soutput}
\end{Schunk}
\cmd{str}\cmd{summary}

On peut maintenant reproduire le tableau de fréquences relatives fourni dans
l'énoncé :
\begin{Schunk}
\begin{Sinput}
> round(xtabs(hdis/total ~ bpress + chol, data=bp), 2)
\end{Sinput}
\begin{Soutput}
         chol
bpress    <200 200-209 210-219 220-244 245-259 260-284 >284
  <117    0.04    0.00    0.00    0.00    0.00    0.05 0.00
  117-126 0.00    0.07    0.04    0.12    0.00    0.23 0.05
  127-136 0.03    0.00    0.10    0.02    0.00    0.08 0.14
  137-146 0.02    0.00    0.00    0.07    0.13    0.06 0.17
  147-156 0.05    0.00    0.00    0.10    0.11    0.25 0.06
  157-166 0.08    0.00    0.00    0.07    0.00    0.15 0.33
  167-186 0.14    0.00    0.00    0.07    0.40    0.38 0.21
  >186    0.20    0.00    0.50    0.10    0.14    0.14 0.14
\end{Soutput}
\end{Schunk}
\cmd{round}\cmd{xtabs}

Puisque l'on ne va s'intéresser qu'à la relation entre pression artérielle
et infarctus, il est nécessaire d'aggréger les données sur le taux de
cholestérol. En d'autres termes, il est nécessaire de cumuler les effectifs
pour chaque niveau de pression artérielle, tous niveaux de cholesterol
confondus. On en profitera également pour renommer les niveaux de la
variable \texttt{bpress} en utilisant les centres des intervalles de classe.
\begin{Schunk}
\begin{Sinput}
> blab2 <- c(111.5,121.5,131.5,141.5,151.5,161.5,176.5,191.5)
> # levels(bp$bpress) <- blab2
> # bp$bpress <- as.numeric(as.character(bp$bpress))
> bp$bpress <- rep(blab2, each=7)
> dfrm <- aggregate(bp[,c("hdis","total")], list(bpress=bp[,"bpress"]), sum)
\end{Sinput}
\end{Schunk}
\cmd{levels}\cmd{aggregate}\cmd{sum}
C'est la dernière commande, \texttt{aggregate}, qui permet d'aggréger les
données : on additionne tous les effectifs (\texttt{sum}) de la variable
\texttt{chol} qui ne figure pas dans la liste des variables que l'on
souhaite conserver pour l'analyse. On en profite pour stocker les résultats
dans une nouvelle base de données nommée \texttt{dfrm}. Un aperçu des
données aggrégées est fourni ci-après :
\begin{Schunk}
\begin{Sinput}
> head(dfrm, 5)
\end{Sinput}
\begin{Soutput}
  bpress hdis total
1  111.5    3   156
2  121.5   17   252
3  131.5   12   284
4  141.5   16   271
5  151.5   12   139
\end{Soutput}
\end{Schunk}
\cmd{head}

Lorsque l'on dispose d'une proportion, $p$, sa valeur sur une échelle dont
les unités sont des \emph{logit} est donnée par la relation
$\log(p/(1-p))$. D'où les commandes suivantes pour convertir les
proportions, calculées comme \verb|hdis/total| en unités logit :
\begin{Schunk}
\begin{Sinput}
> logit <- function(x) log(x/(1-x))
> dfrm$prop <- dfrm$hdis/dfrm$total
> dfrm$logit <- logit(dfrm$hdis/dfrm$total)
\end{Sinput}
\end{Schunk}
\cmd{function}\cmd{log}
On remarquera que l'on a défini une petite fonction permettant de convertir
des valeurs \texttt{x}, qui ici sont supposées être des proportions, en leur
équivalent $\log(x/(1-x))$. On aurait pu écrire de manière équivalente :
\begin{Schunk}
\begin{Sinput}
> log((dfrm$hdis/dfrm$total)/(1-dfrm$hdis/dfrm$total))
\end{Sinput}
\end{Schunk}
\cmd{log}
Le résultat de ces calculs est reproduit ci-après.
\begin{Schunk}
\begin{Sinput}
> dfrm
\end{Sinput}
\begin{Soutput}
  bpress hdis total       prop     logit
1  111.5    3   156 0.01923077 -3.931826
2  121.5   17   252 0.06746032 -2.626372
3  131.5   12   284 0.04225352 -3.120895
4  141.5   16   271 0.05904059 -2.768675
5  151.5   12   139 0.08633094 -2.359280
6  161.5    8    85 0.09411765 -2.264364
7  176.5   16    99 0.16161616 -1.646252
8  191.5    8    43 0.18604651 -1.475907
\end{Soutput}
\end{Schunk}

Le modèle de régression logistique s'écrit de la manière suivante :
\begin{Schunk}
\begin{Sinput}
> glm(cbind(hdis, total-hdis) ~ bpress, data=dfrm, family=binomial)
\end{Sinput}
\end{Schunk}
\cmd{glm}
La formulation utilisée, \verb|cbind(hdis, total-hdis) ~ bpress|, tient
compte du fait que nous disposons de données groupées, et non des réponses
individuelles. La commande \texttt{glm} avec l'option \verb|family=binomial|
correspond à une régression logistique, qui, sans vouloir entrer trop dans
les détails, utilise par défaut la fonction \texttt{logit} comme lien
canonique. 
% On préférera toutefois utiliser la commande équivalente
% \texttt{lrm} du package \texttt{rms}, car celle-ci fournit en règle
% générale beaucoup plus d'informations sur la qualité du modèle, en plus des
% informations que l'on pourrait obtenir à partir de la commande
% \texttt{glm}. La formulation du modèle reste identique :
% <<ex6-4k>>=
% library(rms)
% lrm(cbind(hdis, total-hdis) ~ bpress, data=dfrm)
% @ 
On obtient donc les résultats suivants :
\begin{Schunk}
\begin{Sinput}
> summary(glm(cbind(hdis, total-hdis) ~ bpress, data=dfrm, family=binomial))
\end{Sinput}
\begin{Soutput}
Call:
glm(formula = cbind(hdis, total - hdis) ~ bpress, family = binomial, 
    data = dfrm)
Deviance Residuals: 
    Min       1Q   Median       3Q      Max  
-1.0617  -0.5977  -0.2245   0.2140   1.8501  

Coefficients:
             Estimate Std. Error z value Pr(>|z|)
(Intercept) -6.082033   0.724320  -8.397  < 2e-16
bpress       0.024338   0.004843   5.025 5.03e-07

(Dispersion parameter for binomial family taken to be 1)

    Null deviance: 30.0226  on 7  degrees of freedom
Residual deviance:  5.9092  on 6  degrees of freedom
AIC: 42.61

Number of Fisher Scoring iterations: 4
\end{Soutput}
\end{Schunk}
\cmd{glm}\cmd{summary.glm}
Le résultat précédent comprend les informations essentielles pour répondre à
la question de la significativité statistique de l'association entre
pression artérielle et probabilité d'un infarctus : le coefficient de
régression (sur l'échelle log-odds) vaut 0.024 et est significatif au seuil
usuel de 5~\% (cf. colonne \verb+Pr(>|z|)+).

La probabilité d'avoir un infarctus selon les différents niveaux de pression
artérielle considérés est obtenue comme suit :
\begin{Schunk}
\begin{Sinput}
> glm.res <- glm(cbind(hdis, total-hdis) ~ bpress, data=dfrm, family=binomial)
> predict(glm.res)
\end{Sinput}
\begin{Soutput}
        1         2         3         4         5         6         7         8 
-3.368319 -3.124937 -2.881554 -2.638172 -2.394789 -2.151407 -1.786333 -1.421260 
\end{Soutput}
\end{Schunk}
\cmd{glm}\cmd{predict}
Notons que l'on a stocké les résultats intermédiaires générés par \R sous le
nom \texttt{glm.res} avant d'utiliser la commande \texttt{predict}. Les
prédictions générées par \R sont exprimées sous forme de \emph{logit}, et
l'on peut comparer les logit observés et prédits.
\begin{Schunk}
\begin{Sinput}
> cbind(dfrm, logit.predit=predict(glm.res))
\end{Sinput}
\begin{Soutput}
  bpress hdis total       prop     logit logit.predit
1  111.5    3   156 0.01923077 -3.931826    -3.368319
2  121.5   17   252 0.06746032 -2.626372    -3.124937
3  131.5   12   284 0.04225352 -3.120895    -2.881554
4  141.5   16   271 0.05904059 -2.768675    -2.638172
5  151.5   12   139 0.08633094 -2.359280    -2.394789
6  161.5    8    85 0.09411765 -2.264364    -2.151407
7  176.5   16    99 0.16161616 -1.646252    -1.786333
8  191.5    8    43 0.18604651 -1.475907    -1.421260
\end{Soutput}
\end{Schunk}
\cmd{cbind}\cmd{predict}

Pour représenter graphiquement les proportions d'infarctus observées et
prédites en fonction du niveau de pression artérielle, on a pratiquement
tous les éléments. Il nous manque les prédictions du modèle exprimées sous
forme de proportions, et non sur l'échelle log-odds. Ensuite, on peut
vouloir tracer la courbe de prédiction, c'est-à-dire la probabilité d'avoir
un infarctus en fonction de la pression artérielle, sans limiter cette
dernière aux 8 valeurs observées pour la variable \texttt{bpress}. Voici une
solution possible :
\begin{Schunk}
\begin{Sinput}
> dfrm$prop.predit <- predict(glm.res, type="response") 
> f <- function(x) 1/(1+exp(-(coef(glm.res)[1]+coef(glm.res)[2]*x)))
> xyplot(hdis/total ~ bpress, data=dfrm, aspect=1.2, cex=.8,
+        xlab="Pression artérielle", ylab="Probabilité infarctus",
+        panel=function(x, y, ...) {
+          panel.xyplot(x, y, col="gray30", pch=19, ...)
+          panel.curve(f, lty=3, col="gray70")
+          panel.points(x, dfrm$prop.predit, col="gray70", ...)
+        })
\end{Sinput}
\end{Schunk}
\includegraphics{figs/fig-ex6-4n}
\cmd{predict}\cmd{function}\cmd{exp}\cmd{coef}\cmd{xyplot}
\end{sol}
\end{exo}
%
% Dupont 2009 p. 138
%
\begin{exo}\label{exo:6.5}
Une enquête cas-témoin a porté sur la relation entre la consommation
d'alcool et de tabac et le cancer de l'oesophage chez l'homme (étude "Ille
et Villaine"). Le groupe des cas était composé de 200 patients atteints d'un
cancer de l'oesophage et diagnostiqué entre janvier 1972 et avril 1974. Au
total, 775 témoins de sexe masculin ont été sélectionnés à partir des listes
électorales. Le tableau suivant indique la répartition de l'ensemble des
sujets selon leur consommation journalière d'alcool, en considérant qu'une
consommation supérieure à 80 g est considérée comme un facteur de
risque.\autocite{breslow80} 
\vskip1em

\begin{tabular}{lccc}
\toprule
& \multicolumn{2}{c}{Consommation d'alcool (g/jour)} & \\
\cmidrule(r){2-3}
& $\ge 80$ & $<80$ & Total \\
\midrule
Cas & 96 & 104 & 200 \\
Témoins & 109 & 666 & 775 \\
Total & 205 & 770 & 975 \\
\bottomrule
\end{tabular}
\vskip1em

\begin{description}
%\item[(a)] Quelle est la proportion de personnes considérées à risque dans
%  cet échantillon ?
\item[(a)] Quelle est la valeur de l'odds-ratio et son intervalle de
  confiance à 95~\% (méthode de Woolf) ? Est-ce une bonne estimation du
  risque relatif ? 
\item[(b)] Est-ce que la proportion de consommateurs à risque est la même
  chez les cas et chez les témoins (considérer $\alpha=0.05$) ?
\item[(c)] Construire le modèle de régression logistique permettant de
  tester l'association entre la consommation d'alcool et le statut des
  sujets. Le coefficient de régression est-il significatif ?
\item[(d)] Retrouvez la valeur de l'odds-ratio observé, calculé en (b), et
  son intervalle de confiance à partir des résultats de l'analyse de
  régression.
\end{description}
\begin{sol}
Comme les données brutes (individuelles) ne sont pas disponibles, il faut
travailler directement avec le tableau d'effectifs fourni dans l'énoncé. 
\begin{Schunk}
\begin{Sinput}
> alcool <- matrix(c(666,104,109,96), nr=2, dimnames=list(c("Témoin","Cas"), c("<80",">=80")))
> alcool
\end{Sinput}
\begin{Soutput}
       <80 >=80
Témoin 666  109
Cas    104   96
\end{Soutput}
\end{Schunk}
\cmd{matrix}

Concernant l'odds-ratio, on utilisera la commande \texttt{oddsratio} du
package \texttt{vcd} :
\begin{Schunk}
\begin{Sinput}
> library(vcd)
> oddsratio(alcool, log=FALSE)
\end{Sinput}
\begin{Soutput}
[1] 5.640085
\end{Soutput}
\end{Schunk}
\cmd{oddsratio}
% FIXME:
% Introduire epicalc?
% cc(NULL, NULL, cctable=make2x2(96,109,104,666))
L'option \texttt{log=FALSE} nous assure que le résultat renvoyé correspond
bien à un odds-ratio et non au log odds-ratio. Pour obtenir un intervalle de
confiance asymptotique, on utilise 
\begin{Schunk}
\begin{Sinput}
> confint(oddsratio(alcool, log=FALSE))
\end{Sinput}
\begin{Soutput}
          lwr      upr
[1,] 4.003793 7.945104
\end{Soutput}
\end{Schunk}
\cmd{confint}\cmd{oddsratio}
De même, on pourrait utiliser \verb|summary(oddsratio(alcool))| pour
effectuer un test d'hypothèse sur le log odds-ratio
($H_0:\,\log(\hat\theta)=0$). 

Pour tester l'hypothèse que la proportion de personnes avec une consommation
journalière est $\ge 80$ g est identique chez les cas et les témoins, on
peut utiliser la commande \texttt{prop.test} en indiquant les effectifs
observés à partir du tableau croisé donnée dans l'énoncé.
\begin{Schunk}
\begin{Sinput}
> prop.test(c(96,109), c(200,775), correct=FALSE)
\end{Sinput}
\begin{Soutput}
	2-sample test for equality of proportions without continuity correction
data:  c(96, 109) out of c(200, 775) 
X-squared = 110.2554, df = 1, p-value < 2.2e-16
alternative hypothesis: two.sided 
95 percent confidence interval:
 0.2659162 0.4127935 
sample estimates:
   prop 1    prop 2 
0.4800000 0.1406452 
\end{Soutput}
\end{Schunk}
\cmd{prop.test}
Ce test est exactement équivalent à un test $Z$ pour tester la différence
entre deux proportions estimées à partir des données (si l'on n'utilise pas
de correction de continuité).

Pour le modèle de régression logistique, on a besoin de transformer le
tableau de contingence en un tableau de données où l'on fait clairement
apparaître les deux variables qualitatives (maladie et exposition),
c'est-à-dire un \texttt{data.frame}. Cela peut être réalisé en utilisant une
commande qui permet de transformer les données sous forme tabulaire en
format "long" : la commande \texttt{melt} du package \texttt{reshape}. On en
profitera pour recoder les niveaux de la variable maladie
en 0 (témoins) et 1 (cas), même si ce n'est pas vraiment nécessaire, et
considérer le niveau 0 comme la catégorie de référence (ce qui facilite
l'interprétation des résultats).
\begin{Schunk}
\begin{Sinput}
> library(reshape)
> alcool.df <- melt(alcool)
> names(alcool.df) <- c("maladie", "exposition", "n")
> levels(alcool.df$maladie) <- c(1,0)
> alcool.df$maladie <- relevel(alcool.df$maladie, "0")
\end{Sinput}
\end{Schunk}
\cmd{library}\cmd{melt}\cmd{names}\cmd{levels}\cmd{relevel}

Le modèle de régression logistique
\begin{Schunk}
\begin{Sinput}
> glm.res <- glm(maladie ~ exposition, data=alcool.df, family=binomial, weights=n)
> summary(glm.res)
\end{Sinput}
\begin{Soutput}
Call:
glm(formula = maladie ~ exposition, family = binomial, data = alcool.df, 
    weights = n)
Deviance Residuals: 
     1       2       3       4  
-13.90   20.41  -11.73   12.07  

Coefficients:
               Estimate Std. Error z value Pr(>|z|)
(Intercept)     -1.8569     0.1054 -17.612   <2e-16
exposition>=80   1.7299     0.1752   9.872   <2e-16

(Dispersion parameter for binomial family taken to be 1)

    Null deviance: 989.49  on 3  degrees of freedom
Residual deviance: 893.06  on 2  degrees of freedom
AIC: 897.06

Number of Fisher Scoring iterations: 5
\end{Soutput}
\end{Schunk}
\cmd{glm}\cmd{summary.glm}
Le résultat qui nous intéresse est la ligne associée à
\verb|exposition>=80|, puisqu'elle nous renseigne sur la valeur du
coefficient de régression associé à l'exposition et estimé par \R, avec son
erreur standard, ainsi que la valeur de la statistique de test. Ici, le
coefficient de régression s'interprète comme le log de l'odds-ratio.  Notons
que l'on obtiendrait exactement les mêmes résultats en intervertissant le
rôle des variables dans la formulation précédente, \verb|exposition ~ maladie|.

On peut retrouver l'odds-ratio calculé plus haut à partir du coefficient de
régression associé au facteur d'intérêt (\texttt{exposition}), ainsi que son
intervalle de confiance à 95~\% :
\begin{Schunk}
\begin{Sinput}
> exp(coef(glm.res)[2])
\end{Sinput}
\begin{Soutput}
exposition>=80 
      5.640085 
\end{Soutput}
\begin{Sinput}
> exp(confint(glm.res)[2,])
\end{Sinput}
\begin{Soutput}
   2.5 %   97.5 % 
4.004808 7.964896 
\end{Soutput}
\end{Schunk}
\cmd{exp}\cmd{coef}\cmd{confint}

Une deuxième solution consiste à considérer le nombre de cas et le nombre
total d'individus, comme dans l'exercice~\ref{exo:6.4}.
\begin{Schunk}
\begin{Sinput}
> alcool2 <- data.frame(expos=c("<80",">=80"), cas=c(104,96), total=c(104+666, 96+109))
> summary(glm(cbind(cas, total-cas) ~ expos, data=alcool2, family=binomial))
\end{Sinput}
\begin{Soutput}
Call:
glm(formula = cbind(cas, total - cas) ~ expos, family = binomial, 
    data = alcool2)
Deviance Residuals: 
[1]  0  0

Coefficients:
            Estimate Std. Error z value Pr(>|z|)
(Intercept)  -1.8569     0.1054 -17.612   <2e-16
expos>=80     1.7299     0.1752   9.872   <2e-16

(Dispersion parameter for binomial family taken to be 1)

    Null deviance:  9.6433e+01  on 1  degrees of freedom
Residual deviance: -2.2204e-14  on 0  degrees of freedom
AIC: 16.112

Number of Fisher Scoring iterations: 3
\end{Soutput}
\end{Schunk}
\cmd{data.frame}\cmd{glm}\cmd{summary.glm}

Voici enfin une troisième manière de procéder, toujours avec la même
structure de données.
\begin{Schunk}
\begin{Sinput}
> summary(glm(cas/total ~ expos, data=alcool2, family=binomial, weights=total))
\end{Sinput}
\end{Schunk}
\cmd{summary.glm}\cmd{glm}
Notons que le nombre total de sujets selon l'exposition n'est pas vraiment
utile dans ce dernier cas (on aurait pu utiliser directement le nombre de
témoins). 

% FIXME:
% Autre solution avec rms::lrm
% d <- datadist(alcool.df)
% options(datadist="d")
% lrm(X1 ~ X2, data=alcool.df, weights=value)
% summary(lrm(X1 ~ X2, data=alcool.df, weights=value))
\end{sol}
\end{exo}
\Closesolutionfile{solutions}

\chapter*{Devoir \no 5}
\addcontentsline{toc}{chapter}{Devoir \no 5}

Les exercices sont indépendants. Une seule réponse est correcte pour chaque
question. Lorsque vous ne savez pas répondre, cochez la case correspondante.

\section*{Exercice 1}
Les analyses proposées se basent sur une étude sur les facteurs de risque
d'accidents coronariens \citep{rosenman64}. Les données cas-témoin sont
résumées dans le tableau de contingence suivant : les cas sont définis comme
les personnes ayant eu un accident coronarien durant les 10 années de suivi,
tandis que le facteur d'exposition est l'opacisation de la cornée qui est
supposée être reliée au taux de cholestérol ; sa présence conduit à
considérer la personne comme exposée.
\vskip1em

\begin{tabular}{l|cc|r}
& Exposé & Non-exposé & total \\
\hline
Cas & 102 & 153 & 255 \\
Témoin & 839 & 2058 & 2897 \\
\hline
total & 941 & 2211 & 3152
\end{tabular}
\vskip1em

L'objet de l'analyse est de vérifier si l'obscurcissement de la cornée constitue
effectivement un facteur prédictif du risque de développer un infarctus.
\begin{description}
\item[\bf 1.1] \marginpar{\phantom{text}1.1 $\square$} En notant \texttt{M+/M-} les cas et
  les témoins, et \texttt{E+/E-} les personnes exposées/non-exposées, quelle
  solution peut-on proposer pour stocker ce type de données sous \R ?
  \begin{description}
  \item[A.] \verb|matrix(c(102,153,839,2058), nr=2, byrow=TRUE, dimnames=list(c("M-","M+"),c("E-","E+")))|
  \item[B.] \verb|matrix(c(102,153,839,2058), nr=2, byrow=TRUE, dimnames=list(c("M+","M-"),c("E+","E-")))|
  \item[C.] \verb|matrix(c(102,153,839,2058), nr=2, dimnames=list(c("M+","M-"),c("E+","E-")))|
  \item[D.] \verb|matrix(c(102,153,839,2058), nr=2, dimnames=list(c("M-","M+"),c("E-","E+")))|
  \item[E.] Je ne sais pas.
  \end{description}
\item[\bf 1.2] \marginpar{\phantom{text}1.2 $\square$} Supposons que le tableau de données
  ait été appelé \texttt{tab} sous \R. On souhaite calculer l'odds-ratio à
  l'aide de la commande \texttt{oddsratio} du package \texttt{vcd}. Quelle
  est la commande à utiliser ?
  \begin{description}
  \item[A.] \verb|oddsratio(tab, log=FALSE)|
  \item[B.] \verb|oddsratio(tab)|
  \item[C.] \verb|oddsratio(tab[c(2,1),c(1,2)], log=FALSE)|
  \item[D.] \verb|oddsratio(tab[c(2,1),c(1,2)])|
  \item[E.] Je ne sais pas.
  \end{description}
\item[\bf 1.3] \marginpar{\phantom{text}1.3 $\square$} Pour calculer l'intervalle de
  confiance à 95~\% associé à l'odds-ratio estimé en 1.2 que l'on a
  enregistré dans la variable \texttt{tab.or}, on utilise la commande :
  \begin{description}
  \item[A.] \verb|confint(tab.or)|
  \item[B.] \verb|summary(tab.or, conf.level=0.95)|
  \item[C.] Je ne sais pas.
  \end{description}  
\item[\bf 1.4] \marginpar{\phantom{text}1.4 $\square$} Peut-on retrouver la valeur de
  l'odds-ratio à partir d'une autre commande ? Si oui, laquelle ?
  \begin{description}
  \item[A.] \verb|fisher.test|
  \item[B.] \verb|binom.test|
  \item[C.] \verb|tabstat|
  \item[D.] Ce n'est pas possible.
  \item[E.] Je ne sais pas.
  \end{description}  
\item[\bf 1.5] \marginpar{\phantom{text}1.5 $\square$} Quelle commande utiliserait-on pour
  représenter graphiquement la fréquence relative des cas (uniquement) selon
  le facteur d'exposition ?
  \begin{description}
  \item[A.] \verb|barchart(table(tab["M+",])/sum(tab))|
  \item[B.] \verb|barchart(table(tab[2,])/sum(tab))|
  \item[C.] \verb|barchart(prop.table(tab[2,]))|
  \item[D.] \verb|barchart(prop.table(tab["M+",]))|
  \item[E.] Je ne sais pas.
  \end{description}  
\item[\bf 1.6] \marginpar{\phantom{text}1.6 $\square$} Que renvoit la commande suivante ? 
\begin{verbatim}
(tab[1,1]/sum(tab[,1])) / (tab[1,2]/sum(tab[,2]))
\end{verbatim}
  \begin{description}
  \item[A.] Le risque absolu chez les exposés.
  \item[B.] Le risque absolu chez les non-exposés.      
  \item[C.] Le risque relatif.
  \item[D.] Je ne sais pas.
  \end{description}
\item[\bf 1.7] \marginpar{\phantom{text}1.7 $\square$} Quelle commande du package
  \texttt{epiR} permet de fournir une estimation par intervalle (95~\%) de
  la fraction de risque attribuable (au facteur d'exposition) ?
  \begin{description}
  \item[A.] \verb|epi.2by2|
  \item[B.] \verb|epi.tests|
  \item[C.] Je ne sais pas.
  \end{description}  
\end{description}

\section*{Exercice 2}
Cet exercice est basé sur les mêmes données que celles présentées à
l'exercice précédent.

On souhaite modéliser la relation entre la probabilité d'être défini comme
un cas en fonction de l'exposition préalable à l'aide d'une régression
logistique.
  
\begin{description}
\item[\bf 2.1] \marginpar{\phantom{text}2.1 $\square$} On se propose dans un premier temps
  de transformer le tableau de contingence sous forme d'un
  \texttt{data.frame} comportant trois colonnes décrivant le statut relatif
  à la maladie (M+/M-) et à l'exposition (E+/E-) et les effectifs
  associés. La commande suivante est-elle correcte ? (On supposera que le
  package \texttt{reshape} a été chargé au préalable.)
\begin{verbatim}
tab <- melt(tab, varnames=c("maladie","exposition"))
\end{verbatim}
  \begin{description}
  \item[A.] Oui.
  \item[B.] Non.
  \item[C.] Je ne sais pas.
  \end{description}
\item[\bf 2.2] \marginpar{\phantom{text}2.2 $\square$} En supposant que les données sont à
  présent correctement spécifiées telles que décrit en 2.1, soit
\begin{verbatim}
> tab
  maladie exposition value
1      M+         E+   102
2      M-         E+   839
3      M+         E-   153
4      M-         E-  2058
\end{verbatim}
  la commande permettant d'établir le modèle de régression désiré et de le
  stocker dans une variable appelée \texttt{m} s'écrit :
  \begin{description}
  \item[A.] \verb|m <- glm(maladie ~ exposition, data=tab)|
  \item[B.] \verb|m <- glm(maladie ~ exposition, data=tab, weights=value)|
  \item[C.] \verb|m <- glm(maladie ~ exposition, data=tab, family=binomial, weights=value)|
  \item[D.] Je ne sais pas.
  \end{description}
\item[\bf 2.3] \marginpar{\phantom{text}2.3 $\square$} Pour afficher le résultat des tests
  de nullité des coefficients du modèle \texttt{m} (ordonnée à l'origine et
  pente), on utilise la commande :
  \begin{description}
  \item[A.] \verb|summary|
  \item[B.] \verb|anova|
  \item[C.] Je ne sais pas.
  \end{description}  
\item[\bf 2.4] \marginpar{\phantom{text}2.4 $\square$} On se propose de retrouver la
  valeur de l'odds-ratio calculé à l'exercice~1.2 à partir du modèle
  précédent. Quelle combinaison de commandes est nécessaire pour réaliser
  cette opération ?
  \begin{description}
  \item[A.] \verb|coef| et \verb|log|
  \item[B.] \verb|coef| et \verb|exp|
  \item[C.] \verb|summary| et \verb|exp|
  \item[D.] Je ne sais pas.
  \end{description}
\item[\bf 2.5] \marginpar{\phantom{text}2.5 $\square$} On souhaite calculer la probabilité
  d'être un cas sachant que l'on a été exposé. Quelle commande permet de
  répondre à cette question ?
  \begin{description}
  \item[A.] \verb|predict(m, data.frame(exposition=1), type="link")|
  \item[B.] \verb|predict(m, data.frame(exposition="E+"), type="link")|
  \item[C.] \verb|predict(m, data.frame(exposition=1), type="response")|
  \item[D.] \verb|predict(m, data.frame(exposition="E+"), type="response")|
  \item[E.] Je ne sais pas.
  \end{description}  
\end{description}

\section*{Exercice 3}
Dans une étude diagnostique portant sur 150 patients, on a cherché à
vérifier si un examen radiologique (rayons X) du patient permettait de
prédire correctement un risque de fracture, sachant qu'une opération
chirurgicale intervenant après l'évaluation permettait de confirmer la
présence ou l'absence de fracture chez chaque patient
\citep[p.~280]{peat05}. L'opération chirurgicale tient lieu de "gold
standard" et l'examen radiologique de test de screening. Le tableau suivant
résume les données recueillies à l'issue de l'étude.
\vskip1em

\begin{tabular}{clllll}
\toprule
& & & \multicolumn{2}{c}{Fracture détectée} & \\
\cmidrule(r){4-5}
& & & Présente & Absente & Total \\
\midrule
Résultats Rayons X & Positif & N  & 36 & 24 & 60 \\
(test)             &         & \% & 60.0 & 40.0 & 100.0 \\
                   & Négatif & N  & 8 & 82 & 90 \\
                   &         & \% & 8.9 & 91.1 & 100.0 \\
Total              &         & N  & 44 & 106 & 150 \\
                   &         & \% & 29.3 & 70.7 & 100.0 \\
\bottomrule
\end{tabular}
\vskip1em

On suppose par la suite que le tableau 2x2 (c'est-à-dire sans les marges)
est stocké exactement de la même manière sous \R (résultats du test en
lignes, résultats du diagnostic final en colonnes), et se nomme
\texttt{tab}.
\begin{description}
\item[\bf 3.1] \marginpar{\phantom{text}3.1 $\square$} La commande 
\begin{verbatim}
tab[2,2]/sum(tab[2,])
\end{verbatim}
  fournit une estimation de la valeur prédictive positive ? 
  \begin{description}
  \item[A.] Vrai.
  \item[B.] Faux.
  \item[C.] Je ne sais pas.
  \end{description}  
\item[\bf 3.2] \marginpar{\phantom{text}3.2 $\square$} Soit la valeur de VPN=0.91 et la
  commande suivante :
\begin{verbatim}
0.91 + 1.96 * sqrt(0.91*0.09/150)
\end{verbatim}
  Que produit cette commande ? 
  \begin{description}
  \item[A.] La valeur de l'erreur standard associée à la VPN en utilisant
    une approximation par la loi normale.
  \item[B.] La borne supérieure de l'intervalle de confiance à 90~\% de la
    VPN en utilisant une approximation par la loi normale.
  \item[C.] La borne supérieure de l'intervalle de confiance à 95~\% de la
    VPN en utilisant une approximation par la loi normale.
  \item[D.] Je ne sais pas.
  \end{description}
\item[\bf 3.3] \marginpar{\phantom{text}3.3 $\square$} Peut-on obtenir un résultat
  comparable à partir d'une autre commande \R ?
  \begin{description}
  \item[A.] \verb|fisher.test|
  \item[B.] \verb|prop.test|
  \item[C.] Je ne sais pas.
  \end{description}
\item[\bf 3.4] \marginpar{\phantom{text}3.4 $\square$} Voici le contenu partiel d'une
  sortie produite par \R :
\begin{verbatim}
Point estimates and 95 % CIs:
---------------------------------------------------------
Apparent prevalence                    0.4 (0.32, 0.48)
True prevalence                        0.29 (0.22, 0.37)
Sensitivity                            0.82 (0.67, 0.92)
Specificity                            0.77 (0.68, 0.85)
Positive predictive value              0.6 (0.47, 0.72)
Negative predictive value              0.91 (0.83, 0.96)
---------------------------------------------------------
\end{verbatim}
  Quelle commande du package \texttt{epiR} a permis de produire un tel
  résultat ? 
  \begin{description}
  \item[A.] \verb|epi.2by2|
  \item[B.] \verb|epi.tests|
  \item[C.] Je ne sais pas.
  \end{description}
\end{description}

%--------------------------------------------------------------- Chapter 07 --
\chapter{Données de survie}\label{chap:survival}
\Opensolutionfile{solutions}[solutions7]


\section*{Énoncés}
%
% 
%
\begin{exo}\label{exo:7.1}
Dans un essai contre placebo sur la cirrhose biliaire, la D-penicillamine
(DPCA) a été introduite dans le bras actif sur une cohorte de 312
patients. Au total, 154 patients ont été randomisés dans le bras actif
(variable traitement, \texttt{rx}, 1=Placebo, 2=DPCA). Un ensemble de
données telles que l'âge, des données biologiques et signes cliniques variés
incluant le niveau de bilirubine sérique (\texttt{bilirub}) sont disponibles
dans le fichier \texttt{pbc.txt}.\autocite{vittinghoff05} Le status du
patient est enregistré dans la variable \texttt{status} (0=vivant, 1=décédé)
et la durée de suivi (\texttt{years}) représente le temps écoulé en années
depuis la date de diagnostic.
\begin{description}
\item[(a)] Combien dénombre-t-on d'individus décédés? Quelle proportion de
  ces décès retrouve-t-on dans le bras actif ?  
\item[(b)] Afficher la distribution des durées de suivi des 312 patients, en
  faisant apparaître distinctement les individus décédés. Calculer le temps
  médian (en années) de suivi pour chacun des deux groupes de
  traitement. Combien y'a-t-il d'événements positifs au-delà de 10.5 années
  et quel est le sexe de ces patients ?
\item[(c)] Les 19 patients dont le numéro (\texttt{number}) figure parmi la
  liste suivante ont subi une transplantation durant la période de suivi.
\begin{verbatim}  
5 105 111 120 125 158 183 241 246 247 254 263 264 265 274 288 291
295 297 345 361 362 375 380 383
\end{verbatim}   
  Indiquer leur âge moyen, la distribution selon le sexe et la durée médiane
  de suivi en jours jusqu'à la transplantation.
\item[(d)] Afficher un tableau résumant la distribution des événements à
  risque en fonction du temps, avec la valeur de survie associée.
\item[(e)] Afficher la courbe de Kaplan-Meier avec un intervalle de
  confiance à 95~\%, sans considérer le type de traitement.
\item[(f)] Calculer la médiane de survie et son intervalle de confiance à
  95~\% pour chaque groupe de sujets et afficher les courbes de survie
  correspondantes.
\item[(g)] Effectuer un test du log-rank en considérant comme prédicteur le
  facteur \texttt{rx}. Comparer avec un test de Wilcoxon.
\item[(h)] Effectuer un test du log-rank sur le facteur d'intérêt
  (\texttt{rx}) en stratifiant sur l'âge. On considèrera trois groupe
  d'âge : 40 ans ou moins, entre 40 et 55 ans inclus, plus de 55 ans.
\item[(i)] Retrouver les résultats de l'exercice 1.g avec une régression de
  Cox. 
\end{description}
\begin{sol}
Le chargement des données ne pose pas de difficultés :
\begin{Schunk}
\begin{Sinput}
> pb <- read.table("data/pbc.txt", header=TRUE)
> names(pb)[1:20]
\end{Sinput}
\begin{Soutput}
 [1] "number"   "status"   "rx"       "sex"      "asictes"  "hepatom"  "spiders"  "edema"   
 [9] "bilirub"  "cholest"  "albumin"  "copper"   "alkphos"  "sgot"     "trigli"   "platel"  
[17] "prothrom" "histol"   "age"      "years"   
\end{Soutput}
\end{Schunk}
Pour faciliter la lecture des résultats, recodons d'emblée les variables
traitement (\texttt{rx}) et sexe (\texttt{sex}) comme variables qualitatives
: 
\begin{Schunk}
\begin{Sinput}
> pb$rx <- factor(pb$rx, labels=c("Placebo", "DPCA"))
> table(pb$rx)
\end{Sinput}
\begin{Soutput}
Placebo    DPCA 
    158     154 
\end{Soutput}
\begin{Sinput}
> pb$sex <- factor(pb$sex, labels=c("M","F"))
\end{Sinput}
\end{Schunk}
Le nombre de patients décédés peut être obtenu directement à partir d'un
tableau d'effectif sur la variable \texttt{status} :
\begin{Schunk}
\begin{Sinput}
> prop.table(table(pb$status))
\end{Sinput}
\begin{Soutput}
       0        1 
0.599359 0.400641 
\end{Soutput}
\end{Schunk}
La proportion de décès par groupe de traitement s'obtient en croisant les
deux variables \texttt{status} et \texttt{rx} et en calculant les fréquences
relatives par ligne :
\begin{Schunk}
\begin{Sinput}
> prop.table(with(pb, table(status, rx)), 1)
\end{Sinput}
\begin{Soutput}
      rx
status   Placebo      DPCA
     0 0.4973262 0.5026738
     1 0.5200000 0.4800000
\end{Soutput}
\end{Schunk}
Pour afficher la distribution des temps de suivi, on peut utiliser un simple
diagramme de dispersion dans lequel les coordonnées des points à afficher
sont définis par le numéro d'observation et la durée de suivi depuis le
diagnostique. 
\begin{Schunk}
\begin{Sinput}
> xyplot(number ~ years, data=pb, pch=pb$status, cex=.8)
\end{Sinput}
\end{Schunk}
\includegraphics{figs/fig-ex7-1d}

Le temps médian de suivi par groupe de traitement est obtenu à partir d'une
commande de type \texttt{tapply} en considérant la variable \texttt{rx}
comme variable de classification, soit :
\begin{Schunk}
\begin{Sinput}
> with(pb, tapply(years, rx, median))
\end{Sinput}
\begin{Soutput}
Placebo    DPCA 
5.18820 4.95825 
\end{Soutput}
\end{Schunk}
Le nombre de patients décédés au-delà de 10.5 ans est obtenu comme suit :
\begin{Schunk}
\begin{Sinput}
> with(pb, table(status[years > 10.5]))
\end{Sinput}
\begin{Soutput}
 0  1 
23  4 
\end{Soutput}
\begin{Sinput}
> with(pb, table(sex[years > 10.5 & status == 1]))
\end{Sinput}
\begin{Soutput}
M F 
2 2 
\end{Soutput}
\end{Schunk}
soit 4 patients, dont 2 patients de sexe féminin. Notons que la dernière
commande pourrait également s'écrire 
\begin{Schunk}
\begin{Sinput}
> subset(pb, years > 10.5 & status == 1, sex)
\end{Sinput}
\begin{Soutput}
   sex
24   M
31   F
51   F
66   M
\end{Soutput}
\end{Schunk}
ce qui a l'avantage de fournir les numéros d'observation.

Les patients transplantés figurent normalement parmi les individus vivants à
la date de point. On peut le vérifier aisément à l'aide d'un simple tri à
plat du \texttt{status} de ces patients.
\begin{Schunk}
\begin{Sinput}
> idx <- c(5,105,111,120,125,158,183,241,246,247,254,263,264,265,274,288,291,295,
+          297,345,361,362,375,380,383)
> table(pb$status[pb$number %in% idx])
\end{Sinput}
\begin{Soutput}
 0 
19 
\end{Soutput}
\end{Schunk}
Pour calculer les quantités demandées (âge moyen, sexe, durée de suivi), on
peut dans un premier temps réduire le tableau de données initial à ces seuls
patients :
\begin{Schunk}
\begin{Sinput}
> pb.transp <- subset(pb, number %in% idx, c(age, sex, years))
\end{Sinput}
\end{Schunk}
Ensuite, il s'agit simplement d'appliquer les commandes \texttt{mean},
\texttt{table} et \texttt{median} aux variables sélectionnées pour ces 19
patients.
\begin{Schunk}
\begin{Sinput}
> mean(pb.transp$age)
\end{Sinput}
\begin{Soutput}
[1] 41.17568
\end{Soutput}
\begin{Sinput}
> table(pb.transp$sex)
\end{Sinput}
\begin{Soutput}
 M  F 
 3 16 
\end{Soutput}
\begin{Sinput}
> median(pb.transp$years * 365)
\end{Sinput}
\begin{Soutput}
[1] 1434.012
\end{Soutput}
\end{Schunk}

Pour obtenir un tableau des événements avec la survie associée, il est
nécessaire d'utiliser la commande \texttt{Surv} du package
\texttt{survival}. La commande de base pour recoder des données
événements/temps en données de survie est \texttt{Surv} dont voici un
exemple d'utilisation :
\begin{Schunk}
\begin{Sinput}
> library(survival)
> head(with(pb, Surv(time=years, event=status)))
\end{Sinput}
\begin{Soutput}
[1]  1.0951  12.3203+  2.7707   5.2704   4.1177+  6.8528 
\end{Soutput}
\end{Schunk}
Les données censurées (patients vivant) apparaissent suffixées avec un
\texttt{+}. Un tableau des événements en fonction du temps peut être
construit à l'aide de la commande \texttt{survfit}. On notera que lorsque
l'on ne considère aucun facteur de groupe ou de stratification, il faut
utiliser la syntaxe un peu particulière \verb|Surv(time, event) ~ 1|.
\begin{Schunk}
\begin{Sinput}
> s <- survfit(Surv(years, status) ~ 1, data=pb)
> summary(s)
\end{Sinput}
\end{Schunk}

Une fois que le tableau de survie a été construit, il est très simple
d'afficher l'estimateur de la survie sous la forme d'une courbe de
Kaplan-Meier à l'aide de la commande \texttt{plot}.
\begin{Schunk}
\begin{Sinput}
> plot(s)
\end{Sinput}
\end{Schunk}
\includegraphics{figs/fig-ex7-1m}

La médiane de survie pour chaque bras de traitement peut être obtenue à
l'aide de la commande \texttt{survdiff}. Redéfinissons dans un premier temps
le modèle de survie pour inclure le prédicteur d'intérêt :
\begin{Schunk}
\begin{Sinput}
> s <- survfit(Surv(years, status) ~ rx, data=pb)
\end{Sinput}
\end{Schunk}
Pour obtenir la médiane de survie, il suffit d'afficher le résultat
précédent :
\begin{Schunk}
\begin{Sinput}
> s
\end{Sinput}
\begin{Soutput}
Call: survfit(formula = Surv(years, status) ~ rx, data = pb)
           records n.max n.start events median 0.95LCL 0.95UCL
rx=Placebo     158   158     158     65   8.99    7.07      NA
rx=DPCA        154   154     154     60   9.39    8.46      NA
\end{Soutput}
\end{Schunk}

On utilisera la commande \texttt{plot} pour obtenir les courbes de survie
correspondantes. 
\begin{Schunk}
\begin{Sinput}
> plot(s)
\end{Sinput}
\end{Schunk}
\includegraphics{figs/fig-ex7-1oo}

Le test du log-rank se réalise à partir de la commande \texttt{survdiff},
avec les mêmes notations que dans le cas du calcul de la médiane de survie.
\begin{Schunk}
\begin{Sinput}
> survdiff(Surv(years, status) ~ rx, data=pb)
\end{Sinput}
\begin{Soutput}
Call:
survdiff(formula = Surv(years, status) ~ rx, data = pb)
             N Observed Expected (O-E)^2/E (O-E)^2/V
rx=Placebo 158       65     63.2    0.0502     0.102
rx=DPCA    154       60     61.8    0.0513     0.102

 Chisq= 0.1  on 1 degrees of freedom, p= 0.75 
\end{Soutput}
\end{Schunk}
L'option \texttt{rho} permet de contrôler le type de test réalisé. En
ajoutant \texttt{rho=1} on réalise un test de Wilcoxon au lieu du log-rank
classique :
\begin{Schunk}
\begin{Sinput}
> survdiff(Surv(years, status) ~ rx, data=pb, rho=1)
\end{Sinput}
\begin{Soutput}
Call:
survdiff(formula = Surv(years, status) ~ rx, data = pb, rho = 1)
             N Observed Expected (O-E)^2/E (O-E)^2/V
rx=Placebo 158     49.7     49.0   0.00963    0.0243
rx=DPCA    154     46.8     47.5   0.00993    0.0243

 Chisq= 0  on 1 degrees of freedom, p= 0.876 
\end{Soutput}
\end{Schunk}

Pour réaliser une analyse stratifiée sur l'âge, il est nécessaire de recoder
cette variable numérique en variable qualitative, ce que l'on peut réaliser
à l'aide de la commande \texttt{cut}.
\begin{Schunk}
\begin{Sinput}
> agec <- cut(pb$age, c(26, 40, 55, 79))
\end{Sinput}
\end{Schunk}
On peut effectuer de nouveau le test du log-rank en incluant ce facteur de
stratification additionnel en adaptatnt légèrement le modèle précédent :
\begin{Schunk}
\begin{Sinput}
> survdiff(Surv(years, status) ~ rx + strata(agec), data=pb)
\end{Sinput}
\begin{Soutput}
Call:
survdiff(formula = Surv(years, status) ~ rx + strata(agec), data = pb)
             N Observed Expected (O-E)^2/E (O-E)^2/V
rx=Placebo 158       65       65  1.93e-05   4.1e-05
rx=DPCA    154       60       60  2.09e-05   4.1e-05

 Chisq= 0  on 1 degrees of freedom, p= 0.995 
\end{Soutput}
\end{Schunk}

Enfin, concernant le modèle de Cox, c'est la commande \texttt{coxph} qu'il
faut utiliser.
\begin{Schunk}
\begin{Sinput}
> summary(coxph(Surv(years, status) ~ rx + strata(agec), data=pb))
\end{Sinput}
\begin{Soutput}
Call:
coxph(formula = Surv(years, status) ~ rx + strata(agec), data = pb)
  n= 312, number of events= 125 

            coef exp(coef)  se(coef)     z Pr(>|z|)
rxDPCA 0.0009795 1.0009799 0.1809778 0.005    0.996

       exp(coef) exp(-coef) lower .95 upper .95
rxDPCA     1.001      0.999    0.7021     1.427

Concordance= 0.498  (se = 0.038 )
Rsquare= 0   (max possible= 0.964 )
Likelihood ratio test= 0  on 1 df,   p=0.9957
Wald test            = 0  on 1 df,   p=0.9957
Score (logrank) test = 0  on 1 df,   p=0.9957
\end{Soutput}
\end{Schunk}
\end{sol}
%
% Hill et al. 1996 p. 22
%
\end{exo}
\begin{exo}\label{exo:7.2}
Ci-dessous figurent les durées de rémission en semaines dans un essai
comparant 6-MP à un placebo.\autocite{freireich63} Les observations
censurées (sujet sans rechute aux dernières nouvelles) sont signalées par
une astérisque.
\vskip1em

\begin{tabular}{ll}
\toprule
Placebo & 1 1 2 2 3 4 4 5 5 8 8 8 8 11 11 12 12 \\
        & 15 17 22 23 \\
6-MP    & 6 6 6 6* 7 9* 10 10* 11* 13 16 17* 19* \\
        & 20* 22 23 25* 32* 32* 34* 35* \\
\bottomrule
\end{tabular}
\vskip1em

\begin{description}
\item[(a)] Construire un tableau indiquant le nombre d'exposés et le nombre
  de rechutes en fonction du temps pour le groupe 6-MP.
\item[(b)] Afficher la courbe de survie dans le groupe 6-MP, ainsi que son
  intervalle de confiance à 95~\% (estimateur de Kaplan-Meier).
\item[(c)] Afficher la fonction de risque cumulée (estimateur de Nelson)
  pour le même groupe.
\item[(d)] Calculer la médiane de survie du groupe 6-MP.
\item[(e)] Calculer la survie espérée pour un patient du groupe 6-MP à 15
  semaines. 
\end{description}
\begin{sol}
Dans un premier temps, il est nécessaire de construire un tableau des
données brutes, dans lequel on reportera le traitement, la durée de survie
et la censure (0=vivant/1=décédé).
\begin{Schunk}
\begin{Sinput}
> placebo.time <- c(1,1,2,2,3,4,4,5,5,8,8,8,8,11,11,12,12,15,17,22,23)
> placebo.status <- rep(1, length(placebo.time))
> mp.time <- c(6,6,6,6,7,9,10,10,11,13,16,17,19,20,22,23,25,32,32,34,35)
> mp.status <- c(1,1,1,0,1,0,1,0,0,1,1,0,0,0,1,1,0,0,0,0,0)
> mp <- data.frame(tx=rep(c("Placebo","6-MP"), c(21,21)),
+                  time=c(placebo.time, mp.time), 
+                  status=c(placebo.status, mp.status))
> summary(mp)
\end{Sinput}
\begin{Soutput}
       tx          time           status      
 6-MP   :21   Min.   : 1.00   Min.   :0.0000  
 Placebo:21   1st Qu.: 6.00   1st Qu.:0.0000  
              Median :10.50   Median :1.0000  
              Mean   :12.88   Mean   :0.7143  
              3rd Qu.:18.50   3rd Qu.:1.0000  
              Max.   :35.00   Max.   :1.0000  
\end{Soutput}
\end{Schunk}
On peut vérifier que les événements ont été correctement enregistrés dans ce
nouveau tableau de données en vérifiant le format des données de survie sous
\R :
\begin{Schunk}
\begin{Sinput}
> with(mp, Surv(time, status))
\end{Sinput}
\begin{Soutput}
 [1]  1   1   2   2   3   4   4   5   5   8   8   8   8  11  11  12  12  15  17  22  23   6   6   6 
[25]  6+  7   9+ 10  10+ 11+ 13  16  17+ 19+ 20+ 22  23  25+ 32+ 32+ 34+ 35+
\end{Soutput}
\end{Schunk}

Le tableau des événements dans le groupe 6-MP peut être construit à l'aide
de la commande \texttt{survfit}. On pourrait également limiter les calculs
aux seules observations du groupe 6-MP à l'aide de l'option \texttt{subset}.
\begin{Schunk}
\begin{Sinput}
> s <- survfit(Surv(time, status) ~ tx, data=mp)
> summary(s)
\end{Sinput}
\begin{Soutput}
Call: survfit(formula = Surv(time, status) ~ tx, data = mp)
                tx=6-MP 
 time n.risk n.event survival std.err lower 95% CI upper 95% CI
    6     21       3    0.857  0.0764        0.720        1.000
    7     17       1    0.807  0.0869        0.653        0.996
   10     15       1    0.753  0.0963        0.586        0.968
   13     12       1    0.690  0.1068        0.510        0.935
   16     11       1    0.627  0.1141        0.439        0.896
   22      7       1    0.538  0.1282        0.337        0.858
   23      6       1    0.448  0.1346        0.249        0.807

                tx=Placebo 
 time n.risk n.event survival std.err lower 95% CI upper 95% CI
    1     21       2   0.9048  0.0641      0.78754        1.000
    2     19       2   0.8095  0.0857      0.65785        0.996
    3     17       1   0.7619  0.0929      0.59988        0.968
    4     16       2   0.6667  0.1029      0.49268        0.902
    5     14       2   0.5714  0.1080      0.39455        0.828
    8     12       4   0.3810  0.1060      0.22085        0.657
   11      8       2   0.2857  0.0986      0.14529        0.562
   12      6       2   0.1905  0.0857      0.07887        0.460
   15      4       1   0.1429  0.0764      0.05011        0.407
   17      3       1   0.0952  0.0641      0.02549        0.356
   22      2       1   0.0476  0.0465      0.00703        0.322
   23      1       1   0.0000     NaN           NA           NA
\end{Soutput}
\end{Schunk}
On utilisera le même résultat, contenu dans la variable \texttt{s} pour
afficher la courbe de survie correspondante. Par défaut, lorsqu'il n'y a
qu'un groupe \R affiche automatiquement l'intervalle de confiance à 95~\%.
\begin{Schunk}
\begin{Sinput}
> plot(survfit(Surv(time, status) ~ tx, data=mp, subset=tx=="6-MP"))
\end{Sinput}
\end{Schunk}
\includegraphics{figs/fig-ex7-2d}

Pour afficher la fonction de risque cumulée, on ajoutera l'option
\verb|fun="cumhaz"| à la commande graphique précédente, soit en considérant
les deux groupes :
\begin{Schunk}
\begin{Sinput}
> plot(s, fun="cumhaz")
\end{Sinput}
\end{Schunk}
\includegraphics{figs/fig-ex7-2e}

La médiane de survie est simplement obtenue en affichant la variable
\texttt{s}, c'est-à-dire :
\begin{Schunk}
\begin{Sinput}
> s
\end{Sinput}
\begin{Soutput}
Call: survfit(formula = Surv(time, status) ~ tx, data = mp)
           records n.max n.start events median 0.95LCL 0.95UCL
tx=6-MP         21    21      21      9     23      16      NA
tx=Placebo      21    21      21     21      8       4      12
\end{Soutput}
\end{Schunk}
\end{sol}
\end{exo}
% 
% Everitt p. 348
%
\begin{exo}\label{exo:7.3}
Dans un essai randomisé, on a cherché à comparer deux traitements pour le
cancer de la prostate. Les patients prenaient chaque jour par voie orale
soit 1 mg de diethylstilbestrol (DES, bras actif) soit un placebo, et le
temps de survie est mesuré en mois.\autocite{collett94} La question
d'intérêt est de savoir si la survie diffère entre les deux groupes de
patients, et on négligera les autres variables présentes dans le fichier de
données \texttt{prostate.dat}. 
\begin{description}
\item[(a)] Calculer la médiane de survie pour l'ensemble des patients, et
  par groupe de traitement.
\item[(b)] Quelle est la différence entre les proportions de survie dans les
  deux groupes à 50 mois ?
\item[(c)] Afficher les courbes de survie pour les deux groupes de patients.
\item[(d)] Effectuer un test du log-rank pour tester l'hypothèse selon
  laquelle le traitement par DES a un effet positif sur la survie des
  patients. 
\end{description}
\begin{sol}
Pour importer les données, on utilisera la commande \texttt{read.table}. On
profitera de l'inspection préalable des données pour recoder la variable
\texttt{Treatment} en variable qualitative.
\begin{Schunk}
\begin{Sinput}
> prostate <- read.table("data/prostate.dat", header=TRUE)
> str(prostate)
\end{Sinput}
\begin{Soutput}
'data.frame':	38 obs. of  7 variables:
 $ Treatment: int  1 2 2 1 2 1 1 1 2 1 ...
 $ Time     : int  65 61 60 58 51 51 14 43 16 52 ...
 $ Status   : int  1 1 1 1 1 1 0 1 1 1 ...
 $ Age      : int  67 60 77 64 65 61 73 60 73 73 ...
 $ Haem     : num  13.4 14.6 15.6 16.2 14.1 13.5 12.4 13.6 13.8 11.7 ...
 $ Size     : int  34 4 3 6 21 8 18 7 8 5 ...
 $ Gleason  : int  8 10 8 9 9 8 11 9 9 9 ...
\end{Soutput}
\begin{Sinput}
> head(prostate)
\end{Sinput}
\begin{Soutput}
  Treatment Time Status Age Haem Size Gleason
1         1   65      1  67 13.4   34       8
2         2   61      1  60 14.6    4      10
3         2   60      1  77 15.6    3       8
4         1   58      1  64 16.2    6       9
5         2   51      1  65 14.1   21       9
6         1   51      1  61 13.5    8       8
\end{Soutput}
\begin{Sinput}
> prostate$Treatment <- factor(prostate$Treatment)
> table(prostate$Status)
\end{Sinput}
\begin{Soutput}
 0  1 
 6 32 
\end{Soutput}
\end{Schunk}
Il n'est pas vraiment nécessaire de recoder \texttt{Status} en variable
qualitative puisque l'on va utiliser cette variable conjointement avec
\texttt{Time} pour l'analyse de survie. Dans un premier temps, pour
transformer les données en données de survie, incluant la censure, il est
nécessaire de passer par la commande \texttt{Surv} du package \texttt{survival}.
\begin{Schunk}
\begin{Sinput}
> library(survival)
> with(prostate, Surv(time=Time, event=Status))
\end{Sinput}
\begin{Soutput}
 [1] 65  61  60  58  51  51  14+ 43  16  52  59  55  68  51   2  67  66  66  28  50+ 69+ 67  65  24 
[25] 45  64  61  26+ 42+ 57  70   5  54  36+ 70  67  23  62 
\end{Soutput}
\end{Schunk}

La médiane de survie s'obtient à l'aide de \texttt{survfit}, qui une
commande générale permettant de manipuler les données de survie par la
méthode de Kaplan-Meier ou le modèle de Cox. Si l'on ne tient pas compte des
groupes de traitement (et plus généralement d'un quelconque co-facteur), on
l'utilisera ainsi : 
\begin{Schunk}
\begin{Sinput}
> survfit(Surv(Time, Status) ~ 1, data=prostate)
\end{Sinput}
\begin{Soutput}
Call: survfit(formula = Surv(Time, Status) ~ 1, data = prostate)
records   n.max n.start  events  median 0.95LCL 0.95UCL 
     38      38      38      32      59      54      65 
\end{Soutput}
\end{Schunk}

Pour prendre en compte le facteur traitement dans l'analyse, on utilisera en
revanche : 
\begin{Schunk}
\begin{Sinput}
> survfit(Surv(Time, Status) ~ Treatment, data=prostate)
\end{Sinput}
\begin{Soutput}
Call: survfit(formula = Surv(Time, Status) ~ Treatment, data = prostate)
            records n.max n.start events median 0.95LCL 0.95UCL
Treatment=1      18    18      18     13     59      51      67
Treatment=2      20    20      20     19     60      54      67
\end{Soutput}
\end{Schunk}

Pour afficher les courbes de survie correspondant aux deux groupes de
traitement, on procèdera comme suit :
\begin{Schunk}
\begin{Sinput}
> plot(survfit(Surv(Time, Status) ~ Treatment, data=prostate))
\end{Sinput}
\end{Schunk}
\includegraphics{figs/fig-ex7-3e}

Le test du log-rank peut être obtenu de deux manières différentes. La plus
simple consiste à utiliser la commande \texttt{survdiff} pour comparer deux
courbes de survie à partir de l'estimateur de Kaplan-Meier.
\begin{Schunk}
\begin{Sinput}
> survdiff(Surv(time=Time, event=Status) ~ Treatment, data=prostate)
\end{Sinput}
\begin{Soutput}
Call:
survdiff(formula = Surv(time = Time, event = Status) ~ Treatment, 
    data = prostate)
             N Observed Expected (O-E)^2/E (O-E)^2/V
Treatment=1 18       13       12    0.0854     0.153
Treatment=2 20       19       20    0.0512     0.153

 Chisq= 0.2  on 1 degrees of freedom, p= 0.696 
\end{Soutput}
\end{Schunk}

L'autre solution consiste à utiliser une régression de Cox
\begin{Schunk}
\begin{Sinput}
> summary(coxph(Surv(time=Time, event=Status) ~ Treatment, data=prostate))
\end{Sinput}
\begin{Soutput}
Call:
coxph(formula = Surv(time = Time, event = Status) ~ Treatment, 
    data = prostate)
  n= 38, number of events= 32 

              coef exp(coef) se(coef)      z Pr(>|z|)
Treatment2 -0.1550    0.8564   0.3702 -0.419    0.675

           exp(coef) exp(-coef) lower .95 upper .95
Treatment2    0.8564      1.168    0.4145     1.769

Concordance= 0.521  (se = 0.055 )
Rsquare= 0.005   (max possible= 0.988 )
Likelihood ratio test= 0.17  on 1 df,   p=0.6766
Wald test            = 0.18  on 1 df,   p=0.6754
Score (logrank) test = 0.18  on 1 df,   p=0.6751
\end{Soutput}
\end{Schunk}
\end{sol}
\end{exo}

\Closesolutionfile{solutions}

\chapter*{Devoir \no 6}
\addcontentsline{toc}{chapter}{Devoir \no 6}

Les exercices sont indépendants. Une seule réponse est correcte pour chaque
question. Lorsque vous ne savez pas répondre, cochez la case correspondante.

\section*{Exercice 1}
Considérons les données de survie d'un échantillon de 23 participants
afro-américains, tous de sexe masculin, provenant de l'étude \emph{San
  Francisco Men's Health Study} (1983). Les durées présentées dans le
tableau ci-dessous représentent le temps en mois depuis un diagnostique de
Sida jusqu'au décès du participant ou la fin de la durée de suivi
\citep[p.~130]{selvin08}, pour les participants fumeurs (F) et non-fumeurs
(NF). Les données censurées sont représentées par une astérisque.  \vskip1em

\begin{tabular}{ll}
\toprule
Non-fumeurs & 2* 42* 27* 22 26* 16 31 37 15 30 12* 5 80 \\
            & 29 13 1 14 \\
Fumeurs    & 21* 4 25 8 23 18 \\
\bottomrule
\end{tabular}
\vskip1em

Les données ont été enregistrées sous \R de la même manière qu'à
l'exercice~7.2, c'est-à-dire sous la forme d'un tableau à 3 colonnes :
fumeur (oui/non), temps et statut (en 0/1). Voici un aperçu de
ces données sous \R :
\begin{verbatim}
> head(sfmhs)
  fumeur temps status
1    non     2      0
2    non    42      0
3    non    27      0
4    non    22      1
5    non    26      0
6    non    16      1
\end{verbatim}
\begin{description}
\item[\bf 1.1] \marginpar{\phantom{text}1.1 $\square$} La commande suivante permet-elle
  d'indiquer la proportion de sujets fumeurs encore en vie à la fin de
  l'étude ?
\begin{verbatim}
> with(sfmhs, sum(status[fumeur=="oui"])) 
\end{verbatim}
  \begin{description}
  \item[A.] Oui.
  \item[B.] Non.
  \item[C.] Je ne sais pas.
  \end{description}
\item[\bf 1.2] \marginpar{\phantom{text}1.2 $\square$} On souhaite afficher la table de
  mortalité des non-fumeurs. Quelle commande doit-on utiliser ?
  \begin{description}
  \item[A.] \verb|summary(Surv(temps, status) ~ fumeur, data=sfmhs)|
  \item[B.] \verb|summary(Surv(status, temps) ~ fumeur, data=sfmhs)|
  \item[C.] \verb|summary(survfit(Surv(temps, status) ~ fumeur, data=sfmhs))|
  \item[D.] \verb|summary(survfit(Surv(status, temps) ~ fumeur, data=sfmhs))|
  \item[E.] Je ne sais pas.
  \end{description}
\item[\bf 1.3] \marginpar{\phantom{text}1.3 $\square$} En notant \texttt{st} la variable
  dans laquelle on a stocké les données de survie enregistrées au format \R
  via la commande \texttt{Surv}, la commande suivante permet de fournir le
  même résultat que la commande de l'exercice~1.2 ?
\begin{verbatim}
> summary(survfit(st ~ 1, data=sfmhs, subset=fumeur=="non"))
\end{verbatim}  
  \begin{description}
  \item[A.] Vrai.
  \item[B.] Faux.
  \item[C.] Je ne sais pas.
  \end{description}
\item[\bf 1.4] \marginpar{\phantom{text}1.4 $\square$} Pour afficher la courbe de survie
  du groupe des patients non-fumeurs, sans intervalles de confiance, quelle
  commande peut-on utiliser ? On supposera que la table de mortalité
  calculée en 1.3 a été
  stockée dans une variable appelée \texttt{stab}.
  \begin{description}
  \item[A.] \verb|plot(stab)|
  \item[B.] \verb|plot(stab, confint=0)|
  \item[C.] \verb|plot(stab, conf.int=FALSE)|    
  \item[D.] Je ne sais pas.
  \end{description}
\item[\bf 1.5] \marginpar{\phantom{text}1.5 $\square$} À quelle procédure de test
  correspond la commande suivante ?
\begin{verbatim}
> survdiff(st ~ fumeur, data=sfmhs, rho=1)
\end{verbatim}
  \begin{description}
  \item[A.] Un test du log-rank.
  \item[B.] Un test de Wilcoxon.
  \item[C.] Un test du coefficient de régression d'un modèle de Cox.
  \item[D.] Je ne sais pas.
  \end{description}  
\end{description}

\section*{Exercice 2}
Dans une étude clinique réalisée chez des patients AML, on s'intéresse à
deux facteurs pronostic, l'âge ($< 50$ et $\ge 50$ ans) et le type de
cellularité de la moelle (100~\% ou moins), sur la survie des patients
\citep[p.~275]{lee03}. Les données des 30 patients sont présentées dans le
tableau suivant. Les variables $x_1$ et $x_2$ correspondent à l'âge et la
cellularité, avec $x_1=0$ si le patient a moins de 50 ans et $x_2=0$ lorsque
la cellularité est inférieure à 100~\%. Les censures sont indiquées par le
symbole \texttt{+}, et les durées sont reportées en mois.  
\vskip1em

\begin{tabular}{lcc|lcc}
  \toprule
  Survie & $x_1$ & $x_2$ & Survie & $x_1$ & $x_2$ \\
  \midrule
  18 & 0 & 0 & 8 & 1 & 0 \\
  9 & 0 & 1 & 2 & 1 & 1 \\
  28+ & 0 & 0 & 26+ & 1 & 0 \\
  31 & 0 & 1 & 10 & 1 & 1 \\
  39+ & 0 & 1 & 4 & 1 & 0 \\
  19+ & 0 & 1 & 3 & 1 & 0 \\
  45+ & 0 & 1 & 4 & 1 & 0 \\
  6 & 0 & 1 & 18 & 1 & 1 \\
  8 & 0 & 1 & 8 & 1 & 1 \\
  15 & 0 & 1 & 3 & 1 & 1 \\
  23 & 0 & 0 & 14 & 1 & 1 \\
  28+ & 0 & 0 & 3 & 1 & 0 \\
  7 & 0 & 1 & 13 & 1 & 1 \\
  12 & 1 & 0 & 13 & 1 & 1 \\
  9 & 1 & 0 & 35+ & 1 & 0 \\
  \bottomrule
\end{tabular}
\vskip1em

Voici un aperçu du fichier de données, \texttt{aml.dat}, dans lequel les
données ont été stockées :
\begin{verbatim}
18 0 0 0
8 0 1 0
9 0 0 1
2 0 1 1
28 1 0 0
18 0 0 0
8 0 1 0
9 0 0 1
28 1 0 0
26 1 1 0
\end{verbatim}
Chaque ligne correspond à un individu. La première colonne correspond au
nombre de mois indiqué dans le tableau précédent, la deuxième colonne
correspond à la présence d'une censure (censure = 1), la troisième colonne à
la variable indicatrice codant pour l'âge ($\ge 50$ = 0) et la quatrième et
dernière colonne correspond à la variable indicatrice codant pour la
cellularité (100~\% = 1). Une valeur de 1 pour $x_1$ ou $x_2$ est considérée
comme un indicateur de risque.
\begin{description}
\item[\bf 2.1] \marginpar{\phantom{text}2.1 $\square$} Quelle solution peut-on proposer
  pour importer ces données sous \R ?
  \begin{description}
  \item[A.] \verb|read.csv("aml.dat", header=TRUE, colnames=c("temps","status","age","cell"))|
  \item[B.] \verb|read.csv("aml.dat", colnames=c("temps","status","age","cell"), sep=" ")|
  \item[C.] \verb|read.table("aml.dat", col.names=c("temps","status","age","cell"))|
  \item[D.] Je ne sais pas.
  \end{description}
\item[\bf 2.2] \marginpar{\phantom{text}2.2 $\square$} On supposera par la suite que les
  données ont été importées et stockées dans une variable appelée
  \texttt{aml}. Pour travailler avec ces données sous \R, il est nécessaire
  de recoder la variable \texttt{status} de façon à ce que les valeurs 0
  soient transformées en 1 et, réciproquement, les valeurs 1 en 0. Quelle
  commande ou combinaison de commandes permet de réaliser cette opération ?
  \begin{description}
  \item[A.] \verb|aml$status[aml$status==0] <- 1; aml$status[aml$status==1] <- 0|
  \item[B.] \verb|aml$status <- abs(aml$status-1)|
  \item[C.] \verb|aml$status <- aml$status-1|
  \item[D.] Je ne sais pas.
  \end{description}
\item[\bf 2.3] \marginpar{\phantom{text}2.3 $\square$} Que produit la commande suivante ? 
\begin{verbatim}
> with(aml, table(age, cell))[2,2]
\end{verbatim}
  \begin{description}
  \item[A.] Le nombre d'individus à risque pour $x_1$ (\texttt{age}).
  \item[B.] Le nombre d'individus à risque pour $x_2$ (\texttt{cell}).
  \item[C.] Le nombre d'individus à risque pour $x_1$ et $x_2$.
  \item[D.] Je ne sais pas.
  \end{description}
\item[\bf 2.4] \marginpar{\phantom{text}2.4 $\square$} Le nombre de données censurées par
  groupe d'âge peut être représenté graphiquement sous forme d'un diagramme
  en points à l'aide de la commande :
  \begin{description}
  \item[A.] \verb|dotplot(status ~ age, data=aml)|
  \item[B.] \verb|dotchart(status ~ age, data=aml)|
  \item[C.] \verb|dotplot(xtabs(status ~ age, data=aml))|
  \item[D.] \verb|dotplot(xtabs(~ status + age, data=aml))|
  \item[E.] Je ne sais pas.
  \end{description}  
\item[\bf 2.5] \marginpar{\phantom{text}2.5 $\square$} On souhaite effectuer une
  régression de Cox pour étudier l'effet du facteur âge. Quelle commande est
  la plus appropriée ?
  \begin{description}
  \item[A.] \verb|coxph(status ~ age, data=aml)|
  \item[B.] \verb|coxph(Surv(status, temps) ~ age, data=aml)|
  \item[C.] \verb|coxph(Surv(event=status, time=temps) ~ age, data=aml)|
  \item[D.] Je ne sais pas.
  \end{description}
\item[\bf 2.6] \marginpar{\phantom{text}2.6 $\square$} Voici les résultats retournés par
  \R pour un modèle de Cox incluant les deux prédicteurs :
\begin{verbatim}
      coef exp(coef) se(coef)     z     p
age  1.046      2.85    0.458 2.284 0.022
cell 0.359      1.43    0.440 0.815 0.420
\end{verbatim}
  Lorsque l'on considère simultanément les deux cofacteurs, le risque
  relatif pour un individu âgé de 50 ans ou plus et ayant une cellularité de
  100~\% est de :
  \begin{description}
  \item[A.] $1.05 + 0.36$
  \item[B.] $2.85 + 1.43$
  \item[C.] $\exp(1.05 + 0.36)$
  \item[D.] $\exp(2.85 + 1.43)$    
  \item[E.] Je ne sais pas.
  \end{description}
\item[\bf 2.7] \marginpar{\phantom{text}2.7 $\square$} La commande \texttt{confint},
  appliquée directement sur le modèle \texttt{coxph(...)} renverra :
  \begin{description}
  \item[A.] Les intervalles de confiance associés aux coefficients de
    régression. 
  \item[B.] Les intervalles de confiance associés au rapport de riques instantanés.
  \item[C.] Je ne sais pas.
  \end{description}  
\end{description}

%-------------------------------------- Exos Stata ---------------------------
\blankpage
\blankpage

\thispagestyle{empty}
\centerline{\small Centre d'Enseignement de la Statistique Appliquée, à la Médecine et à la Biologie Médicale}
\vspace*{2cm}
\begin{center}
\centerline{\includegraphics[scale=.55]{cesam}}
\vspace*{2cm}
\begin{minipage}{.75\textwidth}
\begin{mdframed}[style=titlep]
\centerline{\Huge Programme de travail}
\vskip1em
\centerline{\Huge du cours d'informatique du CESAM}
\end{mdframed}
\end{minipage}
\end{center}
\vskip3em
\centerline{\Huge\bf Introduction au logiciel Stata}
\vskip5em
\begin{center}
  \begin{tabular}{ll}
    \textbf{Responsables :} & \\
    Christophe LALANNE & \url{christophe.lalanne@inserm.fr} \\
    Yassin MAZROUI     & \url{yassin.mazroui@upmc.fr} \\
    Pr Mounir MESBAH   & \url{mounir.mesbah@upmc.fr}
  \end{tabular}
\end{center}
\vskip3em
\centerline{\Large \url{http://www.cesam.upmc.fr}}
\vskip3em
\centerline{\LARGE Année Universitaire 2015–2016}
\vfill
\begin{center}
\begin{minipage}{.6\textwidth}
\centering
Adresser toute correspondance à :\\
Université Pierre et Marie Curie – Paris 6
Secrétariat du CESAM – Les Cordeliers
Service Formation Continue, esc. B, 4ème étage,
15 rue de l’école de médecine,
75006 PARIS\\
ou par Courriel à : \url{cesam@upmc.fr}
\end{minipage}
\end{center}  



\blankpage

\chapter*{Calendrier}
\thispagestyle{empty}
\vskip3em

\begin{center}
\begin{tabular}{|l|p{10cm}|l|l|}
\hline
  \multicolumn{4}{|c|}{Module Stata} \\
\hline
  Sem. 11/04 & Élements du langage et statistiques descriptives & Cours 8 &
  Corrigés pp.~\pageref{start:sol1stata}–\pageref{stop:sol1stata}\\
  Sem. 2/05 & Mesures d'association et comparaison de deux variables &
  Cours 9 & Corrigés pp.~\pageref{start:sol2stata}–\pageref{stop:sol2stata}\\
  Sem. 9/05 & Régression linéaire et logistique & Cours 10 & Corrigés pp.~\pageref{start:sol3stata}–\pageref{stop:sol3stata}\\
  Sem. 16/05 & Données de survie & Cours 11 & Corrigés pp.~\pageref{start:sol4stata}–\pageref{stop:sol4stata}\\
\hline
\end{tabular}
\end{center}


\setcounter{page}{1}
\chapter{Élements du langage et statistiques descriptives}

\begin{exo}\label{exo:8.1}
{\footnotesize Identique à l'énoncé 1.1 (p.~\pageref{exo:1.1}), questions
  a–c.}

Un chercheur a recueilli les mesures biologiques suivantes (unités
arbitraires) :
\begin{verbatim}
3.68  2.21  2.45  8.64  4.32  3.43  5.11  3.87
\end{verbatim}
\begin{description}
\item[(a)] Stocker la séquence de mesures dans une variable appelée
  \texttt{x}.  
\item[(b)] Indiquer le nombre d'observations (à l'aide de \R), les valeurs
  minimale et maximale, ainsi que l'étendue.  
\item[(c)] En fait, le chercheur réalise que la valeur 8.64 correspond à une
  erreur de saisie et doit être changée en 3.64. De même, il a un doute sur
  la 7\ieme mesure et décide de la considérer comme une valeur manquante :
  effectuer les transformations correspondantes. 
\end{description}
\end{exo}
\vskip1em

\begin{exo}\label{exo:8.2}
{\footnotesize Identique à l'énoncé 1.2 (p.~\pageref{exo:1.2}), questions a
  et b.}

La charge virale plasmatique permet de décrire la quantité de virus (p.~ex.,
VIH) dans un échantillon de sang. Ce marqueur virologique qui permet de
suivre la progression de l’infection et de mesurer l’efficacité des
traitements est rapporté en nombre de copies par millilitre, et la plupart
des instruments de mesure ont un seuil de détectabilité de 50
copies/ml. Voici une série de mesures, $X$, exprimées en logarithmes (base 10)
collectées sur 20 patients :
\begin{verbatim}
3.64 2.27 1.43 1.77 4.62 3.04 1.01 2.14 3.02 5.62 5.51 5.51 1.01 1.05 4.19
2.63 4.34 4.85 4.02 5.92
\end{verbatim}
Pour rappel, une charge virale de 100 000 copies/ml équivaut à 5 log.
\begin{description}
\item[(a)] Indiquer combien de patients ont une charge virale considérée
  comme non-détectable. 
\item[(b)] Quelle est le niveau de charge virale médian, en copies/ml, pour
  les données considérées comme valides ?
\end{description}
\end{exo}
\vskip1em

\begin{exo}\label{exo:8.3}
{\footnotesize Identique à l'énoncé 1.5 (p.~\pageref{exo:1.5}), questions
  a–d.}  

Le fichier \texttt{anorexia.dat} contient les données d'une étude clinique
chez des patientes anorexiques ayant reçu l'une des trois thérapies
suivantes : thérapie comportementale, thérapie familiale, thérapie
contrôle.\autocite{hand93} 
\begin{description}
\item[(a)] Combien y'a-t-il de patientes au total ? Combien y'a-t-il de
  patientes par groupe de traitement ?
\item[(b)] Les mesures de poids sont en livres. Les convertir en
  kilogrammes.    
\item[(c)] Créer une nouvelle variable contenant les scores de différences
  (\texttt{After} - \texttt{Before}).
\item[(d)] Indiquer la moyenne et l'étendue (min/max) des scores de
  différences par groupe de traitement.
\end{description}
\end{exo}
\vskip1em

\begin{exo}\label{exo:8.4}
{\footnotesize Identique à l'énoncé 2.1 (p.~\pageref{exo:2.1}), questions
  a–d.} 

Une variable quantitative $X$ prend les valeurs suivantes sur un échantillon
de 26 sujets :
\begin{verbatim}
24.9,25.0,25.0,25.1,25.2,25.2,25.3,25.3,25.3,25.4,25.4,25.4,25.4,
25.5,25.5,25.5,25.5,25.6,25.6,25.6,25.7,25.7,25.8,25.8,25.9,26.0
\end{verbatim}
\begin{description}
\item[(a)] Calculer la moyenne, la médiane ainsi que le mode de $X$. 
\item[(b)] Quelle est la valeur de la variance estimée à partir de ces données ? 
\item[(c)] En supposant que les données sont regroupées en 4 classes dont les
  bornes sont : 24.9–25.1, 25.2–25.4, 25.5–25.7, 25.8–26.0, afficher la
  distribution des effectifs par classe sous forme d'un tableau d'effectifs. 
\item[(d)] Représenter la distribution de $X$ sous forme d'histogramme, sans
  considération d'intervalles de classe \emph{a priori}.
\end{description}
\end{exo}
\vskip1em

\begin{exo}\label{exo:8.5}
{\footnotesize Identique à l'énoncé 2.3 (p.~\pageref{exo:2.3}), questions
  a–c.}

Le fichier \texttt{elderly.dat} contient la taille mesurée en cm de 351
personnes âgées de sexe féminin, sélectionnées aléatoirement dans la
population lors d'une étude sur l'ostéoporose. Quelques observations sont
cependant manquantes.
\begin{description}
\item[(a)] Combien y'a t-il d'observations manquantes au total ?
\item[(b)] Donner un intervalle de confiance à 95~\% pour la taille moyenne
  dans cet échantillon, en utilisant une approximation normale.
\item[(c)] Représenter la distribution des tailles observées sous forme
  d'une courbe de densité.  
\end{description}
\end{exo}
\vskip1em

\begin{exo}\label{exo:8.6}
{\footnotesize Identique à l'énoncé 2.4 (p.~\pageref{exo:2.4}), questions
  a–f.} 

Le fichier \texttt{birthwt} est un des jeux de données fournis avec \R. Il
comprend les résultats d'une étude prospective visant à identifier les
facteurs de risque associés à la naissance de bébés dont le poids est
inférieur à la norme (2,5 kg). Les données proviennent de 189 femmes, dont
59 ont accouché d'un enfant en sous-poids. Parmi les variables d'intérêt
figurent l'âge de la mère, le poids de la mère lors des dernières
menstruations, l'ethnicité de la mère et le nombre de visites médicales
durant le premier trimestre de grossesse.\autocite{hosmer89}
Les variables disponibles sont décrites comme suit : \texttt{low} (= 1 si
poids $<2.5$ kg, 0 sinon), \texttt{age} (années), \texttt{lwt} (poids de la
mère en livres), \texttt{race} (ethnicité codée en trois classes, 1 = white,
2 = black, 3 = other), \texttt{smoke} (= 1 si consommation de tabac durant
la grossesse, 0 sinon), \texttt{ptl} (nombre d'accouchements pré-terme
antérieurs), \texttt{ht} (= 1 si antécédent d'hypertension, 0 sinon),
\texttt{ui} (= 1 si manifestation d'irritabilité utérine, 0 sinon),
\texttt{ftv} (nombre de consultations chez le gynécologue durant le premier
trimestre de grossesse), \texttt{bwt} (poids des bébés à la naissance, en
\emph{g}).
\begin{description}
\item[(a)] Recoder les variables \texttt{low}, \texttt{race},
  \texttt{smoke}, \texttt{ui} et \texttt{ht} en variables
  qualitatives, avec des étiquettes ("labels") plus informatives.
\item[(b)] Convertir le poids des mères en \emph{kg}. Indiquer la moyenne, la
  médiane et l'intervalle inter-quartile. Représenter la distribution des
  poids sous forme d'histogramme.
\item[(c)] Indiquer la proportion de mères consommant du tabac durant la
  grossesse, avec un intervalle de confiance à 95~\%. Représenter les
  proportions (en \%) fumeur/non-fumeur sous forme d'un diagramme en
  barres.
\item[(d)] Recoder l'âge des mères en trois classes équilibrées (tercilage)
  et indiquer la proportion d'enfants dont le poids est $<2500$ \emph{g}
  pour chacune des trois classes.
\item[(e)] Construire un tableau d'effectifs ($n$ et \%) pour la variable
  ethnicité (\texttt{race}).  
\item[(f)] Décrire la distribution des variables \texttt{race},
  \texttt{smoke}, \texttt{ui}, \texttt{ht} et \texttt{age} après
  stratification sur la variable \texttt{low}.  
\end{description}
\end{exo}

%--------------------------------------------------------------- Devoir 07 ---
\chapter*{Devoir \no 7}
\addcontentsline{toc}{chapter}{Devoir \no 7}

Les exercices sont indépendants. Une seule réponse est correcte pour chaque
question. Lorsque vous ne savez pas répondre, cochez la case correspondante.

\section*{Exercice 1}
Considérons le petit jeu de données affichés ci-dessous comme une sortie
\Stata (produit avec la commande \verb|list|):
\begin{verbatim}
     +----------------+
     |   x      y   z |
     |----------------|
  1. | 1.2    8.1   1 |
  2. | 2.4   12.4   2 |
  3. | 3.1    6.7   1 |
  4. | 1.8    9.8   2 |
  5. | 5.7   10.3   1 |
     |----------------|
  6. | 6.4   10.8   2 |
  7. | 3.8    9.2   1 |
  8. | 2.6      .   2 |
  9. | 5.7   11.2   1 |
 10. | 3.8   12.7   2 |
     +----------------+
\end{verbatim}
\begin{description}
\item[\bf 1.1] On souhaite remplacer la 6\ieme\ observation de la variable
  \texttt{x} (6.4) par la valeur 5.9. Quelle commande faut-il utiliser ?
  \marginpar{1.1 $\square$} 
  \begin{description}
  \item[A.] \verb|replace x[6] = 5.9|
  \item[B.] \verb|replace x = 5.9 in 6|
  \item[C.] \verb|generate x = 5.9 in 6|
  \item[D.] \verb|generate x = 5.9 in 6, replace|
  \item[E.] Je ne sais pas.
  \end{description}  
\item[\bf 1.2] La commande
\begin{verbatim}
. tab z if y > 10
\end{verbatim}
permet de renvoyer le nombre d'observations pour lequel $y > 10$ pour chaque
modalité de \texttt{z}. \marginpar{1.2 $\square$}
  \begin{description}
  \item[A.] Vrai.
  \item[B.] Faux.
  \item[C.] Je ne sais pas.
  \end{description}  
\item[\bf 1.3] En supposant que l'on dispose de la valeur moyenne des $y$
  dans une variable \texttt{ym} définie comme suit :
  \verb|egen ym = mean(y)|, quelle commande doit-on utiliser pour remplacer
  la valeur manquante présente dans la variable \texttt{y} par la moyenne
  des $y$ ?  \marginpar{1.3 $\square$}
  \begin{description}
  \item[A.] \verb|replace y = ym, if missing(y)|
  \item[B.] \verb|replace y = ym if missing(y)|
  \item[C.] \verb|generate y = ym, if missing(y)|
  \item[D.] \verb|generate y = ym if missing(y)|
  \item[E.] Je ne sais pas.
  \end{description}  
\item[\bf 1.4] Quelle commande a permis de produire le résultat suivant ?
  \marginpar{1.4 $\square$} 
\begin{verbatim}
    Variable |       Obs        Mean    Std. Dev.       Min        Max
-------------+--------------------------------------------------------
           x |        10        3.65    1.776545        1.2        6.4
\end{verbatim}
  \begin{description}
  \item[A.] \verb|describe|
  \item[B.] \verb|summarise|
  \item[C.] \verb|summarize| 
  \item[D.] Je ne sais pas.
  \end{description}
\item[\bf 1.5] La variable \texttt{z} comporte deux valeurs uniques (1 et 2)
  correspondant à deux catégories d'individus : 1 = Homme, 2 = Femme. On
  souhaite associer ces étiquettes, que l'on supposera définies correctement
  dans la variable \texttt{genderlab}, aux valeurs prises par \texttt{z}. La
  commande suivante permet-elle de répondre à la question ? 
  \marginpar{1.5 $\square$}
\begin{verbatim}
. label value genderlab z
\end{verbatim}
  \begin{description}
  \item[A.] Oui.
  \item[B.] Non.
  \item[C.] Je ne sais pas.
  \end{description}  
\item[\bf 1.6] Que permet de réaliser la commande suivante : \marginpar{1.6 $\square$}
\begin{verbatim}
. by z, sort: count if y < 11 & x > 3
\end{verbatim}
  \begin{description}
  \item[A.] Compter le nombre d'observations de la variable pour lesquels $y
    < 11$ et $x > 3$ par modalité de \texttt{z}.
  \item[B.] Trier la variable \texttt{z} par ordre croissant et dénombrer
    les observations pour lesquelles $y < 11$ ou $x > 3$.
  \item[C.] Je ne sais pas.
  \end{description}
\item[\bf 1.7] Quelle commande permet de calculer la médiane et l'intervalle
  inter-quartile de \texttt{x} ?\marginpar{1.7 $\square$}
  \begin{description}
  \item[A.] \verb|summarize x|
  \item[B.] \verb|summarize x, all|
  \item[C.] \verb|tabstat x, stats(median qrange)|
  \item[D.] \verb|tabstat x, stats(median iqr)|
  \item[E.] Je ne sais pas.
  \end{description}  

\end{description}
\section*{Exercice 2}
Considérons les données traitées à l'exercice~\ref{exo:8.3} (décrites
p.~\pageref{exo:8.3}), sans aucune transformation sur les valeurs
numériques. 
\begin{verbatim}
. list in 1/5

     +------------------------+
     | Group   Before   After |
     |------------------------|
  1. |    g1     80.5    82.2 |
  2. |    g1     84.9    85.6 |
  3. |    g1     81.5    81.4 |
  4. |    g1     82.6    81.9 |
  5. |    g1     79.9    76.4 |
     +------------------------+
\end{verbatim}
\begin{description}
\item[\bf 2.1] Supposons que des étiquettes aient été attribuées à chacun
  des trois groupes : \texttt{th.cpt} (initialement \texttt{g1}),
  \texttt{th.fam} (\texttt{g2}), \texttt{ctl} (\texttt{g3}). On souhaite
  identifier les patientes du groupe contrôle ayant un poids initial
  inférieur à 36 kg. Quelle commande doit-on utiliser ?
  \marginpar{2.1 $\square$} 
  \begin{description}
  \item[A.] \verb|list if Group == "ctl" & Before < 36/2.2|
  \item[B.] \verb|list if Group == "ctl" & Before < 36*2.2|
  \item[C.] \verb|list if Group > 2 & Before < 36/2.2|
  \item[D.] \verb|list if Group == 3 & Before < 36*2.2|
  \item[E.] Je ne sais pas.
  \end{description}  
\item[\bf 2.2] Que permet de renvoyer la série de commandes suivante ?
  \marginpar{2.2 $\square$}
\begin{verbatim}
. egen de = max(After - Before) if Group == "g1"
. display de
\end{verbatim}
  \begin{description}
  \item[A.] La différence entre les poids les plus élevés en début et fin
    d'étude chez les patientes du groupe \texttt{g1}.    
  \item[B.] La différence maximale retrouvée entre les poids de début et fin
    d'étude chez les patientes du groupe \texttt{g1}.
  \item[C.] Les numéros d'identification des patientes triés par différence
    de poids entre le début et la fin de l'étude.
  \item[D.] Je ne sais pas.
  \end{description}  
\item[\bf 2.3] On souhaite recoder les poids des patientes en début et fin
  d'étude en 5 classes équilibrées. Quelle commande doit-on utiliser ?
  \marginpar{2.3 $\square$}
  \begin{description}
  \item[A.] \verb|xtile|
  \item[B.] \verb|pctile|
  \item[C.] \verb|cut|
  \item[D.] Je ne sais pas.
  \end{description}  
\item[\bf 2.4] Supposons que les poids des patients en début d'étude
  (variable \texttt{Before}) aient été recodées en 5 classes équilibrées
  comme dans l'exercice précédent. On souhaite à présent recode la variable
  \texttt{After} en utilisant les mêmes intervalles de classe. Quelle
  commande doit-on utiliser ?
  \marginpar{2.4 $\square$}
  \begin{description}
  \item[A.] \verb|xtile|
  \item[B.] \verb|pctile|
  \item[C.] \verb|egen cut|
  \item[D.] Je ne sais pas.
  \end{description}
\item[\bf 2.5] Pour afficher la distribution des effectifs par classe pour
  la variable \texttt{Before} recodée en 5 classes de l'exercice~2.3, que
  l'on appelle \texttt{BeforeCat}, on utilisera la commande : \marginpar{2.5 $\square$}
  \begin{description}
  \item[A.] \verb|histogram BeforeCat, frequency|
  \item[B.] \verb|histogram BeforeCat, frequency discrete|
  \item[C.] \verb|graph bar BeforeCat|
  \item[D.] Je ne sais pas.
  \end{description}
\item[\bf 2.6] Quelle commande permet de reproduire la figure suivante
  (indépendemment du rapport hauteur/largeur et de la couleur) ? \marginpar{2.6 $\square$}
  \begin{center}
    \includegraphics{./figs/dev7_histo}
  \end{center}
  \begin{description}
  \item[A.] \verb|histogram After, frequency addlabels title("Groupe g1")|
  \item[B.] \verb|histogram After, frequency addlabels mtitle("Groupe g1")|
  \item[C.] \verb|histogram After if Group == "g1", frequency addlabels title("Groupe g1")|
  \item[D.] \verb|histogram After if Group == "g1", frequency addlabels mtitle("Groupe g1")|
  \item[C.] Je ne sais pas.
  \end{description}
\item[\bf 2.7] Quelle commande a permis de produire le résultat suivant ? \marginpar{2.7 $\square$}
\begin{verbatim}
--------------------------------------
Type de   |
therapie  | mean(Before)   mean(After)
----------+---------------------------
       g1 |    82.689655     85.696552
       g2 |    81.557692     81.107692
       g3 |    83.229412     90.494118
--------------------------------------
\end{verbatim}
  \begin{description}
  \item[A.] \verb|by Group: summarize Before After, meanonly|
  \item[B.] \verb|by Group, sort: summarize Before After, meanonly|
  \item[C.] \verb|table Group, contents(mean Before mean After)|
  \item[D.] Je ne sais pas.
  \end{description}
\end{description}

\section*{Exercice 3}\label{dev7:exo3}
On dispose d'un fichier de données, \texttt{platelet.txt}, dont un aperçu
des 7 premières lignes est fourni ci-après.
\begin{verbatim}
1	1	359000	396000	B
1	2	200000	184000	A
1	3	149000	151000	A
1	4	235000	242000	B
1	5	174000	177000	B
1	6	271000	203000	A
2	7	180000	199000	B
\end{verbatim}
Ce fichier concerne les données d'un essai clinique multicentrique de
pharmacovigilance portant sur un nouveau produit (A) comparé à un placebo
(B) \citep{chow04}. Le volume total de plaquettes ($\mu$l) a été enregistré
avant et après traitement, pour les deux traitements. Les colonnes du
fichier correspondent dans l'ordre au numéro de centre (\texttt{centre})
(numéroté 1 à 5), au numéro du patient (\texttt{patient}), au volume de
plaquettes avant (\texttt{pre}) et après (\texttt{post}) traitement, et au
type de traitement (\texttt{trt}).

\begin{description}
\item[\bf 3.1] Quelle commande doit-on utiliser pour importer le fichier de
  données sous \Stata ? \marginpar{3.1 $\square$}
  \begin{description}
  \item[A.] \verb|insheet using "platelet.txt", clear|
  \item[B.] \verb|infile centre patient pre post trt using "platelet.txt", clear|
  \item[C.] \verb|infile centre patient pre post str2(trt) using "platelet.txt", clear|
  \item[D.] Je ne sais pas.
  \end{description}
\item[\bf 3.2] On souhaite calculer le volume de plaquettes médian avant et
  après traitement dans les deux derniers centres. Quelle commande doit-on
  utiliser ? \marginpar{3.2 $\square$}
  \begin{description}
  \item[A.] \verb|tabstat pre post, by(centre) if centre > 3 stat(median) nototal|
  \item[B.] \verb|tabstat pre post if centre > 3, by(centre) stat(median) nototal|
  \item[C.] Je ne sais pas.
  \end{description}
\item[\bf 3.3] Pour obtenir un intervalle de confiance à 90~\% de la moyenne
  post-traitement, on utilise la commande : \marginpar{3.3 $\square$}
  \begin{description}
  \item[A.] \verb|ci post, level(90)|
  \item[B.] \verb|ci post, level(0.90)|
  \item[C.] \verb|summarize post, conflevel(90)|
  \item[D.] \verb|summarize post, conflevel(0.90)|
  \item[E.] Je ne sais pas.
  \end{description}
\item[\bf 3.4] On souhaite représenter la distribution des scores de
  différence (\texttt{post} - \texttt{pre}), stockés dans une variable
  appelé \texttt{dv} sous forme d'un histogramme incluant 7 classes
  distinctes. Quelle commande doit-on utiliser ? \marginpar{3.4 $\square$}
  \begin{description}
  \item[A.] \verb|histogram dv, class(7)|
  \item[B.] \verb|histogram dv, start(7)|
  \item[C.] \verb|histogram dv, bin(7)|
  \item[D.] Je ne sais pas.
  \end{description}
\end{description}

%--------------------------------------------------------------- Chapter 09 --
\chapter[Mesures d'association, comparaison de moyennes et de
proportions]{Mesures d'association, comparaison de moyennes et de
  proportions pour deux échantillons ou plus}   

\begin{exo}\label{exo:9.1}
{\footnotesize Identique à l'énoncé 3.1 (p.~\pageref{exo:3.1}), questions
  a–c.} 

On dispose des poids à la naissance d'un échantillon de 50 enfants
présentant un syndrôme de détresse respiratoire idiopathique aïgue. Ce type
de maladie peut entraîner la mort et on a observé 27 décès chez ces
enfants. Les données sont résumées dans le tableau ci-dessous et sont
disponibles dans le fichier \texttt{sirds.dat}, où les 27 premières
observations correspondent au groupe des enfants décédés au moment de
l'étude. \autocite[p.~64]{everitt01}
\vskip1em

\begin{tabular}{ll}
\toprule  
Enfants décédés &
1.050\; 1.175\; 1.230\; 1.310\; 1.500\; 1.600\; 1.720\; 1.750\; 1.770\; 2.275\; 2.500\; 1.030\; 1.100\; 1.185 \\
& 1.225\; 1.262\; 1.295\; 1.300\; 1.550\; 1.820\; 1.890\; 1.940\; 2.200\; 2.270\; 2.440\; 2.560\; 2.730 \\
Enfants vivants &
1.130\; 1.575\; 1.680\; 1.760\; 1.930\; 2.015\; 2.090\; 2.600\; 2.700\; 2.950\; 3.160\; 3.400\; 3.640\; 2.830 \\
& 1.410\; 1.715\; 1.720\; 2.040\; 2.200\; 2.400\; 2.550\; 2.570\; 3.005 \\
\bottomrule
\end{tabular}
\vskip1em

Un chercheur s'intéresse à l'existence éventuelle d'une différence entre le
poids moyen des enfants ayant survécu et celui des enfants décédés des
suites de la maladie. 
\begin{description}
\item[(a)] Réaliser un test $t$ de Student. Peut-on rejeter l'hypothèse nulle
  d'absence de différences entre les deux groupes d'enfants ? 
\item[(b)] Vérifier graphiquement que les conditions d'applications du test
(normalité et homogénéité des variances) sont vérifiées. 
\item[(c)] Quel est l'intervalle de confiance à 95~\% pour la différence de
  moyenne observée ?
\end{description}
\end{exo}
\vskip1em

\begin{exo}\label{exo:9.2}
{\footnotesize Identique à l'énoncé 3.2 (p.~\pageref{exo:3.2}), questions
  a–c.}

La qualité de sommeil de 10 patients a été mesurée avant (contrôle) et après
traitement par un des deux hypnotiques suivants : (1) D. hyoscyamine
hydrobromide et (2) L. hyoscyamine hydrobromide. Le critère de jugement
retenu par les chercheurs était le gain moyen de sommeil (en heures) par
rapport à la durée de sommeil de base
(contrôle). \autocite[p.~20]{student08} Les données sont reportées
ci-dessous et figurent également parmi les jeux de données de base de R
(\verb|data(sleep)|).  
\begin{verbatim}
D. hyoscyamine hydrobromide :
0.7 -1.6 -0.2 -1.2 -0.1  3.4  3.7  0.8  0.0  2.0
L. hyoscyamine hydrobromide :
1.9  0.8  1.1  0.1 -0.1  4.4  5.5  1.6  4.6  3.4
\end{verbatim}

Les chercheurs ont conclu que seule la deuxième molécule avait réellement un
effet soporifique. 
\begin{description}
\item[(a)] Estimer le temps moyen de sommeil pour chacune des deux
  molécules, ainsi que la différence entre ces deux moyennes.
\item[(b)] Afficher la distribution des scores de différence (LHH - DHH)
  sous forme d'un histogramme, en considérant des intervalles de classe
  d'une demi-heure, et indiquer la moyenne et l'écart-type de ces scores de
  différence.
\item[(c)] Vérifier l'exactitude des conclusions à l'aide d'un test de Student.
\end{description}
\end{exo}
\vskip1em

\begin{exo}\label{exo:9.3}
{\footnotesize Identique à l'énoncé 3.4 (p.~\pageref{exo:3.4}), questions
  a–d.}

Dans un essai clinique, on a cherché à évaluer un régime supposé réduire le
nombre de symptômes associé à une maladie bénigne du sein. Un groupe de 229
femmes ayant cette maladie ont été alétoirement réparties en deux
groupes. Le premier groupe a reçu les soins courants, tandis que les
patientes du second groupe suivaient un régime spécial (variable B =
traitement). Après un an, les individus ont été évalués et ont été classés
dans l'une des deux catégories : amélioration ou pas d'amélioration
(variable A = réponse). Les résultats sont résumés dans le tableau suivant,
pour une partie de l'échantillon :\autocite[p.~323]{selvin98}
\vskip1em

\begin{tabular}{l|cc|r}
& régime & pas de régime & total \\
\hline
amélioration & 26 & 21 & 47 \\
pas d'amélioration & 38 & 44 & 82 \\
\hline
total & 64 & 65 & 129
\end{tabular}
\vskip1em

\begin{description}
\item[(a)] Réaliser un test du chi-deux.  
\item[(b)] Quels sont les effectifs théoriques attendus sous une hypothèse
  d'indépendance ?
\item[(c)] Comparer les résultats obtenus en (a) avec ceux d'un test de
  Fisher.
\item[(d)] Donner un intervalle de confiance pour la différence de
  proportion d'amélioration entre les deux groupes de patientes.
\end{description}
\end{exo}
\vskip1em

\begin{exo}\label{exo:9.4}
{\footnotesize Identique à l'énoncé 3.5 (p.~\pageref{exo:3.5}), questions
  a–d.}

Dans un essai clinique, 1360 patients ayant déjà eu un infarctus dy myocarde
ont été assignés à l'un des deux groupes de traitement suivants : prise en
charge par aspirine à faible dose en une seule prise \emph{versus}
placebo. La table ci-après indique le nombre de décès par infarctus lors de
la période de suivi de trois ans :\autocite[p.~72]{agresti02} 
\vskip1em

\begin{tabular}{lccc}
\toprule
& \multicolumn{2}{c}{Infarctus} & \\
\cmidrule(r){2-3}
& Oui & Non & Total \\
\midrule
Placebo & 28 & 656 & 684 \\
Aspirine & 18 & 658 & 676 \\
\bottomrule
\end{tabular}
\vskip1em

\begin{description}
\item[(a)] Calculer la proportion d'infarctus du myocarde dans les deux
  groupes de patients.
\item[(b)] Représenter graphiquement le tableau précédent sous forme d'un
  diagramme en barres ou d'un diagramme en points ("dotplot" de Cleveland).
\item[(c)] Indiquer la valeur de l'odds-ratio ainsi que du risque
  relatif. Pour l'odds-ratio, on considérera comme catégories de référence
  les modalités représentées par la première ligne et la première colonne du
  tableau.  
\item[(d)] À partir de l'intervalle de confiance à 95~\% pour l'odds, quelle
  conclusion peut-on tirer sur l'effet de l'aspirine dans la prévention d'un
  infarctus du myocarde ?
\end{description}
\end{exo}
\vskip1em

\begin{exo}\label{exo:9.5}
{\footnotesize Identique à l'énoncé 4.1 (p.~\pageref{exo:4.1}), questions
  a–d.} 

Dans une étude sur le gène du récepteur à \oe strogènes, des généticiens se
sont intéressés à la relation entre le génotype et l'âge de diagnostic du
cancer du sein. Le génotype était déterminé à partir des deux allèles d'un
polymorphisme de restriction de séquence (1.6 et 0.7 kb), soit trois groupes
de sujets : patients homozygotes pour l'allèle 0.7 kb (0.7/0.7), patients
homozygotes pour l'allèle 1.6 kb (1.6/1.6), et patients hétérozygotes
(1.6/0.7). Les données ont été recueillies sur 59 patientes atteintes d'un
cancer du sein, et sont disponibles dans le fichier
\texttt{polymorphism.dta} (fichier \Stata). Les données moyennes sont
indiquées ci-dessous :\autocite[p.~327]{dupont09}
\vskip1em

\begin{tabular}{lrrrr}
\toprule
& \multicolumn{3}{c}{Génotype} & \\
\cmidrule(r){2-4}
& 1.6/1.6 & 1.6/0.7 & 0.7/0.7 & Total \\
\midrule
Nombre de patients & 14 & 29 & 16 & 59 \\
\emph{Âge lors du diagnostic} & & & & \\
\quad Moyenne & 64.64 & 64.38 & 50.38 & 60.64 \\
\quad Écart-type & 11.18 & 13.26 & 10.64 & 13.49 \\
\quad IC 95~\% & (58.1–71.1) & (59.9–68.9) & (44.3–56.5) & \\
\bottomrule
\end{tabular}
\vskip1em

\begin{description}
\item[(a)] Tester l'hypothèse nulle selon laquelle l'âge de diagnostic ne varie
  pas selon le génotype à l'aide d'une ANOVA. Représenter sous forme
  graphique la distribution des âges pour chaque génotype.
\item[(b)] Les intervalles de confiance présentés dans le tableau ci-dessus ont
  été estimés en supposant l'homogénéité des variances, c'est-à-dire en
  utilisant l'estimé de la variance commune ; donner la valeur de ces
  intervalles de confiance sans supposer l'homoscédasticité. 
\item[(c)] Estimer les différences de moyenne correspondant à l'ensemble des
  combinaisons possibles des trois génotypes, avec une estimation de
  l'intervalle de confiance à 95~\% associé et un test paramétrique
  permettant d'évaluer le degré de significativité de la différence
  observée.
\item[(d)] Représenter graphiquement les moyennes de groupe avec des
  intervalles de confiance à 95~\%.
\end{description}
\end{exo}
\vskip1em

\begin{exo}\label{exo:9.6}
{\footnotesize Identique à l'énoncé 4.2 (p.~\pageref{exo:4.2}), questions
  a–c.}

On a mesuré en fin de traitement chez 18 patients répartis par tirage au
sort en trois groupes de traitement A, B, et C, un paramètre biologique dont
on sait que la distribution est normale. Les résultats sont les suivants :
\vskip1em

\begin{tabular}{ccc}
\toprule
A & B & C \\
\midrule
19.8 & 15.9 & 15.4 \\
20.5 & 19.7 & 17.1 \\
23.7 & 20.8 & 18.2 \\
27.1 & 21.7 & 18.5 \\
29.6 & 22.5 & 19.3 \\
29.9 & 24.0 & 21.2 \\
\bottomrule
\end{tabular}
\vskip1em

\begin{description}
\item[(a)] Réaliser une ANOVA à un facteur.
\item[(b)] Selon le résultat du test, procéder aux comparaisons par paire de
  traitement des moyennes, en appliquant une correction simple de Bonferroni
  (c'est-à-dire où les degrés de significativité estimé sont multipliés par
  le nombre de comparaisons effectuées). Comparer avec de simples tests de
  Student non corrigés pour les comparaisons multiples. 
\item[(c)] D'après des études plus récentes, il s'avère que la normalité des
  distributions parentes peut-être remise en question. Effectuer la
  comparaison des trois groupes par une approche non-paramétrique.
\end{description}
\end{exo}
\vskip1em

\begin{exo}\label{exo:9.7}
{\footnotesize Identique à l'énoncé 4.3 (p.~\pageref{exo:4.3}), questions
  a–e.}

Un service d'obstétrique s'intéresse au poids de nouveaux-nés nés à terme et
âgés de 1 mois. Pour cet échantillon de 550 bébés, on dispose également
d'une information concernant la parité (nombre de frères et soeurs), mais on
sait qu'il n'y aucune relation de gemellité parmi les enfants ayant des
frères et soeurs. L'objet de l'étude est de déterminer si la parité (4
classes) influence le poids des nouveaux-nés à 1 mois. Les données sont
résumées dans le tableau suivant, et elles sont disponibles dans un fichier
SPSS, \texttt{weights.sav}.\autocite[p.~113]{peat05}
\vskip1em

\begin{tabular}{lrrrrr}
\toprule
& \multicolumn{4}{c}{Nombre de frères et soeurs} & Total \\
& 0 & 1 & 2 & $\ge 3$ & \\
\midrule
\emph{Échantillon} & & & & \\ 
Effectif & 180 & 192 & 116 & 62 & 550 \\
Fréquence & 32.7 & 34.9 & 21.1 & 11.3 & 100.0 \\
\emph{Poids (kg)} & & & & \\
Moyenne & 4.26 & 4.39 & 4.46 & 4.43 & \\
Écart-type & 0.62 & 0.59 & 0.61 & 0.54 & \\
(Min–Max) & (2.92–5.75) & (3.17–6.33) & (3.09–6.49) & (3.20–5.48) & \\
\bottomrule
\end{tabular}
\vskip1em

\begin{description}
\item[(a)] Vérifier les données reportées dans le tableau précédent.
\item[(b)] Procéder à une analyse de variance à un facteur. Conclure sur la
  significativité globale et indiquer la part de variance expliquée par le
  modèle.
\item[(c)] Afficher la distribution des poids selon la parité. Procéder à un
  test d'homogénéité des variances (rechercher dans l'aide en ligne le test
  de Levenne). 
\item[(d)] On décide de regrouper les deux dernières catégories (2 et $\ge
  3$). Refaire l'analyse et comparer aux résultats obtenus en (b).
\item[(e)] Réaliser un test de tendance linéaire (par ANOVA) sur les données
  recodées en trois niveaux pour la parité.
\end{description}
\end{exo}

%--------------------------------------------------------------- Devoir 08 ---
\chapter*{Devoir \no 8}
\addcontentsline{toc}{chapter}{Devoir \no 8}

Les exercices sont indépendants. Une seule réponse est correcte pour chaque
question. Lorsque vous ne savez pas répondre, cochez la case correspondante.

\section*{Exercice 1}
Soient les données d'une étude transversale s'intéressant à la transmission
du VIH dans des couples hétérosexuels monogames suite à la transmission de
sang contaminé chez le partenaire male \citep{obrien94}. Dans le tableau
ci-dessous, les colonnes représentent la présence ou absence du virus chez
l'homme et les lignes le diagnostic noté chez la partenaire.
\vskip1em

\begin{tabular}{lccc}
\toprule
& \multicolumn{2}{c}{Diagnostif partenaire} & \\
\cmidrule(r){2-3}
& Positif & Négatif & Total \\
\midrule
HIV+ & 3 & 4 & 7 \\
HIV- & 2 & 22 & 24 \\
\midrule
Total & 5 & 26 & 31 \\
\bottomrule
\end{tabular}
\vskip1em

Les données sous Stata, avant et après assignation d'étiquettes pour les
deux variables qualitatives, sont reportées dans le listing suivant :
\begin{verbatim}
. list in 1/4

     +--------------------+
     | Homme   Femme    N |
     |--------------------|
  1. |     1       1    3 |
  2. |     1       0    2 |
  3. |     0       1    4 |
  4. |     0       0   22 |
     +--------------------+
. list in 1/4

     +----------------------+
     |   Homme   Femme    N |
     |----------------------|
  1. | Positif    HIV+    3 |
  2. | Positif    HIV-    2 |
  3. | Négatif    HIV+    4 |
  4. | Négatif    HIV-   22 |
     +----------------------+
\end{verbatim}
\begin{description}
\item[\bf 1.1] Quelle commande permettrait de retrouver l'effectif total ?
  \marginpar{1.1 $\square$} 
  \begin{description}
  \item[A.] \verb|gen total=sum(N)|
  \item[B.] \verb|egen total=sum(N)|
  \item[C.] \verb|gen total=sum N|
  \item[D.] \verb|egen total=sum N|
  \item[E.] Je ne sais pas.
  \end{description}  
\item[\bf 1.2] On souhaite calculer la proportion de couples pour lesquels
  chaque partenaire a un diagnostique positif. Quelle commande doit-on
  utiliser ? \marginpar{1.2 $\square$} 
  \begin{description}
  \item[A.] \verb|tabulate Femme Homme [fw=N], row|
  \item[B.] \verb|tabulate Femme Homme [fw=N], col|
  \item[C.] \verb|tabulate Femme Homme [fw=N], cell|
  \item[D.] \verb|tabulate Femme Homme [fw=N], tot|
  \item[E.] Je ne sais pas.
  \end{description}  
\item[\bf 1.3] Quelle commande permet de reproduire la figure suivante
  (indépendemment du rapport hauteur/largeur et de la couleur) ?
  \marginpar{1.3 $\square$} 
  \begin{center}
    \includegraphics{./figs/dev8_barchart}
  \end{center}
  \begin{description}
  \item[A.] \verb|graph bar (asis) N, col over(Femme) asyvars over(Homme)|
  \item[B.] \verb|graph bar (asis) N, row over(Femme) asyvars over(Homme)|
  \item[C.] \verb|graph bar (asis) N, over(Femme) asyvars over(Homme)|
  \item[D.] \verb|graph bar (asis) N, over(Femme) asyvars percent over(Homme)|
  \item[E.] Je ne sais pas.
  \end{description}  
\item[\bf 1.4] Quelle commande fournit un résumé numérique équivalent aux
  chiffres présentés dans la figure précédente ? \marginpar{1.4 $\square$} 
  \begin{description}
  \item[A.] \verb|tabulate Femme Homme [fw=N], row|
  \item[B.] \verb|tabulate Femme Homme [fw=N], col|
  \item[C.] \verb|tabulate Femme Homme [fw=N], cell|
  \item[D.] \verb|tabulate Femme Homme [fw=N], tot|
  \item[E.] Je ne sais pas.
  \end{description}  
\item[\bf 1.5] On souhaite comparer les résultats d'un test du $\chi^2$ avec
  celui d'un test exact de Fisher pour vérifier l'existence d'une
  association entre les variables \texttt{Homme} et \texttt{Femme}. Quelle
  commande peut-on utiliser ? \marginpar{1.5 $\square$}
  \begin{description}
  \item[A.] \verb|tabulate Femme Homme [fw=N], nofreq clrchi2 exact|
  \item[B.] \verb|tabulate Femme Homme [fw=N], cchi2 exact|
  \item[C.] \verb|tabulate Femme Homme [fw=N], nofreq chi2 exact|
  \item[D.] Je ne sais pas.
  \end{description}  
\item[\bf 1.6] Quelle commande peut-on utiliser pour calculer l'odds-ratio
  et un intervalle de confiance à 95~\% ? \marginpar{1.6 $\square$}
  \begin{description}
  \item[A.] \verb|tabulate Femme Homme [fw=N], or|
  \item[B.] \verb|tabodds Femme Homme [fw=N], or|
  \item[C.] \verb|tabi 3 4\ 2 22, or|
  \item[D.] \verb|cci 3 4\ 2 22|
  \item[D.] Je ne sais pas.
  \end{description}  
\end{description}

\section*{Exercice 2}
Avec les données décrites à l'exercice~3 du devoir \no~7
(\pageref{dev7:exo3}), dont un aperçu est fourni ci-après, on souhaite
vérifier l'homogénéité du volume total de plaquettes entre les traitements A
(bras actif) et B (placebo) en début d'étude, ainsi qu'aux variations
intra-individuelles entre le début et la fin de l'étude pour chacun des deux
traitements.
\begin{verbatim}
. list in 1/5

     +------------------------------------------+
     | centre   patient      pre     post   trt |
     |------------------------------------------|
  1. |      1         1   359000   396000     B |
  2. |      1         2   200000   184000     A |
  3. |      1         3   149000   151000     A |
  4. |      1         4   235000   242000     B |
  5. |      1         5   174000   177000     B |
     +------------------------------------------+
\end{verbatim}
\begin{description}
\item[\bf 2.1] Pour représenter la distribution des volumes de plaquette
  (\texttt{pre}) sous forme d'histogramme pour chacun des traitements
  (\texttt{trt}), on peut utiliser la commande suivante : 
  \marginpar{2.1 $\square$}
\begin{verbatim}
. histogram pre, over(trt)
\end{verbatim}
  \begin{description}
  \item[A.] Vrai.
  \item[B.] Faux.
  \item[C.] Je ne sais pas.
  \end{description}
\item[\bf 2.2] On souhaite réaliser un test de Student pour vérifier que les
  deux groupes de patients définis par la variable \texttt{trt} sont
  comparables du point de leur volume total moyen de plaquettes. Quelle
  commande doit-on utiliser ?  \marginpar{2.2 $\square$}
  \begin{description}
  \item[A.] \verb|ttest pre, over(trt)|
  \item[B.] \verb|ttest pre, by(trt)|
  \item[C.] Je ne sais pas.
  \end{description}
\item[\bf 2.3] On s'intéresse à la variation du volume total de plaquettes
  en début et sortie d'étude pour le traitement A. On propose de réaliser un
  test de Student. Quelle commande est la plus appropriée ? \marginpar{2.3 $\square$}
  \begin{description}
  \item[A.] \verb|ttest post == pre if trt == "A"|
  \item[B.] \verb|ttest post == pre if trt == "A", unpaired|
  \item[C.] Je ne sais pas.
  \end{description}
\item[\bf 2.4] Voici le résultat d'une commande permettant de calculer un
  intervalle de confiance à 95\% pour le volume moyen de plaquette pour le
  traitement A : \marginpar{2.4 $\square$}
\begin{verbatim}
. ci pre if trt == "A"

    Variable |        Obs        Mean    Std. Err.       [95% Conf. Interval]
-------------+---------------------------------------------------------------
         pre |         15    223066.7    15219.18        190424.8
255708.6
\end{verbatim}
  Les bornes de l'intervalle de confiance présentées ci-dessus sont-elles
  identiques à celles que renverrait la commande \texttt{ttest} utilisée
  dans la question 2.3 ?
  \begin{description}
  \item[A.] Oui.
  \item[B.] Non.
  \item[C.] Je ne sais pas.
  \end{description}
\end{description}

\section*{Exercice 3}
Un investigateur s'intéressant à la fonction respiratoire décide
d'enregistrer le volume expiratoire maximum seconde (mesuré en litres) chez
des sujets fumeurs et non fumeurs. Quatre catégories sont définies \emph{a
  priori} : les non fumeurs, les anciens fumeurs, les nouveaux fumeurs et
les fumeurs de longue date. Un ensemble de 6 personnes est tiré au sort dans
chaque catégorie \citep[p.~34]{mickey04}. Les données présentées dans le
tableau suivant sont disponibles dans un fichier appelé \texttt{vems.dta}.
\vskip1em

\begin{tabular}{lcccccccc}
\toprule
Observation & 1 & 2 & 3 & 4 & 5 & 6 & Moy & Var \\
\midrule
Non fumeurs      & 4.41 & 4.96 & 3.50 & 3.66 & 4.68 & 4.11 & 4.22 & 0.33 \\
Anciens fumeurs  & 3.69 & 3.90 & 3.82 & 4.08 & 3.76 & 4.38 & 3.94 & 0.06 \\
Nouveaux fumeurs & 3.54 & 4.40 & 3.28 & 2.28 & 3.34 & 3.92 & 3.46 & 0.51 \\
Fumeurs          & 2.98 & 2.95 & 2.15 & 3.41 & 3.97 & 3.86 & 3.22 & 0.46 \\
\midrule
Ensemble         & & & & & & & 3.71 & 0.34 \\
\bottomrule
\end{tabular}
\vskip1em

Un aperçu des données après importation sous \Stata est fourni ci-après.
\begin{verbatim}
. list in 1/5

     +--------------------+
     | VEMS     categorie |
     |--------------------|
  1. | 4.41   Non fumeurs |
  2. | 4.96   Non fumeurs |
  3. |  3.5   Non fumeurs |
  4. | 3.66   Non fumeurs |
  5. | 4.68   Non fumeurs |
     +--------------------+
. codebook categorie, compact

Variable   Obs Unique  Mean  Min  Max  Label
-------------------------------------------------------------------------------
categorie   24      4   2.5    1    4  Statut fumeur
-------------------------------------------------------------------------------
\end{verbatim}
\begin{description}
\item[\bf 3.1] Quelle commande doit-on utiliser pour calculer la moyenne et
  la variance dans chaque groupe ? \marginpar{3.1 $\square$}
  \begin{description}
  \item[A.] \verb|summarize VEMS, by(categorie)|
  \item[B.] \verb|by categorie, summarize VEMS|
  \item[C.] \verb|tabstat VEMS, by(categorie) stats(mean sd)|
  \item[D.] \verb|by categorie: tabstat VEMS, mean sd|
  \item[E.] Je ne sais pas.
  \end{description}  
\item[\bf 3.2] Quelle commande permet de reproduire la figure suivante
  (indépendemment du rapport largeur/hauteur et de la couleur) ?
  \marginpar{3.2 $\square$}
\begin{center}
  \includegraphics{./figs/dev8_vemshisto}
\end{center}
\begin{description}
\item[A.] \verb|histogram VEMS|
\item[B.] \verb|histogram VEMS, freq|
\item[C.] \verb|histogram VEMS, discrete|
\item[D.] \verb|histogram VEMS, over(categorie)|
\item[E.] Je ne sais pas.
\end{description}
\item[\bf 3.3] Quelle commande permet de reproduire la figure suivante
  (indépendemment du rapport largeur/hauteur et de la couleur) ?
  \marginpar{3.3 $\square$}
\begin{center}
  \includegraphics{./figs/dev8_vemsdot}
\end{center}
\begin{description}
\item[A.] \verb|dotplot VEMS, over(categorie)|
\item[B.] \verb|dotplot VEMS, by(categorie)|
\item[C.] \verb|stripplot VEMS, over(categorie)|
\item[D.] \verb|stripplot VEMS, by(categorie)|
\item[E.] Je ne sais pas.
\end{description}
\item[\bf 3.4] On souhaite tester le rapport des deux variances les plus
  extrêmes (anciens et nouveaux fumeurs). Quelle commande permet de répondre
  à cette question ? \marginpar{3.4 $\square$}
\begin{description}
\item[A.] \verb+sdtest VEMS if categorie in 2 | categorie in 3+
\item[B.] \verb+sdtest VEMS if categorie in 2 & categorie in 3+
\item[C.] \verb+sdtest VEMS if categorie == 2 & categorie == 3, by(categorie)+
\item[D.] \verb+sdtest VEMS if categorie == 2 | categorie == 3, by(categorie)+
\item[E.] Je ne sais pas.
\end{description}
\item[\bf 3.5] Quelle commande doit-on utiliser pour réaliser une ANOVA à un
  facteur de classification ? \marginpar{3.5 $\square$}
\begin{description}
\item[A.] \verb|oneway VEMS, by(categorie)|
\item[B.] \verb|oneway VEMS, over(categorie)|
\item[C.] \verb|oneway VEMS categorie|
\item[D.] Je ne sais pas.
\end{description}
\item[\bf 3.6] On souhaite réaliser un test de Levene pour vérifier
  l'hypothèse d'homogénéité des variances. Quelle commande doit-on utiliser
  ? \marginpar{3.6 $\square$}
\begin{description}
\item[A.] \verb|vartest VEMS, by(categorie)|
\item[B.] \verb|robtest VEMS, by(categorie)|
\item[C.] \verb|robvar VEMS, by(categorie)|
\item[D.] Je ne sais pas.
\end{description}
\item[\bf 3.7] On peut retrouver le résultat du test $F$ à partir d'une
  approche par régression linéaire. La commande suivante permet d'obtenir un
  tel résultat. \marginpar{3.7 $\square$}
\begin{verbatim}
. regress VEMS i.categorie
\end{verbatim}
\begin{description}
\item[A.] Vrai.
\item[B.] Faux.
\item[C.] Je ne sais pas.
\end{description}
\item[\bf 3.8] En supposant la commande précédente correctement spécifiée,
  on souhaite calculer un intervalle de confiance à 90~\% pour la moyenne
  des fumeurs (4\ieme\ groupe). Quelle commande doit-on utiliser ?
  \marginpar{3.8 $\square$}
\begin{description}
\item[A.] \verb|lincom 4.categorie|
\item[B.] \verb|lincom 4.categorie - 1.categorie|
\item[C.] \verb|lincom 4.categorie + _cons|
\item[D.] Je ne sais pas.
\end{description}
\end{description}

%--------------------------------------------------------------- Chapter 10 ---
\chapter{Régression linéaire et logistique}

\begin{exo}\label{exo:10.1}
{\footnotesize Identique à l'énoncé 5.1 (p.~\pageref{exo:5.1}), questions
  a–e.}

Une étude a porté sur une mesure de malnutrition chez 25 patients âgés de 7
à 23 ans et souffrant de fibrose kystique. On disposait pour ces patients de
différentes informations relatives aux caractéristiques antropométriques
(taille, poids, etc.) et à la fonction pulmonaire. \autocite[p.~180]{everitt01}
Les données sont disponibles dans le fichier \texttt{cystic.dat}.
\begin{description}
\item[(a)] Calculer le coefficient de corrélation linéaire entre les
  variables \texttt{PEmax} et \texttt{Weight}, ainsi que son intervalle de
  confiance à 95~\%.
\item[(b)] Tester si le coefficient de corrélation calculé en (a) peut être
  considéré comme significativement différent de 0.3 au seuil 5~\%.
\item[(c)] Afficher l'ensemble des données numériques sous forme de
  diagrammes de dispersion, soit 45 graphiques arrangés sous forme d'une
  "matrice de dispersion".
\item[(d)] Calculer l'ensemble des corrélations de Pearson et de Spearman
  entre les variables numériques. 
  %Reporter les coefficients de
  %Bravais-Pearson supérieurs à 0.7 en valeur absolue.
\item[(e)] Calculer la corrélation entre \texttt{PEmax} et \texttt{Weight},
  en contrôlant l'âge (\texttt{Age}) (corrélation partielle). Représenter
  graphiquement la covariation entre \texttt{PEmax} et \texttt{Weight} en
  mettant en évidence les deux terciles les plus extrêmes pour la variable
  \texttt{Age}. 
\end{description}
\end{exo}
\vskip1em

\begin{exo}\label{exo:10.2}
{\footnotesize Identique à l'énoncé 5.2 (p.~\pageref{exo:5.2}), questions
  a–e.}

Les données disponibles dans le fichier \texttt{quetelet.csv} renseignent
sur la pression artérielle systolique (\texttt{PAS}), l'indice de Quetelet
(\texttt{QTT}), l'âge (\texttt{AGE}) et la consommation de tabac
(\texttt{TAB}=1 si fumeur, 0 sinon) pour un échantillon de 32 hommes de plus
de 40 ans. 
\begin{description}
\item[(a)] Indiquer la valeur du coefficient de corrélation linéaire entre
  la pression artérielle systolique et l'indice de Quetelet, avec un
  intervalle de confiance à 90~\%.
\item[(b)] Donner les estimations des paramètres de la droite de régression
  linéaire de la pression artérielle sur l'indice de Quetelet.
\item[(c)] Tester si la pente de la droite de régression est différente de 0
  (au seuil 5~\%).
\item[(d)] Représenter graphiquement les variations de pression artérielle
  en fonction de l'indice de Quetelet, en faisant apparaître distinctement
  les fumeurs et les non-fumeurs avec des symboles ou des couleurs
  différentes, et tracer la droite de régression dont les paramètres ont été
  estimés en (b). 
\item[(e)] Refaire l'analyse (b-c) en restreignant l'échantillon aux
  fumeurs.
\end{description}
\end{exo}
\vskip1em

\begin{exo}\label{exo:10.3}
{\footnotesize Identique à l'énoncé 5.3 (p.~\pageref{exo:5.3}), questions
  a–d.}

Dans l'étude Framingham, on dispose de donnée sur la pression artérielle
systolique (\texttt{sbp}) et l'indice de masse corporelle (\texttt{bmi}) de
2047 hommes et 2643 femmes.\autocite[p.~63]{dupont09} On s'intéresse à la
relation entre ces deux variables (après transformation logarithmique) chez
les hommes et chez les femmes séparément.
Les données sont disponibles dans le fichier \texttt{Framingham.csv}.
\begin{description}
\item[(a)] Représenter graphiquement les variations entre pression
  artérielle et IMC (\texttt{bmi}) chez les hommes et chez les femmes.
\item[(b)] Les coefficients de corrélation linéaire estimés chez les hommes
  et chez les femmes sont-ils significativement différents à 5~\% ?
\item[(c)] Estimer les paramètres du modèle de régression linéaire
  considérant la pression artérielle comme variable réponse et l'IMC comme
  variable explicative, pour ces deux sous-échantillons. Donner un
  intervalle de confiance à 95~\% pour l'estimé des pentes respectives.
\item[(d)] Tester l'égalité des deux coefficients de régression associés à
  la pente (au seuil 5~\%).
\end{description}
\end{exo}
\vskip1em

\begin{exo}\label{exo:10.4}
{\footnotesize Identique à l'énoncé 6.1 (p.~\pageref{exo:6.1}) et répondre
  aux questions a–d.} 

On étudie l'effet d'un traitement prophylactique d'un macrolide à faibles
doses (Traitement A) sur les épisodes infectieux chez des patients atteints
de mucoviscidose dans un essai randomisé multicentrique contre placebo
(B). Les résultats sont les suivants :
\vskip1em

\begin{tabular}{lccc}
\toprule
& \multicolumn{2}{c}{Infection} & \\
\cmidrule(r){2-3}
& Non & Oui & Total \\
\midrule
Traitement (A) & 157 & 52 & 209 \\
Placebo (B) & 119 & 103 & 222 \\
Total & 276 & 155 & 431 \\
\bottomrule
\end{tabular}
\vskip1em

\begin{description}
\item[(a)] À partir d'un test du $\chi^2$, que peut-on répondre à la
  question : le traitement permet-il de prévenir la survenue d'épisodes
  infectieux (au seuil $\alpha=0.05$) ? Vérifier que les effectifs
  théoriques sont bien tous supérieurs à 5.
\item[(b)] Conclut-on de la même manière à partir de l'intervalle de
  confiance de l'odds-ratio associé à l'effet traitement ?
\item[(c)] On souhaite vérifier s'il existe une disparité du point de vue
  des pourcentages d'épisodes infectieux en fonction du centre. Les données
  par centre sont indiquées dans le tableau ci-après. Conclure à partir d'un
  test du $\chi^2$.

  \begin{table}[!htb] \hskip40pt
  \begin{minipage}[b]{0.33\linewidth}
  \scalebox{0.65}{\begin{tabular}{|l|r|r|r|}
    \multicolumn{1}{c}{} & \multicolumn{2}{c}{Infection} &  \multicolumn{1}{c}{} \\
    \cline{2-4}
    \multicolumn{1}{c|}{} & Non & Oui & Total \\
    \hline
    Traitement (A) & 51 & 8 & 59 \\
    \hline
    Placebo (B) & 47 & 19 & 66 \\
    \hline
    Total & 98 & 27 & 125 \\
    \hline
    \multicolumn{4}{c}{Centre 1}
  \end{tabular}} 
  \end{minipage} \hspace{0.1cm}
  \begin{minipage}[b]{0.3\linewidth}
  \scalebox{0.65}{\begin{tabular}{|l|r|r|r|}
    \multicolumn{1}{c}{} & \multicolumn{2}{c}{Infection} &  \multicolumn{1}{c}{} \\
    \cline{2-4}
    \multicolumn{1}{c|}{} & Non & Oui & Total \\
    \hline
    Traitement (A) & 91 & 35 & 126 \\
    \hline
    Placebo (B) & 61 & 71 & 132 \\
    \hline
    Total & 152 & 106 & 258 \\
    \hline
    \multicolumn{4}{c}{Centre 2}
  \end{tabular}} 
  \end{minipage} \hspace{0.1cm}
  \begin{minipage}[b]{0.3\linewidth}
  \scalebox{0.65}{\begin{tabular}{|l|r|r|r|}
    \multicolumn{1}{c}{} & \multicolumn{2}{c}{Infection} &  \multicolumn{1}{c}{} \\
    \cline{2-4}
    \multicolumn{1}{c|}{} & Non & Oui & Total \\
    \hline
    Traitement (A) & 15 & 9 & 24 \\
    \hline
    Placebo (B) & 11 & 13 & 24 \\
    \hline
    Total & 26 & 22 & 48 \\
    \hline
    \multicolumn{4}{c}{Centre 3}
  \end{tabular}}
  \end{minipage}
  \end{table}
\item[(d)] À partir du tableau précédent, on cherche à vérifier si l'effet
  traitement est indépendent du centre ou non. On se propose de réaliser un
  test de comparaison entre les deux traitements ajustés sur le centre (test
  de Mantel-Haenszel). Indiquer le résultat du test ainsi que la valeur de
  l'odds-ratio ajusté.
\end{description}
\end{exo}
\vskip1em

\begin{exo}\label{exo:10.5}
{\footnotesize Identique à l'énoncé 6.3 (p.~\pageref{exo:6.3}), questions
  a–e.}

On dispose de données issues d'une étude cherchant à établir la validité
pronostique de la concentration en créatine kinase dans l'organisme sur la
prévention de la survenue d'un infarctus du myocarde.\autocite[p.~115]{rabe-hesketh04}

Les données sont disponibles dans le fichier \texttt{sck.dat} : la première
colonne correspond à la variable créatine kinase (\texttt{ck}), la deuxième
à la variable présence de la maladie (\texttt{pres}) et la dernière à la
variable absence de maladie (\texttt{abs}).
\begin{description}
\item[(a)] Quel est le nombre total de sujets ?
\item[(b)] Calculer les fréquences relatives malades/non-malades, et
  représenter leur évolution en fonction des valeurs de créatine kinase à
  l'aide d'un diagramme de dispersion (points + segments reliant les points).
\item[(c)] À partir d'un modèle de régression logistique dans lequel on
  cherche à prédire la probabilité d'être malade, calculer la valeur de
  \texttt{ck} à partir de laquelle ce modèle prédit que les personnes
  présentent la maladie en considérant une valeur seuil de 0.5
  (si $P(\text{malade})\ge 0.5$ alors \texttt{malade=1}).
\item[(d)] Représenter graphiquement les probabilités d'être malade prédites
  par ce modèle ainsi que les proportions empiriques en fonction des valeurs
  \texttt{ck}. 
\item[(e)] Établir la courbe ROC correspondante, et reporter la
  valeur de l'aire sous la courbe. 
\end{description}
\end{exo}
\vskip1em

\begin{exo}\label{exo:10.6}
{\footnotesize Identique à l'énoncé 6.5 (p.~\pageref{exo:6.5}), questions
  a–d.} 

Une enquête cas-témoin a porté sur la relation entre la consommation
d'alcool et de tabac et le cancer de l'oesophage chez l'homme (étude "Ille
et Villaine"). Le groupe des cas était composé de 200 patients atteints d'un
cancer de l'oesophage et diagnostiqué entre janvier 1972 et avril 1974. Au
total, 775 témoins de sexe masculin ont été sélectionnés à partir des listes
électorales. Le tableau suivant indique la répartition de l'ensemble des
sujets selon leur consommation journalière d'alcool, en considérant qu'une
consommation supérieure à 80 g est considérée comme un facteur de
risque.\autocite{breslow80} 
\vskip1em

\begin{tabular}{lccc}
\toprule
& \multicolumn{2}{c}{Consommation d'alcool (g/jour)} & \\
\cmidrule(r){2-3}
& $\ge 80$ & $<80$ & Total \\
\midrule
Cas & 96 & 104 & 200 \\
Témoins & 109 & 666 & 775 \\
Total & 205 & 770 & 975 \\
\bottomrule
\end{tabular}
\vskip1em

\begin{description}
\item[(a)] Quelle est la valeur de l'odds-ratio et son intervalle de
  confiance à 95~\% (méthode de Woolf) ? Est-ce une bonne estimation du
  risque relatif ? 
\item[(b)] Est-ce que la proportion de consommateurs à risque est la même
  chez les cas et chez les témoins (considérer $\alpha=0.05$) ?
\item[(c)] Construire le modèle de régression logistique permettant de
  tester l'association entre la consommation d'alcool et le statut des
  sujets. Le coefficient de régression est-il significatif ?
\item[(d)] Retrouvez la valeur de l'odds-ratio observé, calculé en (b), et
  son intervalle de confiance à partir des résultats de l'analyse de
  régression.
\end{description}
\end{exo}

%--------------------------------------------------------------- Devoir 09 ---
\chapter*{Devoir \no 9}
\addcontentsline{toc}{chapter}{Devoir \no 9}

Les exercices sont indépendants. Une seule réponse est correcte pour chaque
question. Lorsque vous ne savez pas répondre, cochez la case correspondante.

\section*{Exercice 1}
Dans une étude clinique portant sur le niveau d'intelligence d'enfants
souffrant d'une malformation cardiaque, les investigateurs se sont
intéressés à la variation de QI avant et après une opération
chirurgicale. Les enfants n'étaient pas tirés au sort pour la chirurgie
\citep[p.~399]{mickey04}. Le résumé des données contenues dans le fichier
\texttt{chiriq.dat} est fourni dans le tableau ci-dessous :
\vskip1em

\begin{tabular}{lcccc}
\toprule
& \multicolumn{2}{c}{Sans chirurgie} & \multicolumn{2}{c}{Chirurgie} \\
\cmidrule{2-5}
& Avant & Après & Avant & Après \\
\midrule
Moyenne & 104.25 & 103.63 & 98.63 & 102.04 \\ 
(Ety) & (12.20) & (12.43) & (12.50) & (11.45) \\
\bottomrule
\end{tabular}
\vskip1em

Voici un aperçu des données importées sous \Stata :
\begin{verbatim}
. list in 1/5

     +---------------+
     |   y     x   g |
     |---------------|
  1. | 127   124   0 |
  2. |  82    95   0 |
  3. | 124   127   0 |
  4. | 111   125   0 |
  5. | 100    90   0 |
     +---------------+
\end{verbatim}
Les variables \texttt{x}, \texttt{y} et \texttt{g} désignent,
respectivement, le QI avant et après, et le groupe de traitement
(\texttt{g}=1 pour chirurgie). On fera dans tous les cas l'hypothèse que les
variances sont homogènes pour les quatre échantillons.
\begin{description}
\item[\bf 1.1] Quelle(s) option(s) faut-il rajouter à la commande suivante
  pour calculer les moyennes et écart-types des deux variables \texttt{x} et
  \texttt{y} par groupe de traitement ? \marginpar{1.1 $\square$}
\begin{verbatim}
. tabstat x y, 
\end{verbatim}
\begin{description}
\item[A.] \verb|over(g) stats(mean sd)|
\item[B.] \verb|by(g) stats(mean sd)|
\item[C.] Je ne sais pas.
\end{description}
\item[\bf 1.2] On souhaite vérifier à l'aide d'un test statistique que le QI
  des enfants en début d'étude était comparable entre les deux groupes (avec
  ou sans chirurgie). Quelle commande est la plus appropriée pour répondre à
  cette question ?  \marginpar{1.2 $\square$}
\begin{description}
\item[A.] \verb|by g, sort: ttest x == y|
\item[B.] \verb|ttest x == y if g == 0|
\item[C.] \verb|ttest x, by(g)|
\item[D.] Je ne sais pas.
\end{description}
\item[\bf 1.3] On souhaite afficher les QI individuels en début et fin
  d'étude pour les deux groupes de patients dans deux graphiques
  séparés. Quelle commande peut-on utiliser ? \marginpar{1.3 $\square$}
\begin{description}
\item[A.] \verb|by g, scatter y x|
\item[B.] \verb|twoway scatter y x, by(g)|
\item[C.] \verb|twoway scatter y x, over(g)|
\item[D.] \verb+scatter x y if g == 0 || scatter x y if g == 1+
\item[E.] Je ne sais pas.
\end{description}
\item[\bf 1.4] Quelle est la valeur du coefficient de corrélation linéaire
  entre les mesures avant après dans le groupe ayant subi une intervention
  chirurgicale ? \marginpar{1.4 $\square$}
\begin{description}
\item[A.] \verb|correlate x y, if g == 1|
\item[B.] \verb|correlate x y if g == 1|
\item[C.] \verb|by g, correlate x y|
\item[D.] Je ne sais pas.
\end{description}
\item[\bf 1.5] On souhaite estimer la pente de régression de \texttt{x} sur
  \texttt{y}, de manière indépendante pour les deux groupes de patients. La
  commande suivante est-elle correctement spécifiée ? \marginpar{1.5 $\square$}
\begin{verbatim}
. regress y x g
\end{verbatim}
\begin{description}
\item[A.] Oui.
\item[B.] Non.
\item[C.] Je ne sais pas.
\end{description}
\item[\bf 1.6]  Voici les résultats de la régression de \texttt{y} sur
  \texttt{x} dans le groupe des patients n'ayant pas subi d'intervention
  chirurgicale (\texttt{g}=0). \marginpar{1.6 $\square$}
\begin{verbatim}
      Source |       SS       df       MS              Number of obs =      24
-------------+------------------------------           F(  1,    22) =   12.52
       Model |  1289.31276     1  1289.31276           Prob > F      =  0.0018
    Residual |  2266.31224    22  103.014193           R-squared     =  0.3626
-------------+------------------------------           Adj R-squared =  0.3336
       Total |    3555.625    23  154.592391           Root MSE      =   10.15

------------------------------------------------------------------------------
           y |      Coef.   Std. Err.      t    P>|t|     [95% Conf. Interval]
-------------+----------------------------------------------------------------
           x |   .6135932   .1734403     3.54   0.002     .2539001    .9732863
       _cons |   39.65791   18.19946     2.18   0.040     1.914545    77.40127
------------------------------------------------------------------------------
\end{verbatim}
Quel résultat la série de commandes suivante permet-elle d'obtenir ?
\begin{verbatim}
. egen sx = sd(x) if g == 0
. egen sy = sd(y) if g == 0
. matrix b =e(b)
. svmat b
. di (b1*sx/sy)^2
\end{verbatim}
\begin{description}
\item[A.] La covariance entre les variables \texttt{x} et \texttt{y}.
\item[B.] Le coefficient de corrélation linéaire entre les variables
  \texttt{x} et \texttt{y}.
\item[C.] Le coefficient de détermination pour le modèle reporté ci-dessus. 
\item[D.] Je ne sais pas.
\end{description}
\item[\bf 1.7] La commande suivante permet de renvoyer la valeurs des
  résidus du modèle de régression précédent. \marginpar{1.7 $\square$}
\begin{verbatim}
. predict, residuals
\end{verbatim}
\begin{description}
\item[A.] Vrai.
\item[B.] Faux.
\item[C.] Je ne sais pas.
\end{description}
\end{description}

\section*{Exercice 2}
Considérons les données sur les poids à la naissance décrites à
l'exercice~8.6. On s'intéresse à la relation entre le poids à la naissance,
traité comme variable numérique (\texttt{bwt}) ou qualitative à deux classes
(\texttt{low}=1 si poids $<2.5$ kg). Parmi l'ensemble des facteurs de
risque, on se concentrera sur les antécedents d'hypertension (\texttt{ht}=1
si oui) et le status fumeur de la mère (\texttt{smoke}=1 si oui).

Les données ont été pré-traitées de la même manière, et on dispose donc de
la liste de variables suivantes sous \Stata :
\begin{verbatim}
. list in 1/5

     +---------------------------------------------------------------+
     | low   age   lwt    race   smoke   ptl   ht    ui   ftv    bwt |
     |---------------------------------------------------------------|
  1. |  no    19   182   Black      no     0   no   yes     0   2523 |
  2. |  no    33   155   Other      no     0   no    no     3   2551 |
  3. |  no    20   105   White     yes     0   no    no     1   2557 |
  4. |  no    21   108   White     yes     0   no   yes     2   2594 |
  5. |  no    18   107   White     yes     0   no   yes     0   2600 |
     +---------------------------------------------------------------+
\end{verbatim}
\begin{description}
\item[\bf 2.1] On souhaite afficher les trois tableaux de contingence
  croisant les variables \texttt{low}, \texttt{ht} et \texttt{smoke}. On
  propose d'utiliser la commande : \marginpar{2.1 $\square$}
\begin{verbatim}
. tabulate low ht smoke
\end{verbatim}
Cette commande fournit-elle le résultat escompté ?
\begin{description}
\item[A.] Oui.
\item[B.] Non.
\item[C.] Je ne sais pas.
\end{description}
\item[\bf 2.2] Si l'on souhaitait modéliser la relation entre le poids des
  bébés (en grammes, variable \texttt{bwt}) et les antécedents
  d'hypertension, on pourrait utiliser la commande : \marginpar{2.2 $\square$}
\begin{description}
\item[A.] \verb|oneway bwt i.ht|
\item[B.] \verb|regress bwt ht|
\item[C.] Je ne sais pas.
\end{description}
\item[\bf 2.3] Les deux commandes suivantes fournissent des résultats
  comparables. \marginpar{2.3 $\square$} 
\begin{verbatim}
. regress bwt i.ht
. regress bwt ht
\end{verbatim}
\begin{description}
\item[A.] Vrai.
\item[B.] Faux.
\item[C.] Je ne sais pas.
\end{description}
\item[\bf 2.4] Quelle commande devrait-on utiliser si au lieu d'une
  régression linéaire considérant comme variable réponse le poids des bébés
  comme variable numérique (\texttt{bwt}) on souhaitait travailler avec le
  poids des bébés en deux classes (\texttt{low}), la variable explicative
  restant inchangée ? \marginpar{2.4 $\square$}
\begin{description}
\item[A.] \verb|logistic low ht|
\item[B.] \verb|logis low ht|
\item[C.] \verb|blogit low ht|
\item[D.] Je ne sais pas.
\end{description}
\item[\bf 2.5] Est-il possible à l'aide de la commande précédente d'obtenir
  directement une interprétation de l'effet du facteur d'intérêt
  (\texttt{ht}) en termes d'odds-ratio ? \marginpar{2.5 $\square$}
\begin{description}
\item[A.] Oui.
\item[B.] Non.
\item[C.] Je ne sais pas.
\end{description}
\end{description}

\section*{Exercice 3}
Dans l'essai NINDS, qui est un essai clinique randomisé en double aveugle
avec groupes parallèles : placebo contre rt-PA (activateur plasminogène
recombinant de tissue), on s'est intéressé, d'une part, à l'activité
clinique de rt-PA en termes d'amélioration clinique précoce par rapport au
placebo (phase 1), et, d'autre part, à la proportion de répondeurs avec
récupération intégrale trois mois après traitement entre le groupe rt-PA et
le placebo \cite{chow04}. Une randomisation stratifiée a été retenue, avec
une première stratification sur le centre puis par rapport à la durée entre
le début de la crise et le début du traitement. On distingue donc deux
variables essentielles : la phase de l'essai (1 et 2) et la durée (0–90 ou
91–180 min). Le critère de jugement (amélioration de l'état clinique du
patient) est défini comme une variable binaire (amélioration oui/non) basée
sur un score de sévérité du déficit neurologique.

Voici les données disponibles concernant le nombre de patients pour lesquels
a été observée une amélioration (/effectif total) selon le critère mentionné
ci-dessus : 
\vskip1em
\begin{tabular}{lcrr}
\toprule
  Phase & Durée & rt-PA & Placebo \\
\midrule
  1 & 0–90 & 36/71 & 31/68\\
  1 & 91–180 & 31/73 & 26/79\\
  2 & 0–90 & 51/86 & 30/77\\
  2 & 91–180 & 29/82 & 35/88\\
\bottomrule
\end{tabular}
\vskip1em

\begin{description}
\item[\bf 3.1] De quelle manière est-il le plus naturel et le plus simple de
  saisir ces données sous \Stata afin de réaliser une régression logistique
  en considérant comme variable réponse la réponse au traitement et comme
  variable explicative n'importe lequel des autres facteurs d'étude) ?
  \marginpar{3.1 $\square$} 
\begin{description}
\item[A.] Une liste de 4 variables, \texttt{phase}, \texttt{duree},
  \texttt{rt-PA} et \texttt{Placebo} contenant les effectifs (nombre de
  répondeurs) tels que présentés dans le tableau précédent.
\item[B.] Une liste de 4 variables, \texttt{phase}, \texttt{duree},
  \texttt{rt-PA} et \texttt{Placebo} contenant les fréquences (nombre de
  répondeurs rapporté au total par strate) tels que présentés dans le
  tableau précédent. 
\item[C.] Une liste de 4 variables, \texttt{phase}, \texttt{duree}, type
  de traitement, et les effectifs (nombre de répondeurs) tels que
  présentés dans le tableau précédent.
\item[D.] Une liste de 5 variables, \texttt{phase}, \texttt{duree}, type
  de traitement, et les effectifs (nombre de répondeurs) ainsi que les
  effectifs totaux par strates tels que présentés dans le tableau précédent.
\item[E.] Je ne sais pas.
\end{description}
\item[\bf 3.2] Supposons que l'on ait choisi la représentation suivante :
\begin{verbatim}
. list

     +---------------------------------------+
     | phase    duree        tx    N   Total |
     |---------------------------------------|
  1. |     1     0-90   Placebo   31      68 |
  2. |     1     0-90     rt-PA   36      71 |
  3. |     1   91-180   Placebo   26      79 |
  4. |     1   91-180     rt-PA   31      73 |
  5. |     2     0-90   Placebo   30      77 |
     |---------------------------------------|
  6. |     2     0-90     rt-PA   51      86 |
  7. |     2   91-180   Placebo   35      88 |
  8. |     2   91-180     rt-PA   29      82 |
     +---------------------------------------+
\end{verbatim}
  où \texttt{N} représentent le nombre de répondeurs pour chacune des
  strates considérées dans l'énoncé, et considérons les patients traité par
  rt-PA présentant un signe d'amélioration. Quelle commande permet de
  retrouver le nombre total de ces patients ? \marginpar{3.2 $\square$}
\begin{description}
\item[A.] \verb|tabulate N tx|
\item[B.] \verb|tabulate N if tx=="rt-PA"|
\item[C.] \verb|tabulate tx [fw=N]|
\item[D.] Je ne sais pas.
\end{description}
\item[\bf 3.3] Avec le même échantillon qu'à l'exercice précédent, on
  souhaite combiner les effets du traitement entre les différentes strates
  définie par la phase et la durée. Pour cela, on propose de vérifier
  l'homogénéité des odds-ratio entre les strates. Quelle commande permettrait de
  répondre à cette question (en supposant que la structure de données le
  permet) ?  \marginpar{3.3 $\square$}
\begin{description}
\item[A.] \verb|tabulate|
\item[B.] \verb|tab2|
\item[C.] \verb|cc|
\item[D.] Je ne sais pas.
\end{description}
\item[\bf 3.4] À présent, on souhaite modéliser la relation entre la réponse
  au traitement (en 0/1, telle que définie dans l'énoncé) et le traitement
  (rt-PA ou placebo). Quelle commande peut-on utiliser pour effectuer la
  régression logistique ? \marginpar{3.4 $\square$}
\begin{description}
\item[A.] \verb|logistic|
\item[B.] \verb|logistic| avec l'option \texttt{or}
\item[C.] \verb|blogit|
\item[D.] Je ne sais pas.
\end{description}
\item[\bf 3.5] Voici les résultats d'une régression logistique dans laquelle
  en plus de l'effet traitement on prend en considération la
  phase. \marginpar{3.5 $\square$} 
\begin{verbatim}
Logistic regression for grouped data              Number of obs   =        624
                                                  LR chi2(2)      =       4.14
                                                  Prob > chi2     =     0.1265
Log likelihood = -424.51098                       Pseudo R2       =     0.0048

------------------------------------------------------------------------------
    _outcome |      Coef.   Std. Err.      z    P>|z|     [95% Conf. Interval]
-------------+----------------------------------------------------------------
          tx |   .3271774   .1622558     2.02   0.044     .0091618     .645193
       phase |   .0350803    .162612     0.22   0.829    -.2836335     .353794
       _cons |  -.8238462   .3577782    -2.30   0.021    -1.525079   -.1226138
------------------------------------------------------------------------------
\end{verbatim}
Les valeurs reportées sous la colonne intitulée \texttt{Coef.} représentent
les odds-ratio ajustés associés aux deux facteurs \texttt{tx} et \texttt{phase}.
\begin{description}
\item[A.] Vrai.
\item[B.] Faux.
\item[C.] Je ne sais pas.
\end{description}
\end{description}


%--------------------------------------------------------------- Chapter 11 ---
\chapter{Analyse de données de survie}

\begin{exo}\label{exo:11.1}
{\footnotesize Identique à l'énoncé 7.1 (p.~\pageref{exo:7.1}), questions
  a–i.}

Dans un essai contre placebo sur la cirrhose biliaire, la D-penicillamine
(DPCA) a été introduite dans le bras actif sur une cohorte de 312
patients. Au total, 154 patients ont été randomisés dans le bras actif
(variable traitement, \texttt{rx}, 1=Placebo, 2=DPCA). Un ensemble de
données telles que l'âge, des données biologiques et signes cliniques variés
incluant le niveau de bilirubine sérique (\texttt{bilirub}) sont disponibles
dans le fichier \texttt{pbc.txt}.\autocite{vittinghoff05} Le status du
patient est enregistré dans la variable \texttt{status} (0=vivant, 1=décédé)
et la durée de suivi (\texttt{years}) représente le temps écoulé en années
depuis la date de diagnostic.
\begin{description}
\item[(a)] Combien dénombre-t-on d'individus décédés? Quelle proportion de
  ces décès retrouve-t-on dans le bras actif ?  
\item[(b)] Afficher la distribution des durées de suivi des 312 patients, en
  faisant apparaître distinctement les individus décédés. Calculer le temps
  médian (en années) de suivi pour chacun des deux groupes de
  traitement. Combien y'a-t-il d'événements positifs au-delà de 10.5 années
  et quel est le sexe de ces patients ?
\item[(c)] Les 19 patients dont le numéro (\texttt{number}) figure parmi la
  liste suivante ont subi une transplantation durant la période de suivi.
\begin{verbatim}  
5 105 111 120 125 158 183 241 246 247 254 263 264 265 274 288 291
295 297 345 361 362 375 380 383
\end{verbatim}   
  Indiquer leur âge moyen, la distribution selon le sexe et la durée médiane
  de suivi en jours jusqu'à la transplantation.
\item[(d)] Afficher un tableau résumant la distribution des événements à
  risque en fonction du temps, avec la valeur de survie associée.
\item[(e)] Afficher la courbe de Kaplan-Meier avec un intervalle de
  confiance à 95~\%, sans considérer le type de traitement.
\item[(f)] Calculer la médiane de survie et son intervalle de confiance à
  95~\% pour chaque groupe de sujets et afficher les courbes de survie
  correspondantes.
\item[(g)] Effectuer un test du log-rank en considérant comme prédicteur le
  facteur \texttt{rx}. Comparer avec un test de Wilcoxon.
\item[(h)] Effectuer un test du log-rank sur le facteur d'intérêt
  (\texttt{rx}) en stratifiant sur l'âge. On considèrera trois groupe
  d'âge : 40 ans ou moins, entre 40 et 55 ans inclus, plus de 55 ans.
\item[(i)] Retrouver les résultats de l'exercice 1.g avec une régression de
  Cox. 
\end{description}
\end{exo}
\vskip1em

\begin{exo}\label{exo:11.2}
{\footnotesize Identique à l'énoncé 7.3 (p.~\pageref{exo:7.3}), questions
  a–d}.

Dans un essai randomisé, on a cherché à comparer deux traitements pour le
cancer de la prostate. Les patients prenaient chaque jour par voie orale
soit 1 mg de diethylstilbestrol (DES, bras actif) soit un placebo, et le
temps de survie est mesuré en mois.\autocite{collett94} La question
d'intérêt est de savoir si la survie diffère entre les deux groupes de
patients, et on négligera les autres variables présentes dans le fichier de
données \texttt{prostate.dat}. 
\begin{description}
\item[(a)] Calculer la médiane de survie pour l'ensemble des patients, et
  par groupe de traitement.
\item[(b)] Quelle est la différence entre les proportions de survie dans les
  deux groupes à 50 mois ?
\item[(c)] Afficher les courbes de survie pour les deux groupes de patients.
\item[(d)] Effectuer un test du log-rank pour tester l'hypothèse selon
  laquelle le traitement par DES a un effet positif sur la survie des
  patients. 
\end{description}
\end{exo}

%--------------------------------------------------------------- Devoir 10 ---
\chapter*{Devoir \no 10}
\addcontentsline{toc}{chapter}{Devoir \no 10}

Les exercices sont indépendants. Une seule réponse est correcte pour chaque
question. Lorsque vous ne savez pas répondre, cochez la case correspondante.

\section*{Exercice 1}
Le fichier \texttt{ovarian.csv} contient les données d'une étude sur des
patientes atteintes d'un cancer des ovaires en stade II ou IIIA. L'objectif
de l'étude était de déterminer si le stade était lié ou non à la progression
de la maladie. Les données sous \Stata sont reportées
ci-dessous :
\begin{verbatim}
. list in 1/5

     +--------------------------+
     |  time   censored   grade |
     |--------------------------|
  1. |   .92         no     low |
  2. |  2.93         no     low |
  3. |  5.76         no     low |
  4. |  6.41         no     low |
  5. | 10.16         no     low |
     +--------------------------+
\end{verbatim}
On dispose donc des données sur la durée (en mois) avant évolution du cancer
(\texttt{time}) et d'un indicateur sur la censure éventuelle de
l'observation (\texttt{censored}). Le stade du cancer est reporté dans la
variable \texttt{grade} (\texttt{low}=II, \texttt{high}=IIIA).
\begin{description}
\item[\bf 1.1] Pour convertir les données au format données de survie sous
  \Stata, on peut utiliser la commande : \marginpar{1.1 $\square$}
\begin{verbatim}
. ftset time, failure(censored)
\end{verbatim}
\begin{description}
\item[A.] Vrai.
\item[B.] Faux.
\item[C.] Je ne sais pas.
\end{description}
\item[\bf 1.2] Supposons que l'on souhaite convertir les valeurs prises par
  la variable \texttt{censored} en 1/2 (au lieu de
  \texttt{yes}/\texttt{no}), et stocker ces nouveaux résultats dans une
  variable appelée \texttt{cens}. Quelle commande peut-on utiliser ?
\marginpar{1.2 $\square$}
\begin{description}
\item[A.] \verb|mvencode censored, mv("yes"=1 \ "no"=2)|
\item[B.] \verb|encode censored, generate(cens)|
\item[C.] \verb|recode censored ("yes"=1) ("no"=2), generate(cens)|
\item[D.] \verb|label drop censored| suivi de \verb|gen cens = censored|
\item[E.] Je ne sais pas.
\end{description}
\item[\bf 1.3] On suppose par la suite que les informations de censure sont
  accessibles dans la nouvelle variable \texttt{cens} créée à l'exercice
  précédent, et que l'on a supprimé les anciennes variables de survie
  générées à l'exercice~1.1. Quelle commande peut-on utiliser pour convertir
  les données en données de survie en utilisant la variable \texttt{cens}
  comme indicateur de censure ? \marginpar{1.3 $\square$} 
\begin{description}
\item[A.] \verb|stset time, failure(cens)|
\item[B.] \verb|stset time, failure(cens 1)|
\item[C.] \verb|stset time, failure(cens = 1)|
\item[D.] Je ne sais pas.
\end{description}
\item[\bf 1.4] Quelle commande permet d'afficher une table de mortalité pour
  le jeu de données, dans chacun des deux groupes de patients définis par la
  variable \texttt{grade} ? \marginpar{1.4 $\square$} 
\begin{description}
\item[A.] \verb|sts list|
\item[B.] \verb|sts list, over(grade)|
\item[C.] \verb|sts list, by(grade)|
\item[D.] Je ne sais pas.
\end{description}
\item[\bf 1.5] Quelle commande permet d'afficher la courbe de survie avec un
  intervalle de confiance à 90~\% pour les patients atteints d'un cancer en
  stade II ? \marginpar{1.5 $\square$}
\begin{description}
\item[A.] \verb|sts graph if grade == 1, ci level(.90)|
\item[B.] \verb|sts graph if grade == 1, ci level(90)|
\item[C.] \verb|sts graph if grade == "low", ci level(.90)|
\item[D.] \verb|sts graph if grade == "low", ci level(90)|
\item[E.] Je ne sais pas.
\end{description}
\item[\bf 1.6] La commande suivante \marginpar{1.6 $\square$}
\begin{verbatim}
. sts test grade, w
\end{verbatim}
renvoit le résultat d'un test statistique. Lequel ?
\begin{description}
\item[A.] Test du log-rank.
\item[B.] Test de Wilcoxon.
\item[C.] Test de Tarone-Ware.
\item[D.] Je ne sais pas.
\end{description}
\item[\bf 1.7] Quelle commande a permis de produire la figure suivante
  (indépendemment du rapport hauteur/largeur et de la couleur) ?
  \marginpar{1.7 $\square$}
  \begin{center}
    \includegraphics{./figs/dev10_cumhaz}
  \end{center}
\begin{description}
\item[A.] \verb|sts graph, cumhaz|
\item[B.] \verb|sts graph, haz|
\item[C.] Je ne sais pas.
\end{description}
\end{description}

\section*{Exercice 2}
Dans l'étude UIS, on a cherché à modéliser la durée d'abstinence effective
en sortie d'étude pour des usagers de drogue inclus dans un programme
thérapeutique de courte ou longue durée \cite{hosmer08}. Par tirage au sort,
les patients étaient inclus sur l'un des deux centres (A et B) retenus pour
réaliser l'étude. Les autres informations disponibles, pour chaque patient,
sont les suivantes : âge à l'inclusion, consommation d'héroïne ou de cocaïne
durant les trois mois précédents l'inclusion, nombre de traitements relatifs
à la consommation stupéfiants. Le temps, mesuré en jours, indique la durée
entre la fin du programme thérapeutique et la reprise d'une consommation
régulière de drogue, cette dernière servant également à identifier les
données censurées (1 si consommation drogue observée pendant la période de
suivi, 0 sinon). Les données sous \Stata sont indiquées ci-dessous.
\begin{verbatim}
. list in 1/5

     +-----------------------------------------------------------+
     | ID   age   ndrugtx   treat   site   time   censor   herco |
     |-----------------------------------------------------------|
  1. |  1    39         1       1      0    188        1       3 |
  2. |  2    33         8       1      0     26        1       3 |
  3. |  3    33         3       1      0    207        1       2 |
  4. |  4    32         1       0      0    144        1       3 |
  5. |  5    24         5       1      0    551        0       2 |
     +-----------------------------------------------------------+
\end{verbatim}
La variable \texttt{ID} est le numéro d'identification du patient,
\texttt{age} l'âge du patient en années arrondies à l'entier le plus proche,
\texttt{ndrugtx} le nombre de traitements déjà effectués, \texttt{treat} le
type de programme thérapeutique (0 pour le programme court, 1 pour le
programme long), \texttt{site} le centre (A=0, B=1), \texttt{time} le temps
avant d'observer un retour à une consommation régulière de drogue chez le
patient, \texttt{censor} la présence d'un tel événement et \texttt{herco} la
consommation de cocaïne/héroïne durant les trois mois précédent l'étude
codée 1 pour la consommation des deux substances, 2 pour l'une des deux
substances et 3 pour l'absence de consommation de ces substances. 

Au total, il y a 628 patients, et on dénombre 5 valeurs manquantes pour la
variable \texttt{age} et 17 valeurs manquantes pour la variable
\texttt{ndrugtx}. 
\begin{description}
\item[\bf 2.1] On souhaite dénombrer les événements (\texttt{censor}) dans
  chacun des deux sites, pour les personnes âgées de plus de 25 ans. La
  commande suivante est-elle correctement spécifiée ? \marginpar{2.1 $\square$}
\begin{verbatim}
. tabulate censor site if age > 25
\end{verbatim}
\begin{description}
\item[A.] Oui.
\item[B.] Non.
\item[C.] Je ne sais pas.
\end{description}
\item[\bf 2.2] Pour tester l'égalité des courbes de survie associée à chacun
  des deux programmes thérapeutiques, on peut utiliser la commande :
  \marginpar{2.2 $\square$}
\begin{description}
\item[A.] \verb|sts test treat|
\item[B.] \verb|sts test treat, trend|
\item[C.] \verb|stci, by(treat)|
\item[D.] Je ne sais pas.
\end{description}
\item[\bf 2.3] Quelle commande a permis de produire la figure suivante
  (indépendemment du rapport hauteur/largeur etd e la couleur) ?
  \marginpar{2.3 $\square$} 
  \begin{center}
    \includegraphics{./figs/dev10_kaplan}
  \end{center}
\begin{description}
\item[A.] \verb|sts graph, by(treat) cens|
\item[B.] \verb|sts graph, over(treat) cens ci|
\item[C.] \verb|sts graph, by(treat) ci|
\item[D.] Je ne sais pas.
\end{description}
\item[\bf 2.4] On souhaite réaliser une régression de Cox pour étudier
  l'effet de la consommation d'héroïne ou de cocaïne et afficher le
  coefficient de régression et son intervalle de confiance à 95~\%. Quelle
  commande doit-on utiliser ? \marginpar{2.4 $\square$}
\begin{description}
\item[A.] \verb|sts cox herco|
\item[B.] \verb|stcox herco, nohr|
\item[C.] \verb|stcox herco, noshow|
\item[D.] Je ne sais pas.
\end{description}
\item[\bf 2.5] Quelle commande pourrait-on utiliser pour vérifier si
  l'hypothèse de risque proportionnels est vérifiée dans le modèle précédent
  ? \marginpar{2.5 $\square$}
\begin{description}
\item[A.] \verb|predict double cs, csnell|
\item[B.] \verb|estat concordance|
\item[C.] \verb|sts graph, by(herco)|
\item[D.] Je ne sais pas.
\end{description}
\end{description}


\cleardoublepage

\blankpage
\blankpage

% -------------------------------------------------- Corrigés R ----------------
\thispagestyle{empty}

\centerline{\small Centre d'Enseignement de la Statistique Appliquée, à la Médecine et à la Biologie Médicale}
\vspace*{2cm}
\begin{center}
\centerline{\includegraphics[scale=.55]{cesam}}
\vspace*{2cm}
\begin{minipage}{.75\textwidth}
\begin{mdframed}[style=titlep]
\centerline{\Huge Corrigés des exercices}
\vskip1em
\centerline{\Huge du cours d'informatique du CESAM}
\end{mdframed}
\end{minipage}
\end{center}
\vskip3em
\centerline{\Huge\bf Introduction au logiciel R}
\vskip5em
\begin{center}
  \begin{tabular}{ll}
    \textbf{Responsables :} & \\
    Christophe LALANNE & \url{christophe.lalanne@inserm.fr} \\
    Yassin MAZROUI     & \url{yassin.mazroui@upmc.fr} \\
    Pr Mounir MESBAH   & \url{mounir.mesbah@upmc.fr}
  \end{tabular}
\end{center}
\vskip3em
\centerline{\Large \url{http://www.cesam.upmc.fr}}
\vskip3em
\centerline{\LARGE Année Universitaire 2015–2016}
\vfill
\begin{center}
\begin{minipage}{.6\textwidth}
\centering
Adresser toute correspondance à :\\
Université Pierre et Marie Curie – Paris 6
Secrétariat du CESAM – Les Cordeliers
Service Formation Continue, esc. B, 4ème étage,
15 rue de l’école de médecine,
75006 PARIS\\
ou par Courriel à : \url{cesam@upmc.fr}
\end{minipage}
\end{center}

\blankpage

% remove chapter prefix because we no longer want chapter name.
\titleformat{\chapter}
  {\normalfont\huge\bfseries}{}{20pt}{\huge}


\setcounter{page}{1}

\chapter*{Semaine 1\markboth{Corrigés de la semaine 1}{}}
\label{start:sol1}
\input{solutions1} 
\label{stop:sol1}

\chapter*{Semaine 2\markboth{Corrigés de la semaine 2}{}}
\label{start:sol2}
\input{solutions2} 
\label{stop:sol2}

\chapter*{Semaine 3\markboth{Corrigés de la semaine 3}{}}
\label{start:sol3}
\input{solutions3} 
\label{stop:sol3}

\chapter*{Semaine 4\markboth{Corrigés de la semaine 4}{}}
\label{start:sol4}
\input{solutions4} 
\label{stop:sol4}

\chapter*{Semaine 5\markboth{Corrigés de la semaine 5}{}}
\label{start:sol5}
\input{solutions5} 
\label{stop:sol5}

\chapter*{Semaine 6\markboth{Corrigés de la semaine 6}{}}
\label{start:sol6}
\input{solutions6} 
\label{stop:sol6}

\chapter*{Semaine 7\markboth{Corrigés de la semaine 7}{}}
\label{start:sol7}
\input{solutions7} 
\label{stop:sol7}

\cleardoublepage

\blankpage
\blankpage

% -------------------------------------------------- Corrigés Stata ----------------
\thispagestyle{empty}

\centerline{\small Centre d'Enseignement de la Statistique Appliquée, à la Médecine et à la Biologie Médicale}
\vspace*{2cm}
\begin{center}
\centerline{\includegraphics[scale=.55]{cesam}}
\vspace*{2cm}
\begin{minipage}{.75\textwidth}
\begin{mdframed}[style=titlep]
\centerline{\Huge Corrigés des exercices}
\vskip1em
\centerline{\Huge du cours d'informatique du CESAM}
\end{mdframed}
\end{minipage}
\end{center}
\vskip3em
\centerline{\Huge\bf Introduction au logiciel Stata}
\vskip5em
\begin{center}
  \begin{tabular}{ll}
    \textbf{Responsables :} & \\
    Christophe LALANNE & \url{christophe.lalanne@inserm.fr} \\
    Yassin MAZROUI     & \url{yassin.mazroui@upmc.fr} \\
    Pr Mounir MESBAH   & \url{mounir.mesbah@upmc.fr}
  \end{tabular}
\end{center}
\vskip3em
\centerline{\Large \url{http://www.cesam.upmc.fr}}
\vskip3em
\centerline{\LARGE Année Universitaire 2015–2016}
\vfill
\begin{center}
\begin{minipage}{.6\textwidth}
\centering
Adresser toute correspondance à :\\
Université Pierre et Marie Curie – Paris 6
Secrétariat du CESAM – Les Cordeliers
Service Formation Continue, esc. B, 4ème étage,
15 rue de l’école de médecine,
75006 PARIS\\
ou par Courriel à : \url{cesam@upmc.fr}
\end{minipage}
\end{center}



\blankpage

% remove chapter prefix because we no longer want chapter name.
\titleformat{\chapter}
  {\normalfont\huge\bfseries}{}{20pt}{\huge}
  

\setcounter{page}{1}
%---------------------------------------------------------------- Séance 08 --
\chapter*{Semaine 8\markboth{Corrigés de la semaine 8}{}}

\soln{\ref{exo:8.1}} \Stata dispose d'un éditeur de données qui se présente
sous la forme d'un tableur (à l'image d'Excel), mais on peut saisir
dirctement des données à l'aide de la commande \texttt{input}. Après avoir
donné le nom de la ou des variables (lorsqu'il y a plusieurs variables, on
sépare leur nom par un espace), on presse sur la touche \textsf{Entrée}
et on saisit les observations (idem, lorsqu'il y a plusieurs variables on
saisit les observations ou valeurs observées en les séparant par un
espace). Lorsque la saisie est terminé, on saisit le mot \texttt{end} et on
presse sur \textsf{Entrée} pour indiquer à \Stata que la saisie est
terminée. \label{para:edit}
\begin{verbatim}
. input x

             x
  1. 3.68
  2. 2.21
  3. 2.45
  4. 8.64
  5. 4.32
  6. 3.43
  7. 5.11
  8. 3.87
  9. end
\end{verbatim}
On peut vérifier la saisie en affichant les valeurs 3 ou 4 premières valeurs
de \texttt{x} :
\begin{verbatim}
. list in 1/4

     +------+
     |    x |
     |------|
  1. | 3.68 |
  2. | 2.21 |
  3. | 2.45 |
  4. | 8.64 |
     +------+
\end{verbatim}
Les commandes \texttt{describe} ou \texttt{summarize} fournissent toutes les
deux le nombre d'observations contenues dans la variable \texttt{x} (comme
il n'y a qu'une seule variable, il n'est même pas nécessaire de spécifier le
nom de la variable lorsque l'on utilise ces commandes).
\begin{verbatim}
. summarize

    Variable |       Obs        Mean    Std. Dev.       Min        Max
-------------+--------------------------------------------------------
           x |         8     4.21375    2.019526       2.21       8.64
\end{verbatim}
Pour corriger la valeur erronée (8.64), on utilisera \texttt{replace} en
sépcifiant la position de l'observation dans la variable (son rang
numérique) :
\begin{verbatim}
. replace x = 3.64 in 4
(1 real change made)
\end{verbatim}
On procèdera de même pour recoder la 7\ieme\ observation en valeur
manquante, sachant que les valeurs manquantes simples sont codées à l'aide
d'un point :
\begin{verbatim}
. replace x = . in 7   
(1 real change made, 1 to missing)
\end{verbatim}
D'où le résumé numérique final suivant :
\begin{verbatim}
. summarize

    Variable |       Obs        Mean    Std. Dev.       Min        Max
-------------+--------------------------------------------------------
           x |         7    3.371429    .7656246       2.21       4.32
\end{verbatim}
% 
%
%
\soln{\ref{exo:8.2}} On peut utiliser la même solution qu'à l'exercice
précédent pour saisir les données, c'est-à-dire utiliser la commande
\texttt{input} ou bien se servir de l'éditeur de données qui se présente sous
la forme d'un tableur :

\includegraphics{./figs/stata_editX}

On en profitera pour nommer la variable (\texttt{var1} par défaut pour la
première colonne du tableur). Le seuil de détection en logarithme vaut :
\begin{verbatim}
. display log10(50)
1.69897
\end{verbatim}
On peut donc compter le nombre d'observation ne remplissant pas la condition
$X>\log(50)$ à l'aide de la commande \texttt{count}.
\begin{verbatim}
. count if X <= log10(50)
    4
\end{verbatim}
D'où le calcul de la charge virale médiane en ne considérant que les données
au-dessus de la limite de détection :
\begin{verbatim}
. egen Xm = median(10^X) if X > log10(50)
. display round(Xm)
12980
\end{verbatim}
Pour calculer la médiane des $X_i$ remplissant la condition $X_i>\log(50)$,
on a utilisé directement la commande \texttt{egen} qui fournit un certain
nombre de fonctions de transformations et de calcul de base (voir l'aide en
ligne, \verb|help egen|).
%
%
%
\soln{\ref{exo:8.3}}
Pour lire les données contenues dans le fichier \texttt{anorexia.data} dont
un aperçu est fourni ci-dessous
\begin{verbatim}
Group Before After
g1 80.5  82.2
g1 84.9  85.6
g1 81.5  81.4
g1 82.6  81.9
g1 79.9  76.4
\end{verbatim}
on va utiliser la commande \texttt{infile} à partir d'un fichier \og
dictionnaire\fg\ qui décrit la structure des données. Ce fichier porte
généralement le même nom que le fichier source de données, et possède
l'extension \texttt{dct}. En voici le listing complet
(\texttt{anorexia.dct}) :
\begin{verbatim}
infile dictionary using anorexia.dat {
  _first(2)
  str2 Group "Type de therapie"
  double Before "Avant"
  double After "Apres"
}
\end{verbatim}
Dans celui-ci, on indique que les observations commencent à la 2\ieme\ ligne
(on ignore la ligne d'en-tête), et que l'on a trois variables,
\texttt{Group} (variable qualitative), \texttt{Before} et \texttt{After}
(variables numériques). Notons que l'on a associé des étiquettes de
description pour ces trois variables. Il suffit ensuite d'utiliser la
commande \texttt{infile} en fournissant le nom de ce fichier dictionnaire
(inutile de préciser l'extension du fichier).
\begin{verbatim}
. infile using anorexia

infile dictionary using anorexia.dat {
  _first(2)
  str2 Group "Type de therapie"
  double Before "Avant"
  double After "Apres"
}

(72 observations read)
\end{verbatim}
On peut ensuite vérifier que les données ont été importées correctement à
l'aide de \texttt{describe}. Cette commande fournit par ailleurs le nombre
d'observations disponibles dans le tableau de données ($N=72$).
\begin{verbatim}
. describe

Contains data
  obs:            72                          
 vars:             3                          
 size:         1,296                          
----------------------------------------------------------------------------------------
              storage  display     value
variable name   type   format      label      variable label
----------------------------------------------------------------------------------------
Group           str2   %9s                    Type de therapie
Before          double %10.0g                 Avant
After           double %10.0g                 Apres
----------------------------------------------------------------------------------------
Sorted by:  
     Note:  dataset has changed since last saved
\end{verbatim}

Pour obtenir les effectifs par type de thérapie, on utilise un simple tri à
plat à l'aide de la commande \texttt{tabulate}.
\begin{verbatim}
. tabulate Group

    Type de |
   therapie |      Freq.     Percent        Cum.
------------+-----------------------------------
         g1 |         29       40.28       40.28
         g2 |         26       36.11       76.39
         g3 |         17       23.61      100.00
------------+-----------------------------------
\end{verbatim}

La transformation d'unités pour les poids ne pose pas de problème
spécifique, mais il faut décider si l'on crée de nouvelles variables
(\texttt{generate}) ou si l'on remplace les valeurs existantes
(\texttt{replace}). Ici, on remplacera les valeurs existantes :
\begin{verbatim}
. replace Before = Before/2.2
(72 real changes made)
. replace After = After/2.2
(72 real changes made)
\end{verbatim}

Pour les scores de différences, on crée cette fois-ci une nouvelles variable
à l'aide de la commande \texttt{generate}.
\begin{verbatim}
. generate diff = After - Before
. summarize diff

    Variable |       Obs        Mean    Std. Dev.       Min        Max
-------------+--------------------------------------------------------
        diff |        72    1.256313    3.628908  -5.545455   9.772727
\end{verbatim}

La commande \texttt{summarize} peut être utilisée pour fournir un résumé
numérique pour chacun des groupes de traitement, par exemple 
\verb|by Group, sort: summarize diff|. Mais, pour calculer spécifiquement
certains indicateurs descriptifs, il est plus commode d'utiliser la comamnde
\texttt{tabstat} à laquelle on fournit la variable réponse et le facteur de
classification, ainsi que les statistiques recherchées via
\texttt{stats()}. \label{para:tabstat}
\begin{verbatim}
. tabstat diff, by(Group) stats(mean min max)

Summary for variables: diff
     by categories of: Group (Type de therapie)

 Group |      mean       min       max
-------+------------------------------
    g1 |  1.366771 -4.136364       9.5
    g2 | -.2045454 -5.545455  7.227273
    g3 |  3.302139 -2.409091  9.772727
-------+------------------------------
 Total |  1.256313 -5.545455  9.772727
--------------------------------------
\end{verbatim}
%
%
%
\soln{\ref{exo:8.4}} Pour la saisie des données, on procéder de la même
manière qu'à l'exercice~8.1, p.~\pageref{para:edit}, ou tout simplement
saisir les données dans un fichier au format texte. Supposons que les
données aient été entrées dans un fichier appelé \verb|saisie_x.txt| dont
voici un aperçu :
\begin{verbatim}
24.9 25.0 25.0 25.1 25.2 ...
\end{verbatim}
Les valeurs de $X$ sont simplement séparées par un espace. Dans ce cas, les
données peuvent être lues et importées dans \Stata avec la commande \texttt{infile}.
\begin{verbatim}
. infile x using "saisie_x.txt"             
(26 observations read)
\end{verbatim}
On n'oubliera pas d'ajouter l'option \texttt{clear} si des données existe
déjà dans l'espace de travail de \Stata.

La commande \texttt{tabstat} peut être utilisée pour calculer la moyenne et
la médiane des observations. On utilisera par contre la commande
\texttt{egen} avec la fonction \texttt{mode} pour calculer les valeurs
modales de $X$. Comme il y a plusieurs modes, on demandera à \Stata de
renvoyer la valeur la plus basse des deux modes. Notons que la commande
\texttt{egen} pourrait également être utilisée pour calculer la moyenne et
la médiane de $X$. 
\begin{verbatim}
. tabstat x, stats(mean median)

    variable |      mean       p50
-------------+--------------------
           x |  25.44615     25.45
----------------------------------
. egen xmode = mode(x), minmode
. display xmode
25.4
\end{verbatim}

Concernant la variance, on a directement accès à l'écart-type à partir de 
\verb|summarize x|, mais on peut également la calculer avec \texttt{egen}
comme le carré de l'écart-type estimé à partir de l'échantillon, puis
l'afficher sur la console des résultats :
\begin{verbatim}
. egen varx = sd(x)
. di varx^2
.0793846
\end{verbatim}

Pour le recodage en 4 classes d'intervalles pré-définis pour la variable
\texttt{x}, on peut toujours utiliser la fonction \texttt{cut} avec
\texttt{egen}. On peut ensuite vérifier que les intervalles de classe sont
bien respectés à l'aide de \texttt{table} qui est plus souple que la
commande \texttt{summarize} et permet de préciser la liste des valeurs
statistiques à afficher.
\begin{verbatim}
. egen xc = cut(x), at(24.8,25.2,25.5,25.8,26.1) label
. table xc, contents(min x max x)

----------------------------------
       xc |     min(x)      max(x)
----------+-----------------------
    24.8- |       24.9        25.1
    25.2- |       25.2        25.4
    25.5- |       25.5        25.8
    25.8- |       25.9          26
----------------------------------
\end{verbatim}
En revanche, le tableau d'effectifs peut s'obtenir directement avec
\texttt{tabulate}, par exemple :
\begin{verbatim}
. tabulate xc, plot

         xc |      Freq.
------------+------------+-----------------------------------------------------
      24.8- |          4 |****
      25.2- |          9 |*********
      25.5- |         11 |***********
      25.8- |          2 |**
------------+------------+-----------------------------------------------------
      Total |         26
\end{verbatim}
Enfin, pour la représentation sous forme d'histogramme, \Stata dispose de ses
propres algorithmes de calcul pour la détermination du nombre de classes à
construire, tout comme R. Tout est géré à partir de la commande
\texttt{histogram}. 
\begin{verbatim}
. histogram x, frequency
(bin=5, start=24.9, width=.22000008)
\end{verbatim}
Ne pas oublier l'option \texttt{frequency} si l'on souhaite afficher les
effectifs plutôt que la densité (qui est le choix par défaut).

\includegraphics{./figs/stata_histx}
%
%
%
\soln{\ref{exo:8.5}} Pour importer les données stockées dans un simple
fichier texte, on utilise la commande \verb|infile|.
\begin{verbatim}
. infile tailles using "elderly.dat", clear
\end{verbatim}
Ici, on notera que l'on ne précise pas la manière dont sont codées les
valeurs manquantes car le "." est le format utilisé par défaut par \Stata
(pour les variables numériques uniquement).

Il existe plusieurs commandes qui facilitent la détection et
l'identification des valeurs manquantes. Parmi celles disponibles par défaut
sur \Stata, on distingue \verb|codebook| qui fournit un résumé de la
variable, ainsi que les fonctions de base qui permettent de tabuler ou
compter les observations répondant à un certain critère.
\begin{verbatim}
. count if tailles == .
    5
\end{verbatim}

Pour obtenir la taille moyenne et son intervalle de confiance à 95~\%
associé, il suffit d'entrer la commande suivante :
\begin{verbatim}
. ci tailles

    Variable |        Obs        Mean    Std. Err.       [95% Conf. Interval]
-------------+---------------------------------------------------------------
     tailles |        346    159.8324    .3232401        159.1966    160.4681
\end{verbatim}

Enfin, pour afficher la distribution des tailles sous forme d'une courbe de
densité, on utilise la commande \texttt{histogram} avec l'option
\texttt{kdensity}. Le degré de lissage peut être contrôlé à l'aide de
l'option \texttt{kdenopts} ; par exemple, ajouter \verb|kdenopts(gauss width(1))| 
à la commande ci-dessous produirait une courbe \og moins lisse\fg\
(c'est-à-dire s'ajustant plus aux données).
\begin{verbatim}
. histogram tailles, kdensity
(bin=18, start=142, width=2)
\end{verbatim}

\includegraphics{./figs/stata_kdens}

%
%
%
\soln{\ref{exo:8.6}} Les données sur les poids à la naissance de Hosmer \&
Lemeshow (1989) peuvent être obtenues directement au format \Stata grâce à la
commande \verb|webuse lbw|, mais par souci de simplicité on utilisera les
mêmes données que celles traitées avec \R. Celles-ci ont été exportées au
format texte dans le fichier \texttt{birthwt.dat}, et elles peuvent être
importées ainsi : 
\begin{verbatim}
. infile low age lwt race smoke ptl ht ui ftv bwt using "birthwt.dat", clear
\end{verbatim}

On peut vérifier avec la commande \verb|- describe -| que les variables sont
bien toutes au format numérique (colonne \texttt{storage type}).
\begin{verbatim}
. describe

Contains data
  obs:           189                          
 vars:            10                          
 size:         7,560                          
-----------------------------------------------------------------------------------------
              storage  display     value
variable name   type   format      label      variable label
-----------------------------------------------------------------------------------------
low             float  %9.0g                  
age             float  %9.0g                  
lwt             float  %9.0g                  
race            float  %9.0g                  
smoke           float  %9.0g                  
ptl             float  %9.0g                  
ht              float  %9.0g                  
ui              float  %9.0g                  
ftv             float  %9.0g                  
bwt             float  %9.0g                  
-----------------------------------------------------------------------------------------
Sorted by:  
     Note:  dataset has changed since last saved
\end{verbatim}
Pour associer de nouvelles étiquettes aux variables \texttt{low},
\texttt{race}, \texttt{smoke}, \texttt{ui} et \texttt{ht}, il suffit de
définir des "labels" et de les associer aux variables en question (cela ne
change le format de représentation des variables en mémoire, qui restent
stockées sous forme de nombre).
\begin{verbatim}
. label define yesno 0 "no" 1 "yes"
. label define ethn 1 "White" 2 "Black" 3 "Other"
. label values low smoke ui ht yesno
. label values race ethn
\end{verbatim}
On peut vérifier que le tableau de données a bien été mis à jour
\begin{verbatim}
. list in 1/5

     +---------------------------------------------------------------+
     | low   age   lwt    race   smoke   ptl   ht    ui   ftv    bwt |
     |---------------------------------------------------------------|
  1. |  no    19   182   Black      no     0   no   yes     0   2523 |
  2. |  no    33   155   Other      no     0   no    no     3   2551 |
  3. |  no    20   105   White     yes     0   no    no     1   2557 |
  4. |  no    21   108   White     yes     0   no   yes     2   2594 |
  5. |  no    18   107   White     yes     0   no   yes     0   2600 |
     +---------------------------------------------------------------+
\end{verbatim}

La conversion du poids des mères en \emph{kg} ne pose pas de problème
particulier, et ici on remplacera directement les données disponibles à
l'aide de la commande \texttt{replace}
\begin{verbatim}
. replace lwt = lwt/2.2
(189 real changes made)
\end{verbatim}
Les indicateurs de tendance centrale et de dispersion relative sont obtenus
à partir de la commande \verb|summarize|.
\begin{verbatim}
. summarize lwt, detail

                             lwt
-------------------------------------------------------------
      Percentiles      Smallest
 1%     38.63636       36.36364
 5%     42.72727       38.63636
10%     44.54546       38.63636       Obs                 189
25%           50       40.45454       Sum of Wgt.         189

50%           55                      Mean           59.00673
                        Largest       Std. Dev.      13.89972
75%     63.63636       104.0909
90%     77.27273       106.8182       Variance       193.2022
95%     85.90909       109.5455       Skewness       1.390855
99%     109.5455       113.6364       Kurtosis       5.309181
\end{verbatim}
On a ajouté l'option \texttt{detail} car la commande \verb|summarize lwt| ne
fournit pas la médiane. On a cependant vu une alternative possible avec la
commande \texttt{tabstat} dans l'exercice~1.5
(p.~\pageref{para:tabstat}). Pour obtenir l'intervalle inter-quartile, on
peut le calculer ainsi :
\begin{verbatim}
. egen lwtiqr = iqr(lwt)
. display lwtiqr
13.636364
\end{verbatim}
Enfin, un histogramme du poids des mères est construit à l'aide de la
commande \verb|histogram|. Ici, sans autre option un histogramme de
densité sera construit.
\begin{verbatim}
. histogram lwt
(bin=13, start=36.363636, width=5.9440557)
\end{verbatim}

\includegraphics{./figs/stata_birthwt}

Concernant la proportion de mères ayant fumé durant la grossesse et
le calcul de l'intervalle de confiance à 95~\% associé, on peut utiliser la
commande \verb|prtest|, qui comme dans le cas de la commande \R
\texttt{prop.test} suppose de grands échantillons :
\begin{verbatim}
. tabulate smoke

      smoke |      Freq.     Percent        Cum.
------------+-----------------------------------
         no |        115       60.85       60.85
        yes |         74       39.15      100.00
------------+-----------------------------------
      Total |        189      100.00
. prtest smoke == .5

One-sample test of proportion                  smoke: Number of obs =      189
------------------------------------------------------------------------------
    Variable |       Mean   Std. Err.                     [95% Conf. Interval]
-------------+----------------------------------------------------------------
       smoke |   .3915344   .0355036                      .3219487    .4611201
------------------------------------------------------------------------------
    p = proportion(smoke)                                         z =  -2.9823
Ho: p = 0.5

     Ha: p < 0.5                 Ha: p != 0.5                   Ha: p > 0.5
 Pr(Z < z) = 0.0014         Pr(|Z| > |z|) = 0.0029          Pr(Z > z) = 0.9986
\end{verbatim}
Le test utilisé par \Stata diffère du $\chi^2$ de \R mais ces deux tests
fournissent essentiellement les mêmes résultats. Ici, la proportion estimée
vaut 0.392 avec un intervalle de confiance de $[0.322;0.461]$. Si l'on ne
souhaite pas utiliser l'approximation pour grands échantillons, on peut
utiliser \verb|bintest| à la place.

La commande \verb|graph bar| permet de construire assez facilement des
diagrammes en barres. 
\begin{verbatim}
. gen one = 1
. graph bar (sum) one, over(smoke) ytitle("Effectif")
\end{verbatim}

\includegraphics{./figs/stata_birthwt2}

Pour générer les terciles, on utilise la commande \verb|- xtile -| en
spécifiant le nombre de quantile désiré, \texttt{nq(3)}. 
\begin{verbatim}
. xtile ageter = age, nq(3)
. tabulate ageter

3 quantiles |
     of age |      Freq.     Percent        Cum.
------------+-----------------------------------
          1 |         69       36.51       36.51
          2 |         66       34.92       71.43
          3 |         54       28.57      100.00
------------+-----------------------------------
      Total |        189      100.00
\end{verbatim}
On notera que \Stata n'attribue pas automatiquement des labels aux catégories
générées, mais simplement un rang (ici, ${1,2,3}$). On peut toutefois
obtenir les bornes des intervalles des terciles de deux manières. La
première solution, 
\begin{verbatim}
. _pctile age, n(3)
. return list

scalars:
                 r(r1) =  20
                 r(r2) =  25
\end{verbatim}
renvoit les bornes inférieures (exclues) des deux derniers intervalles, d'où
les trois intervalles : (14,20], (20,25] et (25,45], la borne inférieure du
1\ier intervalle étant obtenue comme le \texttt{min} de la variable
\texttt{age}. La seconde méthode consiste à fournir un résumé plus détaillé
de la variable \texttt{ageter} :
\begin{verbatim}
. table ageter, contents(freq min age min age)

----------------------------------------------
3         |
quantiles |
of age    |      Freq.    min(age)    min(age)
----------+-----------------------------------
        1 |         69          14          14
        2 |         66          21          21
        3 |         54          26          26
----------------------------------------------
\end{verbatim}

On notera qu'une commande telle que \texttt{cut} permettrait de générer des
groupes approximativement équivalents en termes d'effectifs, mais cela ne
correspond pas tout à fait au résultat souhaité.
\begin{verbatim}
. egen ageter = cut(age), group(3) label
\end{verbatim}

Le croisement de cette nouvelle variable avec la variable indicatrice de
sous-poids donne les résultats suivants, exprimées en termes de proportions
(par colonne) :
\begin{verbatim}
. tabulate ageter low, col nofreq

         3 |
 quantiles |          low
    of age |        no        yes |     Total
-----------+----------------------+----------
         1 |     35.38      38.98 |     36.51 
         2 |     33.08      38.98 |     34.92 
         3 |     31.54      22.03 |     28.57 
-----------+----------------------+----------
     Total |    100.00     100.00 |    100.00
\end{verbatim}

Un tri à plat de la variable \texttt{race} est effectuée de la même manière
à partir de la commande \verb|tabulate|.
\begin{verbatim}
. tabulate race

       race |      Freq.     Percent        Cum.
------------+-----------------------------------
      White |         96       50.79       50.79
      Black |         26       13.76       64.55
      Other |         67       35.45      100.00
------------+-----------------------------------
      Total |        189      100.00
\end{verbatim}


Enfin, pour résumer la distribution des variables selon la variable
indicatrice \texttt{low}, on peut procéder en deux temps. D'abord, on résume
les variables quantitatives :
\begin{verbatim}
. by low, sort: summarize age lwt ptl ftv

-------------------------------------------------------------------------------------------------------------------
-> low = no

    Variable |       Obs        Mean    Std. Dev.       Min        Max
-------------+--------------------------------------------------------
         age |       130    23.66154    5.584522         14         45
         lwt |       130    60.59091    14.42001   38.63636   113.6364
         ptl |       130    .1307692    .4556019          0          3
         ftv |       130    .8384615    1.069694          0          6

-------------------------------------------------------------------------------------------------------------------
-> low = yes

    Variable |       Obs        Mean    Std. Dev.       Min        Max
-------------+--------------------------------------------------------
         age |        59    22.30508    4.511496         14         34
         lwt |        59    55.51618     12.0724   36.36364   90.90909
         ptl |        59    .3389831    .5448875          0          2
         ftv |        59    .6949153    1.038139          0          4
\end{verbatim}
Puis, on procède de même avec les variables qualitatives :
\begin{verbatim}
. by low, sort: tab1 race smoke ht ui
\end{verbatim}


%---------------------------------------------------------------- Séance 09 --
\chapter*{Semaine 9\markboth{Corrigés de la semaine 9}{}}

\soln{\ref{exo:9.1}} On rappelle que seuls les valeurs numériques des poids
à la naissance sont disponibles dans le fichier \texttt{sirds.dat}, et qu'il
nous faut contruire la variable de groupement (enfants décédés \emph{versus}
vivants). Une manière de procéder consiste à générer une variable prenant
deux valeurs puis à associer des étiquettes ("labels") aux valeurs
distinctes prises par cette variable. Par exemple,
\begin{verbatim}
. infile poids using "sirds.dat", clear
. gen status = 0
. replace status = 1 if _n>27
. label define labstatus 0 "décédé" 1 "vivant"
. label values status labstatus
\end{verbatim}

Il est possible d'obtenir un résumé numérique de la variable \texttt{poids}
pour chaque modalité de la variable \texttt{status} nouvellement créée à
l'aide de la commande \verb|- summarize -|.
\begin{verbatim}
. by status, sort: summarize poids

-> status = décédé

    Variable |       Obs        Mean    Std. Dev.       Min        Max
-------------+--------------------------------------------------------
       poids |        27    1.691741    .5176473       1.03       2.73

-----------------------------------------------------------------------------------------
-> status = vivant

    Variable |       Obs        Mean    Std. Dev.       Min        Max
-------------+--------------------------------------------------------
       poids |        23    2.307391    .6647229       1.13       3.64
\end{verbatim}

Pour réaliser un histogramme de la distribution des poids à la naissance en
fonction du status clinique, on peut procéder ainsi :
\begin{verbatim}
. histogram poids, frequency by(status)
\end{verbatim}
\includegraphics{./figs/stata_sirds}

Enfin, le test de Student est réalisé en utilisant la commande
\texttt{ttest}, qui par défaut suppose l'homogénéité des variances vérifiée :
\begin{verbatim}
. ttest poids, by(status)

Two-sample t test with equal variances
------------------------------------------------------------------------------
   Group |     Obs        Mean    Std. Err.   Std. Dev.   [95% Conf. Interval]
---------+--------------------------------------------------------------------
  décédé |      27    1.691741    .0996213    .5176473    1.486966    1.896515
  vivant |      23    2.307391    .1386043    .6647229    2.019944    2.594839
---------+--------------------------------------------------------------------
combined |      50     1.97494    .0934493     .660786    1.787147    2.162733
---------+--------------------------------------------------------------------
    diff |           -.6156506    .1673084               -.9520467   -.2792545
------------------------------------------------------------------------------
    diff = mean(décédé) - mean(vivant)                            t =  -3.6797
Ho: diff = 0                                     degrees of freedom =       48

    Ha: diff < 0                 Ha: diff != 0                 Ha: diff > 0
 Pr(T < t) = 0.0003         Pr(|T| > |t|) = 0.0006          Pr(T > t) = 0.9997
\end{verbatim}
\Stata reporte les degrés de significativité correspondant aux deux types de
tests, uni- et bilatéral. Ici, pour le test bilatéral, le degré de
significativité estimé est de 0.0006 ; ce résultat est arrondi, mais il est
possible d'obtenir une valeur plus précise à l'aide de la commande suivante :
\begin{verbatim}
. display r(p)
.00059019
\end{verbatim}
La commande ci-dessus doit être entrée immédiatement après l'appel à la
commande \verb|ttest|.
%
%
%
\soln{\ref{exo:9.2}} Les données de l'étude sur le sommeil servant de base à
l'article de Student ne sont pas disponibles sous \Stata. On peut en revanche
les saisir manuellement à l'aide de la commande \verb|input|, comme on l'a
vu p.~\pageref{para:edit}. Dans le cas présent, on va donc créer deux
variables, \texttt{DHH} et \texttt{LHH}, correspondant aux temps de sommeil
enregistrées pour les molécules D. hyoscyamine hydrobromide et
L. hyoscyamine hydrobromide, respectivement. 
\begin{verbatim}
. clear all
. input DHH LHH

           DHH        LHH
  1. 0.7 1.9
  2. -1.6 0.8
  3. -0.2 1.1
  4. -1.2 0.1
  5. -0.1 -0.1
  6. 3.4 4.4
  7. 3.7 5.5
  8. 0.8 1.6
  9. 0.0 4.6
 10. 2.0 3.4
 11. end
\end{verbatim}
On peut vérifier que la saisie donne bien les résultats attendus en
affichant les 5 premières observations :
\begin{verbatim}
. list in 1/5

     +------------+
     |  DHH   LHH |
     |------------|
  1. |   .7   1.9 |
  2. | -1.6    .8 |
  3. |  -.2   1.1 |
  4. | -1.2    .1 |
  5. |  -.1   -.1 |
     +------------+
\end{verbatim}

Les moyennes par groupe de traitement sont obtenues à l'aide de
\texttt{tabstat}, qui sans autre option renvoit la moyenne des variables
listées dans la commande.
\begin{verbatim}
. tabstat DHH LHH, save

   stats |       DHH       LHH
---------+--------------------
    mean |       .75      2.33
------------------------------
\end{verbatim}
L'option \texttt{save} utilisée ci-dessus permet de stocker temporairement
les résultats renvoyés par la commande \texttt{tabstat}, ce qui permet de
les réutiliser pour calculer la différence de moyennes entre les deux
molécules. 
\begin{verbatim}
. return list

matrices:
          r(StatTotal) :  1 x 2
. matrix m = r(StatTotal)
. mat li m

m[1,2]
            DHH        LHH
mean  .75000001       2.33
. display m[1,2] - m[1,1]
1.58
\end{verbatim}
Comme on l'a vu avec R, la manipulation de la variable auxiliaire
\texttt{m}, dans laquelle on a stocké les deux moyennes de groupe, se fait
en appelant ses éléments par numéro de position (la moyenne pour le groupe
LHH se trouve en seconde position, donc peut être utilisée comme
\verb|m[1,2]|). 

On calculera les scores de différences par simple soustraction, et on
stockera les résultats dans une nouvelle variable comme indiqué ci-après.
\begin{verbatim}
. gen sdif = LHH - DHH
. tabstat sdif, stats(mean sd)

    variable |      mean        sd
-------------+--------------------
        sdif |      1.58  1.229995
----------------------------------
\end{verbatim}
Enfin, pour afficher la distribution des scores de différences sous forme
d'un histogramme en imposant des intervalles de classe de 0.5 heure, il
faudra ajouter les options \texttt{bin(8)} (8 intervalles au total) et
\texttt{start(0)} (en débutant à 0).
\begin{verbatim}
. histogram sdif, percent bin(8) start(0)
(bin=8, start=0, width=.57499999)
\end{verbatim}

\includegraphics{./figs/stata_sdif}

Pour réaliser un test $t$ pour données appariées, on utilise toujours la
commande \verb|ttest|, avec cette fois-ci une syntaxe légèrement
différente :
\begin{verbatim}
. ttest DHH == LHH

Paired t test
------------------------------------------------------------------------------
Variable |     Obs        Mean    Std. Err.   Std. Dev.   [95% Conf. Interval]
---------+--------------------------------------------------------------------
     DHH |      10         .75    .5657345     1.78901   -.5297804     2.02978
     LHH |      10        2.33    .6331666    2.002249    .8976776    3.762322
---------+--------------------------------------------------------------------
    diff |      10       -1.58    .3889587    1.229995   -2.459886   -.7001143
------------------------------------------------------------------------------
     mean(diff) = mean(DHH - LHH)                                 t =  -4.0621
 Ho: mean(diff) = 0                              degrees of freedom =        9

 Ha: mean(diff) < 0           Ha: mean(diff) != 0           Ha: mean(diff) > 0
 Pr(T < t) = 0.0014         Pr(|T| > |t|) = 0.0028          Pr(T > t) = 0.9986
\end{verbatim}

Enfin, on peut représenter les gains moyens de temps de sommeil pour
chaque molécule à l'aide d'un diagramme en barres.
\begin{verbatim}
. graph hbar DHH LHH, bargap(20)
\end{verbatim}
\includegraphics{./figs/stata_student}

Ici, on a choisi de représenter les données sous forme de barres
horizontales (\texttt{hbar} au lieu de \texttt{bar}), sachant que par défaut
\Stata calcule automatiquement les moyennes conditionnelles. Pour afficher la
médiane, on utiliserait la commande \verb|graph hbar (median) DHH LHH|.
% 
%
%
\soln{\ref{exo:9.3}} \Stata dispose de commandes appelées "immédiates"
permettant de calculer les statistiques de test associées à des données
saisies directement à l'invite de commande, c'est-à-dire sans passer par
l'importation d'un fichier de données ou la saisie d'un tableau
d'effectifs. Dans le cas présent, on peut répondre aux trois questions
posées à partir de la même commande, \texttt{tabi} (à ne pas confondre
avec la commande externe \texttt{chitesti} qui réalise des tests
d'ajustement à une loi donnée).  
\begin{verbatim}
. tabi 26 21\ 38 44, exact chi2 expected

+--------------------+
| Key                |
|--------------------|
|     frequency      |
| expected frequency |
+--------------------+

           |          col
       row |         1          2 |     Total
-----------+----------------------+----------
         1 |        26         21 |        47 
           |      23.3       23.7 |      47.0 
-----------+----------------------+----------
         2 |        38         44 |        82 
           |      40.7       41.3 |      82.0 
-----------+----------------------+----------
     Total |        64         65 |       129 
           |      64.0       65.0 |     129.0 

          Pearson chi2(1) =   0.9632   Pr = 0.326
           Fisher's exact =                 0.364
   1-sided Fisher's exact =                 0.212
\end{verbatim}
\emph{À noter} : \Stata ne fournit pas de correction de continuité de Yates pour ce
type de test parmi les commandes de base.  
%
%
%
\soln{\ref{exo:9.4}} Comme dans l'exercice précédent, on pourrait travailler
à l'aide de commandes "immédiates". Toutefois, il est parfois utile de
recréer directement le tableau d'effectif sous \Stata. Voici une façon de
procéder. \label{exo:3.5stata}

La première étape consiste à créer trois variables qui permettront de
stocker, pour chaque combinaison de traitement (placebo ou aspirine) et de
réponse (infarctus ou pas d'infarctus), les effectifs observés. Les deux
variables \texttt{traitement} et \texttt{infarctus} sont des variables
binaires (deux modalités, codées 0 et 1). 
\begin{verbatim}
. clear all
. input traitement infarctus N

     traitem~t  infarctus          N
  1. 0 1 28
  2. 0 0 656
  3. 1 1 18
  4. 1 0 658
  5. end
\end{verbatim}
Ensuite, on associe des labels aux modalités des variables nouvellement
créées. On peut ensuite vérifier que les données ont bien été enregistrées
dans le format attendu.
\begin{verbatim}
. label define tx 0 "Placebo" 1 "Aspirine"
. label values traitement tx
. label define ouinon 0 "Non" 1 "Oui"
. label values infarctus ouinon
. list

     +---------------------------+
     | traite~t   infarc~s     N |
     |---------------------------|
  1. |  Placebo        Oui    28 |
  2. |  Placebo        Non   656 |
  3. | Aspirine        Oui    18 |
  4. | Aspirine        Non   658 |
     +---------------------------+
\end{verbatim}

Les diagrammes en barres sont prouits avec la commande \verb|- graph bar -|.
\begin{verbatim}
. graph bar (asis) N, over(infarctus) asyvars over(traitement) legend(title("Infarctus"))
\end{verbatim}
L'option \texttt{asyvars} insérée entre les deux variables de classification
est indispensable si l'on souhaite que les barres aient des couleurs
différentes selon le type de variable considéré.

\includegraphics{./figs/stata_infarctus}
% FIXME:
% display as percent
% catplot infarctus traitement [fw=N], percent recast(bar)

Pour représenter les proportions d'infarctus selon le type de traitement, on
peut utiliser une commande externe de \Stata (à installer de la manière
suivante : \verb|ssc install catplot|) qui est bien commode pour la
représentation graphique des données catégorielles. Son usage est le suivant :
\begin{verbatim}
. catplot infarctus traitement [fw=N] if infarctus==1, percent
\end{verbatim}

\includegraphics{./figs/stata_infarctus2}

On verra dans le chapitre consacré aux données épidémiologiques comment
estimer l'odds-ratio et son intervalle de confiance à partir de la commande
\texttt{cc}. Dans le cas présent, on se contentera d'utiliser
\texttt{tabodds} qui fournit une estimation des odds et de
l'odds-ratio. Pour ce dernier, il faut préciser l'option \texttt{or} comme
indiqué ci-après ainsi que la catégorie de référence (\texttt{base}).
\begin{verbatim}
. tabodds infarctus traitement [fweight=N], or base(2)

---------------------------------------------------------------------------
   infarctus |  Odds Ratio       chi2       P>chi2     [95% Conf. Interval]
-------------+-------------------------------------------------------------
         Non |           .          .           .              .          .
         Oui |           .          .           .              .          .
---------------------------------------------------------------------------
Test of homogeneity (equal odds): chi2(1)  =     2.13
                                  Pr>chi2  =   0.1446

Score test for trend of odds:     chi2(1)  =     2.13
                                  Pr>chi2  =   0.1446
\end{verbatim}
%
%
%
\soln{\ref{exo:9.5}} Le chargement des données ne pose pas vraiment de
problème puisque les données sont déjà disponibles au format \Stata.
\begin{verbatim}
. use polymorphism.dta
. by genotype: summarize age

-----------------------------------------------------------------------------------------
-> genotype = 1.6/1.6

    Variable |       Obs        Mean    Std. Dev.       Min        Max
-------------+--------------------------------------------------------
         age |        14    64.64286    11.18108         43         79

-----------------------------------------------------------------------------------------
-> genotype = 1.6/0.7

    Variable |       Obs        Mean    Std. Dev.       Min        Max
-------------+--------------------------------------------------------
         age |        29    64.37931    13.25954         38         87

-----------------------------------------------------------------------------------------
-> genotype = 0.7/0.7

    Variable |       Obs        Mean    Std. Dev.       Min        Max
-------------+--------------------------------------------------------
         age |        16      50.375    10.63877         33         72
\end{verbatim}
Les intervalles de confiance reportés ci-dessus ne sont pas calculés à
partir de l'ANOVA réalisée ci-après, mais en considérant une approximation à
la loi normale. On peut en revanche obtenir un résumé descriptif plus
détaillé en ajoutant l'option \texttt{detail}: 
\begin{verbatim}
. by genotype: summarize age, detail
\end{verbatim}

Il est possible d'afficher la distribution des âges pour chacun des
génotypes à l'aide de boites à moustaches.
\begin{verbatim}
. graph box age, over(genotype)
\end{verbatim}

\includegraphics{./figs/stata_polymsm}

Si l'on souhaite représenter la distribution des âges à l'aide
d'histogrammes, on peut utiliser la commande \texttt{histogram} de la
manière suivante :
\begin{verbatim}
. histogram age, by(genotype, col(3))
\end{verbatim}
L'option \verb|by(genotype, col(3))| permet d'afficher les distributions
conditionnellement au génotype et d'aligner les histogrammes horizontalement
(c'est-à-dire sur 3 colonnes).

\includegraphics{./figs/stata_histby}

La commande \texttt{oneway} permet de réaliser une analyse de variance à un
effet fixe, comme suit :
\begin{verbatim}
. oneway age genotype

                        Analysis of Variance
    Source              SS         df      MS            F     Prob > F
------------------------------------------------------------------------
Between groups      2315.73355      2   1157.86678      7.86     0.0010
 Within groups      8245.79187     56   147.246283
------------------------------------------------------------------------
    Total           10561.5254     58   182.095266

Bartlett's test for equal variances:  chi2(2) =   1.0798  Prob>chi2 = 0.583
\end{verbatim}
Le tableau d'ANOVA produit par \Stata est sensiblement identique à celui
fourni par \R, à ceci près que \Stata reporte également les totaux pour les
sommes de carrés, carrés moyens et degrés de liberté associés. On notera
également que \Stata fournit le résultat d'un test de Bartlett concernant
l'hypothèse d'homogénéité des variances. Le test de Levene pour tester
l'homogénéité des variances peut être obtenu à l'aide d'une approche un peu
différente : la commande \verb|- robvar -| qui fournit un ensemble de tests
robustes pour l'égalité des variances.
\begin{verbatim}
. robvar age, by(genotype)

            |     Summary of Age at Diagnosis
   Genotype |        Mean   Std. Dev.       Freq.
------------+------------------------------------
    1.6/1.6 |   64.642857   11.181077          14
    1.6/0.7 |    64.37931   13.259535          29
    0.7/0.7 |      50.375   10.638766          16
------------+------------------------------------
      Total |   60.644068   13.494268          59

W0  =  0.83032671   df(2, 56)     Pr > F = 0.44120161

W50 =  0.60460508   df(2, 56)     Pr > F = 0.54981692

W10 =  0.79381598   df(2, 56)     Pr > F = 0.45713722
\end{verbatim}
Le test de Levene est reporté sous la statistique de test \texttt{W0}.

Enfin, notons que le résumé numérique produit à l'étape précédente (moyennes
conditionnelles) peut être reproduit partiellement (sans les intervalles de
confiance) en ajoutant l'option \texttt{tabulate} :
\begin{verbatim}
. oneway age genotype, tabulate
\end{verbatim}
Si l'on souhaite former les intervalles de confiance associés aux moyennes
de groupe à partir des résultats de l'ANOVA, c'est-à-dire en considérant une
estimation de la résiduelle basée sur la variance commune, on peut procéder
comme dans le cas \R, en utilisant le carré moyen de l'erreur
(\texttt{Within groups}) et le quantile de référence de la loi $t$ (0.975)
que l'on peut obtenir comme ceci sous \Stata (ici pour le premier génotype,
1.6/1.6) :
\begin{verbatim}
. display invttail(14-3, 0.025)
2.2009852
\end{verbatim}

Plus généralement, on pourrait générer les bornes inférieures et supérieures
des intervalles de confiance à 95~\% des trois moyennes de groupe à partir
des données aggrégées, comme le montre l'exemple suivant (on ne tient plus
compte de l'estimé de la variance commune dans ce cas) :
\begin{verbatim}
. collapse (mean) agem=age (sd) ages=age (count) n=age, by(genotype)
. generate agelci = agem - invttail(n-1, 0.025)*(ages/sqrt(n))
. generate ageuci = agem + invttail(n-1, 0.025)*(ages/sqrt(n))
\end{verbatim}
et une commande telle que \verb|twoway (bar agem genotype) (rcap ageuci agelci genotype)|
permettrait d'afficher les résultats sous forme graphique. Une solution
alternative et plus commode est d'utiliser la commande \texttt{serrbar}  qui
permet d'afficher une série de moyennes associées à leur erreurs
standard. Comme on souhaite utiliser les intervalles de confiance plutôt que
les erreurs standard, il suffit juste d'ajouter une petite modification à
l'usage standard. En utilisant les calculs précédents, on calcule la
demi-largeur de l'IC (sachant que celui-ci est symétrique autour de la moyenne)
\begin{verbatim}
. gen ageb = agem-agelci
. serrbar agem ageb genotype, xlabel(1 "1.6/1.6" 2 "1.6/0.7" 3 "0.7/0.7")
\end{verbatim}

\includegraphics{./figs/stata_sebar}


On notera que les différences de moyennes peuvent être obtenues en formant
des contrastes spécifiques à partir d'un modèle de régression qui donne des
résultats équivalents au modèle d'ANOVA, sous réserve de coder la variable
qualitative \texttt{genotype} sous forme d'une matrice d'indicatrices à
l'aide de l'opérateur \verb|i.*|. On ne présente pas les résultats de
l'appel à la commande \texttt{regress} puisque ce qui nous intéresse est
simplement d'exploiter les résultats qu'elle sauvegarde (cf. \verb|e()|).
\begin{verbatim}
. regress age i.genotype
\end{verbatim}

Pour la différence de moyennes entre génotype 0.7/0.7 et 1.6/0.7, par
exemple, on utiliserait la commande suivante :
\begin{verbatim}
. lincom 3.genotype - 2.genotype

 ( 1)  - 2.genotype + 3.genotype = 0

------------------------------------------------------------------------------
         age |      Coef.   Std. Err.      t    P>|t|     [95% Conf. Interval]
-------------+----------------------------------------------------------------
         (1) |  -14.00431   3.778935    -3.71   0.000    -21.57443   -6.434194
------------------------------------------------------------------------------
\end{verbatim}
On procédera de même pour les deux autres différences de moyennes, 0.7/0.7 -
1.6/1.6 et 1.6/0.7 - 1.6/1.6.

En fait, la même approche (par régression) nous permettrait d'obtenir les
intervalles de confiance à 95~\% pour chaque moyenne de groupe :
\begin{verbatim}
. lincom _cons + 1.genotype

 ( 1)  1b.genotype + _cons = 0

------------------------------------------------------------------------------
         age |      Coef.   Std. Err.      t    P>|t|     [95% Conf. Interval]
-------------+----------------------------------------------------------------
         (1) |   64.64286   3.243084    19.93   0.000     58.14618    71.13953
------------------------------------------------------------------------------
\end{verbatim}

\soln{\ref{exo:4.2}}
Les données peuvent être saisies manuellement dans \Stata (ou à l'aide de
l'éditeur de données, ou \texttt{Data Editor}), comme présentées dans le
tableau de l'énoncé, c'est-à-dire trois variables organisées en colonnes :

\includegraphics{./figs/stata_dataentry}

Pour utiliser la commande \texttt{oneway}, il est toutefois nécessaire de
réarranger ces trois variables de façon à avoir une variable réponse et une
variable décrivant le facteur de classification (à trois classes, A, B et
C). Pour cela, on utilise la commande \texttt{stack} qui permet de
concaténer (verticalement) des variables arrangées en colonnes.
\begin{verbatim}
. stack A B C, into(pb) clear
. rename _stack tx
. label define txlab 1 "A" 2 "B" 3 "C"
. label values tx txlab
. list in 1/5

     +-----------+
     | tx     pb |
     |-----------|
  1. |  A   19.8 |
  2. |  A   20.5 |
  3. |  A   23.7 |
  4. |  A   27.1 |
  5. |  A   29.6 |
     +-----------+
\end{verbatim}

Concernant le résumé descriptif des données, on pourra procéder comme suit :
\begin{verbatim}
. tabstat pb, stats(mean sd n) by(tx)

Summary for variables: pb
     by categories of: tx 

    tx |      mean        sd         N
-------+------------------------------
     A |      25.1  4.438468         6
     B |  20.76667  2.798333         6
     C |  18.28333  1.965113         6
-------+------------------------------
 Total |  21.38333  4.199335        18
--------------------------------------
. graph box pb, over(tx)
\end{verbatim}

\includegraphics{./figs/stata_pbtx}

L'ANOVA à un facteur est réalisée comme dans l'exercice précédent avec la
commande \texttt{oneway}. Notons que l'on a ajouté l'option
\texttt{bonferroni} pour tester automatiquement les paires de moyennes entre
elles. Le résultat du test F est significatif et le test de Bartlett
concernant l'homogénéité des variances suggère que cette hypothèse est
raisonnable au vu des données. Les tests multiples protégés permettent de
préciser 
\begin{verbatim}
. oneway pb tx, bonferroni

                        Analysis of Variance
    Source              SS         df      MS            F     Prob > F
------------------------------------------------------------------------
Between groups      142.823331      2   71.4116655      6.82     0.0078
 Within groups      156.961681     15   10.4641121
------------------------------------------------------------------------
    Total           299.785012     17   17.6344125

Bartlett's test for equal variances:  chi2(2) =   3.0035  Prob>chi2 = 0.223

                            Comparison of pb by tx
                                (Bonferroni)
Row Mean-|
Col Mean |          A          B
---------+----------------------
       B |   -4.33333
         |      0.105
         |
       C |   -6.81667   -2.48333
         |      0.007      0.610
\end{verbatim}

Si l'hypothèse de normalité doit être remise en cause, on peut alors
utiliser la commande \texttt{kwallis} pour réaliser une ANOVA sur les rangs
(approche non-paramétrique).
\begin{verbatim}
. kwallis pb, by(tx)

Kruskal-Wallis equality-of-populations rank test

  +---------------------+
  | tx | Obs | Rank Sum |
  |----+-----+----------|
  |  A |   6 |    82.00 |
  |  B |   6 |    59.00 |
  |  C |   6 |    30.00 |
  +---------------------+

chi-squared =     7.942 with 2 d.f.
probability =     0.0189

chi-squared with ties =     7.942 with 2 d.f.
probability =     0.0189
\end{verbatim}
Le résultat ci-dessus est cohérent avec le résultat de l'ANOVA
paramétrique. Pour la comparaison des paires de variables, on utilisera un
test de Wilcoxon (échantillons indépendants), en corrigeant la $p$-valeur
par le nombre de tests réalisés (ici, 3). On peut soit utiliser la commande
\texttt{ranksum}, par exemple \verb+ranksum pb if tx == 1 | tx == 2, by(tx)+, 
pour comparer les deux premiers traitements, ou alors la commande
\texttt{kwallis2} qui fournit les mêmes résultats que \texttt{kwallis} mais
également les tests post-hoc associés. Il est toutefois nécessaire
d'installer ce package (\texttt{findit kwallis2}, puis suivre la procédure
d'installation indiquée).
\begin{verbatim}
. kwallis2 pb, by(tx)

One-way analysis of variance by ranks (Kruskal-Wallis Test)

tx       Obs   RankSum  RankMean 
--------------------------------
  1        6     82.00     13.67
  2        6     59.00      9.83
  3        6     30.00      5.00

Chi-squared (uncorrected for ties) =     7.942 with    2 d.f. (p = 0.01886)
Chi-squared (corrected for ties)   =     7.942 with    2 d.f. (p = 0.01886)

Multiple comparisons between groups
-----------------------------------
(Adjusted p-value for significance is 0.008333)

Ho: pb(tx==1) = pb(tx==2)
    RankMeans difference =      3.83  Critical value =      7.38
    Prob = 0.106805 (NS)

Ho: pb(tx==1) = pb(tx==3)
    RankMeans difference =      8.67  Critical value =      7.38
    Prob = 0.002463 (S)

Ho: pb(tx==2) = pb(tx==3)
    RankMeans difference =      4.83  Critical value =      7.38
    Prob = 0.058424 (NS)
\end{verbatim}

\soln{\ref{exo:4.3}}
Il n'existe pas de solution très commode pour importer un fichier SPSS sous
\Stata directement, sauf lorsque l'on travaille sous Windows (voir la
commande \texttt{usespss}). Le fichier contenant les données,
\texttt{weights.sav}, a été exporté depuis R au format \Stata à l'aide des
commandes suivantes :
\begin{verbatim}
library(foreign)
weights <- read.spss("weights.sav", to.data.frame=TRUE)
write.dta(weights, file="weights.dta")
\end{verbatim}
Depuis \Stata, on peut donc l'importer très simplement à l'aide de la
commande \texttt{use} :
\begin{verbatim}
. use "weights.dta", clear
. list in 1/5

     +----------------------------------------------------------------+
     |   ID   WEIGHT   LENGTH   HEADC   GENDER   EDUCATIO      PARITY |
     |----------------------------------------------------------------|
  1. | L001     3.95     55.5    37.5   Female   tertiary   3 or more |
  2. | L003     4.63       57    38.5   Female   tertiary   Singleton |
  3. | L004     4.75       56    38.5     Male     year12   2 sibling |
  4. | L005     3.92       56      39     Male   tertiary   One sibli |
  5. | L006     4.56       55    39.5     Male     year10   2 sibling |
     +----------------------------------------------------------------+
\end{verbatim}

Le tableau d'effectifs et de fréquences relatives pour la variable
\texttt{PARITY} s'obtient à partir de la commande \texttt{tabulate} :
\begin{verbatim}
. tabulate PARITY

            PARITY |      Freq.     Percent        Cum.
-------------------+-----------------------------------
         Singleton |        180       32.73       32.73
       One sibling |        192       34.91       67.64
        2 siblings |        116       21.09       88.73
3 or more siblings |         62       11.27      100.00
-------------------+-----------------------------------
             Total |        550      100.00
\end{verbatim}
Les moyennes et écarts-type du poids selon la taille de la fratrie
s'obtiennent ainsi :
\begin{verbatim}
. tabstat WEIGHT, stats(mean sd) by(PARITY) format(%9.2f)

Summary for variables: WEIGHT
     by categories of: PARITY (PARITY)

          PARITY |      mean        sd
-----------------+--------------------
       Singleton |      4.26      0.62
     One sibling |      4.39      0.59
      2 siblings |      4.46      0.61
3 or more siblin |      4.43      0.54
-----------------+--------------------
           Total |      4.37      0.60
--------------------------------------
\end{verbatim}
L'option \verb|format(%9.2f)| permet de limiter l'affichage des nombres à
deux décimales.

L'ANOVA à un facteur se réalise comme dans les exercices précédents, à
l'aide de la commande \texttt{oneway} et en fournissant la variable réponse
et la variable qualitative décrivant les groupes à comparer.
\begin{verbatim}
. oneway WEIGHT PARITY

                        Analysis of Variance
    Source              SS         df      MS            F     Prob > F
------------------------------------------------------------------------
Between groups       3.4769599      3   1.15898663      3.24     0.0219
 Within groups      195.364879    546   .357811134
------------------------------------------------------------------------
    Total           198.841839    549   .362189142

Bartlett's test for equal variances:  chi2(3) =   1.9085  Prob>chi2 = 0.592
\end{verbatim}
Comme on l'a vu avec R, on peut calculer la part de variance expliquée à
partir des sommes de carré reportées ci-dessus. On peut également utiliser
la commande \texttt{anova} qui renvoit, outre un tableau d'analyse de
variance, le coefficient de détermination associé au modèle.
\begin{verbatim}
. anova WEIGHT PARITY

                           Number of obs =     550     R-squared     =  0.0175
                           Root MSE      = .598173     Adj R-squared =  0.0121

                  Source |  Partial SS    df       MS           F     Prob > F
              -----------+----------------------------------------------------
                   Model |   3.4769599     3  1.15898663       3.24     0.0219
                         |
                  PARITY |   3.4769599     3  1.15898663       3.24     0.0219
                         |
                Residual |  195.364879   546  .357811134   
              -----------+----------------------------------------------------
                   Total |  198.841839   549  .362189142
\end{verbatim}

Pour afficher les données sous forme d'histogrammes pour chaque groupe, il
faut utiliser la commande \texttt{histogram} avec l'option \texttt{by} pour
définir le facteur de classification. L'option \texttt{freq} permet quant à
elle d'afficher des effectifs plutôt que des proportions (ou densités).
\begin{verbatim}
. histogram WEIGHT, by(PARITY) freq
\end{verbatim}

\includegraphics{./figs/stata_histweight}

Pour afficher afficher un diagramme de dispersion, comme sous R, on peut
utiliser une commande externe, telle que \texttt{stripplot} (\texttt{ssc
  install stripplot}), par exemple
\begin{verbatim}
. stripplot WEIGHT, over(PARITY) stack height(.4) center vertical width(.3)
\end{verbatim}
ou plus simplement 
\begin{verbatim}
. twoway scatter WEIGHT PARITY, jitter(3) xlabel(1 "Singleton" 2 "One sibling" 3 "2 siblings" 4 "3 more")
\end{verbatim}

\includegraphics{./figs/stata_stripheight}

La commande \texttt{oneway} affiche par défaut le résultat d'un test de
Bartlett pour comparer les variances des groupes entre elles. Si l'on
souhaite utiliser un test de Levene, il faut utiliser la commande
\texttt{robvar} qui fournit le résultat du test de Levene (\texttt{W0}) :
\begin{verbatim}
. robvar WEIGHT, by(PARITY)

            |          Summary of WEIGHT
     PARITY |        Mean   Std. Dev.       Freq.
------------+------------------------------------
  Singleton |   4.2589445   .61950106         180
  One sibli |   4.3886979   .59258231         192
  2 sibling |   4.4600862   .60519991         116
  3 or more |   4.4341935   .53526321          62
------------+------------------------------------
      Total |   4.3664182   .60182152         550

W0  =  0.63850632   df(3, 546)     Pr > F = 0.59046456

W50 =  0.64415596   df(3, 546)     Pr > F = 0.58688851

W10 =  0.64584135   df(3, 546)     Pr > F = 0.58582455
\end{verbatim}

Il existe plusieurs façon de recoder des variables sous \Stata, mais dans le
cas présent le plus simple pour aggréger les deux dernières classes
(\texttt{2 siblings} et \texttt{3 or more}) consiste à générer une deuxième
variable, \texttt{PARITY2}, de la manière suivante :
\begin{verbatim}
. recode PARITY (1=1) (2=2) (3/4=3), gen(PARITY2)
\end{verbatim}
Pour vérifier que la conversion s'est déroulée correctement, on affichera un
simple tableau de contingence croisant les effectifs des deux variables :
\begin{verbatim}
. tabulate PARITY PARITY2

                   |    RECODE of PARITY (PARITY)
            PARITY |         1          2          3 |     Total
-------------------+---------------------------------+----------
         Singleton |       180          0          0 |       180 
       One sibling |         0        192          0 |       192 
        2 siblings |         0          0        116 |       116 
3 or more siblings |         0          0         62 |        62 
-------------------+---------------------------------+----------
             Total |       180        192        178 |       550
\end{verbatim}

Les résultats de l'ANOVA à un facteur considérant la variable nouvellement
créée sont reportés ci-dessous :
\begin{verbatim}
. oneway WEIGHT PARITY2

                        Analysis of Variance
    Source              SS         df      MS            F     Prob > F
------------------------------------------------------------------------
Between groups      3.44987154      2   1.72493577      4.83     0.0083
 Within groups      195.391968    547   .357206522
------------------------------------------------------------------------
    Total           198.841839    549   .362189142

Bartlett's test for equal variances:  chi2(2) =   0.7963  Prob>chi2 = 0.672
\end{verbatim}
Pour le test de la tendance linéaire, on présente deux manières (méthode des
contrastes et régression linéaire). Pour générer un contraste testant la
tendance linéaire, il faut explicitement demander à \Stata d'effectuer une
régression en considérant la variable \texttt{PARITY2} comme une variable
qualitative (passage par des variables indicatrices codant pour les deux
dernières modalités du facteur). On utilisera dans ce cas la commande de
post-estimation \texttt{contrast}. Comme le résultat de la régression sur la
variable qualitative ne nous intéresse pas vraiment, on demandera à \Stata de
ne pas afficher les résultats à l'aide de l'instruction \verb|quietly :|.
\begin{verbatim}
. quietly: regress WEIGHT i.PARITY2
. contrast p.PARITY2, noeffects

Contrasts of marginal linear predictions

Margins      : asbalanced

------------------------------------------------
             |         df           F        P>F
-------------+----------------------------------
     PARITY2 |
   (linear)  |          1        9.25     0.0025
(quadratic)  |          1        0.40     0.5288
      Joint  |          2        4.83     0.0083
             |
    Residual |        547
------------------------------------------------
\end{verbatim}
Le contraste d'intérêt apparaît sur la ligne intitulée \verb|(linear)|.

Pour l'approche par régression linéaire simple, la commande est plus simple :
\begin{verbatim}
. regress WEIGHT PARITY2

      Source |       SS       df       MS              Number of obs =     550
-------------+------------------------------           F(  1,   548) =    9.27
       Model |  3.30800821     1  3.30800821           Prob > F      =  0.0024
    Residual |  195.533831   548   .35681356           R-squared     =  0.0166
-------------+------------------------------           Adj R-squared =  0.0148
       Total |  198.841839   549  .362189142           Root MSE      =  .59734

------------------------------------------------------------------------------
      WEIGHT |      Coef.   Std. Err.      t    P>|t|     [95% Conf. Interval]
-------------+----------------------------------------------------------------
     PARITY2 |   .0961272   .0315707     3.04   0.002     .0341129    .1581415
       _cons |   4.174513   .0679786    61.41   0.000     4.040983    4.308044
------------------------------------------------------------------------------
\end{verbatim}
Le test de tendance correspond au test associé à la pente de la droite de
régression, ici le coefficient \texttt{PARITY2}.


%---------------------------------------------------------------- Séance 10 --
\chapter*{Semaine 10\markboth{Corrigés de la semaine 10}{}}

\soln{\ref{exo:5.1}}
Les données tabulées sont disponibles dans un fichier texte que l'on peut
importer à l'aide de la commande \texttt{insheet}. Toutefois, on peut
également importer les données avec \texttt{infile} qui a l'avantage de
fonctionner même dans le cas où les données ne sont pas issue d'un tableau
de type Excel.
\begin{verbatim}
. infile int (Sub Age Sex) Height Weight BMP FEV RV FRC TLC PEmax using "cystic.dat" in 2/26, clear
\end{verbatim}
Dans l'instruction ci-dessus, on a explicitement demandé à \Stata de coder
les trois premières variables sous forme d'entiers et non de nombres réels,
ce qui dans le présent ne change pas grand-chose mais permet d'économiser de
l'espace mémoire pour les gros jeux de données.
Comme on l'a fait avec R, on recodera d'emblée la variable \texttt{Sex} en
variable qualitative pour éviter toute confusion.
\begin{verbatim}
. label define labsex 0 "M" 1 "F"
. label values Sex labsex
. tabulate Sex

        Sex |      Freq.     Percent        Cum.
------------+-----------------------------------
          M |         14       56.00       56.00
          F |         11       44.00      100.00
------------+-----------------------------------
      Total |         25      100.00
\end{verbatim}

L'estimation du coefficient de corrélation de Bravais-Pearson se fait avec
la commande \texttt{correlate}, mais celle-ci ne fournit pas d'options pour
l'intervalle de confiance. On peut utiliser des commandes externes comme
\texttt{ci2} qui fonctionne sur le même principe que la commande interne
\texttt{ci} (pour les moyennes et proportions) dans le cas paramétrique ou
non-paramétrique (Spearman) ou bien \texttt{corrci}, uniquement pour le cas
paramétrique. \label{cmd:corrci}
\begin{verbatim}
. correlate PEmax Weight
(obs=25)

             |    PEmax   Weight
-------------+------------------
       PEmax |   1.0000
      Weight |   0.6363   1.0000
. corrci PEmax Weight

(obs=25)

                correlation and 95% limits
PEmax  Weight      0.636    0.322    0.824
\end{verbatim}

Pour tester ce coefficient de corrélation contre l'alternative $H_0:
\rho=0.3$, il faut
% FIXME:
% change to compute test against H_0: \rho=0.3
\begin{verbatim}
. pwcorr PEmax Weight, sig

             |    PEmax   Weight
-------------+------------------
       PEmax |   1.0000 
             |
             |
      Weight |   0.6363   1.0000 
             |   0.0006
             |
\end{verbatim}

Pour afficher l'ensemble des diagrammes de dispersion, on utilise la
commande \texttt{graph matrix} en sépcifiant la liste de variables que l'on
souhaite voir figurer dans le graphique. Ici, on omettra la variable
\texttt{Sex} : 
\begin{verbatim}
. graph matrix Age Height-PEmax
\end{verbatim}

\includegraphics{./figs/stata_splom}

Les corrélations pour chaque paire de variables s'obtiennent avec la
commande \texttt{pwcorr}, et on utilisera la même notation pour indiquer les
variables qui nous intéressent.
\begin{verbatim}
. pwcorr Age Height-PEmax

             |      Age   Height   Weight      BMP      FEV       RV      FRC      TLC      PEmax
-------------+-----------------------------------------------------------------------------------
         Age |   1.0000 
      Height |   0.9261   1.0000 
      Weight |   0.9065   0.9221   1.0000 
         BMP |   0.3778   0.4408   0.6703   1.0000 
         FEV |   0.2945   0.3167   0.4492   0.5455   1.0000 
          RV |  -0.5519  -0.5695  -0.6234  -0.5824  -0.6659   1.0000 
         FRC |  -0.6394  -0.6243  -0.6182  -0.4344  -0.6651   0.9106   1.0000 
         TLC |  -0.4734  -0.4595  -0.4214  -0.3633  -0.4425   0.5899   0.7056    1.0000 
       PEmax |   0.6135   0.5992   0.6363   0.2295   0.4534  -0.3156  -0.4172   -0.1805   1.0000
\end{verbatim}
Pour les corrélations de Spearman, on remplacera \texttt{pwcorr} par
\texttt{spearman}.
\begin{verbatim}
. spearman Age Height-PEmax, stats(rho)
(obs=25)

             |      Age   Height   Weight      BMP      FEV       RV      FRC      TLC    PEmax
-------------+---------------------------------------------------------------------------------
         Age |   1.0000 
      Height |   0.9335   1.0000 
      Weight |   0.9013   0.9619   1.0000 
         BMP |   0.5091   0.5734   0.7264   1.0000 
         FEV |   0.2975   0.4256   0.4637   0.5623   1.0000 
          RV |  -0.5815  -0.6221  -0.7005  -0.6917  -0.6830   1.0000 
         FRC |  -0.7185  -0.6642  -0.6706  -0.5547  -0.6044   0.8547   1.0000 
         TLC |  -0.4926  -0.4734  -0.4846  -0.4935  -0.4398   0.5895   0.6721   1.0000 
       PEmax |   0.5198   0.5920   0.4881   0.2224   0.3140  -0.3089  -0.3835  -0.1482   1.0000
\end{verbatim}

% FIXME:
% filtrage correlations elevees à faire.

La corrélation partielle entre les variables \texttt{PEmax} et
\texttt{Weight} en tenant compte de \texttt{Age} est obtenue à l'aide de la
commande \texttt{pcorr} ; la variable d'intérêt doit être située en première
position dans la liste des variables et par défaut \Stata affiche l'ensemble
des coefficients de corrélation partielle pour les autres variables.
\begin{verbatim}
. pcorr PEmax Weight Age
(obs=25)

Partial and semipartial correlations of PEmax with

               Partial   Semipartial      Partial   Semipartial   Significance
   Variable |    Corr.         Corr.      Corr.^2       Corr.^2          Value
------------+-----------------------------------------------------------------
     Weight |   0.2405        0.1899       0.0578        0.0361         0.2577
        Age |   0.1126        0.0869       0.0127        0.0076         0.6002
\end{verbatim}
Pour réaliser un tercilage de la variable \texttt{Age}, on peut procéder comme à
l'exercice~\ref{exo:2.4}, c'est-à-dire en créant une variable dérivée, avec
\texttt{egen}, et en utilisant la fonction \texttt{cut}. Mais on peut
directement utiliser la commande \texttt{xtile} qui est plus simple d'emploi :
\begin{verbatim}
. xtile Age3 = Age, nq(3)
\end{verbatim}
Enfin, pour le diagramme de dispersion, voici une première solution :
\begin{verbatim}
. scatter PEmax Weight if Age3 != 2, mlab(Age3)
\end{verbatim}
On pourrait également (2\ieme\ solution) enchaîner deux appels à la commande
\texttt{scatter}, en restreignant à chaque fois l'échantillon aux seules
classe de \texttt{Age3} qui nous intéressent (1 et 3, en l'occurence).
\begin{verbatim}
. scatter PEmax Weight if Age3 == 1, msymbol(circle) || scatter PEmax Weight if Age3 == 3, msymbol(square) 
  legend(label(1 "1st tercile") label(2 "3rd tercile"))
\end{verbatim}

\includegraphics{./figs/stata_tercile}

\soln{\ref{exo:5.2}}
Le chargement des données ne pose pas de difficultés particulières car
celles-ci ont été exportées depuis Excel et sont au format CSV. On veillera
cependant à préciser le type de délimiteur de champ, ici un point-virgule/
\begin{verbatim}
. insheet using "/Users/chl/Documents/Tutors/cesam-r/tex/quetelet.csv", delim(";") clear
(5 vars, 32 obs)
. describe

Contains data
  obs:            32                          
 vars:             5                          
 size:           320                          
-----------------------------------------------------------------------------------------
              storage  display     value
variable name   type   format      label      variable label
-----------------------------------------------------------------------------------------
id              byte   %8.0g                  ID
pas             int    %8.0g                  PAS
qtt             str5   %9s                    QTT
age             byte   %8.0g                  AGE
tab             byte   %8.0g                  TAB
-----------------------------------------------------------------------------------------
Sorted by:
\end{verbatim}
Après l'importation, on remarque que la variable \texttt{qtt} n'est pas
reconnue en tant que nombre mais a été codée sous forme de chaîne de
caractère. Ceci s'explique par le fait que la partie décimale est séparée de
la partie entière par une virgule et non un point. Il faut donc convertir cette
variable en nombre, ce que l'on peut réaliser avec la commande
\texttt{destring} et l'option \texttt{dpcomma}.
\begin{verbatim}
. destring qtt, dpcomma replace
qtt has all characters numeric; replaced as double
\end{verbatim}
Ensuite, on recode la variable \texttt{tab} en variable qualitative avec des
étiquettes plus informatives. 
\begin{verbatim}
. label define ltab 0 "NF" 1 "F"
. label values tab ltab
. list in 1/5

     +------------------------------+
     | id   pas     qtt   age   tab |
     |------------------------------|
  1. |  1   135   2.876    45    NF |
  2. |  2   122   3.251    41    NF |
  3. |  3   130     3.1    49    NF |
  4. |  4   148   3.768    52    NF |
  5. |  5   146   2.979    54     F |
     +------------------------------+
\end{verbatim}
Enfin, le résumé numérique des variable s'obtient avec la commande \texttt{summarize}.
\begin{verbatim}
. summarize pas-tab

    Variable |       Obs        Mean    Std. Dev.       Min        Max
-------------+--------------------------------------------------------
         pas |        32    144.5313    14.39755        120        180
         qtt |        32    3.441094    .4970781      2.368      4.637
         age |        32       53.25    6.956083         41         65
         tab |        32      .53125    .5070073          0          1
\end{verbatim}
Comme la variable \texttt{tab} est internellement codée en 0/1, la moyenne
correspond à la fréquence relative des fumeurs, soit 53~\%.

Le coefficeint de corrélation liénaire entre les variables \texttt{pas} et
\texttt{qtt} peut être estimé, ainsi que son intervalle de confiance, avec
la commande \texttt{corrci} (p.~\pageref{cmd:corrci}). L'option
\texttt{level(90)} permet de spécifier le niveau de confiance.
\begin{verbatim}
. corrci pas qtt, level(90)

(obs=32)

          correlation and 90% limits
pas qtt      0.742    0.571    0.851
\end{verbatim}

La commande \texttt{regress} permet d'effectuer une régression linéaire
pour une variable réponse (placée en premier dans la liste des variables) et
une ou plusieurs variables explicatives. On l'utilise comme suit :
\begin{verbatim}
. regress pas qtt

      Source |       SS       df       MS              Number of obs =      32
-------------+------------------------------           F(  1,    30) =   36.75
       Model |  3537.94574     1  3537.94574           Prob > F      =  0.0000
    Residual |  2888.02301    30  96.2674337           R-squared     =  0.5506
-------------+------------------------------           Adj R-squared =  0.5356
       Total |  6425.96875    31  207.289315           Root MSE      =  9.8116

------------------------------------------------------------------------------
         pas |      Coef.   Std. Err.      t    P>|t|     [95% Conf. Interval]
-------------+----------------------------------------------------------------
         qtt |   21.49167   3.545147     6.06   0.000     14.25151    28.73182
       _cons |    70.5764   12.32187     5.73   0.000     45.41179    95.74101
------------------------------------------------------------------------------
\end{verbatim}
Par défaut, on obtient un tableau d'analyse de variance pour le modèle de
régression et un tableau des coefficients du modèle, ici l'ordonnée à
l'origine (70.58) et la pente de la droite de régression (21.49). Le test
de Student associé à la pente permet d'évaluer sa significativité au vu des
données. Si l'on souhaite manipuler les coefficients de régression, il est
possible de les extraire du tableau de résultats comme indiqué ci-dessous :
\begin{verbatim}
. matrix b = e(b)
. svmat b
. di "pente = " b1 ", ordonnée origine = " b2
pente = 21.491669, ordonnée origine = 70.576401
\end{verbatim}

En ce qui concerne l'affichage du diagramme de dispersion avec les deux
droites de régression superposées, on utilise une combinaison de
\texttt{lfit} (pour tracer la droite de régression) et \texttt{scatter}
(pour afficher les observations). L'inconvénient est qu'il est nécessaire de
construire manuellement la légende. On notera également que l'on impose que
les droites de régression soient présentées sur toute l'étendue de la
variable \texttt{qtt}, d'où l'option \texttt{range(2, 5)}. \label{para:twoway}
\begin{verbatim}
. twoway lfit pas qtt if tab == 0, range(2 5) lpattern(dot) || 
  scatter pas qtt if tab == 0, msymbol(square) || 
  lfit pas qtt if tab == 1, range(2 5) ||
  scatter pas qtt if tab == 1, msymbol(circle) 
  legend(label(1 "") label(2 "NF") label(3 "") label(4 "F"))
\end{verbatim}

\includegraphics{./figs/stata_lfit}

La régression de \texttt{qtt} sur \texttt{pas} en restreignant l'analyse aux
seules observations du groupe fumeur (\verb|tab == 1| ou \verb|tab == "F"|) 
ne pose pas de problème car on vient de le voir dans l'application
précdente, il suffit d'ajouter une option \texttt{if tab == 1} :
\begin{verbatim}
. regress pas qtt if tab == 1

      Source |       SS       df       MS              Number of obs =      17
-------------+------------------------------           F(  1,    15) =   19.40
       Model |  2088.16977     1  2088.16977           Prob > F      =  0.0005
    Residual |  1614.30082    15  107.620055           R-squared     =  0.5640
-------------+------------------------------           Adj R-squared =  0.5349
       Total |  3702.47059    16  231.404412           Root MSE      =  10.374

------------------------------------------------------------------------------
         pas |      Coef.   Std. Err.      t    P>|t|     [95% Conf. Interval]
-------------+----------------------------------------------------------------
         qtt |   20.11804   4.567193     4.40   0.001      10.3833    29.85278
       _cons |   79.25533   15.76837     5.03   0.000     45.64585    112.8648
------------------------------------------------------------------------------
\end{verbatim}

\soln{\ref{exo:5.3}}
Puisque les données sont disponibles au format CSV \og classique\fg\
(utilisant la virgule comme séparateur de champ), on peut les importer très
simplement à l'aide de la commande \texttt{insheet}. 
\begin{verbatim}
. insheet using "/Users/chl/Documents/Tutors/cesam-r/tex/Framingham.csv"
(10 vars, 4699 obs)
. describe, simple
sex       sbp       dbp       scl       chdfate   followup  age       bmi   month     id
. list in 1/5

     +------------------------------------------------------------------------+
     | sex   sbp   dbp   scl   chdfate   followup   age    bmi   month     id |
     |------------------------------------------------------------------------|
  1. |   1   120    80   267         1         18    55     25       8   2642 |
  2. |   1   130    78   192         1         35    53   28.4      12   4627 |
  3. |   1   144    90   207         1        109    61   25.1       8   2568 |
  4. |   1    92    66   231         1        147    48   26.2      11   4192 |
  5. |   1   162    98   271         1        169    39   28.4      11   3977 |
     +------------------------------------------------------------------------+
\end{verbatim}
Pour faciliter l'interprétation et la lecture des tableaux de résultats, on
associera des étiquettes plus informatives à la variable qualitative \texttt{sex}.
\begin{verbatim}
. label define labsex 1 "M" 2 "F"
. label values sex labsex
. tabulate sex

        sex |      Freq.     Percent        Cum.
------------+-----------------------------------
          M |      2,049       43.61       43.61
          F |      2,650       56.39      100.00
------------+-----------------------------------
      Total |      4,699      100.00
\end{verbatim}
Pour afficher un résumé du nombre de données manquantes pour chacune des
variables de ce tableau de données, on peut utiliser la commande suivante :
\begin{verbatim}
. misstable summarize
                                                               Obs<.
                                                +------------------------------
               |                                | Unique
      Variable |     Obs=.     Obs>.     Obs<.  | values        Min         Max
  -------------+--------------------------------+------------------------------
           scl |        33               4,666  |    248        115         568
           bmi |         9               4,690  |    248       16.2        57.6
  -----------------------------------------------------------------------------
\end{verbatim}
On voit donc que les variables \texttt{scl} et \texttt{bmi} incluent 33 et 9
valeurs manquantes, respectivement. D'où le tableau corrigé des effectifs
par sexe :
\begin{verbatim}
. tabulate sex if bmi < .

        sex |      Freq.     Percent        Cum.
------------+-----------------------------------
          M |      2,047       43.65       43.65
          F |      2,643       56.35      100.00
------------+-----------------------------------
      Total |      4,690      100.00
\end{verbatim}

Sous \Stata, il n'est pas possible d'utiliser des marqueurs transparents dans
un diagramme de dispersion. Dans le cas où le nombre d'observations est
élevé, Il est donc préférable de modifier le type de symbole par défaut
(disque) et d'utiliser de petits cercles (\texttt{oh} ou \texttt{Oh}).
\begin{verbatim}
. scatter sbp bmi, by(sex) msymbol(Oh)
\end{verbatim}

\includegraphics{./figs/stata_overplotting}

Le coefficient de corrélation linéaire entre les variables \texttt{sbp} et
\texttt{bmi} selon le sexe peut être obtenu à l'aide de \texttt{correlate}
après stratification sur le facteur \texttt{sex}. On notera que l'option
\texttt{by} figure en premier et qu'il est nécessaire de trier les données
dans un premier temps, d'où l'ajout de la commande \texttt{sort}
immédiatement après la stratification.
\begin{verbatim}
. by sex, sort: correlate sbp bmi

----------------------------------------------------------------------------------------
-> sex = M
(obs=2047)

             |      sbp      bmi
-------------+------------------
         sbp |   1.0000
         bmi |   0.2364   1.0000


----------------------------------------------------------------------------------------
-> sex = F
(obs=2643)

             |      sbp      bmi
-------------+------------------
         sbp |   1.0000
         bmi |   0.3736   1.0000
\end{verbatim}

% FIXME:
% test correlation by sex

Pour le modèle de régression, on va d'abord transformer les variables
\texttt{bmi} (variable explicative) et \texttt{sbp} (variable réponse) à
l'aide d'une transformation logarithmique.
\begin{verbatim}
. gen logbmi = log(bmi)
(9 missing values generated)
. gen logsbp = log(sbp)
\end{verbatim}
On peut ensuite vérifier visuellement à l'aide d'un histogramme (ou d'un
QQ-plot) que cette transformation a bien permis de ramener les distributions
de ces deux variables proches de la normale. Pour afficher simultanément les
quatre histogrammes, on peut sauver chacune des figures au format \Stata
(\texttt{gph}) et ensuite les combiner à l'aide de la commande \texttt{graph combine}.
\begin{verbatim}
. histogram bmi, saving(gphbmi)
(bin=36, start=16.200001, width=1.1499999)
(file gphbmi.gph saved)
. histogram logbmi, saving(gphlogbmi)
(bin=36, start=2.7850113, width=.03523642)
(file gphlogbmi.gph saved)
. histogram sbp, saving(gphsbp)
(bin=36, start=80, width=5.2777778)
(file gphsbp.gph saved)
. histogram logsbp, saving(gphlogsbp)
(bin=36, start=4.3820267, width=.03378876)
(file gphlogsbp.gph saved)
. graph combine gphbmi.gph gphlogbmi.gph gphsbp.gph gphlogsbp.gph
\end{verbatim}

\includegraphics{./figs/stata_combine}

Le modèle de régression stratifié par sexe ne pose pas de problème majeur,
et contrairement à R il n'est pas nécessaire de calculer les intervalles de
confiance pour les pentes avec une commande séparée car ceux-ci sont
directement fournis dans le tableau de résultats renvoyés par \Stata.
\begin{verbatim}
. regress logsbp logbmi if sex == 1, noheader
------------------------------------------------------------------------------
      logsbp |      Coef.   Std. Err.      t    P>|t|     [95% Conf. Interval]
-------------+----------------------------------------------------------------
      logbmi |    .272646   .0232152    11.74   0.000     .2271182    .3181739
       _cons |   3.988043   .0754584    52.85   0.000      3.84006    4.136026
------------------------------------------------------------------------------

. regress logsbp logbmi if sex == 2, noheader
------------------------------------------------------------------------------
      logsbp |      Coef.   Std. Err.      t    P>|t|     [95% Conf. Interval]
-------------+----------------------------------------------------------------
      logbmi |   .3985947   .0185464    21.49   0.000     .3622278    .4349616
       _cons |   3.593017   .0597887    60.10   0.000     3.475779    3.710254
------------------------------------------------------------------------------
\end{verbatim}

\soln{\ref{exo:6.1}}
Dans un premier temps, il est nécessaire de construire le tableau
d'effectifs donné dans l'énoncé. On s'inspirera de la méthode manuelle vue à
l'exercice~\ref{exo:3.5stata} (p.~\pageref{exo:3.5stata}). Par contre, on va
saisir directement les labels des variables et non des codes
numériques. Pour cela, il est nécessaire d'indiquer à \Stata quel est le
format de ces labels à l'aide de l'instruction \texttt{str} que l'on suffixe
par le nombre de caractères que l'on souhaite utiliser pour le codage des
modalités des variables.
\begin{verbatim}
. clear all
. input str1 traitement str3 infection N

     traitem~t  infection          N
  1. "A" "Non" 157
  2. "B" "Non" 119
  3. "A" "Oui" 52
  4. "B" "Oui" 103
  5. end
. list

     +---------------------------+
     | traite~t   infect~n     N |
     |---------------------------|
  1. |        A        Oui   157 |
  2. |        B        Oui   119 |
  3. |        A        Non    52 |
  4. |        B        Non   103 |
     +---------------------------+
\end{verbatim}


Voici donc le tableau de l'énoncé, avec le test du $\chi^2$ associé :
\begin{verbatim}
. tabulate traitement infection [fweight=N], chi

           |       infection
traitement |       Non        Oui |     Total
-----------+----------------------+----------
         A |       157         52 |       209 
         B |       119        103 |       222 
-----------+----------------------+----------
     Total |       276        155 |       431 

          Pearson chi2(1) =  21.6401   Pr = 0.000
\end{verbatim}
Si l'on souhaite une valeur plus précise pour le degré de significativité du
test, on peut utiliser les informations renvoyées (de manière invisible) par
la commande précédente :
\begin{verbatim}
. return list

scalars:
                  r(N) =  431
                  r(r) =  2
                  r(c) =  2
               r(chi2) =  21.64010539408598
                  r(p) =  3.28902266933e-06
\end{verbatim}
D'où pour le "petit $p$", 
\begin{verbatim}
. display %10.9f r(p)
0.000003289
\end{verbatim}
La commande ci-dessus demande à \Stata d'afficher le résultat sous forme d'un
nombre avec 9 décimales.

Concernant les effectifs théoriques, on utilise exactement la même commande
que précédemment mais en ajoutant l'option \texttt{expected}.
\begin{verbatim}
. tabulate traitement infection [fweight=N], chi expected nofreq

           |       infection
traitement |       Non        Oui |     Total
-----------+----------------------+----------
         A |     133.8       75.2 |     209.0 
         B |     142.2       79.8 |     222.0 
-----------+----------------------+----------
     Total |     276.0      155.0 |     431.0 

          Pearson chi2(1) =  21.6401   Pr = 0.000
\end{verbatim}
Notons que l'on a supprimé l'affichage des effectifs observés grâce à
l'option \texttt{nofreq}.

% FIXME:
% tabodds infection traitement [fw=N], or
Pour calculer la valeur de l'odds-ratio, on utilisera la commande
\texttt{cc}. Toutefois, les commandes "epitab" nécessitent que les variables
soient codées sous forme binaire (0 = non-exposé/non-malade, 1 =
exposé/malade). La saisie des données effectuée à l'étape précédente n'étant
pas compatible avec ce format, il est nécessaire de recoder les données
comme dans l'exercice~\ref{exo:3.5stata}. Par exemple, en utilisant la
commande \texttt{input} :
\begin{verbatim}
. input traitement infection N

     traitem~t  infection          N
  1. 1 0 157
  2. 0 0 119
  3. 1 1 52
  4. 0 1 103
  5. end
. label define tx 0 "Placebo" 1 "Traitement"
. label values traitement tx
. label define ouinon 0 "Non" 1 "Oui"
. label values infection ouinon
\end{verbatim}
Ensuite, il est possible de procéder à l'estimation de l'odds-ratio
\begin{verbatim}
. cc infection traitement [fweight=N], woolf exact

                 | traitement             |             Proportion
                 |   Exposed   Unexposed  |      Total     Exposed
-----------------+------------------------+------------------------
           Cases |        52         103  |        155       0.3355
        Controls |       157         119  |        276       0.5688
-----------------+------------------------+------------------------
           Total |       209         222  |        431       0.4849
                 |                        |
                 |      Point estimate    |    [95% Conf. Interval]
                 |------------------------+------------------------
      Odds ratio |         .3826603       |    .2540087    .5764721 (Woolf)
 Prev. frac. ex. |         .6173397       |    .4235279    .7459913 (Woolf)
 Prev. frac. pop |         .3511679       |
                 +-------------------------------------------------
                                  1-sided Fisher's exact P = 0.0000
                                  2-sided Fisher's exact P = 0.0000
\end{verbatim}
On remarquera que cette fois-ci on a précisé l'option \texttt{exact} pour
obtenir un test de Fisher au lieu de l'approximation par la loi du $\chi^2$
pour le test d'hypothèse sur le tableau.

Si l'on tient compte des données par centre, il est nécessaire de
reconstruire les tableaux d'effectifs, en ne considérant que les marges
colonnes des tableaux d'effectifs donnés dans l'énoncé. La saisie des
données avec \texttt{input} ne pose pas de difficultés particulières.
\begin{verbatim}
. input infection centre N

     infection     centre          N
  1. 0 1 98
  2. 1 1 27
  3. 0 2 152
  4. 1 2 106
  5. 0 3 26
  6. 1 3 22
  7. end
. tabulate infection centre [fweight=N], chi

           |              centre
 infection |         1          2          3 |     Total
-----------+---------------------------------+----------
         0 |        98        152         26 |       276 
         1 |        27        106         22 |       155 
-----------+---------------------------------+----------
     Total |       125        258         48 |       431 

          Pearson chi2(2) =  16.1673   Pr = 0.000
\end{verbatim}
Une fois de plus on adopte le format rapide de saisie des données aggrégées
: 2 variables (épisodes infectieux oui/non, \no\ de centre) et les effectifs
associés au croisement de chacun des niveaux de ces variables. L'option
\texttt{fweight} permet ensuite d'appliquer un test du $\chi^2$ via la
commande \texttt{tabulate} en pondérant le tableau à deux entrées par les
effectifs \texttt{N}. 

Pour réaliser un test de Mantel-Haenszel, il est nécessaire de revenir à
l'ensemble des données (3 tableaux $2\times 2$) et d'utiliser la commande
\texttt{cc} en précisant le facteur de stratification grâce à l'option
\texttt{by}. Voici une manière de procéder, en considérant trois variables
\texttt{tx} (traitement A (1) ou B (0)), \texttt{inf} (infection non (0)/oui (1)) et
\texttt{cen} (centre, 1 à 3). Attention à bien coder les classes des deux
variables de classification en 0 et 1.
\begin{verbatim}
. input tx inf cen N

            tx        inf        cen          N
  1. 1 0 1 51
  2. 1 1 1 8
  3. 0 0 1 47
  4. 0 1 1 19
--%<----
 13. end
\end{verbatim}
On peut vérifier la structure des données à l'aide de la commande
\texttt{table}, en suivant exactement le même principe pour les options
(variables de classification, facteur de pondération et variable de
stratification). On en profitera pour ajouter des étiquettes plus
informatives aux modalités des variables de classification.
\begin{verbatim}
. label define txlab 0 "A" 1 "B"
. label define inflab 0 "Non" 1 "Oui"
. label values tx txlab
. label value inf inflab
. table tx inf [fw=N], by(cen)

----------------------
cen and   |    inf    
tx        |  Non   Oui
----------+-----------
1         |
        A |    8    51
        B |   19    47
----------+-----------
2         |
        A |   35    91
        B |   71    61
----------+-----------
3         |
        A |    9    15
        B |   13    11
----------------------
\end{verbatim}
Concernant le test de Mantel-Haenszel, on obtient les résultats suivants :
\begin{verbatim}
. cc tx inf [freq=N], by(cen)

             cen |       OR       [95% Conf. Interval]   M-H Weight
-----------------+-------------------------------------------------
               1 |   .3880289      .1346123    1.04317        7.752 (exact)
               2 |   .3304442      .1899288   .5726148     25.04264 (exact)
               3 |   .5076923      .1370954   1.857073       4.0625 (exact)
-----------------+-------------------------------------------------
           Crude |   .3826603      .2483246   .5877322              (exact)
    M-H combined |   .3620925      .2379264   .5510569              
-------------------------------------------------------------------
Test of homogeneity (M-H)      chi2(2) =     0.47  Pr>chi2 = 0.7898

                   Test that combined OR = 1:
                                Mantel-Haenszel chi2(1) =     23.01
                                                Pr>chi2 =    0.0000
\end{verbatim}
%
%
%
\soln{\ref{exo:6.3}}
Pour le chargement des données qui se présentent sous la forme de données
tabulées dans un fichier texte sans ligne d'en-tête, on utilisera la
commande \texttt{infile} en spécificiant le nom des variables. On notera que
\Stata indique le nombre de lignes ("observations") lues, ce qui permet de
d'assurer de l'intégrité des données si l'on connaît le nombre
d'observations à l'avance.
\begin{verbatim}
. infile ck pres abs using sck.dat
(13 observations read)
\end{verbatim}

Le nombre total de sujets correspond à la somme des valeurs dans les
variables \texttt{pres} et \texttt{abs}. Le plus simple est donc de faire la
somme de l'ensemble de ces valeurs pour avoir l'effectif total :
\begin{verbatim}
. generate tot = pres+abs
. egen ntot = sum(tot)
. display ntot
360
\end{verbatim}

%% Concernant la représentation des fréquences relatives des variables
%% \texttt{pres} et \texttt(abs}, il est nécessaire de \og construire\fg\ deux
%% nouvelles variables. Voici une solution possible :
%% \begin{verbatim}
%% . quietly: summarize pres
%% . scalar npres = r(sum)
%% . gen ppres = pres / npres
%% \end{verbatim}

Il existe plusieurs commandes pour construire un modèle de régression
logistique sous \Stata. Dans le cas des données dites "groupées", on peut
utiliser la commande \texttt{blogit} qui nécessite de connaître les
effectifs pour chacune des deux classes de la variable binaire à prédire, ou
plus exactement les effectifs de la classe "positive" et les effectifs
totaux. Bien qu'on présente son usage ici, on verra qu'il est préférable
d'utiliser un autre type de commande par la suite.
\footnote{\url{http://www.stata.com/support/faqs/statistics/logistic-regression-with-grouped-data/}}
\begin{verbatim}
. blogit pres tot ck

Logistic regression for grouped data              Number of obs   =        360
                                                  LR chi2(1)      =     283.15
                                                  Prob > chi2     =     0.0000
Log likelihood = -93.886407                       Pseudo R2       =     0.6013

------------------------------------------------------------------------------
    _outcome |      Coef.   Std. Err.      z    P>|z|     [95% Conf. Interval]
-------------+----------------------------------------------------------------
          ck |   .0351044   .0040812     8.60   0.000     .0271053    .0431035
       _cons |  -2.326272   .2993611    -7.77   0.000    -2.913009   -1.739535
------------------------------------------------------------------------------
\end{verbatim}
La lecture de ce type de sortie ne présente pas de problème particulier : on
tiendra compte du fait que le coefficient de régression associé à la
variable d'étude \texttt{ck} est exprimé sur une échelle log odds ; pour
obtenir la valeur de l'odds-ratio associé, il est donc nécessaire de prendre
l'exponentielle de la valeur retournée (.0351044), par exemple
\begin{verbatim}
. display exp(_b[ck])
1.0357278
\end{verbatim}

Pour afficher sur un même graphique les proportions empiriques (malades) et
les valeurs prédites par le modèle (sous forme d'une ligne brisée), il est
nécessaire de calculer ces deux quantités.
\begin{verbatim}
. gen prop = pres/tot
. predict pred, p
. label variable prop "observed"
. label variable pred "predicted"
. graph twoway (line pred ck) (scatter prop ck), ytitle("Probabilité")
\end{verbatim}

\includegraphics{./figs/stata_sck}

La commande \texttt{predict} comprend le nom dela variable (\texttt{pred}
dans laquelle on souhaite stocker les résultats, et le type de prédictions
que l'on souhaite réaliser (\texttt{p}, pour probabilités). Un diagramme de
dispersion composé de deux séries d'instructions délimitées par des
parenthèses est ensuite généré à l'aide de la commande \texttt{twoway}.

La commande \texttt{blogit} permet de travailler avec des données
groupées. Il existe une autre manière de construire des modèles de
régression logistique, pour données individuelles ou groupées. En
particulier, les comamndes \texttt{logit} et \texttt{logistic} fournissent
des résultats additionnels. Dans la suite, on va transformer le jeu de
données initial (données groupées) en données individuelles, puis ré-estimer
les paramètres du même modèle de régression logistique. Les commandes
suivantes permettent de générer autant de lignes qu'il y a d'unités
statistiques (nombre d'observations tel que calculé et stocké dans
\texttt{ntot}) ainsi qu'une variable binaire, \texttt{infct}, codant pour
l'observation de l'événement (malade/non-malade).
\begin{verbatim}
. expand tot
(347 observations created)
. bysort ck: gen infct = _n <= pres
. logit infct ck, nolog

Logistic regression                               Number of obs   =        360
                                                  LR chi2(1)      =     283.15
                                                  Prob > chi2     =     0.0000
Log likelihood = -93.886407                       Pseudo R2       =     0.6013

------------------------------------------------------------------------------
       infct |      Coef.   Std. Err.      z    P>|z|     [95% Conf. Interval]
-------------+----------------------------------------------------------------
          ck |   .0351044   .0040812     8.60   0.000     .0271053    .0431035
       _cons |  -2.326272   .2993611    -7.77   0.000    -2.913009   -1.739535
------------------------------------------------------------------------------
\end{verbatim}

Parmi les commandes de post-estimation disponible, la commande
\texttt{classification} fournit automatiquement un tableau de synthèse des
unités correctement ou incorrectement classées selon le critère choisi (par
défaut, $P(\texttt{malade})>0.5$).
\begin{verbatim}
. estat classification

Logistic model for infct

              -------- True --------
Classified |         D            ~D  |      Total
-----------+--------------------------+-----------
     +     |       215            16  |        231
     -     |        15           114  |        129
-----------+--------------------------+-----------
   Total   |       230           130  |        360

Classified + if predicted Pr(D) >= .5
True D defined as infct != 0
--------------------------------------------------
Sensitivity                     Pr( +| D)   93.48%
Specificity                     Pr( -|~D)   87.69%
Positive predictive value       Pr( D| +)   93.07%
Negative predictive value       Pr(~D| -)   88.37%
--------------------------------------------------
False + rate for true ~D        Pr( +|~D)   12.31%
False - rate for true D         Pr( -| D)    6.52%
False + rate for classified +   Pr(~D| +)    6.93%
False - rate for classified -   Pr( D| -)   11.63%
--------------------------------------------------
Correctly classified                        91.39%
--------------------------------------------------
\end{verbatim}

\begin{verbatim}
. estat classification, cutoff(.15)

Logistic model for infct

              -------- True --------
Classified |         D            ~D  |      Total
-----------+--------------------------+-----------
     +     |       228            42  |        270
     -     |         2            88  |         90
-----------+--------------------------+-----------
   Total   |       230           130  |        360

Classified + if predicted Pr(D) >= .15
True D defined as infct != 0
--------------------------------------------------
Sensitivity                     Pr( +| D)   99.13%
Specificity                     Pr( -|~D)   67.69%
Positive predictive value       Pr( D| +)   84.44%
Negative predictive value       Pr(~D| -)   97.78%
--------------------------------------------------
False + rate for true ~D        Pr( +|~D)   32.31%
False - rate for true D         Pr( -| D)    0.87%
False + rate for classified +   Pr(~D| +)   15.56%
False - rate for classified -   Pr( D| -)    2.22%
--------------------------------------------------
Correctly classified                        87.78%
--------------------------------------------------
\end{verbatim}

ROC curve
\begin{verbatim}
. lroc

Logistic model for infct

number of observations =      360
area under ROC curve   =   0.9593
\end{verbatim}

\includegraphics{./figs/stata_sck2}

Sensibilité/spécificité
\begin{verbatim}
. lsens
\end{verbatim}

\includegraphics{./figs/stata_sck3}

% 
%
%
\soln{\ref{exo:6.5}}
Les données ont été sauvegardées dans un format compact (3 colonnes
indiquant la présence ou non d'un cancer, le niveau de consommation
d'alcool, et les effectifs associés).
\begin{verbatim}
. insheet using "cc_oesophage.csv", clear
. label define yesno 0 "No" 1 "Yes" 
. label values cancer yesno 
. label define dose 0 "< 80g" 1 ">= 80g"
. label values alcohol dose
. list

     +-----------------------------+
     | cancer   alcohol   patients |
     |-----------------------------|
  1. |     No     < 80g        666 |
  2. |    Yes     < 80g        104 |
  3. |     No    >= 80g        109 |
  4. |    Yes    >= 80g         96 |
     +-----------------------------+
\end{verbatim}
Les commandes précédentes permettent de charger le fichier de données et
d'associer aux modalités des variables (\texttt{cancer} et \texttt{alcohol})
des noms plus informatifs. Pour obtenir le nombre total de patients, on peut
utiliser la commande suivante :
\begin{verbatim}
. egen ntot = sum(patients)
. display ntot
975
\end{verbatim}

La proportion d'individus à risque, c'est-à-dire ayant une consommation
journalière d'alcool $\ge 80$ g s'obtient à partir d'un simple tableau
d'effectifs croisant les variables \texttt{cancer} et \texttt{alcohol} (il
faut indiquer comment remplir les cellules en ajoutant une option
\texttt{weight}), et en demandant les profils lignes, c'est-à-dire les les
fréquences relatives par ligne.
\begin{verbatim}
. tabulate cancer alcohol [fweight=patients], row

+----------------+
| Key            |
|----------------|
|   frequency    |
| row percentage |
+----------------+

           |        alcohol
    cancer |     < 80g     >= 80g |     Total
-----------+----------------------+----------
        No |       666        109 |       775 
           |     85.94      14.06 |    100.00 
-----------+----------------------+----------
       Yes |       104         96 |       200 
           |     52.00      48.00 |    100.00 
-----------+----------------------+----------
     Total |       770        205 |       975 
           |     78.97      21.03 |    100.00
\end{verbatim}

Le calcul de l'odds-ratio peut se faire à l'aide de l'une des commandes
\Stata dites "immédiates", \texttt{cc}, en gardant à l'esprit qu'il faut bien
tenir compte de la colonne \texttt{patients}, comme précédemment :
\begin{verbatim}
. cc cancer alcohol [fweight=patients], woolf

                 | alcohol                |             Proportion
                 |   Exposed   Unexposed  |      Total     Exposed
-----------------+------------------------+------------------------
           Cases |        96         104  |        200       0.4800
        Controls |       109         666  |        775       0.1406
-----------------+------------------------+------------------------
           Total |       205         770  |        975       0.2103
                 |                        |
                 |      Point estimate    |    [95% Conf. Interval]
                 |------------------------+------------------------
      Odds ratio |         5.640085       |    4.000589    7.951467 (Woolf)
 Attr. frac. ex. |         .8226977       |    .7500368    .8742371 (Woolf)
 Attr. frac. pop |         .3948949       |
                 +-------------------------------------------------
                               chi2(1) =   110.26  Pr>chi2 = 0.0000
\end{verbatim}

Si l'on dispose du tableau d'effectifs, c'est-à-dire la répartition des 975
sujets dans les quatre cellules du tableau croisant l'exposition à l'alcool
et le statut cas-témoin, il est également possible d'utiliser le calculateur
pour odds-ratio dans les études cas-témoins accessible par le menu
\textsf{Statistics} $\rhd$ \textsf{Epidemiology and related} $\rhd$
\textsf{Tables for epidemiologists}.

\includegraphics{./figs/stata_ORcalculator}

Pour tester l'hypothèse que la proportion de personnes avec une consommation
journalière d'alcool $\ge 80$ g est identique chez les cas ($p_1$) et les
témoins ($p_1$), on peut procéder comme suit :
\begin{verbatim}
. prtesti 96 0.4800 109 0.1406

Two-sample test of proportions                     x: Number of obs =       96
                                                   y: Number of obs =      109
------------------------------------------------------------------------------
    Variable |       Mean   Std. Err.      z    P>|z|     [95% Conf. Interval]
-------------+----------------------------------------------------------------
           x |        .48   .0509902                      .3800611    .5799389
           y |      .1406   .0332949                      .0753433    .2058567
-------------+----------------------------------------------------------------
        diff |      .3394   .0608978                      .2200424    .4587576
             |  under Ho:   .0641131     5.29   0.000
------------------------------------------------------------------------------
        diff = prop(x) - prop(y)                                  z =   5.2938
    Ho: diff = 0

    Ha: diff < 0                 Ha: diff != 0                 Ha: diff > 0
 Pr(Z < z) = 1.0000         Pr(|Z| < |z|) = 0.0000          Pr(Z > z) = 0.0000
\end{verbatim}

Une autre façon de tester cette hypothèse consiste à remarquer que
l'hypothèse précédente, $H_0:\, \pi_0=\pi_1$, n'est vraie que si
l'odds-ratio vaut 1. D'où l'idée d'exploiter directement le test du $\chi^2$
pour l'odds-ratio donné par la commande \texttt{cc} (ici, $\chi^2=110.26$,
$p<0.001$). 

Le modèle de régression logistique, comme les autres modèles de régression
sous \Stata, se formule ainsi : 

\begin{verbatim}
. logistic alcohol cancer [freq=patients]

Logistic regression                               Number of obs   =        975
                                                  LR chi2(1)      =      96.43
                                                  Prob > chi2     =     0.0000
Log likelihood =  -453.2224                       Pseudo R2       =     0.0962

------------------------------------------------------------------------------
     alcohol | Odds Ratio   Std. Err.      z    P>|z|     [95% Conf. Interval]
-------------+----------------------------------------------------------------
      cancer |   5.640085   .9883492     9.87   0.000     4.000589    7.951467
       _cons |   .1636637   .0169104   -17.52   0.000     .1336604    .2004018
------------------------------------------------------------------------------
\end{verbatim}

Par défaut, \Stata présente les résultats (coefficients du modèle) sous forme
d'odds-ratio, avec leurs intervalles de confiance associés. Si l'on souhaite
obtenir directement les coefficients de régression (sur l'échelle du log
odds), il faut utiliser la commande \texttt{logit} après avoir utilisé
\texttt{logistic}. On pourrait également utiliser directement une commande de
type \verb|logistic alcohol cancer [freq=patients]|.
\begin{verbatim}
. logit

Logistic regression                               Number of obs   =        975
                                                  LR chi2(1)      =      96.43
                                                  Prob > chi2     =     0.0000
Log likelihood =  -453.2224                       Pseudo R2       =     0.0962

------------------------------------------------------------------------------
     alcohol |      Coef.   Std. Err.      z    P>|z|     [95% Conf. Interval]
-------------+----------------------------------------------------------------
      cancer |   1.729899   .1752366     9.87   0.000     1.386442    2.073356
       _cons |  -1.809942   .1033238   -17.52   0.000    -2.012453   -1.607431
------------------------------------------------------------------------------
\end{verbatim}

%---------------------------------------------------------------- Séance 11 --
\chapter*{Semaine 11\markboth{Corrigés de la semaine 11}{}}

\soln{\ref{exo:7.1}}
Le format texte du fichier de données \texttt{prostate.dat} est identique à
celui du fichier \texttt{pbc.txt} de l'exercice précédent, à ceci près que
les champs sont séparés par un simple espace. On utilisera donc la commande
\texttt{insheet} :
\begin{verbatim}
. insheet using "prostate.dat", delimiter(" ")
(7 vars, 38 obs)
. list in 1/5

     +--------------------------------------------------------+
     | treatm~t   time   status   age   haem   size   gleason |
     |--------------------------------------------------------|
  1. |        1     65        0    67   13.4     34         8 |
  2. |        2     61        0    60   14.6      4        10 |
  3. |        2     60        0    77   15.6      3         8 |
  4. |        1     58        0    64   16.2      6         9 |
  5. |        2     51        0    65   14.1     21         9 |
     +--------------------------------------------------------+
\end{verbatim}
Le nombre de patients vivants à la date de point est obtenu à partir de
\texttt{tabulate} :
\begin{verbatim}
. tabulate status

     Status |      Freq.     Percent        Cum.
------------+-----------------------------------
          0 |         32       84.21       84.21
          1 |          6       15.79      100.00
------------+-----------------------------------
      Total |         38      100.00
\end{verbatim}
soit 32 personnes encore en vie à la fin de la durée de suivi.

Comme dans l'exercice~7.1, il est nécessaire d'indiquer à \Stata quelle
variable sert à identifier les événements (\texttt{status}) et le temps
(\texttt{time}). On utilisera la commande \texttt{stset} de la manière
suivante :
\begin{verbatim}
. stset time, failure(status)

     failure event:  status != 0 & status < .
obs. time interval:  (0, time]
 exit on or before:  failure

------------------------------------------------------------------------------
       38  total obs.
        0  exclusions
------------------------------------------------------------------------------
       38  obs. remaining, representing
        6  failures in single record/single failure data
     1890  total analysis time at risk, at risk from t =         0
                             earliest observed entry t =         0
                                  last observed exit t =        70
\end{verbatim}

La médiane de survie selon le traitement est disponible à partir de la
commande \texttt{stci}, en précisant le percentile d'intérêt (ici,
\texttt{p(50)}). 
\begin{verbatim}
. stci, by(treatment) p(50)

         failure _d:  status
   analysis time _t:  time

             |    no. of 
treatment    |  subjects         50%     Std. Err.     [95% Conf. Interval]
-------------+-------------------------------------------------------------
           1 |        18          69      .3043484           42          .
           2 |        20           .             .            .          .
-------------+-------------------------------------------------------------
       total |        38           .             .           69          .
\end{verbatim}

Pour afficher les courbes de survie pour chacun des bras de traitement, on
utilisera la commande \texttt{sts graph} en spécifiant le facteur de
classification à l'aide de l'option \texttt{by}. L'affichage des données
censurées se fait à l'aide de l'option \texttt{censored}.
\begin{verbatim}
. sts graph, by(treatment) censored(s)
\end{verbatim}

\includegraphics{./figs/stata_ststwoway}

Le test du log-rank s'effectue avec la commande \texttt{sts test}. Notons
que l'on n'a besoin de spécifier que le facteur traitement.
\begin{verbatim}
. sts test treatment

         failure _d:  status
   analysis time _t:  time


Log-rank test for equality of survivor functions

          |   Events         Events
treatment |  observed       expected
----------+-------------------------
1         |         5           2.47
2         |         1           3.53
----------+-------------------------
Total     |         6           6.00

                chi2(1) =       4.42
                Pr>chi2 =     0.0355
\end{verbatim}

\soln{\ref{exo:7.3}}
Le fichier de données est un fichier texte avec des tabulations comme
séparateur de champ. On peut l'importer sous \Stata en utilisant la commande
\texttt{insheet}. Pour afficher le nom des variables après importation, il
suffit d'utiliser \texttt{describe} avec l'option \texttt{simple}.
\begin{verbatim}
. insheet using "pbc.txt", tab
(28 vars, 312 obs)
. describe, simple
number    rx        asictes   spiders   bilirub   albumin   alkphos   trigli
prothrom  age       sample    logalbu   _st       _t
status    sex       hepatom   edema     cholest   copper    sgot      platel
histol    years     logbili   logprot   _d        _t0
\end{verbatim}
Après recodage des étiquettes des variables \texttt{rx} et \texttt{sex}, 
\begin{verbatim}
. label define trt 1 "Placebo" 2 "DPCA"
. label define sexe 0 "M" 1 "F"
. label values rx trt
. label values sex sexe
\end{verbatim}
on peut vérifier la proportion de patients décédés (\texttt{status}, 0 =
vivant et 1 = décédé) et leur répartition selon le groupe de traitement à
l'aide de tris simple et croisé. Pour le tri croisé, on ajoutera l'option
\texttt{row} pour obtenir les fréquences relatives par status.
\begin{verbatim}
. tabulate status

     status |      Freq.     Percent        Cum.
------------+-----------------------------------
          0 |        187       59.94       59.94
          1 |        125       40.06      100.00
------------+-----------------------------------
      Total |        312      100.00

. tabulate status rx, row

+----------------+
| Key            |
|----------------|
|   frequency    |
| row percentage |
+----------------+

           |          rx
    status |   Placebo       DPCA |     Total
-----------+----------------------+----------
         0 |        93         94 |       187 
           |     49.73      50.27 |    100.00 
-----------+----------------------+----------
         1 |        65         60 |       125 
           |     52.00      48.00 |    100.00 
-----------+----------------------+----------
     Total |       158        154 |       312 
           |     50.64      49.36 |    100.00
\end{verbatim}

Pour afficher la distribution des temps de suivi, on utilisera un simple
diagramme de dispersion comme on l'a vu dans le cas de R. Pour faire
appraître distinctement les observations selon le status (0 ou 1), on
pourrait très bien superposer deux séries de points sur le même graphique
(cf. exercice~5.2, p.~\pageref{para:twoway}). Voici une autre manière de
procéder :
\begin{verbatim}
. separate number, by(status)

              storage  display     value
variable name   type   format      label      variable label
-----------------------------------------------------------------------------------------
number0         int    %8.0g                  number, status == 0
number1         int    %8.0g                  number, status == 1

. twoway scatter number0 number1 years, msymbol(S O)
\end{verbatim}
La première commande permet en fait de séparer les numéros de patients en
fonction du status afin d'afficher les deux séries d'observation en fonction
de la durée de suivi en années.

\includegraphics{./figs/stata_followup}

La médiane de la durée de suivi par groupe de traitement peut s'obtenir avec
la commande \texttt{tabstat} en opérant par groupe grâce à l'option
\texttt{by}. 
\begin{verbatim}
. tabstat years, by(rx) stats(median) nototal

Summary for variables: years
     by categories of: rx 

     rx |       p50
--------+----------
Placebo |    5.1882
   DPCA |   4.95825
-------------------
\end{verbatim}

Le nombre de décès enregistrés au-delà de 10.5 années de suivi s'obtient
avec un simple tri à plat par la commande \texttt{tabulate} :
\begin{verbatim}
. tabulate status if years > 10.49

     status |      Freq.     Percent        Cum.
------------+-----------------------------------
          0 |         23       85.19       85.19
          1 |          4       14.81      100.00
------------+-----------------------------------
      Total |         27      100.00
\end{verbatim}
de même que le sexe des patients décédé après cette période :
\begin{verbatim}
. tabulate sex if years > 10.49 & status == 1

        sex |      Freq.     Percent        Cum.
------------+-----------------------------------
          M |          2       50.00       50.00
          F |          2       50.00      100.00
------------+-----------------------------------
      Total |          4      100.00
\end{verbatim}

Concernant l'analyse des patients transplantés, on peut également procéder
comme on l'a fait avec R, c'est-à-dire restreindre le tableau de données à
ces seuls patients. Comme \Stata ne permet de travailler qu'avec un seul
tableau de données à la fois, il est toutefois nécessaire de sauvegarder
temporairement les données actuelles avant de créer un nouveau tableau.
\begin{verbatim}
. preserve
. egen idx = anymatch(number), values(5 105 111 120 125 158 183 241 246 247 254 263 264
  265 274 288 291 295 297 345 361 362 375 380 383)
. keep if idx
(293 observations deleted)
. gen days = years*365
. tabstat age sex days, stats(mean median sum)

   stats |       age       sex      days
---------+------------------------------
    mean |  41.17568  .8421053  1507.177
     p50 |   40.9008         1  1434.012
     sum |  782.3379        16  28636.37
----------------------------------------
\end{verbatim}
La première commande permet de construire une liste des individus que l'on
souhaite utiliser pour filtrer le tableau de données d'origine (sur la base
des numéros de sujet contenus dans la variable \texttt{number}). Ensuite, on
applique le calcul des statistiques descriptives à l'aide d'une commande
\texttt{tabstat}. Une fois les calculs terminés, on peut restorer les
données d'origine de la manière suivante :
\begin{verbatim}
. restore
\end{verbatim}

Comme R, \Stata utilise ses propres conventions pour le codage des données de
survie. Les commandes essentielles sont ainsi : \texttt{stset} pour définir
la façon dont les événements sont enregistrés et le temps d'observation,
\texttt{sts} pour calculer un tableau de survie à partir de l'estimateur de
Kaplan-Meier. Voici comment appliquer ces commandes pour construire le
tableau et la courbe de survie, sans considération du facteur traitement.
\begin{verbatim}
. stset years, failure(status)

     failure event:  status != 0 & status < .
obs. time interval:  (0, years]
 exit on or before:  failure

------------------------------------------------------------------------------
      312  total obs.
        0  exclusions
------------------------------------------------------------------------------
      312  obs. remaining, representing
      125  failures in single record/single failure data
 1713.854  total analysis time at risk, at risk from t =         0
                             earliest observed entry t =         0
                                  last observed exit t =   12.4736

. sts list
\end{verbatim}
La deuxième commande affiche le tableau demandé. Pour la courbe de survie,
on utilisera :
\begin{verbatim}
. sts graph, ci censored(single)
\end{verbatim}
L'option \texttt{ci} permet d'afficher l'intervalle de confiance à 95~\%
pour l'estimateur de KM.

\includegraphics{./figs/stata_stsgraph}


%-------------------------------------- Exos SAS  ---------------------------
\blankpage
\blankpage

\thispagestyle{empty}
\centerline{\small Centre d'Enseignement de la Statistique Appliquée, à la Médecine et à la Biologie Médicale}
\vspace*{2cm}
\begin{center}
\centerline{\includegraphics[scale=.55]{cesam}}
\vspace*{2cm}
\begin{minipage}{.75\textwidth}
\begin{mdframed}[style=titlep]
\centerline{\Huge Programme de travail}
\vskip1em
\centerline{\Huge du cours d'informatique du CESAM}
\end{mdframed}
\end{minipage}
\end{center}
\vskip3em
\centerline{\Huge\bf Introduction au logiciel SAS}
\vskip5em
\begin{center}
  \begin{tabular}{ll}
    \textbf{Responsables :} & \\
    Christophe LALANNE & \url{christophe.lalanne@inserm.fr} \\
    Yassin MAZROUI     & \url{yassin.mazroui@upmc.fr} \\
    Pr Mounir MESBAH   & \url{mounir.mesbah@upmc.fr}
  \end{tabular}
\end{center}
\vskip3em
\centerline{\Large \url{http://www.cesam.upmc.fr}}
\vskip3em
\centerline{\LARGE Année Universitaire 2015–2016}
\vfill
\begin{center}
\begin{minipage}{.6\textwidth}
\centering
Adresser toute correspondance à :\\
Université Pierre et Marie Curie – Paris 6
Secrétariat du CESAM – Les Cordeliers
Service Formation Continue, esc. B, 4ème étage,
15 rue de l’école de médecine,
75006 PARIS\\
ou par Courriel à : \url{cesam@upmc.fr}
\end{minipage}
\end{center}  



\blankpage


\chapter*{Calendrier}
\thispagestyle{empty}
\vskip3em

\begin{center}
\begin{tabular}{|l|p{10cm}|l|l|}
\hline
  \multicolumn{4}{|c|}{Introduction au logiciel SAS} \\
\hline
  Sem. 13/04 & Élements du langage et statistiques descriptives & Corrigés
  pp.~\pageref{start:sol8b}–\pageref{stop:sol8b} & Devoir
  \no 7\\
  Sem. 20/04 & Mesures d'association et comparaison de deux variables &
  Corrigés pp.~\pageref{start:sol9b}–\pageref{stop:sol9b} &
  Devoir \no 8\\
  Sem. 27/04 & Régression linéaire et logistique & Corrigés pp.~\pageref{start:sol10b}–\pageref{stop:sol10b} & Devoir \no 9\\
  Sem. 04/05 & Données de survie & Corrigés pp.~\pageref{start:sol11b}–\pageref{stop:sol11b} & Devoir \no 10\\
\hline
\end{tabular}
\end{center}

\titleformat{\chapter}
  {\normalfont\huge\bfseries}{\chaptertitlename\ \thechapter.}{20pt}{\huge}

\setcounter{page}{1}
\setcounter{chapter}{7}
\chapter{Élements du langage et statistiques descriptives}

\begin{exo}\label{exo:8.1}
{\footnotesize Identique à l'énoncé 1.1 (p.~\pageref{exo:1.1}), questions
  a–c.}

Un chercheur a recueilli les mesures biologiques suivantes (unités
arbitraires) :
\begin{verbatim}
3.68  2.21  2.45  8.64  4.32  3.43  5.11  3.87
\end{verbatim}
\begin{description}
\item[(a)] Stocker la séquence de mesures dans une variable appelée
  \texttt{x}.  
\item[(b)] Indiquer le nombre d'observations (à l'aide de \R), les valeurs
  minimale et maximale, ainsi que l'étendue.  
\item[(c)] En fait, le chercheur réalise que la valeur 8.64 correspond à une
  erreur de saisie et doit être changée en 3.64. De même, il a un doute sur
  la 7\ieme mesure et décide de la considérer comme une valeur manquante :
  effectuer les transformations correspondantes. 
\end{description}
\end{exo}
\vskip1em

\begin{exo}\label{exo:8.2}
{\footnotesize Identique à l'énoncé 1.2 (p.~\pageref{exo:1.2}), questions a
  et b.}

La charge virale plasmatique permet de décrire la quantité de virus (p.~ex.,
VIH) dans un échantillon de sang. Ce marqueur virologique qui permet de
suivre la progression de l’infection et de mesurer l’efficacité des
traitements est rapporté en nombre de copies par millilitre, et la plupart
des instruments de mesure ont un seuil de détectabilité de 50
copies/ml. Voici une série de mesures, $X$, exprimées en logarithmes (base 10)
collectées sur 20 patients :
\begin{verbatim}
3.64 2.27 1.43 1.77 4.62 3.04 1.01 2.14 3.02 5.62 5.51 5.51 1.01 1.05 4.19
2.63 4.34 4.85 4.02 5.92
\end{verbatim}
Pour rappel, une charge virale de 100 000 copies/ml équivaut à 5 log.
\begin{description}
\item[(a)] Indiquer combien de patients ont une charge virale considérée
  comme non-détectable. 
\item[(b)] Quelle est le niveau de charge virale médian, en copies/ml, pour
  les données considérées comme valides ?
\end{description}
\end{exo}
\vskip1em

\begin{exo}\label{exo:8.3}
{\footnotesize Identique à l'énoncé 1.5 (p.~\pageref{exo:1.5}), questions
  a–d.}  

Le fichier \texttt{anorexia.dat} contient les données d'une étude clinique
chez des patientes anorexiques ayant reçu l'une des trois thérapies
suivantes : thérapie comportementale, thérapie familiale, thérapie
contrôle.\autocite{hand93} 
\begin{description}
\item[(a)] Combien y'a-t-il de patientes au total ? Combien y'a-t-il de
  patientes par groupe de traitement ?
\item[(b)] Les mesures de poids sont en livres. Les convertir en
  kilogrammes.    
\item[(c)] Créer une nouvelle variable contenant les scores de différences
  (\texttt{After} - \texttt{Before}).
\item[(d)] Indiquer la moyenne et l'étendue (min/max) des scores de
  différences par groupe de traitement.
\end{description}
\end{exo}
\vskip1em

\begin{exo}\label{exo:8.4}
{\footnotesize Identique à l'énoncé 2.1 (p.~\pageref{exo:2.1}), questions
  a–d.} 

Une variable quantitative $X$ prend les valeurs suivantes sur un échantillon
de 26 sujets :
\begin{verbatim}
24.9,25.0,25.0,25.1,25.2,25.2,25.3,25.3,25.3,25.4,25.4,25.4,25.4,
25.5,25.5,25.5,25.5,25.6,25.6,25.6,25.7,25.7,25.8,25.8,25.9,26.0
\end{verbatim}
\begin{description}
\item[(a)] Calculer la moyenne, la médiane ainsi que le mode de $X$. 
\item[(b)] Quelle est la valeur de la variance estimée à partir de ces données ? 
\item[(c)] En supposant que les données sont regroupées en 4 classes dont les
  bornes sont : 24.9–25.1, 25.2–25.4, 25.5–25.7, 25.8–26.0, afficher la
  distribution des effectifs par classe sous forme d'un tableau d'effectifs. 
\item[(d)] Représenter la distribution de $X$ sous forme d'histogramme, sans
  considération d'intervalles de classe \emph{a priori}.
\end{description}
\end{exo}
\vskip1em

\begin{exo}\label{exo:8.5}
{\footnotesize Identique à l'énoncé 2.3 (p.~\pageref{exo:2.3}), questions
  a–c.}

Le fichier \texttt{elderly.dat} contient la taille mesurée en cm de 351
personnes âgées de sexe féminin, sélectionnées aléatoirement dans la
population lors d'une étude sur l'ostéoporose. Quelques observations sont
cependant manquantes.
\begin{description}
\item[(a)] Combien y'a t-il d'observations manquantes au total ?
\item[(b)] Donner un intervalle de confiance à 95~\% pour la taille moyenne
  dans cet échantillon, en utilisant une approximation normale.
\item[(c)] Représenter la distribution des tailles observées sous forme
  d'une courbe de densité.  
\end{description}
\end{exo}
\vskip1em

\begin{exo}\label{exo:8.6}
{\footnotesize Identique à l'énoncé 2.4 (p.~\pageref{exo:2.4}), questions
  a–f.} 

Le fichier \texttt{birthwt} est un des jeux de données fournis avec \R. Il
comprend les résultats d'une étude prospective visant à identifier les
facteurs de risque associés à la naissance de bébés dont le poids est
inférieur à la norme (2,5 kg). Les données proviennent de 189 femmes, dont
59 ont accouché d'un enfant en sous-poids. Parmi les variables d'intérêt
figurent l'âge de la mère, le poids de la mère lors des dernières
menstruations, l'ethnicité de la mère et le nombre de visites médicales
durant le premier trimestre de grossesse.\autocite{hosmer89}
Les variables disponibles sont décrites comme suit : \texttt{low} (= 1 si
poids $<2.5$ kg, 0 sinon), \texttt{age} (années), \texttt{lwt} (poids de la
mère en livres), \texttt{race} (ethnicité codée en trois classes, 1 = white,
2 = black, 3 = other), \texttt{smoke} (= 1 si consommation de tabac durant
la grossesse, 0 sinon), \texttt{ptl} (nombre d'accouchements pré-terme
antérieurs), \texttt{ht} (= 1 si antécédent d'hypertension, 0 sinon),
\texttt{ui} (= 1 si manifestation d'irritabilité utérine, 0 sinon),
\texttt{ftv} (nombre de consultations chez le gynécologue durant le premier
trimestre de grossesse), \texttt{bwt} (poids des bébés à la naissance, en
\emph{g}).
\begin{description}
\item[(a)] Recoder les variables \texttt{low}, \texttt{race},
  \texttt{smoke}, \texttt{ui} et \texttt{ht} en variables
  qualitatives, avec des étiquettes ("labels") plus informatives.
\item[(b)] Convertir le poids des mères en \emph{kg}. Indiquer la moyenne, la
  médiane et l'intervalle inter-quartile. Représenter la distribution des
  poids sous forme d'histogramme.
\item[(c)] Indiquer la proportion de mères consommant du tabac durant la
  grossesse, avec un intervalle de confiance à 95~\%. Représenter les
  proportions (en \%) fumeur/non-fumeur sous forme d'un diagramme en
  barres.
\item[(d)] Recoder l'âge des mères en trois classes équilibrées (tercilage)
  et indiquer la proportion d'enfants dont le poids est $<2500$ \emph{g}
  pour chacune des trois classes.
\item[(e)] Construire un tableau d'effectifs ($n$ et \%) pour la variable
  ethnicité (\texttt{race}).  
\item[(f)] Décrire la distribution des variables \texttt{race},
  \texttt{smoke}, \texttt{ui}, \texttt{ht} et \texttt{age} après
  stratification sur la variable \texttt{low}.  
\end{description}
\end{exo}

%--------------------------------------------------------------- Devoir 07 ---
\chapter*{Devoir \no 7}
\addcontentsline{toc}{chapter}{Devoir \no 7}

Les exercices sont indépendants. Une seule réponse est correcte pour chaque
question. Lorsque vous ne savez pas répondre, cochez la case correspondante.

\section*{Exercice 1}
\section*{Exercice 2}
\section*{Exercice 3}\label{dev7:exo3}

%--------------------------------------------------------------- Chapter 09 --
\chapter[Mesures d'association, comparaison de moyennes et de
proportions]{Mesures d'association, comparaison de moyennes et de
  proportions pour deux échantillons ou plus}   

\begin{exo}\label{exo:9.1}
{\footnotesize Identique à l'énoncé 3.1 (p.~\pageref{exo:3.1}), questions
  a–c.} 

On dispose des poids à la naissance d'un échantillon de 50 enfants
présentant un syndrôme de détresse respiratoire idiopathique aïgue. Ce type
de maladie peut entraîner la mort et on a observé 27 décès chez ces
enfants. Les données sont résumées dans le tableau ci-dessous et sont
disponibles dans le fichier \texttt{sirds.dat}, où les 27 premières
observations correspondent au groupe des enfants décédés au moment de
l'étude. \autocite[p.~64]{everitt01}
\vskip1em

\begin{tabular}{ll}
\toprule  
Enfants décédés &
1.050\; 1.175\; 1.230\; 1.310\; 1.500\; 1.600\; 1.720\; 1.750\; 1.770\; 2.275\; 2.500\; 1.030\; 1.100\; 1.185 \\
& 1.225\; 1.262\; 1.295\; 1.300\; 1.550\; 1.820\; 1.890\; 1.940\; 2.200\; 2.270\; 2.440\; 2.560\; 2.730 \\
Enfants vivants &
1.130\; 1.575\; 1.680\; 1.760\; 1.930\; 2.015\; 2.090\; 2.600\; 2.700\; 2.950\; 3.160\; 3.400\; 3.640\; 2.830 \\
& 1.410\; 1.715\; 1.720\; 2.040\; 2.200\; 2.400\; 2.550\; 2.570\; 3.005 \\
\bottomrule
\end{tabular}
\vskip1em

Un chercheur s'intéresse à l'existence éventuelle d'une différence entre le
poids moyen des enfants ayant survécu et celui des enfants décédés des
suites de la maladie. 
\begin{description}
\item[(a)] Réaliser un test $t$ de Student. Peut-on rejeter l'hypothèse nulle
  d'absence de différences entre les deux groupes d'enfants ? 
\item[(b)] Vérifier graphiquement que les conditions d'applications du test
(normalité et homogénéité des variances) sont vérifiées. 
\item[(c)] Quel est l'intervalle de confiance à 95~\% pour la différence de
  moyenne observée ?
\end{description}
\end{exo}
\vskip1em

\begin{exo}\label{exo:9.2}
{\footnotesize Identique à l'énoncé 3.2 (p.~\pageref{exo:3.2}), questions
  a–c.}

La qualité de sommeil de 10 patients a été mesurée avant (contrôle) et après
traitement par un des deux hypnotiques suivants : (1) D. hyoscyamine
hydrobromide et (2) L. hyoscyamine hydrobromide. Le critère de jugement
retenu par les chercheurs était le gain moyen de sommeil (en heures) par
rapport à la durée de sommeil de base
(contrôle). \autocite[p.~20]{student08} Les données sont reportées
ci-dessous et figurent également parmi les jeux de données de base de R
(\verb|data(sleep)|).  
\begin{verbatim}
D. hyoscyamine hydrobromide :
0.7 -1.6 -0.2 -1.2 -0.1  3.4  3.7  0.8  0.0  2.0
L. hyoscyamine hydrobromide :
1.9  0.8  1.1  0.1 -0.1  4.4  5.5  1.6  4.6  3.4
\end{verbatim}

Les chercheurs ont conclu que seule la deuxième molécule avait réellement un
effet soporifique. 
\begin{description}
\item[(a)] Estimer le temps moyen de sommeil pour chacune des deux
  molécules, ainsi que la différence entre ces deux moyennes.
\item[(b)] Afficher la distribution des scores de différence (LHH - DHH)
  sous forme d'un histogramme, en considérant des intervalles de classe
  d'une demi-heure, et indiquer la moyenne et l'écart-type de ces scores de
  différence.
\item[(c)] Vérifier l'exactitude des conclusions à l'aide d'un test de Student.
\end{description}
\end{exo}
\vskip1em

\begin{exo}\label{exo:9.3}
{\footnotesize Identique à l'énoncé 3.4 (p.~\pageref{exo:3.4}), questions
  a–d.}

Dans un essai clinique, on a cherché à évaluer un régime supposé réduire le
nombre de symptômes associé à une maladie bénigne du sein. Un groupe de 229
femmes ayant cette maladie ont été alétoirement réparties en deux
groupes. Le premier groupe a reçu les soins courants, tandis que les
patientes du second groupe suivaient un régime spécial (variable B =
traitement). Après un an, les individus ont été évalués et ont été classés
dans l'une des deux catégories : amélioration ou pas d'amélioration
(variable A = réponse). Les résultats sont résumés dans le tableau suivant,
pour une partie de l'échantillon :\autocite[p.~323]{selvin98}
\vskip1em

\begin{tabular}{l|cc|r}
& régime & pas de régime & total \\
\hline
amélioration & 26 & 21 & 47 \\
pas d'amélioration & 38 & 44 & 82 \\
\hline
total & 64 & 65 & 129
\end{tabular}
\vskip1em

\begin{description}
\item[(a)] Réaliser un test du chi-deux.  
\item[(b)] Quels sont les effectifs théoriques attendus sous une hypothèse
  d'indépendance ?
\item[(c)] Comparer les résultats obtenus en (a) avec ceux d'un test de
  Fisher.
\item[(d)] Donner un intervalle de confiance pour la différence de
  proportion d'amélioration entre les deux groupes de patientes.
\end{description}
\end{exo}
\vskip1em

\begin{exo}\label{exo:9.4}
{\footnotesize Identique à l'énoncé 3.5 (p.~\pageref{exo:3.5}), questions
  a–d.}

Dans un essai clinique, 1360 patients ayant déjà eu un infarctus dy myocarde
ont été assignés à l'un des deux groupes de traitement suivants : prise en
charge par aspirine à faible dose en une seule prise \emph{versus}
placebo. La table ci-après indique le nombre de décès par infarctus lors de
la période de suivi de trois ans :\autocite[p.~72]{agresti02} 
\vskip1em

\begin{tabular}{lccc}
\toprule
& \multicolumn{2}{c}{Infarctus} & \\
\cmidrule(r){2-3}
& Oui & Non & Total \\
\midrule
Placebo & 28 & 656 & 684 \\
Aspirine & 18 & 658 & 676 \\
\bottomrule
\end{tabular}
\vskip1em

\begin{description}
\item[(a)] Calculer la proportion d'infarctus du myocarde dans les deux
  groupes de patients.
\item[(b)] Représenter graphiquement le tableau précédent sous forme d'un
  diagramme en barres ou d'un diagramme en points ("dotplot" de Cleveland).
\item[(c)] Indiquer la valeur de l'odds-ratio ainsi que du risque
  relatif. Pour l'odds-ratio, on considérera comme catégories de référence
  les modalités représentées par la première ligne et la première colonne du
  tableau.  
\item[(d)] À partir de l'intervalle de confiance à 95~\% pour l'odds, quelle
  conclusion peut-on tirer sur l'effet de l'aspirine dans la prévention d'un
  infarctus du myocarde ?
\end{description}
\end{exo}
\vskip1em

\begin{exo}\label{exo:9.5}
{\footnotesize Identique à l'énoncé 4.1 (p.~\pageref{exo:4.1}), questions
  a–d.} 

Dans une étude sur le gène du récepteur à \oe strogènes, des généticiens se
sont intéressés à la relation entre le génotype et l'âge de diagnostic du
cancer du sein. Le génotype était déterminé à partir des deux allèles d'un
polymorphisme de restriction de séquence (1.6 et 0.7 kb), soit trois groupes
de sujets : patients homozygotes pour l'allèle 0.7 kb (0.7/0.7), patients
homozygotes pour l'allèle 1.6 kb (1.6/1.6), et patients hétérozygotes
(1.6/0.7). Les données ont été recueillies sur 59 patientes atteintes d'un
cancer du sein, et sont disponibles dans le fichier
\texttt{polymorphism.dta} (fichier \Stata). Les données moyennes sont
indiquées ci-dessous :\autocite[p.~327]{dupont09}
\vskip1em

\begin{tabular}{lrrrr}
\toprule
& \multicolumn{3}{c}{Génotype} & \\
\cmidrule(r){2-4}
& 1.6/1.6 & 1.6/0.7 & 0.7/0.7 & Total \\
\midrule
Nombre de patients & 14 & 29 & 16 & 59 \\
\emph{Âge lors du diagnostic} & & & & \\
\quad Moyenne & 64.64 & 64.38 & 50.38 & 60.64 \\
\quad Écart-type & 11.18 & 13.26 & 10.64 & 13.49 \\
\quad IC 95~\% & (58.1–71.1) & (59.9–68.9) & (44.3–56.5) & \\
\bottomrule
\end{tabular}
\vskip1em

\begin{description}
\item[(a)] Tester l'hypothèse nulle selon laquelle l'âge de diagnostic ne varie
  pas selon le génotype à l'aide d'une ANOVA. Représenter sous forme
  graphique la distribution des âges pour chaque génotype.
\item[(b)] Les intervalles de confiance présentés dans le tableau ci-dessus ont
  été estimés en supposant l'homogénéité des variances, c'est-à-dire en
  utilisant l'estimé de la variance commune ; donner la valeur de ces
  intervalles de confiance sans supposer l'homoscédasticité. 
\item[(c)] Estimer les différences de moyenne correspondant à l'ensemble des
  combinaisons possibles des trois génotypes, avec une estimation de
  l'intervalle de confiance à 95~\% associé et un test paramétrique
  permettant d'évaluer le degré de significativité de la différence
  observée.
\item[(d)] Représenter graphiquement les moyennes de groupe avec des
  intervalles de confiance à 95~\%.
\end{description}
\end{exo}
\vskip1em

\begin{exo}\label{exo:9.6}
{\footnotesize Identique à l'énoncé 4.2 (p.~\pageref{exo:4.2}), questions
  a–c.}

On a mesuré en fin de traitement chez 18 patients répartis par tirage au
sort en trois groupes de traitement A, B, et C, un paramètre biologique dont
on sait que la distribution est normale. Les résultats sont les suivants :
\vskip1em

\begin{tabular}{ccc}
\toprule
A & B & C \\
\midrule
19.8 & 15.9 & 15.4 \\
20.5 & 19.7 & 17.1 \\
23.7 & 20.8 & 18.2 \\
27.1 & 21.7 & 18.5 \\
29.6 & 22.5 & 19.3 \\
29.9 & 24.0 & 21.2 \\
\bottomrule
\end{tabular}
\vskip1em

\begin{description}
\item[(a)] Réaliser une ANOVA à un facteur.
\item[(b)] Selon le résultat du test, procéder aux comparaisons par paire de
  traitement des moyennes, en appliquant une correction simple de Bonferroni
  (c'est-à-dire où les degrés de significativité estimé sont multipliés par
  le nombre de comparaisons effectuées). Comparer avec de simples tests de
  Student non corrigés pour les comparaisons multiples. 
\item[(c)] D'après des études plus récentes, il s'avère que la normalité des
  distributions parentes peut-être remise en question. Effectuer la
  comparaison des trois groupes par une approche non-paramétrique.
\end{description}
\end{exo}
\vskip1em

\begin{exo}\label{exo:9.7}
{\footnotesize Identique à l'énoncé 4.3 (p.~\pageref{exo:4.3}), questions
  a–e.}

Un service d'obstétrique s'intéresse au poids de nouveaux-nés nés à terme et
âgés de 1 mois. Pour cet échantillon de 550 bébés, on dispose également
d'une information concernant la parité (nombre de frères et soeurs), mais on
sait qu'il n'y aucune relation de gemellité parmi les enfants ayant des
frères et soeurs. L'objet de l'étude est de déterminer si la parité (4
classes) influence le poids des nouveaux-nés à 1 mois. Les données sont
résumées dans le tableau suivant, et elles sont disponibles dans un fichier
SPSS, \texttt{weights.sav}.\autocite[p.~113]{peat05}
\vskip1em

\begin{tabular}{lrrrrr}
\toprule
& \multicolumn{4}{c}{Nombre de frères et soeurs} & Total \\
& 0 & 1 & 2 & $\ge 3$ & \\
\midrule
\emph{Échantillon} & & & & \\ 
Effectif & 180 & 192 & 116 & 62 & 550 \\
Fréquence & 32.7 & 34.9 & 21.1 & 11.3 & 100.0 \\
\emph{Poids (kg)} & & & & \\
Moyenne & 4.26 & 4.39 & 4.46 & 4.43 & \\
Écart-type & 0.62 & 0.59 & 0.61 & 0.54 & \\
(Min–Max) & (2.92–5.75) & (3.17–6.33) & (3.09–6.49) & (3.20–5.48) & \\
\bottomrule
\end{tabular}
\vskip1em

\begin{description}
\item[(a)] Vérifier les données reportées dans le tableau précédent.
\item[(b)] Procéder à une analyse de variance à un facteur. Conclure sur la
  significativité globale et indiquer la part de variance expliquée par le
  modèle.
\item[(c)] Afficher la distribution des poids selon la parité. Procéder à un
  test d'homogénéité des variances (rechercher dans l'aide en ligne le test
  de Levenne). 
\item[(d)] On décide de regrouper les deux dernières catégories (2 et $\ge
  3$). Refaire l'analyse et comparer aux résultats obtenus en (b).
\item[(e)] Réaliser un test de tendance linéaire (par ANOVA) sur les données
  recodées en trois niveaux pour la parité.
\end{description}
\end{exo}

%--------------------------------------------------------------- Devoir 08 ---
\chapter*{Devoir \no 8}
\addcontentsline{toc}{chapter}{Devoir \no 8}

Les exercices sont indépendants. Une seule réponse est correcte pour chaque
question. Lorsque vous ne savez pas répondre, cochez la case correspondante.

\section*{Exercice 1}

\section*{Exercice 2}

\section*{Exercice 3}

%--------------------------------------------------------------- Chapter 10 ---
\chapter{Régression linéaire et logistique}

\begin{exo}\label{exo:10.1}
{\footnotesize Identique à l'énoncé 5.1 (p.~\pageref{exo:5.1}), questions
  a–e.}

Une étude a porté sur une mesure de malnutrition chez 25 patients âgés de 7
à 23 ans et souffrant de fibrose kystique. On disposait pour ces patients de
différentes informations relatives aux caractéristiques antropométriques
(taille, poids, etc.) et à la fonction pulmonaire. \autocite[p.~180]{everitt01}
Les données sont disponibles dans le fichier \texttt{cystic.dat}.
\begin{description}
\item[(a)] Calculer le coefficient de corrélation linéaire entre les
  variables \texttt{PEmax} et \texttt{Weight}, ainsi que son intervalle de
  confiance à 95~\%.
\item[(b)] Tester si le coefficient de corrélation calculé en (a) peut être
  considéré comme significativement différent de 0.3 au seuil 5~\%.
\item[(c)] Afficher l'ensemble des données numériques sous forme de
  diagrammes de dispersion, soit 45 graphiques arrangés sous forme d'une
  "matrice de dispersion".
\item[(d)] Calculer l'ensemble des corrélations de Pearson et de Spearman
  entre les variables numériques. 
  %Reporter les coefficients de
  %Bravais-Pearson supérieurs à 0.7 en valeur absolue.
\item[(e)] Calculer la corrélation entre \texttt{PEmax} et \texttt{Weight},
  en contrôlant l'âge (\texttt{Age}) (corrélation partielle). Représenter
  graphiquement la covariation entre \texttt{PEmax} et \texttt{Weight} en
  mettant en évidence les deux terciles les plus extrêmes pour la variable
  \texttt{Age}. 
\end{description}
\end{exo}
\vskip1em

\begin{exo}\label{exo:10.2}
{\footnotesize Identique à l'énoncé 5.2 (p.~\pageref{exo:5.2}), questions
  a–e.}

Les données disponibles dans le fichier \texttt{quetelet.csv} renseignent
sur la pression artérielle systolique (\texttt{PAS}), l'indice de Quetelet
(\texttt{QTT}), l'âge (\texttt{AGE}) et la consommation de tabac
(\texttt{TAB}=1 si fumeur, 0 sinon) pour un échantillon de 32 hommes de plus
de 40 ans. 
\begin{description}
\item[(a)] Indiquer la valeur du coefficient de corrélation linéaire entre
  la pression artérielle systolique et l'indice de Quetelet, avec un
  intervalle de confiance à 90~\%.
\item[(b)] Donner les estimations des paramètres de la droite de régression
  linéaire de la pression artérielle sur l'indice de Quetelet.
\item[(c)] Tester si la pente de la droite de régression est différente de 0
  (au seuil 5~\%).
\item[(d)] Représenter graphiquement les variations de pression artérielle
  en fonction de l'indice de Quetelet, en faisant apparaître distinctement
  les fumeurs et les non-fumeurs avec des symboles ou des couleurs
  différentes, et tracer la droite de régression dont les paramètres ont été
  estimés en (b). 
\item[(e)] Refaire l'analyse (b-c) en restreignant l'échantillon aux
  fumeurs.
\end{description}
\end{exo}
\vskip1em

\begin{exo}\label{exo:10.3}
{\footnotesize Identique à l'énoncé 5.3 (p.~\pageref{exo:5.3}), questions
  a–d.}

Dans l'étude Framingham, on dispose de donnée sur la pression artérielle
systolique (\texttt{sbp}) et l'indice de masse corporelle (\texttt{bmi}) de
2047 hommes et 2643 femmes.\autocite[p.~63]{dupont09} On s'intéresse à la
relation entre ces deux variables (après transformation logarithmique) chez
les hommes et chez les femmes séparément.
Les données sont disponibles dans le fichier \texttt{Framingham.csv}.
\begin{description}
\item[(a)] Représenter graphiquement les variations entre pression
  artérielle et IMC (\texttt{bmi}) chez les hommes et chez les femmes.
\item[(b)] Les coefficients de corrélation linéaire estimés chez les hommes
  et chez les femmes sont-ils significativement différents à 5~\% ?
\item[(c)] Estimer les paramètres du modèle de régression linéaire
  considérant la pression artérielle comme variable réponse et l'IMC comme
  variable explicative, pour ces deux sous-échantillons. Donner un
  intervalle de confiance à 95~\% pour l'estimé des pentes respectives.
\item[(d)] Tester l'égalité des deux coefficients de régression associés à
  la pente (au seuil 5~\%).
\end{description}
\end{exo}
\vskip1em

\begin{exo}\label{exo:10.4}
{\footnotesize Identique à l'énoncé 6.1 (p.~\pageref{exo:6.1}) et répondre
  aux questions a–d.} 

On étudie l'effet d'un traitement prophylactique d'un macrolide à faibles
doses (Traitement A) sur les épisodes infectieux chez des patients atteints
de mucoviscidose dans un essai randomisé multicentrique contre placebo
(B). Les résultats sont les suivants :
\vskip1em

\begin{tabular}{lccc}
\toprule
& \multicolumn{2}{c}{Infection} & \\
\cmidrule(r){2-3}
& Non & Oui & Total \\
\midrule
Traitement (A) & 157 & 52 & 209 \\
Placebo (B) & 119 & 103 & 222 \\
Total & 276 & 155 & 431 \\
\bottomrule
\end{tabular}
\vskip1em

\begin{description}
\item[(a)] À partir d'un test du $\chi^2$, que peut-on répondre à la
  question : le traitement permet-il de prévenir la survenue d'épisodes
  infectieux (au seuil $\alpha=0.05$) ? Vérifier que les effectifs
  théoriques sont bien tous supérieurs à 5.
\item[(b)] Conclut-on de la même manière à partir de l'intervalle de
  confiance de l'odds-ratio associé à l'effet traitement ?
\item[(c)] On souhaite vérifier s'il existe une disparité du point de vue
  des pourcentages d'épisodes infectieux en fonction du centre. Les données
  par centre sont indiquées dans le tableau ci-après. Conclure à partir d'un
  test du $\chi^2$.

  \begin{table}[!htb] \hskip40pt
  \begin{minipage}[b]{0.33\linewidth}
  \scalebox{0.65}{\begin{tabular}{|l|r|r|r|}
    \multicolumn{1}{c}{} & \multicolumn{2}{c}{Infection} &  \multicolumn{1}{c}{} \\
    \cline{2-4}
    \multicolumn{1}{c|}{} & Non & Oui & Total \\
    \hline
    Traitement (A) & 51 & 8 & 59 \\
    \hline
    Placebo (B) & 47 & 19 & 66 \\
    \hline
    Total & 98 & 27 & 125 \\
    \hline
    \multicolumn{4}{c}{Centre 1}
  \end{tabular}} 
  \end{minipage} \hspace{0.1cm}
  \begin{minipage}[b]{0.3\linewidth}
  \scalebox{0.65}{\begin{tabular}{|l|r|r|r|}
    \multicolumn{1}{c}{} & \multicolumn{2}{c}{Infection} &  \multicolumn{1}{c}{} \\
    \cline{2-4}
    \multicolumn{1}{c|}{} & Non & Oui & Total \\
    \hline
    Traitement (A) & 91 & 35 & 126 \\
    \hline
    Placebo (B) & 61 & 71 & 132 \\
    \hline
    Total & 152 & 106 & 258 \\
    \hline
    \multicolumn{4}{c}{Centre 2}
  \end{tabular}} 
  \end{minipage} \hspace{0.1cm}
  \begin{minipage}[b]{0.3\linewidth}
  \scalebox{0.65}{\begin{tabular}{|l|r|r|r|}
    \multicolumn{1}{c}{} & \multicolumn{2}{c}{Infection} &  \multicolumn{1}{c}{} \\
    \cline{2-4}
    \multicolumn{1}{c|}{} & Non & Oui & Total \\
    \hline
    Traitement (A) & 15 & 9 & 24 \\
    \hline
    Placebo (B) & 11 & 13 & 24 \\
    \hline
    Total & 26 & 22 & 48 \\
    \hline
    \multicolumn{4}{c}{Centre 3}
  \end{tabular}}
  \end{minipage}
  \end{table}
\item[(d)] À partir du tableau précédent, on cherche à vérifier si l'effet
  traitement est indépendent du centre ou non. On se propose de réaliser un
  test de comparaison entre les deux traitements ajustés sur le centre (test
  de Mantel-Haenszel). Indiquer le résultat du test ainsi que la valeur de
  l'odds-ratio ajusté.
\end{description}
\end{exo}
\vskip1em

\begin{exo}\label{exo:10.5}
{\footnotesize Identique à l'énoncé 6.3 (p.~\pageref{exo:6.3}), questions
  a–e.}

On dispose de données issues d'une étude cherchant à établir la validité
pronostique de la concentration en créatine kinase dans l'organisme sur la
prévention de la survenue d'un infarctus du myocarde.\autocite[p.~115]{rabe-hesketh04}

Les données sont disponibles dans le fichier \texttt{sck.dat} : la première
colonne correspond à la variable créatine kinase (\texttt{ck}), la deuxième
à la variable présence de la maladie (\texttt{pres}) et la dernière à la
variable absence de maladie (\texttt{abs}).
\begin{description}
\item[(a)] Quel est le nombre total de sujets ?
\item[(b)] Calculer les fréquences relatives malades/non-malades, et
  représenter leur évolution en fonction des valeurs de créatine kinase à
  l'aide d'un diagramme de dispersion (points + segments reliant les points).
\item[(c)] À partir d'un modèle de régression logistique dans lequel on
  cherche à prédire la probabilité d'être malade, calculer la valeur de
  \texttt{ck} à partir de laquelle ce modèle prédit que les personnes
  présentent la maladie en considérant une valeur seuil de 0.5
  (si $P(\text{malade})\ge 0.5$ alors \texttt{malade=1}).
\item[(d)] Représenter graphiquement les probabilités d'être malade prédites
  par ce modèle ainsi que les proportions empiriques en fonction des valeurs
  \texttt{ck}. 
\item[(e)] Établir la courbe ROC correspondante, et reporter la
  valeur de l'aire sous la courbe. 
\end{description}
\end{exo}
\vskip1em

\begin{exo}\label{exo:10.6}
{\footnotesize Identique à l'énoncé 6.5 (p.~\pageref{exo:6.5}), questions
  a–d.} 

Une enquête cas-témoin a porté sur la relation entre la consommation
d'alcool et de tabac et le cancer de l'oesophage chez l'homme (étude "Ille
et Villaine"). Le groupe des cas était composé de 200 patients atteints d'un
cancer de l'oesophage et diagnostiqué entre janvier 1972 et avril 1974. Au
total, 775 témoins de sexe masculin ont été sélectionnés à partir des listes
électorales. Le tableau suivant indique la répartition de l'ensemble des
sujets selon leur consommation journalière d'alcool, en considérant qu'une
consommation supérieure à 80 g est considérée comme un facteur de
risque.\autocite{breslow80} 
\vskip1em

\begin{tabular}{lccc}
\toprule
& \multicolumn{2}{c}{Consommation d'alcool (g/jour)} & \\
\cmidrule(r){2-3}
& $\ge 80$ & $<80$ & Total \\
\midrule
Cas & 96 & 104 & 200 \\
Témoins & 109 & 666 & 775 \\
Total & 205 & 770 & 975 \\
\bottomrule
\end{tabular}
\vskip1em

\begin{description}
\item[(a)] Quelle est la valeur de l'odds-ratio et son intervalle de
  confiance à 95~\% (méthode de Woolf) ? Est-ce une bonne estimation du
  risque relatif ? 
\item[(b)] Est-ce que la proportion de consommateurs à risque est la même
  chez les cas et chez les témoins (considérer $\alpha=0.05$) ?
\item[(c)] Construire le modèle de régression logistique permettant de
  tester l'association entre la consommation d'alcool et le statut des
  sujets. Le coefficient de régression est-il significatif ?
\item[(d)] Retrouvez la valeur de l'odds-ratio observé, calculé en (b), et
  son intervalle de confiance à partir des résultats de l'analyse de
  régression.
\end{description}
\end{exo}

%--------------------------------------------------------------- Devoir 09 ---
\chapter*{Devoir \no 9}
\addcontentsline{toc}{chapter}{Devoir \no 9}

Les exercices sont indépendants. Une seule réponse est correcte pour chaque
question. Lorsque vous ne savez pas répondre, cochez la case correspondante.

\section*{Exercice 1}

\section*{Exercice 2}

\section*{Exercice 3}


%--------------------------------------------------------------- Chapter 11 ---
\chapter{Analyse de données de survie}

\begin{exo}\label{exo:11.1}
{\footnotesize Identique à l'énoncé 7.1 (p.~\pageref{exo:7.1}), questions
  a–i.}

Dans un essai contre placebo sur la cirrhose biliaire, la D-penicillamine
(DPCA) a été introduite dans le bras actif sur une cohorte de 312
patients. Au total, 154 patients ont été randomisés dans le bras actif
(variable traitement, \texttt{rx}, 1=Placebo, 2=DPCA). Un ensemble de
données telles que l'âge, des données biologiques et signes cliniques variés
incluant le niveau de bilirubine sérique (\texttt{bilirub}) sont disponibles
dans le fichier \texttt{pbc.txt}.\autocite{vittinghoff05} Le status du
patient est enregistré dans la variable \texttt{status} (0=vivant, 1=décédé)
et la durée de suivi (\texttt{years}) représente le temps écoulé en années
depuis la date de diagnostic.
\begin{description}
\item[(a)] Combien dénombre-t-on d'individus décédés? Quelle proportion de
  ces décès retrouve-t-on dans le bras actif ?  
\item[(b)] Afficher la distribution des durées de suivi des 312 patients, en
  faisant apparaître distinctement les individus décédés. Calculer le temps
  médian (en années) de suivi pour chacun des deux groupes de
  traitement. Combien y'a-t-il d'événements positifs au-delà de 10.5 années
  et quel est le sexe de ces patients ?
\item[(c)] Les 19 patients dont le numéro (\texttt{number}) figure parmi la
  liste suivante ont subi une transplantation durant la période de suivi.
\begin{verbatim}  
5 105 111 120 125 158 183 241 246 247 254 263 264 265 274 288 291
295 297 345 361 362 375 380 383
\end{verbatim}   
  Indiquer leur âge moyen, la distribution selon le sexe et la durée médiane
  de suivi en jours jusqu'à la transplantation.
\item[(d)] Afficher un tableau résumant la distribution des événements à
  risque en fonction du temps, avec la valeur de survie associée.
\item[(e)] Afficher la courbe de Kaplan-Meier avec un intervalle de
  confiance à 95~\%, sans considérer le type de traitement.
\item[(f)] Calculer la médiane de survie et son intervalle de confiance à
  95~\% pour chaque groupe de sujets et afficher les courbes de survie
  correspondantes.
\item[(g)] Effectuer un test du log-rank en considérant comme prédicteur le
  facteur \texttt{rx}. Comparer avec un test de Wilcoxon.
\item[(h)] Effectuer un test du log-rank sur le facteur d'intérêt
  (\texttt{rx}) en stratifiant sur l'âge. On considèrera trois groupe
  d'âge : 40 ans ou moins, entre 40 et 55 ans inclus, plus de 55 ans.
\item[(i)] Retrouver les résultats de l'exercice 1.g avec une régression de
  Cox. 
\end{description}
\end{exo}
\vskip1em

\begin{exo}\label{exo:11.2}
{\footnotesize Identique à l'énoncé 7.3 (p.~\pageref{exo:7.3}), questions
  a–d}.

Dans un essai randomisé, on a cherché à comparer deux traitements pour le
cancer de la prostate. Les patients prenaient chaque jour par voie orale
soit 1 mg de diethylstilbestrol (DES, bras actif) soit un placebo, et le
temps de survie est mesuré en mois.\autocite{collett94} La question
d'intérêt est de savoir si la survie diffère entre les deux groupes de
patients, et on négligera les autres variables présentes dans le fichier de
données \texttt{prostate.dat}. 
\begin{description}
\item[(a)] Calculer la médiane de survie pour l'ensemble des patients, et
  par groupe de traitement.
\item[(b)] Quelle est la différence entre les proportions de survie dans les
  deux groupes à 50 mois ?
\item[(c)] Afficher les courbes de survie pour les deux groupes de patients.
\item[(d)] Effectuer un test du log-rank pour tester l'hypothèse selon
  laquelle le traitement par DES a un effet positif sur la survie des
  patients. 
\end{description}
\end{exo}

%--------------------------------------------------------------- Devoir 10 ---
\chapter*{Devoir \no 10}
\addcontentsline{toc}{chapter}{Devoir \no 10}

Les exercices sont indépendants. Une seule réponse est correcte pour chaque
question. Lorsque vous ne savez pas répondre, cochez la case correspondante.

\section*{Exercice 1}

\section*{Exercice 2}


\cleardoublepage

\blankpage
\blankpage

% -------------------------------------------------- Corrigés SAS ----------------
\thispagestyle{empty}

\centerline{\small Centre d'Enseignement de la Statistique Appliquée, à la Médecine et à la Biologie Médicale}
\vspace*{2cm}
\begin{center}
\centerline{\includegraphics[scale=.55]{cesam}}
\vspace*{2cm}
\begin{minipage}{.75\textwidth}
\begin{mdframed}[style=titlep]
\centerline{\Huge Corrigés des exercices}
\vskip1em
\centerline{\Huge du cours d'informatique du CESAM}
\end{mdframed}
\end{minipage}
\end{center}
\vskip3em
\centerline{\Huge\bf Introduction au logiciel SAS}
\vskip5em
\begin{center}
  \begin{tabular}{ll}
    \textbf{Responsables :} & \\
    Christophe LALANNE & \url{christophe.lalanne@inserm.fr} \\
    Yassin MAZROUI     & \url{yassin.mazroui@upmc.fr} \\
    Pr Mounir MESBAH   & \url{mounir.mesbah@upmc.fr}
  \end{tabular}
\end{center}
\vskip3em
\centerline{\Large \url{http://www.cesam.upmc.fr}}
\vskip3em
\centerline{\LARGE Année Universitaire 2015–2016}
\vfill
\begin{center}
\begin{minipage}{.6\textwidth}
\centering
Adresser toute correspondance à :\\
Université Pierre et Marie Curie – Paris 6
Secrétariat du CESAM – Les Cordeliers
Service Formation Continue, esc. B, 4ème étage,
15 rue de l’école de médecine,
75006 PARIS\\
ou par Courriel à : \url{cesam@upmc.fr}
\end{minipage}
\end{center}



\blankpage

% remove chapter prefix because we no longer want chapter name.
\titleformat{\chapter}
  {\normalfont\huge\bfseries}{}{20pt}{\huge}
  

\setcounter{page}{1}
%---------------------------------------------------------------- Séance 08 --
\chapter*{Semaine 8\markboth{Corrigés de la semaine 8}{}}

\soln{\ref{exo:8.1}}
Pour saisir les données sous \SAS, on effectue dans un premier temps une
étape \texttt{DATA} dans laquelle les données individuelles sont saisie
manuellement à l'aide de la commande \texttt{input} pour indiquer le nom de
la variable, suivi de \texttt{cards} qui place les données directement dans
le programme \SAS.
\begin{verbatim}
DATA Exercice1_1;
INPUT x;
CARDS;
3.68
2.21
2.45
8.64
4.32
3.43
5.11
3.87
;
RUN;
\end{verbatim}

Le nombre d'observations ainsi que les valeurs des observations extrêmes de
la distribution empirique sont obtenues à partir de la commande \texttt{PROC
  SUMMARY}, en précisant les options correspondantes (\texttt{n},
\texttt{min} et \texttt{max} et \texttt{range}).
\begin{verbatim}
PROC SUMMARY DATA=Exercice1_1 PRINT n min max range; VAR x; RUN;
\end{verbatim}

Comme le jeu de données est limité en termes d'effectifs, le plus simple
consiste à modifier l'étape \texttt{DATA} précédente en effectuant les
modifications demandées : (a) remplacement de la valeur 8.64 par 3.64, et
(b) recodage de la valeur 5.11 en point (".") qui est le symbole par défaut
utilisé par \SAS pour représenter les données manquantes.
\begin{verbatim}
DATA Exercice1_1;
INPUT x;
CARDS;
3.68
2.21
2.45
3.64    /* (a) */
4.32
3.43
.       /* (b) */
3.87
;
RUN;
\end{verbatim}
%
%
%
\soln{\ref{exo:8.2}}
La saisie des données s'effectue de la même manière qu'à l'exercice
précédent, c'est-à-dire à partir d'une étape \texttt{DATA} :
\begin{verbatim}
DATA Exercice1_2;
INPUT X;
detect=1;
IF X <= log10(50) THEN detect=0;
CARDS;
3.64
2.27
1.43
1.77
4.62
3.04
1.01
2.14
3.02
5.62
5.51
5.51
1.01
1.05
4.19
2.63
4.34
4.85
4.02
5.92
;
RUN;
\end{verbatim}
On notera une petite différence par rapport à l'exercice~\ref{exo:1.1}, à
savoir que l'on définit également une variable \texttt{detect} qui prend la
valeur 0 si $X\le \log_{10}(50)$ et 1 autrement. Ceci est effectué avant
l'instruction \texttt{cards} qui signifie le début de la sauvegarde des
données numériques en mémoire.

Le nombre de patients avec une charge virale indétectable s'obtient à partir
d'un simple tri à plat de la variable \texttt{detect} grâce à la commande
\texttt{PROC FREQ}.
\begin{verbatim}
PROC FREQ DATA=exercice1_2; TABLES detect; RUN;
\end{verbatim}

Pour calculer la charge virale médiane pour les seules observations valides
(c'est-à-dire ayant une valeur au-dessus du seuil limite de détection), il
est nécessaire de sélectionner les unités statistiques remplissant les
conditions de validité et d'utiliser la commande \texttt{PROC SUMMARY}.
\begin{verbatim}
DATA detect; SET exercice1_2;
Y=exp(X*log(10));
IF detect=1; RUN;
PROC SUMMARY DATA=detect PRINT median; VAR Y; RUN;
\end{verbatim}
%
%
%
\soln{\ref{exo:8.3}}
Pour lire les données contenues dans le fichier \texttt{anorexia.data} dont
un aperçu est fourni ci-dessous
\begin{verbatim}
Group Before After
g1 80.5  82.2
g1 84.9  85.6
g1 81.5  81.4
g1 82.6  81.9
g1 79.9  76.4
\end{verbatim}
on va utiliser la commande \texttt{infile} dans l'étape \texttt{DATA}, en
précisant que la lecture des données doit commencer à la 2\ieme\ ligne
(\texttt{firstobs=2}) et que les données sont séparées par des espaces (en
nombre variable).
\begin{verbatim}
DATA anorexia;
INFILE "C:\data\anorexia.dat" firstobs=2 dlm="09"X;
INPUT  groupe $ 1-2  before 3-7   after 8-13;
RUN;
\end{verbatim}
On peut très bien associer des étiquettes plus informatives aux modalités
prises par la variable qualitative \texttt{Group} à l'aide d'une étape
\texttt{FORMAT} :
\begin{verbatim}
PROC FORMAT;
VALUE Therapie
  1='Thérapie comportementale'
  2='Thérapie familiale'
  3='Thérapie contrôle'
;
RUN;
\end{verbatim}

Pour obtenir les effectifs par type de thérapie, on utilise un simple tri à
plat à l'aide de la commande \texttt{PROC FREQ}. L'affichage des résultats
peut être personnalisé en utilisant le renommage des groupes réalisé à
l'étape précédente.
\begin{verbatim}
PROC FREQ DATA=anorexia; TABLES groupe; FORMAT groupe therapie.; RUN;
\end{verbatim}

La transformation d'unités pour les poids ne pose pas de problème
spécifique, mais il faut décider si l'on crée de nouvelles variables ou si
l'on remplace les valeurs existantes. Ici, on créera deux nouvelles
variables, \verb|before_kg| et \verb|after_kg| :
\begin{verbatim}
DATA anorexia; SET anorexia;
before_kg=before/2.2;
after_kg=after/2.2;
RUN;
\end{verbatim}

Pour les scores de différences, on crée également une nouvelles variable
à l'aide d'une étape \texttt{DATA} :
\begin{verbatim}
DATA anorexia; SET anorexia;
diff=after_kg-before_kg;
RUN;
\end{verbatim}
et l'on résumé la distribution des scores de différence (moyenne et étendue)
par groupe de traitement à l'aide de \texttt{PROC SUMMARY}. On notera que
puisque l'on fait intervenir une variable de groupement, il est nécessaire
dans un premier temps de trier les données par classe (variable
\texttt{groupe}). 
\begin{verbatim}
PROC SORT DATA=anorexia; BY groupe; RUN;

PROC SUMMARY DATA=anorexia PRINT n mean min max range; VAR diff; BY groupe; FORMAT groupe therapie.; RUN;
\end{verbatim}
%
%
%
\soln{\ref{exo:8.4}}
La saisie des données brutes se fera comme dans les exercices~8.1 et 8.2 à
l'aide d'une étape \texttt{DATA} (commande \texttt{cards}).
\begin{verbatim}
DATA X;
INPUT X;
CARDS;
24.9
25.0
25.0
25.1
25.2
25.2
25.3
25.3
25.3
25.4
25.4
25.4
25.4
25.5
25.5
25.5
25.5
25.6
25.6
25.6
25.7
25.7
25.8
25.8
25.9
26.0
;
RUN;
\end{verbatim}

Les indicateurs de tendance centrale (moyenne, médiane et mode) peuvent être
calculés et affichés à partir de la commande \texttt{PROC SUMMARY}.
\begin{verbatim}
PROC SUMMARY DATA=X PRINT mean median mode; VAR X; RUN;
\end{verbatim}

Quant à la variance, on changera simplement les options de calcul dans
\texttt{PROC SUMMARY} :
\begin{verbatim}
PROC SUMMARY DATA=X PRINT var; VAR X; RUN;
\end{verbatim}

Pour le recodage en 4 classes d'intervalles pré-définis pour la variable
\texttt{X}, on crée une variable auxiliaire \verb|X_classes| à laquelle on
affecte les valeurs 1, 2, 3 et 4 selon les valeurs prises par \texttt{X}.
\begin{verbatim}
DATA X; SET X;
/* 24.9-25.1, 25.2-25.4, 25.5-25.7, 25.8-26.0 */

X_classes=1;
IF X ge 25.2 THEN X_classes=2;
IF X ge 25.5 THEN X_classes=3;
IF X ge 25.8 THEN X_classes=4;
RUN;
\end{verbatim}
Le tableau d'effectifs peut s'obtenir directement avec une commande
\texttt{tables} dans \texttt{PROC FREQ} :
\begin{verbatim}
PROC FREQ DATA=X; TABLES X_classes; RUN;
\end{verbatim}
Pour faciliter la lecture des résultats, on peut tout à fait associer des
étiquettes plus informatives aux quatre classes avec \texttt{PROC FORMAT}
et associer celles-ci au tableau de résultat renvoyé par \texttt{tables}.
\begin{verbatim}
PROC FORMAT;
    VALUE classes   24.9-25.1="Intervalle 1"
                    25.2-25.4="Intervalle 2"
                    25.5-25.7="Intervalle 3"
                    25.8-26.0="Intervalle 4";
RUN;

PROC FREQ DATA=X; TABLES X; FORMAT X classes.; RUN;
\end{verbatim}

Enfin, pour la représentation sous forme d'histogramme, \SAS dispose de ses
propres algorithmes de calcul pour la détermination du nombre de classes à
construire, tout comme R. Tout est géré à partir de la commande
\texttt{PROC GCHART}.
\begin{verbatim}
PROC GCHART DATA=X; Hbar X; RUN;  
\end{verbatim}

On pourrait également utiliser un diagramme en barres orientées
verticalement en précisant l'option \texttt{Vbar}.
\begin{verbatim}
PROC GCHART DATA=X; Vbar X; RUN;
\end{verbatim}
%
%
%
\soln{\ref{exo:8.5}}
Pour importer les données stockées dans un simple fichier texte, on utilise
la commande \verb|infile| dans une étape \texttt{DATA}.
\begin{verbatim}
DATA elderly;
INFILE  "C:\data\elderly.dat" dlm="09"X ;
INPUT  taille @@ ;
RUN;
\end{verbatim}
Ici, on notera que l'on ne précise pas la manière dont sont codées les
valeurs manquantes car le "." est le format utilisé par défaut par \SAS. une
autre solution consisterait à introduire une commande \texttt{datalines}
après la commande \texttt{input} et à copier/coller les données du fichier
texte. 

Le nombre total d'observations manquantes peut s'obtenir à partir de
\texttt{PROC SUMMARY} en spécifiant l'option \texttt{nmiss}.
\begin{verbatim}
PROC SUMMARY DATA=elderly PRINT nmiss; VAR taille; RUN;
\end{verbatim}

Pour obtenir la taille moyenne et son intervalle de confiance à 95~\%
associé, il suffit d'entrer la commande suivante :
\begin{verbatim}
PROC SUMMARY DATA=elderly PRINT clm uclm lclm; VAR taille; RUN;
\end{verbatim}

Enfin, pour afficher la distribution des tailles sous forme d'une courbe de
densité, on utilise la commande \texttt{PROC UNIVARIATE} et
\texttt{HISTOGRAM} avec l'option \texttt{/ kernel}. Le degré de lissage peut
être contrôlé à l'aide de l'option \texttt{C}.
\begin{verbatim}
PROC UNIVARIATE DATA=elderly; VAR taille; HISTOGRAM taille / kernel; RUN;
\end{verbatim}
%
%
%
\soln{\ref{exo:8.6}}
Les données sur les poids à la naissance de Hosmer \& Lemeshow (1989) ont
été exportées au format texte dans le fichier \texttt{birthwt.dat}, et elles
peuvent être importées ainsi :
\begin{verbatim}
PROC IMPORT OUT= WORK.BIRTHWT
            DATAFILE= "C:\data\birthwt.dat"
            DBMS=DLM REPLACE;
     DELIMITER='20'x;
     GETNAMES=NO;
     DATAROW=1;
RUN;

DATA birthwt1; SET birthwt;
  low = var1;
  age = var2;
  lwt = var3 ;
  race = var4 ;
  smoke = var5;
  ptl = var6 ;
  ht = var7 ;
  ui = var8 ;
  ftw = var9 ;
  bwt = var10 ;
  DROP var1-var10;
RUN;
\end{verbatim}
La première étape (\texttt{PROC IMPORT}) consiste à définir les options pour
l'importation des données : les noms de variables ne figurent pas dans le
fichier de données (\texttt{GETNAMES=NO}) et la lecture des données doit
commencer dès la 1\iere\ ligne du fichier (\texttt{DATAROW=1}), sachant que
les données individuelles sont séparées par des espaces. La seconde étape
(\texttt{DATA}) consiste à assigner des noms de variables au tableau de
données \texttt{WORK.BIRTHWT} généré par \SAS. On notera qu'il est
nécessaire de supprimer les anciens noms de variable après renommage, d'où
l'usage de la commande \texttt{DROP}.

Pour associer de nouvelles étiquettes aux variables \texttt{low},
\texttt{race}, \texttt{smoke}, \texttt{ui} et \texttt{ht}, il suffit de
définir des "labels" et de les associer aux variables en question (cela ne
change le format de représentation des variables en mémoire, qui restent
stockées sous forme de nombre).
\begin{verbatim}
PROC FORMAT;
      VALUE  low 1="Poids inférieur à 2.5 Kg"
                 0="Poids supérieur à 2.5 Kg";
      VALUE  ethnicite 1="White"
                       2="Black"
                       3="Other";
      VALUE  tabac 1="consommation tabac durant grossesse"
                   0="Pas de consommation tabac durant grossesse";
      VALUE  Hypert 1="Antecedent d hypertension"
                    0="Pas d antecedent d hypertension";
      VALUE  uterine 1="Manisfestation d irritabilite uterine"
                     0="Pas de manisfestation d irritabilite uterine";
RUN;
\end{verbatim}
On pourra vérifier que ces modifications ont bien été prises en compte en
utilisant une comamnde telle que \texttt{PROC FREQ} pour afficher les
tableaux d'effectifs associés à ces variables qualitatives.
\begin{verbatim}
PROC FREQ DATA=birthwt1; 
  TABLES low race smoke ui ht;
  FORMAT low low. race ethnicite. smoke tabac. ht hypert. ui uterine.;
RUN;
\end{verbatim}

La conversion du poids des mères en \emph{kg} ne pose pas de problème
particulier, et ici on remplacera directement les données disponibles dans
une étape \texttt{DATA}.
\begin{verbatim}
DATA birthwt2; SET birthwt1;
  lwt=lwt/2.2; 
RUN;
\end{verbatim}

Les indicateurs de tendance centrale et de dispersion relative sont obtenus
à partir de la commande \texttt{PROC SUMMARY}, en spécifiant les options
adéqautes. 
\begin{verbatim}
PROC SUMMARY DATA=birthwt2 PRINT mean median qrange; VAR lwt; RUN;
\end{verbatim}

Enfin, un histogramme du poids des mères est construit à l'aide de la
commande \texttt{PROC GCHART}. Ici, sans autre option un histogramme de
densité sera construit.
\begin{verbatim}
PROC GCHART DATA=birthwt2; Hbar lwt; RUN;

PROC UNIVARIATE DATA=birthwt2;
  VAR lwt;
  HISTOGRAM lwt / kernel;
RUN;
\end{verbatim}

Concernant la proportion de mères ayant fumé durant la grossesse et
le calcul de l'intervalle de confiance à 95~\% associé, on peut utiliser la
commande \texttt{PROC SUMMARY}, qui comme dans le cas de la commande \R
\texttt{prop.test} suppose de grands échantillons :
\begin{verbatim}
PROC SUMMARY DATA=birthwt2 PRINT clm uclm lclm; VAR smoke; RUN;
\end{verbatim}

Les diagrammes en barres peuvent être affichés, selon une orientation
verticale ou horizontale, à l'aide de \texttt{PROC GCHART}, avec une syntaxe
identique au cas des histogrammes pour des variables numériques.
\begin{verbatim}
PROC GCHART DATA=birthwt2; Vbar smoke; RUN;
PROC GCHART DATA=birthwt2; Hbar smoke; RUN;
\end{verbatim}

Pour générer les terciles, on classera dans un premier temps les
observations par ordre croissant, puis on découpera les valeurs en trois
sous-effectifs égaux (de taille $189/3=63$).
\begin{verbatim}
/* PROC UNIVARIATE DATA=birthwt2; VAR lwt; RUN; */
PROC RANK DATA=birthwt2 OUT=ranking; VAR age; RANKS ordre; RUN;

DATA birthwt3; SET ranking;
age_classe=1;
IF ordre gt 189/3 THEN age_classe=2; 
IF ordre gt 2*(189/3) THEN age_classe=3;
RUN;
\end{verbatim}
Le croisement de cette nouvelle variable avec la variable indicatrice de
sous-poids donne les résultats suivants, exprimées en termes de proportions :
\begin{verbatim}
PROC FREQ DATA=birthwt3; TABLE age_classe*low; RUN;
\end{verbatim}

Un tri à plat de la variable \texttt{race} est effectuée de la même manière
à partir de la commande \texttt{PROC FREQ}..
\begin{verbatim}
PROC FREQ DATA=birthwt3; TABLES race; FORMAT race ethnicite.; RUN;
\end{verbatim}

Enfin, pour résumer la distribution des variables selon la variable
indicatrice \texttt{low}, on reprendra la commande \texttt{PROC FREQ} en
sépcifiant la liste des variables d'intérêt, 
\verb|(race smoke ui ht age_classe)|, à croiser avec \verb|low| via
l'opérateur \texttt{*}.
\begin{verbatim}
PROC FREQ DATA=birthwt3; 
  TABLES (race smoke ui ht age_classe)*low;
  FORMAT low low. race ethnicite. smoke tabac. ht hypert. ui uterine.;
RUN;
\end{verbatim}

%---------------------------------------------------------------- Séance 09 --
\chapter*{Semaine 9\markboth{Corrigés de la semaine 9}{}}

\soln{\ref{exo:9.1}} On rappelle que seuls les valeurs numériques des poids
à la naissance sont disponibles dans le fichier \texttt{sirds.dat}, et qu'il
nous faut contruire la variable de groupement (enfants décédés \emph{versus}
vivants). Une manière de procéder consiste à effectuer deux étapes
\texttt{DATA} en incluant une variable qualitative indiquant le status, en
plus des données brutes insérer par copier/coller.
\begin{verbatim}
DATA DCD;
INPUT poids @@;
deces=1;
DATALINES;
1.050 1.175 1.230 1.310 1.500 1.600 1.720 1.750 1.770 2.275
2.500 1.030 1.100 1.185 1.225 1.262 1.295 1.300 1.550 1.820
1.890 1.940 2.200 2.270 2.440 2.560 2.730
;
RUN;

DATA VIV;
INPUT poids @@;
deces=0;
CARDS;
1.130 1.575 1.680 1.760 1.930 2.015 2.090 2.600 2.700 2.950
3.160 3.400 3.640 2.830 1.410 1.715 1.720 2.040 2.200 2.400
2.550 2.570 3.005
;
RUN;

DATA DRIA; SET DCD VIV; RUN;
\end{verbatim}

Le test de Student est réalisé en utilisant la commande \texttt{PROC TTEST},
qui par défaut fournit les résultats sous l'hypothèse d'homoskédasticité ou
d'hétéroskédasticité, ainsi que les résumés descriptifs numériques et
graphiques pour la distribution du poids selon le status clinique. La
variable de classification est introduite après l'instruction \texttt{CLASS}
et la variable réponse après l'instruction \texttt{VAR}. Par défaut, le test
reporté est bilatéral.
\begin{verbatim}
PROC TTEST DATA=DRIA PLOTS=all;
  CLASS deces;
  VAR poids;
RUN;
\end{verbatim}

\includegraphics{./figs/sas_ttest}
%
%
%
\soln{\ref{exo:9.2}}
Les données de l'étude sur le sommeil servant de base à
l'article de Student peuvent être importées sous \SAS comme à l'exercice
précédent, c'est-à-dire en combinant les résultats de deux étapes
\texttt{DATA}. On en profitera lors de la dernière étape pour créer une
variable auxiliaire pour les scores de différence.
\begin{verbatim}
DATA DHH;
INPUT GMSD @@;
CARDS;
0.7 -1.6 -0.2 -1.2 -0.1 3.4 3.7 0.8 0.0 2.0
;
RUN;

DATA LHH;
INPUT GMSL @@;
CARDS;
1.9 0.8 1.1 0.1 -0.1 4.4 5.5 1.6 4.6 3.4
;
RUN;

DATA HH; MERGE DHH LHH; diff_GMS=GMSL-GMSD; RUN;
\end{verbatim}

Un résumé numérique pour l'ensemble des variables numériques (\texttt{GMSD},
\texttt{GMSL}, et \verb|diff_GMS|) peut être obtenu avec \texttt{PROC SUMMARY} 
de la manière suivante :
\begin{verbatim}
PROC SUMMARY DATA=HH PRINT n mean var lclm uclm; VAR GMSD GMSL diff_GMS; RUN;
\end{verbatim}

Les gains moyens de temps de sommeil pour chaque molécule peuvent être
représentés à l'aide d'un diagramme en barres grâce à \texttt{PROC GCHART}.
\begin{verbatim}
PROC GCHART DATA=hh; Hbar diff_gms / midpoints=(0 0.5 1 1.5 2 2.5 3 3.5 4 4.5 5 5.5 6); RUN;
\end{verbatim}

Enfin, pour réaliser un test $t$ pour données appariées, on utilisera toujours la
commande \texttt{PROC TTEST}, mais en spécifiant l'option \texttt{PAIRED},
comme indiqué ci-après.
\begin{verbatim}
PROC TTEST DATA=hh;
  PAIRED GMSD*GMSL;
RUN;
\end{verbatim}

\includegraphics{./figs/sas_ttestpaired}
%
%
%
\soln{\ref{exo:9.3}}
Dans les cas des données dites "groupées", ou plus généralement d'un tableau
de contingence quelconque, on se contente généralement de saisir le tableau
d'effectif en faisant apparaître distinctement les variables de
classification et les effectifs associés. Sous \SAS, on peut procéder ainsi :
\begin{verbatim}
DATA symptom;
INPUT regime amelioration effectif;
CARDS;
1 1 26
0 1 21
1 0 38
0 0 44
;
RUN;

PROC FORMAT; VALUE ouinon 1="Oui" 0="Non"; RUN;
\end{verbatim}

Ensuite, on peut répondre aux trois questions à partir d'une même commande,
\texttt{PROC FREQ}, en se rappelant qu'il est nécessaire de renseigner
l'option \texttt{WEIGHT} pour indiquer une pondération par les effectifs.
\begin{verbatim}
PROC FREQ DATA=symptom ORDER=data;
  TABLES amelioration * regime / chisq;
  WEIGHT effectif;
  FORMAT regime amelioration ouinon.;
RUN;
\end{verbatim}

\includegraphics{./figs/sas_regime}
%
%
%
\soln{\ref{exo:9.4}}
Comme dans l'exercice précédent, le plus simple pour travailler avec ce type
de tableau à deux entrées est de créer une struture de donnes où l'on fait
correspondre les effectifs pour chacun des croisement des modalités des deux
facteurs d'étude (infarctus et traitement) dans une étape \texttt{DATA}.
\begin{verbatim}
DATA myocarde;
INPUT Infractus traitement nombre;
CARDS;
1 1 28
1 2 18
2 1 656
2 2 658
;
RUN;

PROC FORMAT; 
  VALUE ouinon 1="Oui" 2 = "Non";
  VALUE treat 1=" Placebo" 2 = "Aspirine";
RUN;
\end{verbatim}

Ensuite, on peut répondre aux trois questions à partir d'une même commande,
\texttt{PROC FREQ} et l'option \texttt{/ all}, en se rappelant qu'il est
nécessaire de renseigner l'option \texttt{WEIGHT} pour indiquer une
pondération par les effectifs.
\begin{verbatim}
PROC FREQ DATA=myocarde;
  TABLES traitement*Infractus / all;
  WEIGHT nombre;
  FORMAT Infractus ouinon. traitement treat.;
RUN;
\end{verbatim}
%
%
%
\soln{\ref{exo:9.5}}
Les données au format \Stata peuvent être importées sous \SAS à l'aide de
\texttt{PROC IMPORT} en précisant le type de source de données, ici
\texttt{DBMS=STATA}.
\begin{verbatim}
PROC IMPORT OUT= WORK.polymorphism
            DATAFILE= "C:\data\polymorphism.dta"
            DBMS=STATA REPLACE;
RUN;
\end{verbatim}

La commande \texttt{PROC GLM} est utilisée pour les modèles linéaires
(ANOVA, régression linéaire simple ou multiple, ANCOVA). Dans le cas d'une
ANOVA, on indique le facteur de classification après l'instruction
\texttt{CLASS} et le modèle d'ANOVA après l'instruction \texttt{MODEL}. Un
modèle d'ANOVA à un facteur s'écrira donc \texttt{y=g} où \texttt{y} désigne
la variable réponse et \texttt{g} le facteur d'étude. 
\begin{verbatim}
PROC GLM DATA=polymorphism; 
  CLASS genotype; 
  MODEL age=genotype; 
RUN;
\end{verbatim}
Le tableau de variance retourné par la commande inclut les sommes de carrés
pour le facteur (\texttt{Model}) et la résiduelle (\texttt{Error}), ainsi
que la somme de carrés totale et le coefficient de détermination ($\eta^2$)
reflétant la part de variance expliquée par le modèle. On notera que \SAS
fournit par défaut deux estimations des sommes de carrés pour le facteur
d'étude (type I et III), mais celles-ci sont identiques dans le cas d'un
plan complet équilibré.

La commande \texttt{PROC GLM} fournit également des sorties graphiques,
incluant une représentation de la distribution de l'âge selon le génotype
sous forme de boîtes à moustaches.

\includegraphics{./figs/sas_boxplot}

Il est également possible de représenter la distribution des âges selon le
génotype sous forme de diagramme en barres
\begin{verbatim}
PROC SORT DATA=polymorphism; BY genotype; RUN;

PROC GCHART DATA=polymorphism; Vbar age; BY genotype; RUN;
\end{verbatim}

\includegraphics{./figs/sas_genotype}

Les moyennes de groupes, avec leur intervalles de confiance à 95~\%, peuvent
être obtenues grâce à \texttt{PROC SUMMARY}. La variable de conditionnement
est spécifiée après l'instruction \texttt{BY}.
\begin{verbatim}
PROC SORT DATA=polymorphism; BY genotype; RUN;

PROC SUMMARY DATA=polymorphism PRINT n mean stddev ucl lcl; VAR age; BY genotype; RUN;
\end{verbatim}

En ce qui concerne les comparaisons multiples, on utilise toujours
\texttt{PROC GLM}, en ajoutant cette fois l'option 
\verb|MEANS genotype / BON CLDIFF| qui fournit les tests corrigés par la
méthode de Bonferroni (également appelé tests de Dunn sous \SAS). 
\begin{verbatim}
PROC GLM DATA=polymorphism; 
  CLASS genotype; 
  MODEL age=genotype;
  MEANS genotype / BON CLDIFF;
RUN;
\end{verbatim}
L'option
\texttt{CLDIFF} permet de représenter les résultats portant sur les
différences de moyennes sous forme d'intervalles de confiance, alors que
\texttt{CLM} fait de même pour les moyennes de groupe (intervalles de
confiance simultanés).
\begin{verbatim}
PROC GLM DATA=polymorphism PLOT=MEANPLOT(CLBAND); 
  CLASS genotype; 
  MODEL age=genotype;
  MEANS genotype / BON CLM;
RUN;
\end{verbatim}
%
%
%
\soln{\ref{exo:9.6}}
Pour importer le tableau de données sous \SAS, on procèdera comme dans les
exercices précédents : création de deux tableaux de données dans des étapes
\texttt{DATA} et association entre les deux à l'aide de la commande
\texttt{MERGE}.
\begin{verbatim}
DATA bio1;
INPUT pb @@;
CARDS;
19.8       15.9        15.4
20.5       19.7        17.1
23.7       20.8        18.2
27.1       21.7        18.5
29.6       22.5        19.3
29.9       24.0        21.2
;
RUN;

DATA bio2;
INPUT groupe $ @@;
CARDS;
A           B           C
A           B           C
A           B           C
A           B           C
A           B           C
A           B           C
;
RUN;

DATA biologie; MERGE bio1 bio2; RUN;
\end{verbatim}

À la place de \texttt{PROC GLM}, on peut utiliser la commande \texttt{ANOVA}
pour réaliser une ANOVA à un facteur, la syntaxe générale étant similaire
dans les deux cas (utilisation des instructions \texttt{CLASS} et
\texttt{MODEL}). 
\begin{verbatim}
PROC ANOVA DATA=biologie; 
  CLASS groupe; 
  MODEL pb=groupe;
RUN;
\end{verbatim}

Pour comparer les paires de moyennes entre elles, on rajoutera une
instructions \texttt{MEANS}, en spécifiant l'option \verb|/ BON| pour
travailler avec une correction de type Bonferroni.
\begin{verbatim}
PROC ANOVA DATA=biologie; 
  CLASS groupe; 
  MODEL pb=groupe;
  MEANS groupe / BON;
RUN;
\end{verbatim}

Enfin, si l'on remet en cause la normalité des distributions parentes, la
comparaison des trois groupes peut être effectuée par une ANOVA sur les
rangs (Kruskal-Wallis) à partir de la commande \texttt{PROC NPAR1WAY} dans
laquelle on indique le facteur de classification (\texttt{CLASS}) et la
variable réponse (\texttt{VAR}).
\begin{verbatim}
PROC NPAR1WAY DATA=biologie; 
  CLASS groupe; 
  VAR pb; 
RUN;
\end{verbatim}
%
%
%
\soln{\ref{exo:9.7}}
Le fichier contenant les données, \texttt{weights.sav}, a été exporté depuis
R au format \Stata à l'aide des commandes suivantes :
\begin{verbatim}
library(foreign)
weights <- read.spss("weights.sav", to.data.frame=TRUE)
write.dta(weights, file="weights.dta")
\end{verbatim}
On peut donc l'importer avec \texttt{PROC IMPORT}, en renseignant le type de
source de données (\texttt{DBMS=STATA}), comme on l'a fait pour l'exercice~9.5.
\begin{verbatim}
PROC IMPORT OUT= WORK.weight
            DATAFILE= "C:\data\weights.dta"
            DBMS=STATA REPLACE;

RUN;
\end{verbatim}

Le tableau d'effectifs et de fréquences relatives pour la variable
\texttt{parity} s'obtient à partir de la commande \texttt{PROC FREQ} en
spécifiant l'instruction \texttt{TABLES} pour la variable d'intérêt :
\begin{verbatim}
PROC CONTENTS DATA=weight; RUN;

PROC FREQ DATA=weight; TABLES parity; RUN;
\end{verbatim}

Les moyennes et écarts-type du poids selon la taille de la fratrie
s'obtiennent ainsi :
\begin{verbatim}
PROC SORT DATA=weight; BY parity; RUN;

PROC SUMMARY DATA=weight PRINT n mean stddev min max; VAR weight; BY parity; RUN;
\end{verbatim}

L'ANOVA à un facteur se réalise comme dans les exercices précédents, à
l'aide par exemple de la commande \texttt{PROC GLM} et en fournissant la
variable réponse et la variable qualitative décrivant les groupes à
comparer.
\begin{verbatim}
PROC GLM DATA=weight; 
  CLASS parity; 
  MODEL weight=parity; 
RUN;
\end{verbatim}

Pour tester l'homogénéité des variance, il est nécessaire de rajouter
l'option \verb|/ HOVTEST| dans une instruction \texttt{MEANS}, comme indiqué
ci-dessous :
\begin{verbatim}
PROC GLM DATA=weight; 
  CLASS parity; 
  MODEL weight=parity;
  MEANS parity / HOVTEST;
RUN;
\end{verbatim}
Par défaut, le test utilisé est un test de Levene. Pour obtenir un test de
Bartlett, il suffit de modifier l'option ainsi : \verb|HOVTEST=BARTLETT|.

Pour le recodage de la variable \texttt{parity} en 3 classes, voici une
solution possible sous \SAS :
\begin{verbatim}
DATA weight1; SET weight; Newparity=parity+1-1; RUN;

PROC FREQ DATA=weight1; TABLES newparity; RUN;

DATA weight2; SET weight1;
  Nparity=newparity; 
  IF newparity GE 3 THEN Nparity=3;
RUN;

PROC FREQ DATA=weight2; TABLES Nparity; RUN;
\end{verbatim}

Le modèle d'analyse de variance peut être à nouveau estimé sur ces nouvelles
données (\texttt{weight2}) avec \texttt{PROC GLM}.
\begin{verbatim}
PROC GLM DATA=weight2; 
  CLASS Nparity; 
  MODEL weight=Nparity;
  MEANS Nparity / HOVTEST;
RUN;
\end{verbatim}

Le test de tendance linéaire peut être réalisé à partir d'une régression
linéaire simple. On peut utiliser \texttt{PROC GLM}, comme dans le cas de
l'ANOVA, mais en omettant l'instruction \texttt{CLASS} qui indique à \SAS de
traiter la variable explicative comme une variable qualitative.
\begin{verbatim}
PROC GLM DATA=weight2;  
  MODEL weight=Nparity;
RUN;
\end{verbatim}

\end{document}
