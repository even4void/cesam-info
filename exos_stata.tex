\chapter{Élements du langage et statistiques descriptives}

\begin{exo}\label{exo:8.1}
Reprendre l'énoncé 1.1 (p.~\pageref{exo:1.1}) et répondre aux questions a–c.
\end{exo}
\vskip1em

\begin{exo}\label{exo:8.2}
Reprendre l'énoncé 1.2 (p.~\pageref{exo:1.2}) et répondre aux questions a et
b.
\end{exo}
\vskip1em

\begin{exo}\label{exo:8.3}
Reprendre l'énoncé 1.5 (p.~\pageref{exo:1.5}) et répondre aux questions a–d.
\end{exo}
\vskip1em

\begin{exo}\label{exo:8.4}
Reprendre l'énoncé 2.1 (p.~\pageref{exo:2.1}) et répondre aux questions a–d.
\end{exo}
\vskip1em

\begin{exo}\label{exo:8.5}
Reprendre l'énoncé 2.3 (p.~\pageref{exo:2.3}) et répondre aux questions a–c.
\end{exo}
\vskip1em

\begin{exo}\label{exo:8.6}
Reprendre l'énoncé 2.4 (p.~\pageref{exo:2.4}) et répondre aux questions a–f.
\end{exo}

%--------------------------------------------------------------- Devoir 07 ---
\chapter*{Devoir \no 7}
\addcontentsline{toc}{chapter}{Devoir \no 7}

Les exercices sont indépendants. Une seule réponse est correcte pour chaque
question. Lorsque vous ne savez pas répondre, cochez la case correspondante.

\section*{Exercice 1}
Considérons le petit jeu de données affichés ci-dessous comme une sortie
\Stata (produit avec la commande \verb|list|):
\begin{verbatim}
     +----------------+
     |   x      y   z |
     |----------------|
  1. | 1.2    8.1   1 |
  2. | 2.4   12.4   2 |
  3. | 3.1    6.7   1 |
  4. | 1.8    9.8   2 |
  5. | 5.7   10.3   1 |
     |----------------|
  6. | 6.4   10.8   2 |
  7. | 3.8    9.2   1 |
  8. | 2.6      .   2 |
  9. | 5.7   11.2   1 |
 10. | 3.8   12.7   2 |
     +----------------+
\end{verbatim}
\begin{description}
\item[\bf 1.1] On souhaite remplacer la 6\ieme\ observation de la variable
  \texttt{x} (6.4) par la valeur 5.9. Quelle commande faut-il utiliser ?
  \marginpar{1.1 $\square$} 
  \begin{description}
  \item[A.] \verb|replace x[6] = 5.9|
  \item[B.] \verb|replace x = 5.9 in 6|
  \item[C.] \verb|generate x = 5.9 in 6|
  \item[D.] \verb|generate x = 5.9 in 6, replace|
  \item[E.] Je ne sais pas.
  \end{description}  
\item[\bf 1.2] La commande
\begin{verbatim}
. tab z if y > 10
\end{verbatim}
permet de renvoyer le nombre d'observations pour lequel $y > 10$ pour chaque
modalité de \texttt{z}. \marginpar{1.2 $\square$}
  \begin{description}
  \item[A.] Vrai.
  \item[B.] Faux.
  \item[C.] Je ne sais pas.
  \end{description}  
\item[\bf 1.3] En supposant que l'on dispose de la valeur moyenne des $y$
  dans une variable \texttt{ym} définie comme suit :
  \verb|egen ym = mean(y)|, quelle commande doit-on utiliser pour remplacer
  la valeur manquante présente dans la variable \texttt{y} par la moyenne
  des $y$ ?  \marginpar{1.3 $\square$}
  \begin{description}
  \item[A.] \verb|replace y = ym, if missing(y)|
  \item[B.] \verb|replace y = ym if missing(y)|
  \item[C.] \verb|generate y = ym, if missing(y)|
  \item[D.] \verb|generate y = ym if missing(y)|
  \item[E.] Je ne sais pas.
  \end{description}  
\item[\bf 1.4] Quelle commande a permis de produire le résultat suivant ?
  \marginpar{1.4 $\square$} 
\begin{verbatim}
    Variable |       Obs        Mean    Std. Dev.       Min        Max
-------------+--------------------------------------------------------
           x |        10        3.65    1.776545        1.2        6.4
\end{verbatim}
  \begin{description}
  \item[A.] \verb|describe|
  \item[B.] \verb|summarise|
  \item[C.] \verb|summarize| 
  \item[D.] Je ne sais pas.
  \end{description}
\item[\bf 1.5] La variable \texttt{z} comporte deux valeurs uniques (1 et 2)
  correspondant à deux catégories d'individus : 1 = Homme, 2 = Femme. On
  souhaite associer ces étiquettes, que l'on supposera définies correctement
  dans la variable \texttt{genderlab}, aux valeurs prises par \texttt{z}. La
  commande suivante permet-elle de répondre à la question ? 
  \marginpar{1.5 $\square$}
\begin{verbatim}
. label value genderlab z
\end{verbatim}
  \begin{description}
  \item[A.] Oui.
  \item[B.] Non.
  \item[C.] Je ne sais pas.
  \end{description}  
\item[\bf 1.6] Que permet de réaliser la commande suivante : \marginpar{1.6 $\square$}
\begin{verbatim}
. by z, sort: count if y < 11 & x > 3
\end{verbatim}
  \begin{description}
  \item[A.] Compter le nombre d'observations de la variable pour lesquels $y
    < 11$ et $x > 3$ par modalité de \texttt{z}.
  \item[B.] Trier la variable \texttt{z} par ordre croissant et dénombrer
    les observations pour lesquelles $y < 11$ ou $x > 3$.
  \item[C.] Je ne sais pas.
  \end{description}  

\end{description}
\section*{Exercice 2}

\section*{Exercice 3}



\chapter[Mesures d'association, comparaison de moyennes et de
proportions]{Mesures d'association, comparaison de moyennes et de
  proportions pour deux échantillons ou plus}   

\begin{exo}\label{exo:9.1}
Reprendre l'énoncé 3.1 (p.~\pageref{exo:3.1}) et répondre aux questions a–c.
\end{exo}
\vskip1em

\begin{exo}\label{exo:9.2}
Reprendre l'énoncé 3.2 (p.~\pageref{exo:3.2}) et répondre aux questions a–c.
\end{exo}
\vskip1em

\begin{exo}\label{exo:9.3}
Reprendre l'énoncé 3.4 (p.~\pageref{exo:3.4}) et répondre aux questions a–d.
\end{exo}
\vskip1em

\begin{exo}\label{exo:9.4}
Reprendre l'énoncé 3.5 (p.~\pageref{exo:3.5}) et répondre aux questions a–d.
\end{exo}
\vskip1em

\begin{exo}\label{exo:9.5}
Reprendre l'énoncé 4.1 (p.~\pageref{exo:4.1}) et répondre aux questions a–d.
\end{exo}
\vskip1em

\begin{exo}\label{exo:9.6}
Reprendre l'énoncé 4.2 (p.~\pageref{exo:4.2}) et répondre aux questions a–c.
\end{exo}
\vskip1em

\begin{exo}\label{exo:9.7}
Reprendre l'énoncé 4.3 (p.~\pageref{exo:4.3}) et répondre aux questions a–e.
\end{exo}

%--------------------------------------------------------------- Devoir 08 ---
\chapter*{Devoir \no 8}
\addcontentsline{toc}{chapter}{Devoir \no 8}

Les exercices sont indépendants. Une seule réponse est correcte pour chaque
question. Lorsque vous ne savez pas répondre, cochez la case correspondante.

\section*{Exercice 1}

\section*{Exercice 2}

\section*{Exercice 3}
Un investigateur s'intéressant à la fonction respiratoire décide
d'enregistrer le volume expiratoire maximum seconde (mesuré en litres) chez
des sujets fumeurs et non fumeurs. Quatre catégories sont définies \emph{a
  priori} : les non fumeurs, les anciens fumeurs, les nouveaux fumeurs et
les fumeurs de longue date. Un ensemble de 6 personnes est tiré au sort dans
chaque catégorie \citep[p.~34]{mickey04}. Les données présentées dans le
tableau suivant sont disponibles dans le fichier \texttt{vems.dta}.
\vskip1em

\begin{tabular}{lcccccccc}
\toprule
Observation & 1 & 2 & 3 & 4 & 5 & 6 & Moy & Var \\
\midrule
Non fumeurs      & 4.41 & 4.96 & 3.50 & 3.66 & 4.68 & 4.11 & 4.22 & 0.33 \\
Anciens fumeurs  & 3.69 & 3.90 & 3.82 & 4.08 & 3.76 & 4.38 & 3.94 & 0.06 \\
Nouveaux fumeurs & 3.54 & 4.40 & 3.28 & 2.28 & 3.34 & 3.92 & 3.46 & 0.51 \\
Fumeurs          & 2.98 & 2.95 & 2.15 & 3.41 & 3.97 & 3.86 & 3.22 & 0.46 \\
\midrule
Ensemble         & & & & & & & 3.71 & 0.34 \\
\bottomrule
\end{tabular}
\vskip1em

Un aperçu des données après importation sous \Stata est fourni ci-dessous
\begin{verbatim}
. list in 1/5

     +--------------------+
     | VEMS     categorie |
     |--------------------|
  1. | 4.41   Non fumeurs |
  2. | 4.96   Non fumeurs |
  3. |  3.5   Non fumeurs |
  4. | 3.66   Non fumeurs |
  5. | 4.68   Non fumeurs |
     +--------------------+
. codebook categorie, compact

Variable   Obs Unique  Mean  Min  Max  Label
-------------------------------------------------------------------------------
categorie   24      4   2.5    1    4  Statut fumeur
-------------------------------------------------------------------------------
\end{verbatim}
\begin{description}
\item[\bf 3.1] Quelle commande doit-on utiliser pour calculer la moyenne et
  la variance dans chaque groupe ? \marginpar{3.1 $\square$}
  \begin{description}
  \item[A.] \verb|summarize VEMS, by(categorie)|
  \item[B.] \verb|by categorie, summarize VEMS|
  \item[C.] \verb|tabstat VEMS, by(categorie) stats(mean sd)|
  \item[D.] \verb|by categorie: tabstat VEMS, mean sd|
  \item[E.] Je ne sais pas.
  \end{description}  
\item[\bf 3.2] Quelle commande permet de reproduire la figure suivante
  (indépendemment du rapport largeur/hauteur et de la couleur) ?
  \marginpar{3.2 $\square$}
\begin{center}
  \includegraphics{./figs/dev8_vemshisto}
\end{center}
\begin{description}
\item[A.] \verb|histogram VEMS|
\item[B.] \verb|histogram VEMS, freq|
\item[C.] \verb|histogram VEMS, discrete|
\item[D.] \verb|histogram VEMS, over(categorie)|
\item[E.] Je ne sais pas.
\end{description}
\item[\bf 3.3] Quelle commande permet de reproduire la figure suivante
  (indépendemment du rapport largeur/hauteur et de la couleur) ?
  \marginpar{3.3 $\square$}
\begin{center}
  \includegraphics{./figs/dev8_vemsdot}
\end{center}
\begin{description}
\item[A.] \verb|dotplot VEMS, over(categorie)|
\item[B.] \verb|dotplot VEMS, by(categorie)|
\item[C.] \verb|stripplot VEMS, over(categorie)|
\item[D.] \verb|stripplot VEMS, by(categorie)|
\item[E.] Je ne sais pas.
\end{description}
\item[\bf 3.4] On souhaite tester le rapport des deux variances les plus
  extrêmes (anciens et nouveaux fumeurs). Quelle commande permet de répondre
  à cette question ? \marginpar{3.4 $\square$}
\begin{description}
\item[A.] \verb+sdtest VEMS if categorie in 2 | categorie in 3+
\item[B.] \verb+sdtest VEMS if categorie in 2 & categorie in 3+
\item[C.] \verb+sdtest VEMS if categorie == 2 & categorie == 3, by(categorie)+
\item[D.] \verb+sdtest VEMS if categorie == 2 | categorie == 3, by(categorie)+
\item[E.] Je ne sais pas.
\end{description}
\item[\bf 3.5] Quelle commande doit-on utiliser pour réaliser une ANOVA à un
  facteur de classification ? \marginpar{3.5 $\square$}
\begin{description}
\item[A.] \verb|oneway VEMS by categorie|
\item[B.] \verb|oneway VEMS, over(categorie)|
\item[C.] \verb|oneway VEMS categorie|
\item[D.] Je ne sais pas.
\end{description}
\item[\bf 3.6] On souhaite obtenir les intervalles de confiance à 95~\% pour
  les quatre moyennes de groupe, en utilisant la loi de Student comme
  distribution asymptotique. Quelle commande doit-on utiliser ?
  \marginpar{3.6 $\square$}

\item[\bf 3.7] 
\end{description}

%--------------------------------------------------------------- Chapter 10 ---
\chapter{Régression linéaire et logistique}

\begin{exo}\label{exo:10.1}
Reprendre l'énoncé 5.1 (p.~\pageref{exo:5.1}) et répondre aux questions a–e.
\end{exo}
\vskip1em

\begin{exo}\label{exo:10.2}
Reprendre l'énoncé 5.2 (p.~\pageref{exo:5.2}) et répondre aux questions a–e.
\end{exo}
\vskip1em

\begin{exo}\label{exo:10.3}
Reprendre l'énoncé 5.3 (p.~\pageref{exo:5.3}) et répondre aux questions a–d.
\end{exo}
\vskip1em

\begin{exo}\label{exo:10.4}
Reprendre l'énoncé 6.1 (p.~\pageref{exo:6.1}) et répondre aux questions a–d.
\end{exo}
\vskip1em

\begin{exo}\label{exo:10.5}
Reprendre l'énoncé 6.3 (p.~\pageref{exo:6.3}) et répondre aux questions a–f.
\end{exo}
\vskip1em

\begin{exo}\label{exo:10.6}
Reprendre l'énoncé 6.5 (p.~\pageref{exo:6.5}) et répondre aux questions a–e.
\end{exo}

%--------------------------------------------------------------- Devoir 09 ---
\chapter*{Devoir \no 9}
\addcontentsline{toc}{chapter}{Devoir \no 9}

Les exercices sont indépendants. Une seule réponse est correcte pour chaque
question. Lorsque vous ne savez pas répondre, cochez la case correspondante.

\section*{Exercice 1}

\section*{Exercice 2}

\section*{Exercice 3}

%--------------------------------------------------------------- Chapter 11 ---
\chapter{Analyse de données de survie}

\begin{exo}\label{exo:11.1}
Reprendre l'énoncé 7.1 (p.~\pageref{exo:7.1}) et répondre aux questions a–d.
\end{exo}
\vskip1em

\begin{exo}\label{exo:11.2}
Reprendre l'énoncé 7.3 (p.~\pageref{exo:7.3}) et répondre aux questions a–i.
\end{exo}

%--------------------------------------------------------------- Devoir 10 ---
\chapter*{Devoir \no 10}
\addcontentsline{toc}{chapter}{Devoir \no 10}

Les exercices sont indépendants. Une seule réponse est correcte pour chaque
question. Lorsque vous ne savez pas répondre, cochez la case correspondante.

\section*{Exercice 1}

\section*{Exercice 2}

\section*{Exercice 3}
