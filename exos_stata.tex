\chapter{Élements du langage et statistiques descriptives}

\begin{exo}\label{exo:8.1}
{\footnotesize Identique à l'énoncé 1.1 (p.~\pageref{exo:1.1}), questions
  a–c.}

Un chercheur a recueilli les mesures biologiques suivantes (unités
arbitraires) :
\begin{verbatim}
3.68  2.21  2.45  8.64  4.32  3.43  5.11  3.87
\end{verbatim}
\begin{description}
\item[(a)] Stocker la séquence de mesures dans une variable appelée
  \texttt{x}.  
\item[(b)] Indiquer le nombre d'observations (à l'aide de \R), les valeurs
  minimale et maximale, ainsi que l'étendue.  
\item[(c)] En fait, le chercheur réalise que la valeur 8.64 correspond à une
  erreur de saisie et doit être changée en 3.64. De même, il a un doute sur
  la 7\ieme mesure et décide de la considérer comme une valeur manquante :
  effectuer les transformations correspondantes. 
\end{description}
\end{exo}
\vskip1em

\begin{exo}\label{exo:8.2}
{\footnotesize Identique à l'énoncé 1.2 (p.~\pageref{exo:1.2}), questions a
  et b.}

La charge virale plasmatique permet de décrire la quantité de virus (p.~ex.,
VIH) dans un échantillon de sang. Ce marqueur virologique qui permet de
suivre la progression de l’infection et de mesurer l’efficacité des
traitements est rapporté en nombre de copies par millilitre, et la plupart
des instruments de mesure ont un seuil de détectabilité de 50
copies/ml. Voici une série de mesures, $X$, exprimées en logarithmes (base 10)
collectées sur 20 patients :
\begin{verbatim}
3.64 2.27 1.43 1.77 4.62 3.04 1.01 2.14 3.02 5.62 5.51 5.51 1.01 1.05 4.19
2.63 4.34 4.85 4.02 5.92
\end{verbatim}
Pour rappel, une charge virale de 100 000 copies/ml équivaut à 5 log.
\begin{description}
\item[(a)] Indiquer combien de patients ont une charge virale considérée
  comme non-détectable. 
\item[(b)] Quelle est le niveau de charge virale médian, en copies/ml, pour
  les données considérées comme valides ?
\end{description}
\end{exo}
\vskip1em

\begin{exo}\label{exo:8.3}
{\footnotesize Identique à l'énoncé 1.5 (p.~\pageref{exo:1.5}), questions
  a–d.}  

Le fichier \texttt{anorexia.dat} contient les données d'une étude clinique
chez des patientes anorexiques ayant reçu l'une des trois thérapies
suivantes : thérapie comportementale, thérapie familiale, thérapie
contrôle.\autocite{hand93} 
\begin{description}
\item[(a)] Combien y'a-t-il de patientes au total ? Combien y'a-t-il de
  patientes par groupe de traitement ?
\item[(b)] Les mesures de poids sont en livres. Les convertir en
  kilogrammes.    
\item[(c)] Créer une nouvelle variable contenant les scores de différences
  (\texttt{After} - \texttt{Before}).
\item[(d)] Indiquer la moyenne et l'étendue (min/max) des scores de
  différences par groupe de traitement.
\end{description}
\end{exo}
\vskip1em

\begin{exo}\label{exo:8.4}
{\footnotesize Identique à l'énoncé 2.1 (p.~\pageref{exo:2.1}), questions
  a–d.} 

Une variable quantitative $X$ prend les valeurs suivantes sur un échantillon
de 26 sujets :
\begin{verbatim}
24.9,25.0,25.0,25.1,25.2,25.2,25.3,25.3,25.3,25.4,25.4,25.4,25.4,
25.5,25.5,25.5,25.5,25.6,25.6,25.6,25.7,25.7,25.8,25.8,25.9,26.0
\end{verbatim}
\begin{description}
\item[(a)] Calculer la moyenne, la médiane ainsi que le mode de $X$. 
\item[(b)] Quelle est la valeur de la variance estimée à partir de ces données ? 
\item[(c)] En supposant que les données sont regroupées en 4 classes dont les
  bornes sont : 24.9–25.1, 25.2–25.4, 25.5–25.7, 25.8–26.0, afficher la
  distribution des effectifs par classe sous forme d'un tableau d'effectifs. 
\item[(d)] Représenter la distribution de $X$ sous forme d'histogramme, sans
  considération d'intervalles de classe \emph{a priori}.
\end{description}
\end{exo}
\vskip1em

\begin{exo}\label{exo:8.5}
{\footnotesize Identique à l'énoncé 2.3 (p.~\pageref{exo:2.3}), questions
  a–c.}

Le fichier \texttt{elderly.dat} contient la taille mesurée en cm de 351
personnes âgées de sexe féminin, sélectionnées aléatoirement dans la
population lors d'une étude sur l'ostéoporose. Quelques observations sont
cependant manquantes.
\begin{description}
\item[(a)] Combien y'a t-il d'observations manquantes au total ?
\item[(b)] Donner un intervalle de confiance à 95~\% pour la taille moyenne
  dans cet échantillon, en utilisant une approximation normale.
\item[(c)] Représenter la distribution des tailles observées sous forme
  d'une courbe de densité.  
\end{description}
\end{exo}
\vskip1em

\begin{exo}\label{exo:8.6}
{\footnotesize Identique à l'énoncé 2.4 (p.~\pageref{exo:2.4}), questions
  a–f.} 

Le fichier \texttt{birthwt} est un des jeux de données fournis avec \R. Il
comprend les résultats d'une étude prospective visant à identifier les
facteurs de risque associés à la naissance de bébés dont le poids est
inférieur à la norme (2,5 kg). Les données proviennent de 189 femmes, dont
59 ont accouché d'un enfant en sous-poids. Parmi les variables d'intérêt
figurent l'âge de la mère, le poids de la mère lors des dernières
menstruations, l'ethnicité de la mère et le nombre de visites médicales
durant le premier trimestre de grossesse.\autocite{hosmer89}
Les variables disponibles sont décrites comme suit : \texttt{low} (= 1 si
poids $<2.5$ kg, 0 sinon), \texttt{age} (années), \texttt{lwt} (poids de la
mère en livres), \texttt{race} (ethnicité codée en trois classes, 1 = white,
2 = black, 3 = other), \texttt{smoke} (= 1 si consommation de tabac durant
la grossesse, 0 sinon), \texttt{ptl} (nombre d'accouchements pré-terme
antérieurs), \texttt{ht} (= 1 si antécédent d'hypertension, 0 sinon),
\texttt{ui} (= 1 si manifestation d'irritabilité utérine, 0 sinon),
\texttt{ftv} (nombre de consultations chez le gynécologue durant le premier
trimestre de grossesse), \texttt{bwt} (poids des bébés à la naissance, en
\emph{g}).
\begin{description}
\item[(a)] Recoder les variables \texttt{low}, \texttt{race},
  \texttt{smoke}, \texttt{ui} et \texttt{ht} en variables
  qualitatives, avec des étiquettes ("labels") plus informatives.
\item[(b)] Convertir le poids des mères en \emph{kg}. Indiquer la moyenne, la
  médiane et l'intervalle inter-quartile. Représenter la distribution des
  poids sous forme d'histogramme.
\item[(c)] Indiquer la proportion de mères consommant du tabac durant la
  grossesse, avec un intervalle de confiance à 95~\%. Représenter les
  proportions (en \%) fumeur/non-fumeur sous forme d'un diagramme en
  barres.
\item[(d)] Recoder l'âge des mères en trois classes équilibrées (tercilage)
  et indiquer la proportion d'enfants dont le poids est $<2500$ \emph{g}
  pour chacune des trois classes.
\item[(e)] Construire un tableau d'effectifs ($n$ et \%) pour la variable
  ethnicité (\texttt{race}).  
\item[(f)] Décrire la distribution des variables \texttt{race},
  \texttt{smoke}, \texttt{ui}, \texttt{ht} et \texttt{age} après
  stratification sur la variable \texttt{low}.  
\end{description}
\end{exo}

%--------------------------------------------------------------- Devoir 07 ---
\chapter*{Devoir \no 7}
\addcontentsline{toc}{chapter}{Devoir \no 7}

Les exercices sont indépendants. Une seule réponse est correcte pour chaque
question. Lorsque vous ne savez pas répondre, cochez la case correspondante.

\section*{Exercice 1}
Considérons le petit jeu de données affichés ci-dessous comme une sortie
\Stata (produit avec la commande \verb|list|):
\begin{verbatim}
     +----------------+
     |   x      y   z |
     |----------------|
  1. | 1.2    8.1   1 |
  2. | 2.4   12.4   2 |
  3. | 3.1    6.7   1 |
  4. | 1.8    9.8   2 |
  5. | 5.7   10.3   1 |
     |----------------|
  6. | 6.4   10.8   2 |
  7. | 3.8    9.2   1 |
  8. | 2.6      .   2 |
  9. | 5.7   11.2   1 |
 10. | 3.8   12.7   2 |
     +----------------+
\end{verbatim}
\begin{description}
\item[\bf 1.1] On souhaite remplacer la 6\ieme\ observation de la variable
  \texttt{x} (6.4) par la valeur 5.9. Quelle commande faut-il utiliser ?
  \marginpar{1.1 $\square$} 
  \begin{description}
  \item[A.] \verb|replace x[6] = 5.9|
  \item[B.] \verb|replace x = 5.9 in 6|
  \item[C.] \verb|generate x = 5.9 in 6|
  \item[D.] \verb|generate x = 5.9 in 6, replace|
  \item[E.] Je ne sais pas.
  \end{description}  
\item[\bf 1.2] La commande
\begin{verbatim}
. tab z if y > 10
\end{verbatim}
permet de renvoyer le nombre d'observations pour lequel $y > 10$ pour chaque
modalité de \texttt{z}. \marginpar{1.2 $\square$}
  \begin{description}
  \item[A.] Vrai.
  \item[B.] Faux.
  \item[C.] Je ne sais pas.
  \end{description}  
\item[\bf 1.3] En supposant que l'on dispose de la valeur moyenne des $y$
  dans une variable \texttt{ym} définie comme suit :
  \verb|egen ym = mean(y)|, quelle commande doit-on utiliser pour remplacer
  la valeur manquante présente dans la variable \texttt{y} par la moyenne
  des $y$ ?  \marginpar{1.3 $\square$}
  \begin{description}
  \item[A.] \verb|replace y = ym, if missing(y)|
  \item[B.] \verb|replace y = ym if missing(y)|
  \item[C.] \verb|generate y = ym, if missing(y)|
  \item[D.] \verb|generate y = ym if missing(y)|
  \item[E.] Je ne sais pas.
  \end{description}  
\item[\bf 1.4] Quelle commande a permis de produire le résultat suivant ?
  \marginpar{1.4 $\square$} 
\begin{verbatim}
    Variable |       Obs        Mean    Std. Dev.       Min        Max
-------------+--------------------------------------------------------
           x |        10        3.65    1.776545        1.2        6.4
\end{verbatim}
  \begin{description}
  \item[A.] \verb|describe|
  \item[B.] \verb|summarise|
  \item[C.] \verb|summarize| 
  \item[D.] Je ne sais pas.
  \end{description}
\item[\bf 1.5] La variable \texttt{z} comporte deux valeurs uniques (1 et 2)
  correspondant à deux catégories d'individus : 1 = Homme, 2 = Femme. On
  souhaite associer ces étiquettes, que l'on supposera définies correctement
  dans la variable \texttt{genderlab}, aux valeurs prises par \texttt{z}. La
  commande suivante permet-elle de répondre à la question ? 
  \marginpar{1.5 $\square$}
\begin{verbatim}
. label value genderlab z
\end{verbatim}
  \begin{description}
  \item[A.] Oui.
  \item[B.] Non.
  \item[C.] Je ne sais pas.
  \end{description}  
\item[\bf 1.6] Que permet de réaliser la commande suivante : \marginpar{1.6 $\square$}
\begin{verbatim}
. by z, sort: count if y < 11 & x > 3
\end{verbatim}
  \begin{description}
  \item[A.] Compter le nombre d'observations de la variable pour lesquels $y
    < 11$ et $x > 3$ par modalité de \texttt{z}.
  \item[B.] Trier la variable \texttt{z} par ordre croissant et dénombrer
    les observations pour lesquelles $y < 11$ ou $x > 3$.
  \item[C.] Je ne sais pas.
  \end{description}
\item[\bf 1.7] Quelle commande permet de calculer la médiane et l'intervalle
  inter-quartile de \texttt{x} ?\marginpar{1.7 $\square$}
  \begin{description}
  \item[A.] \verb|summarize x|
  \item[B.] \verb|summarize x, all|
  \item[C.] \verb|tabstat x, stats(median qrange)|
  \item[D.] \verb|tabstat x, stats(median iqr)|
  \item[E.] Je ne sais pas.
  \end{description}  

\end{description}
\section*{Exercice 2}
Considérons les données traitées à l'exercice~\ref{exo:8.3} (décrites
p.~\pageref{exo:1.5}), sans aucune transformation sur les valeurs
numériques. 
\begin{verbatim}
. list in 1/5

     +------------------------+
     | Group   Before   After |
     |------------------------|
  1. |    g1     80.5    82.2 |
  2. |    g1     84.9    85.6 |
  3. |    g1     81.5    81.4 |
  4. |    g1     82.6    81.9 |
  5. |    g1     79.9    76.4 |
     +------------------------+
\end{verbatim}
\begin{description}
\item[\bf 2.1] Supposons que des étiquettes aient été attribuées à chacun
  des trois groupes : \texttt{th. cpt} (initialement \texttt{g1}),
  \texttt{th. fam.} (\texttt{g2}), \texttt{ctl} (\texttt{g3}). On souhaite
  identifier les patientes du groupe contrôle ayant un poids initial
  inférieur à 36 kg. Quelle commande doit-on utiliser ?
  \marginpar{2.1 $\square$} 
  \begin{description}
  \item[A.] \verb|list if Group == "ctl" & Before < 36/2.2|
  \item[B.] \verb|list if Group == "ctl" & Before < 36*2.2|
  \item[C.] \verb|list if Group > 2 & Before < 36/2.2|
  \item[D.] \verb|list if Group == 3 & Before < 36*2.2|
  \item[E.] Je ne sais pas.
  \end{description}  
\item[\bf 2.2] Que permet de renvoyer la série de commandes suivante ?
  \marginpar{2.2 $\square$}
\begin{verbatim}
. egen de = max(After - Before) if Group == "g1"
. display de
\end{verbatim}
  \begin{description}
  \item[A.] La différence entre les poids les plus élevés en début et fin
    d'étude chez les patientes du groupe \texttt{g1}.    
  \item[B.] La différence maximale retrouvée entre les poids de début et fin
    d'étude chez les patientes du groupe \texttt{g1}.
  \item[C.] Les numéros d'identification des patientes triés par différence
    de poids entre le début et la fin de l'étude.
  \item[D.] Je ne sais pas.
  \end{description}  
\item[\bf 2.3] On souhaite recoder les poids des patientes en début et fin
  d'étude en 5 classes équilibrées. Quelle commande doit-on utiliser ?
  \marginpar{2.3 $\square$}
  \begin{description}
  \item[A.] \verb|xtile|
  \item[B.] \verb|pctile|
  \item[C.] \verb|cut|
  \item[D.] Je ne sais pas.
  \end{description}  
\item[\bf 2.4] Supposons que les poids des patients en début d'étude
  (variable \texttt{Before}) aient été recodées en 5 classes équilibrées
  comme dans l'exercice précédent. On souhaite à présent recode la variable
  \texttt{After} en utilisant les mêmes intervalles de classe. Quelle
  commande doit-on utiliser ?
  \marginpar{2.4 $\square$}
  \begin{description}
  \item[A.] \verb|xtile|
  \item[B.] \verb|pctile|
  \item[C.] \verb|egen cut|
  \item[D.] Je ne sais pas.
  \end{description}
\item[\bf 2.5] Pour afficher la distribution des effectifs par classe pour
  la variable \texttt{Before} recodée en 5 classes de l'exercice~2.3, que
  l'on appelle \texttt{BeforeCat}, on utilisera la commande : \marginpar{2.5 $\square$}
  \begin{description}
  \item[A.] \verb|histogram BeforeCat, frequency|
  \item[B.] \verb|histogram BeforeCat, frequency discrete|
  \item[C.] \verb|graph bar BeforeCat|
  \item[D.] Je ne sais pas.
  \end{description}
\item[\bf 2.6] Quelle commande permet de reproduire la figure suivante
  (indépendemment du rapport hauteur/largeur et de la couleur) ? \marginpar{2.6 $\square$}
  \begin{center}
    \includegraphics{./figs/dev7_histo}
  \end{center}
  \begin{description}
  \item[A.] \verb|histogram After, frequency addlabels title("Groupe g1")|
  \item[B.] \verb|histogram After, frequency addlabels mtitle("Groupe g1")|
  \item[C.] \verb|histogram After if Group == "g1", frequency addlabels title("Groupe g1")|
  \item[D.] \verb|histogram After if Group == "g1", frequency addlabels mtitle("Groupe g1")|
  \item[C.] Je ne sais pas.
  \end{description}
\item[\bf 2.7] Quelle commande a permis de produire le résultat suivant ? \marginpar{2.7 $\square$}
\begin{verbatim}
--------------------------------------
Type de   |
therapie  | mean(Before)   mean(After)
----------+---------------------------
       g1 |    82.689655     85.696552
       g2 |    81.557692     81.107692
       g3 |    83.229412     90.494118
--------------------------------------
\end{verbatim}
  \begin{description}
  \item[A.] \verb|by Groupe: summarize Before After, meanonly|
  \item[B.] \verb|by Groupe, sort: summarize Before After, meanonly|
  \item[C.] \verb|table Group, contents(mean Before mean After)|
  \item[D.] Je ne sais pas.
  \end{description}
\end{description}

\section*{Exercice 3}\label{dev7:exo3}
On dispose d'un fichier de données, \texttt{platelet.txt}, dont un aperçu
des 7 premières lignes est fourni ci-après.
\begin{verbatim}
1	1	359000	396000	B
1	2	200000	184000	A
1	3	149000	151000	A
1	4	235000	242000	B
1	5	174000	177000	B
1	6	271000	203000	A
2	7	180000	199000	B
\end{verbatim}
Ce fichier concerne les données d'un essai clinique multicentrique de
pharmacovigilance portant sur un nouveau produit (A) comparé à un placebo
(B) \citep{chow04}. Le volume total de plaquettes ($\mu$l) a été enregistré
avant et après traitement, pour les deux traitements. Les colonnes du
fichier correspondent dans l'ordre au numéro de centre (\texttt{centre}), au
numéro du patient (\texttt{patient}), au volume de plaquettes avant
(\texttt{pre}) et après (\texttt{post}) traitement, et au type de traitement
(\texttt{trt}).

\begin{description}
\item[\bf 3.1] Quelle commande doit-on utiliser pour importer le fichier de
  données sous \Stata ? \marginpar{3.1 $\square$}
  \begin{description}
  \item[A.] \verb|insheet centre patient pre post trt using "platelet.txt", clear|
  \item[B.] \verb|infile centre patient pre post trt using "platelet.txt", clear|
  \item[C.] \verb|infile centre patient pre post str2(trt) using "platelet.txt", clear|
  \item[D.] Je ne sais pas.
  \end{description}
\item[\bf 3.2] On souhaite calculer le volume de plaquettes médian avant et
  après traitement dans les deux derniers centres. Quelle commande doit-on
  utiliser ? \marginpar{3.2 $\square$}
  \begin{description}
  \item[A.] \verb|tabstat pre post, by(centre) if centre > 3 stat(median) nototal|
  \item[B.] \verb|tabstat pre post if centre > 3, by(centre) stat(median) nototal|
  \item[C.] Je ne sais pas.
  \end{description}
\item[\bf 3.3] Pour obtenir un intervalle de confiance à 95~\% de la moyenne
  post-traitement, on utilise la commande : \marginpar{3.3 $\square$}
  \begin{description}
  \item[A.] \verb|ci post, level(90)|
  \item[B.] \verb|ci post, level(0.90)|
  \item[C.] \verb|summarize post, conflevel(90)|
  \item[D.] \verb|summarize post, conflevel(0.90)|
  \item[E.] Je ne sais pas.
  \end{description}
\item[\bf 3.4] On souhaite représenter la distribution des scores de
  différence (\texttt{post} - \texttt{pre}), stockés dans une variable
  appelé \texttt{dv} sous forme d'un histogramme incluant 7 classes
  distinctes. Quelle commande doit-on utiliser ? \marginpar{3.4 $\square$}
  \begin{description}
  \item[A.] \verb|histogram dv, class(7)|
  \item[B.] \verb|histogram dv, start(7)|
  \item[C.] \verb|histogram dv, bin(7)|
  \item[D.] Je ne sais pas.
  \end{description}
\end{description}

%--------------------------------------------------------------- Chapter 09 --
\chapter[Mesures d'association, comparaison de moyennes et de
proportions]{Mesures d'association, comparaison de moyennes et de
  proportions pour deux échantillons ou plus}   

\begin{exo}\label{exo:9.1}
{\footnotesize Identique à l'énoncé 3.1 (p.~\pageref{exo:3.1}), questions
  a–c.} 

On dispose des poids à la naissance d'un échantillon de 50 enfants
présentant un syndrôme de détresse respiratoire idiopathique aïgue. Ce type
de maladie peut entraîner la mort et on a observé 27 décès chez ces
enfants. Les données sont résumées dans le tableau ci-dessous et sont
disponibles dans le fichier \texttt{sirds.dat}, où les 27 premières
observations correspondent au groupe des enfants décédés au moment de
l'étude. \autocite[p.~64]{everitt01}
\vskip1em

\begin{tabular}{ll}
\toprule  
Enfants décédés &
1.050\; 1.175\; 1.230\; 1.310\; 1.500\; 1.600\; 1.720\; 1.750\; 1.770\; 2.275\; 2.500\; 1.030\; 1.100\; 1.185 \\
& 1.225\; 1.262\; 1.295\; 1.300\; 1.550\; 1.820\; 1.890\; 1.940\; 2.200\; 2.270\; 2.440\; 2.560\; 2.730 \\
Enfants vivants &
1.130\; 1.575\; 1.680\; 1.760\; 1.930\; 2.015\; 2.090\; 2.600\; 2.700\; 2.950\; 3.160\; 3.400\; 3.640\; 2.830 \\
& 1.410\; 1.715\; 1.720\; 2.040\; 2.200\; 2.400\; 2.550\; 2.570\; 3.005 \\
\bottomrule
\end{tabular}
\vskip1em

Un chercheur s'intéresse à l'existence éventuelle d'une différence entre le
poids moyen des enfants ayant survécu et celui des enfants décédés des
suites de la maladie. 
\begin{description}
\item[(a)] Réaliser un test $t$ de Student. Peut-on rejeter l'hypothèse nulle
  d'absence de différences entre les deux groupes d'enfants ? 
\item[(b)] Vérifier graphiquement que les conditions d'applications du test
(normalité et homogénéité des variances) sont vérifiées. 
\item[(c)] Quel est l'intervalle de confiance à 95~\% pour la différence de
  moyenne observée ?
\end{description}
\end{exo}
\vskip1em

\begin{exo}\label{exo:9.2}
{\footnotesize Identique à l'énoncé 3.2 (p.~\pageref{exo:3.2}), questions
  a–c.}

La qualité de sommeil de 10 patients a été mesurée avant (contrôle) et après
traitement par un des deux hypnotiques suivants : (1) D. hyoscyamine
hydrobromide et (2) L. hyoscyamine hydrobromide. Le critère de jugement
retenu par les chercheurs était le gain moyen de sommeil (en heures) par
rapport à la durée de sommeil de base
(contrôle). \autocite[p.~20]{student08} Les données sont reportées
ci-dessous et figurent également parmi les jeux de données de base de R
(\verb|data(sleep)|).  
\begin{verbatim}
D. hyoscyamine hydrobromide :
0.7 -1.6 -0.2 -1.2 -0.1  3.4  3.7  0.8  0.0  2.0
L. hyoscyamine hydrobromide :
1.9  0.8  1.1  0.1 -0.1  4.4  5.5  1.6  4.6  3.4
\end{verbatim}

Les chercheurs ont conclu que seule la deuxième molécule avait réellement un
effet soporifique. 
\begin{description}
\item[(a)] Estimer le temps moyen de sommeil pour chacune des deux
  molécules, ainsi que la différence entre ces deux moyennes.
\item[(b)] Afficher la distribution des scores de différence (LHH - DHH)
  sous forme d'un histogramme, en considérant des intervalles de classe
  d'une demi-heure, et indiquer la moyenne et l'écart-type de ces scores de
  différence.
\item[(c)] Vérifier l'exactitude des conclusions à l'aide d'un test de Student.
\end{description}
\end{exo}
\vskip1em

\begin{exo}\label{exo:9.3}
{\footnotesize Identique à l'énoncé 3.4 (p.~\pageref{exo:3.4}), questions
  a–d.}

Dans un essai clinique, on a cherché à évaluer un régime supposé réduire le
nombre de symptômes associé à une maladie bénigne du sein. Un groupe de 229
femmes ayant cette maladie ont été alétoirement réparties en deux
groupes. Le premier groupe a reçu les soins courants, tandis que les
patientes du second groupe suivaient un régime spécial (variable B =
traitement). Après un an, les individus ont été évalués et ont été classés
dans l'une des deux catégories : amélioration ou pas d'amélioration
(variable A = réponse). Les résultats sont résumés dans le tableau suivant,
pour une partie de l'échantillon :\autocite[p.~323]{selvin98}
\vskip1em

\begin{tabular}{l|cc|r}
& régime & pas de régime & total \\
\hline
amélioration & 26 & 21 & 47 \\
pas d'amélioration & 38 & 44 & 82 \\
\hline
total & 64 & 65 & 129
\end{tabular}
\vskip1em

\begin{description}
\item[(a)] Réaliser un test du chi-deux.  
\item[(b)] Quels sont les effectifs théoriques attendus sous une hypothèse
  d'indépendance ?
\item[(c)] Comparer les résultats obtenus en (a) avec ceux d'un test de
  Fisher.
\item[(d)] Donner un intervalle de confiance pour la différence de
  proportion d'amélioration entre les deux groupes de patientes.
\end{description}
\end{exo}
\vskip1em

\begin{exo}\label{exo:9.4}
{\footnotesize Identique à l'énoncé 3.5 (p.~\pageref{exo:3.5}), questions
  a–d.}

Dans un essai clinique, 1360 patients ayant déjà eu un infarctus dy myocarde
ont été assignés à l'un des deux groupes de traitement suivants : prise en
charge par aspirine à faible dose en une seule prise \emph{versus}
placebo. La table ci-après indique le nombre de décès par infarctus lors de
la période de suivi de trois ans :\autocite[p.~72]{agresti02} 
\vskip1em

\begin{tabular}{lccc}
\toprule
& \multicolumn{2}{c}{Infarctus} & \\
\cmidrule(r){2-3}
& Oui & Non & Total \\
\midrule
Placebo & 28 & 656 & 684 \\
Aspirine & 18 & 658 & 676 \\
\bottomrule
\end{tabular}
\vskip1em

\begin{description}
\item[(a)] Calculer la proportion d'infarctus du myocarde dans les deux
  groupes de patients.
\item[(b)] Représenter graphiquement le tableau précédent sous forme d'un
  diagramme en barres ou d'un diagramme en points ("dotplot" de Cleveland).
\item[(c)] Indiquer la valeur de l'odds-ratio ainsi que du risque
  relatif. Pour l'odds-ratio, on considérera comme catégories de référence
  les modalités représentées par la première ligne et la première colonne du
  tableau.  
\item[(d)] À partir de l'intervalle de confiance à 95~\% pour l'odds, quelle
  conclusion peut-on tirer sur l'effet de l'aspirine dans la prévention d'un
  infarctus du myocarde ?
\end{description}
\end{exo}
\vskip1em

\begin{exo}\label{exo:9.5}
{\footnotesize Identique à l'énoncé 4.1 (p.~\pageref{exo:4.1}), questions
  a–d.} 

Dans une étude sur le gène du récepteur à \oe strogènes, des généticiens se
sont intéressés à la relation entre le génotype et l'âge de diagnostic du
cancer du sein. Le génotype était déterminé à partir des deux allèles d'un
polymorphisme de restriction de séquence (1.6 et 0.7 kb), soit trois groupes
de sujets : patients homozygotes pour l'allèle 0.7 kb (0.7/0.7), patients
homozygotes pour l'allèle 1.6 kb (1.6/1.6), et patients hétérozygotes
(1.6/0.7). Les données ont été recueillies sur 59 patientes atteintes d'un
cancer du sein, et sont disponibles dans le fichier
\texttt{polymorphism.dta} (fichier \Stata). Les données moyennes sont
indiquées ci-dessous :\autocite[p.~327]{dupont09}
\vskip1em

\begin{tabular}{lrrrr}
\toprule
& \multicolumn{3}{c}{Génotype} & \\
\cmidrule(r){2-4}
& 1.6/1.6 & 1.6/0.7 & 0.7/0.7 & Total \\
\midrule
Nombre de patients & 14 & 29 & 16 & 59 \\
\emph{Âge lors du diagnostic} & & & & \\
\quad Moyenne & 64.64 & 64.38 & 50.38 & 60.64 \\
\quad Écart-type & 11.18 & 13.26 & 10.64 & 13.49 \\
\quad IC 95~\% & (58.1–71.1) & (59.9–68.9) & (44.3–56.5) & \\
\bottomrule
\end{tabular}
\vskip1em

\begin{description}
\item[(a)] Tester l'hypothèse nulle selon laquelle l'âge de diagnostic ne varie
  pas selon le génotype à l'aide d'une ANOVA. Représenter sous forme
  graphique la distribution des âges pour chaque génotype.
\item[(b)] Les intervalles de confiance présentés dans le tableau ci-dessus ont
  été estimés en supposant l'homogénéité des variances, c'est-à-dire en
  utilisant l'estimé de la variance commune ; donner la valeur de ces
  intervalles de confiance sans supposer l'homoscédasticité. 
\item[(c)] Estimer les différences de moyenne correspondant à l'ensemble des
  combinaisons possibles des trois génotypes, avec une estimation de
  l'intervalle de confiance à 95~\% associé et un test paramétrique
  permettant d'évaluer le degré de significativité de la différence
  observée.
\item[(d)] Représenter graphiquement les moyennes de groupe avec des
  intervalles de confiance à 95~\%.
\end{description}
\end{exo}
\vskip1em

\begin{exo}\label{exo:9.6}
{\footnotesize Identique à l'énoncé 4.2 (p.~\pageref{exo:4.2}), questions
  a–c.}

On a mesuré en fin de traitement chez 18 patients répartis par tirage au
sort en trois groupes de traitement A, B, et C, un paramètre biologique dont
on sait que la distribution est normale. Les résultats sont les suivants :
\vskip1em

\begin{tabular}{ccc}
\toprule
A & B & C \\
\midrule
19.8 & 15.9 & 15.4 \\
20.5 & 19.7 & 17.1 \\
23.7 & 20.8 & 18.2 \\
27.1 & 21.7 & 18.5 \\
29.6 & 22.5 & 19.3 \\
29.9 & 24.0 & 21.2 \\
\bottomrule
\end{tabular}
\vskip1em

\begin{description}
\item[(a)] Réaliser une ANOVA à un facteur.
\item[(b)] Selon le résultat du test, procéder aux comparaisons par paire de
  traitement des moyennes, en appliquant une correction simple de Bonferroni
  (c'est-à-dire où les degrés de significativité estimé sont multipliés par
  le nombre de comparaisons effectuées). Comparer avec de simples tests de
  Student non corrigés pour les comparaisons multiples. 
\item[(c)] D'après des études plus récentes, il s'avère que la normalité des
  distributions parentes peut-être remise en question. Effectuer la
  comparaison des trois groupes par une approche non-paramétrique.
\end{description}
\end{exo}
\vskip1em

\begin{exo}\label{exo:9.7}
{\footnotesize Identique à l'énoncé 4.3 (p.~\pageref{exo:4.3}), questions
  a–e.}

Un service d'obstétrique s'intéresse au poids de nouveaux-nés nés à terme et
âgés de 1 mois. Pour cet échantillon de 550 bébés, on dispose également
d'une information concernant la parité (nombre de frères et soeurs), mais on
sait qu'il n'y aucune relation de gemellité parmi les enfants ayant des
frères et soeurs. L'objet de l'étude est de déterminer si la parité (4
classes) influence le poids des nouveaux-nés à 1 mois. Les données sont
résumées dans le tableau suivant, et elles sont disponibles dans un fichier
SPSS, \texttt{weights.sav}.\autocite[p.~113]{peat05}
\vskip1em

\begin{tabular}{lrrrrr}
\toprule
& \multicolumn{4}{c}{Nombre de frères et soeurs} & Total \\
& 0 & 1 & 2 & $\ge 3$ & \\
\midrule
\emph{Échantillon} & & & & \\ 
Effectif & 180 & 192 & 116 & 62 & 550 \\
Fréquence & 32.7 & 34.9 & 21.1 & 11.3 & 100.0 \\
\emph{Poids (kg)} & & & & \\
Moyenne & 4.26 & 4.39 & 4.46 & 4.43 & \\
Écart-type & 0.62 & 0.59 & 0.61 & 0.54 & \\
(Min–Max) & (2.92–5.75) & (3.17–6.33) & (3.09–6.49) & (3.20–5.48) & \\
\bottomrule
\end{tabular}
\vskip1em

\begin{description}
\item[(a)] Vérifier les données reportées dans le tableau précédent.
\item[(b)] Procéder à une analyse de variance à un facteur. Conclure sur la
  significativité globale et indiquer la part de variance expliquée par le
  modèle.
\item[(c)] Afficher la distribution des poids selon la parité. Procéder à un
  test d'homogénéité des variances (rechercher dans l'aide en ligne le test
  de Levenne). 
\item[(d)] On décide de regrouper les deux dernières catégories (2 et $\ge
  3$). Refaire l'analyse et comparer aux résultats obtenus en (b).
\item[(e)] Réaliser un test de tendance linéaire (par ANOVA) sur les données
  recodées en trois niveaux pour la parité.
\end{description}
\end{exo}

%--------------------------------------------------------------- Devoir 08 ---
\chapter*{Devoir \no 8}
\addcontentsline{toc}{chapter}{Devoir \no 8}

Les exercices sont indépendants. Une seule réponse est correcte pour chaque
question. Lorsque vous ne savez pas répondre, cochez la case correspondante.

\section*{Exercice 1}
Soient les données d'une étude transversale s'intéressant à la transmission
du VIH dans des couples hétérosexuels monogames suite à la transmission de
sang contaminé chez le partenaire male \citep{obrien94}. Dans le tableau
ci-dessous, les colonnes représentent la présence ou absence du virus chez
l'homme et les lignes le diagnostic noté chez la partenaire.
\vskip1em

\begin{tabular}{lccc}
\toprule
& \multicolumn{2}{c}{Diagnostif partenaire} & \\
\cmidrule(r){2-3}
& Positif & Négatif & Total \\
\midrule
HIV+ & 3 & 4 & 7 \\
HIV- & 2 & 22 & 24 \\
\midrule
Total & 5 & 26 & 31 \\
\bottomrule
\end{tabular}
\vskip1em

Les données sous Stata, avant et après assignation d'étiquettes pour les
deux variables qualitatives, sont reportées dans le listing suivant :
\begin{verbatim}
. list

     +--------------------+
     | Homme   Femme    N |
     |--------------------|
  1. |     1       1    3 |
  2. |     1       0    2 |
  3. |     0       1    4 |
  4. |     0       0   22 |
     +--------------------+
----%<------------
. list

     +----------------------+
     |   Homme   Femme    N |
     |----------------------|
  1. | Positif    HIV+    3 |
  2. | Positif    HIV-    2 |
  3. | Négatif    HIV+    4 |
  4. | Négatif    HIV-   22 |
     +----------------------+
\end{verbatim}
\begin{description}
\item[\bf 1.1] Quelle commande permettrait de retrouver l'effectif total ?
  \marginpar{1.1 $\square$} 
  \begin{description}
  \item[A.] \verb|gen total=sum(N)|
  \item[B.] \verb|egen total=sum(N)|
  \item[C.] \verb|gen total=sum N|
  \item[D.] \verb|egen total=sum N|
  \item[E.] Je ne sais pas.
  \end{description}  
\item[\bf 1.2] On souhaite calculer la proportion de couples pour lesquels
  chaque partenaire a un diagnostique positif. Quelle commande doit-on
  utiliser ? \marginpar{1.2 $\square$} 
  \begin{description}
  \item[A.] \verb|tabulate Femme Homme [fw=N], row|
  \item[B.] \verb|tabulate Femme Homme [fw=N], col|
  \item[C.] \verb|tabulate Femme Homme [fw=N], cell|
  \item[D.] \verb|tabulate Femme Homme [fw=N], tot|
  \item[E.] Je ne sais pas.
  \end{description}  
\item[\bf 1.3] Quelle commande permet de reproduire la figure suivante
  (indépendemment du rapport hauteur/largeur et de la couleur) ?
  \marginpar{1.3 $\square$} 
  \begin{center}
    \includegraphics{./figs/dev8_barchart}
  \end{center}
  \begin{description}
  \item[A.] \verb|graph bar (asis) N, col over(Femme) asyvars over(Homme)|
  \item[B.] \verb|graph bar (asis) N, row over(Femme) asyvars over(Homme)|
  \item[C.] \verb|graph bar (asis) N, over(Femme) asyvars over(Homme)|
  \item[D.] \verb|graph bar (asis) N, over(Femme) asyvars percent over(Homme)|
  \item[E.] Je ne sais pas.
  \end{description}  
\item[\bf 1.4] Quelle commande fournit un résumé numérique équivalent aux
  chiffres présentés dans la figure précédente ? \marginpar{1.4 $\square$} 
  \begin{description}
  \item[A.] \verb|tabulate Femme Homme [fw=N], row|
  \item[B.] \verb|tabulate Femme Homme [fw=N], col|
  \item[C.] \verb|tabulate Femme Homme [fw=N], cell|
  \item[D.] \verb|tabulate Femme Homme [fw=N], tot|
  \item[E.] Je ne sais pas.
  \end{description}  
\item[\bf 1.5] On souhaite comparer les résultats d'un test du $\chi^2$ avec
  celui d'un test exact de Fisher pour vérifier l'existence d'une
  association entre les variables \texttt{Homme} et \texttt{Femme}. Quelle
  commande peut-on utiliser ? \marginpar{1.5 $\square$}
  \begin{description}
  \item[A.] \verb|tabulate Femme Homme [fw=N], nofreq clrchi2 exact|
  \item[B.] \verb|tabulate Femme Homme [fw=N], cchi2 exact|
  \item[C.] \verb|tabulate Femme Homme [fw=N], nofreq chi2 exact|
  \item[D.] Je ne sais pas.
  \end{description}  
\item[\bf 1.6] Quelle commande peut-on utiliser pour calculer l'odds-ratio
  et un intervalle de confiance à 95~\% ? \marginpar{1.6 $\square$}
  \begin{description}
  \item[A.] \verb|tabulate Femme Homme [fw=N], or|
  \item[B.] \verb|tabodds Femme Homme [fw=N], or|
  \item[C.] \verb|tabi 3 4\ 2 22, or|
  \item[D.] \verb|cci 3 4\ 2 22|
  \item[D.] Je ne sais pas.
  \end{description}  
\end{description}

\section*{Exercice 2}
Avec les données décrites à l'exercice~3 du devoir \no~7
(\pageref{dev7:exo3}), dont un aperçu est fourni ci-après, on souhaite
vérifier l'homogénéité du volume total de plaquettes entre les traitements A
(bras actif) et B (placebo) en début d'étude, ainsi qu'aux variations
intra-individuelles entre le début et la fin de l'étude pour chacun des deux
traitements.
\begin{verbatim}
. list in 1/5

     +------------------------------------------+
     | centre   patient      pre     post   trt |
     |------------------------------------------|
  1. |      1         1   359000   396000     B |
  2. |      1         2   200000   184000     A |
  3. |      1         3   149000   151000     A |
  4. |      1         4   235000   242000     B |
  5. |      1         5   174000   177000     B |
     +------------------------------------------+
\end{verbatim}
\begin{description}
\item[\bf 2.1] Pour représenter la distribution des volumes de plaquette
  (\texttt{pre}) sous forme d'histogramme pour chacun des traitements
  (\texttt{trt}), on peut utiliser la commande suivante : 
  \marginpar{2.1 $\square$}
\begin{verbatim}
. histogram pre, over(trt)
\end{verbatim}
  \begin{description}
  \item[A.] Vrai.
  \item[B.] Faux.
  \item[C.] Je ne sais pas.
  \end{description}
\item[\bf 2.2] On souhaite réaliser un test de Student pour vérifier que les
  deux groupes de patients définis par la variable \texttt{trt} sont
  comparables du point de leur volume total moyen de plaquettes. Quelle
  commande doit-on utiliser ?  \marginpar{2.2 $\square$}
  \begin{description}
  \item[A.] \verb|ttest pre, over(trt)|
  \item[B.] \verb|ttest pre, by(trt)|
  \item[C.] Je ne sais pas.
  \end{description}
\item[\bf 2.3] On s'intéresse à la variation du volume total de plaquettes
  en début et sortie d'étude pour le traitement A. On propose de réaliser un
  test de Student. Quelle commande est la plus appropriée ? \marginpar{2.3 $\square$}
  \begin{description}
  \item[A.] \verb|ttest post == pre if trt == "A"|
  \item[B.] \verb|ttest post == pre if trt == "A", unpaired|
  \item[C.] Je ne sais pas.
  \end{description}
\item[\bf 2.4] Voici le résultat d'une commande permettant de calculer un
  intervalle de confiance à 95\% pour le volume moyen de plaquette pour le
  traitement A : \marginpar{2.4 $\square$}
\begin{verbatim}
. ci pre if trt == "A"

    Variable |        Obs        Mean    Std. Err.       [95% Conf. Interval]
-------------+---------------------------------------------------------------
         pre |         15    223066.7    15219.18        190424.8
255708.6
\end{verbatim}
  Les bornes de l'intervalle de confiance présentées ci-dessus sont-elles
  identiques à celles que renverrait la commande \texttt{ttest} utilisée
  dans la question 2.3 ?
  \begin{description}
  \item[A.] Oui.
  \item[B.] Non.
  \item[C.] Je ne sais pas.
  \end{description}
\end{description}

\section*{Exercice 3}
Un investigateur s'intéressant à la fonction respiratoire décide
d'enregistrer le volume expiratoire maximum seconde (mesuré en litres) chez
des sujets fumeurs et non fumeurs. Quatre catégories sont définies \emph{a
  priori} : les non fumeurs, les anciens fumeurs, les nouveaux fumeurs et
les fumeurs de longue date. Un ensemble de 6 personnes est tiré au sort dans
chaque catégorie \citep[p.~34]{mickey04}. Les données présentées dans le
tableau suivant sont disponibles dans le fichier \texttt{vems.dta}.
\vskip1em

\begin{tabular}{lcccccccc}
\toprule
Observation & 1 & 2 & 3 & 4 & 5 & 6 & Moy & Var \\
\midrule
Non fumeurs      & 4.41 & 4.96 & 3.50 & 3.66 & 4.68 & 4.11 & 4.22 & 0.33 \\
Anciens fumeurs  & 3.69 & 3.90 & 3.82 & 4.08 & 3.76 & 4.38 & 3.94 & 0.06 \\
Nouveaux fumeurs & 3.54 & 4.40 & 3.28 & 2.28 & 3.34 & 3.92 & 3.46 & 0.51 \\
Fumeurs          & 2.98 & 2.95 & 2.15 & 3.41 & 3.97 & 3.86 & 3.22 & 0.46 \\
\midrule
Ensemble         & & & & & & & 3.71 & 0.34 \\
\bottomrule
\end{tabular}
\vskip1em

Un aperçu des données après importation sous \Stata est fourni ci-après.
\begin{verbatim}
. list in 1/5

     +--------------------+
     | VEMS     categorie |
     |--------------------|
  1. | 4.41   Non fumeurs |
  2. | 4.96   Non fumeurs |
  3. |  3.5   Non fumeurs |
  4. | 3.66   Non fumeurs |
  5. | 4.68   Non fumeurs |
     +--------------------+
. codebook categorie, compact

Variable   Obs Unique  Mean  Min  Max  Label
-------------------------------------------------------------------------------
categorie   24      4   2.5    1    4  Statut fumeur
-------------------------------------------------------------------------------
\end{verbatim}
\begin{description}
\item[\bf 3.1] Quelle commande doit-on utiliser pour calculer la moyenne et
  la variance dans chaque groupe ? \marginpar{3.1 $\square$}
  \begin{description}
  \item[A.] \verb|summarize VEMS, by(categorie)|
  \item[B.] \verb|by categorie, summarize VEMS|
  \item[C.] \verb|tabstat VEMS, by(categorie) stats(mean sd)|
  \item[D.] \verb|by categorie: tabstat VEMS, mean sd|
  \item[E.] Je ne sais pas.
  \end{description}  
\item[\bf 3.2] Quelle commande permet de reproduire la figure suivante
  (indépendemment du rapport largeur/hauteur et de la couleur) ?
  \marginpar{3.2 $\square$}
\begin{center}
  \includegraphics{./figs/dev8_vemshisto}
\end{center}
\begin{description}
\item[A.] \verb|histogram VEMS|
\item[B.] \verb|histogram VEMS, freq|
\item[C.] \verb|histogram VEMS, discrete|
\item[D.] \verb|histogram VEMS, over(categorie)|
\item[E.] Je ne sais pas.
\end{description}
\item[\bf 3.3] Quelle commande permet de reproduire la figure suivante
  (indépendemment du rapport largeur/hauteur et de la couleur) ?
  \marginpar{3.3 $\square$}
\begin{center}
  \includegraphics{./figs/dev8_vemsdot}
\end{center}
\begin{description}
\item[A.] \verb|dotplot VEMS, over(categorie)|
\item[B.] \verb|dotplot VEMS, by(categorie)|
\item[C.] \verb|stripplot VEMS, over(categorie)|
\item[D.] \verb|stripplot VEMS, by(categorie)|
\item[E.] Je ne sais pas.
\end{description}
\item[\bf 3.4] On souhaite tester le rapport des deux variances les plus
  extrêmes (anciens et nouveaux fumeurs). Quelle commande permet de répondre
  à cette question ? \marginpar{3.4 $\square$}
\begin{description}
\item[A.] \verb+sdtest VEMS if categorie in 2 | categorie in 3+
\item[B.] \verb+sdtest VEMS if categorie in 2 & categorie in 3+
\item[C.] \verb+sdtest VEMS if categorie == 2 & categorie == 3, by(categorie)+
\item[D.] \verb+sdtest VEMS if categorie == 2 | categorie == 3, by(categorie)+
\item[E.] Je ne sais pas.
\end{description}
\item[\bf 3.5] Quelle commande doit-on utiliser pour réaliser une ANOVA à un
  facteur de classification ? \marginpar{3.5 $\square$}
\begin{description}
\item[A.] \verb|oneway VEMS, by(categorie)|
\item[B.] \verb|oneway VEMS, over(categorie)|
\item[C.] \verb|oneway VEMS categorie|
\item[D.] Je ne sais pas.
\end{description}
\item[\bf 3.6] On souhaite réaliser un test de Levene pour vérifier
  l'hypothèse d'homogénéité des variances. Quelle commande doit-on utiliser
  ? \marginpar{3.6 $\square$}
\begin{description}
\item[A.] \verb|vartest VEMS, by(categorie)|
\item[B.] \verb|robtest VEMS, by(categorie)|
\item[C.] \verb|robvar VEMS, by(categorie)|
\item[D.] Je ne sais pas.
\end{description}
\item[\bf 3.7] On peut retrouver le résultat du test $F$ à partir d'une
  approche par régression linéaire. La commande suivante permet d'obtenir un
  tel résultat. \marginpar{3.7 $\square$}
\begin{verbatim}
. regress VEMS i.categorie
\end{verbatim}
\begin{description}
\item[A.] Vrai.
\item[B.] Faux.
\item[C.] Je ne sais pas.
\end{description}
\item[\bf 3.8] En supposant la commande précédente correctement spécifiée,
  on souhaite calculer un intervalle de confiance à 90~\% pour la moyenne
  des fumeurs (4\ieme\ groupe). Quelle commande doit-on utiliser ?
  \marginpar{3.8 $\square$}
\begin{description}
\item[A.] \verb|lincom 4.categorie|
\item[B.] \verb|lincom 4.categorie - 1.categorie|
\item[C.] \verb|lincom 4.categorie + _cons|
\item[D.] Je ne sais pas.
\end{description}
\end{description}

%--------------------------------------------------------------- Chapter 10 ---
\chapter{Régression linéaire et logistique}

\begin{exo}\label{exo:10.1}
{\footnotesize Identique à l'énoncé 5.1 (p.~\pageref{exo:5.1}), questions
  a–e.}

Une étude a porté sur une mesure de malnutrition chez 25 patients âgés de 7
à 23 ans et souffrant de fibrose kystique. On disposait pour ces patients de
différentes informations relatives aux caractéristiques antropométriques
(taille, poids, etc.) et à la fonction pulmonaire. \autocite[p.~180]{everitt01}
Les données sont disponibles dans le fichier \texttt{cystic.dat}.
\begin{description}
\item[(a)] Calculer le coefficient de corrélation linéaire entre les
  variables \texttt{PEmax} et \texttt{Weight}, ainsi que son intervalle de
  confiance à 95~\%.
\item[(b)] Tester si le coefficient de corrélation calculé en (a) peut être
  considéré comme significativement différent de 0.3 au seuil 5~\%.
\item[(c)] Afficher l'ensemble des données numériques sous forme de
  diagrammes de dispersion, soit 45 graphiques arrangés sous forme d'une
  "matrice de dispersion".
\item[(d)] Calculer l'ensemble des corrélations de Pearson et de Spearman
  entre les variables numériques. 
  %Reporter les coefficients de
  %Bravais-Pearson supérieurs à 0.7 en valeur absolue.
\item[(e)] Calculer la corrélation entre \texttt{PEmax} et \texttt{Weight},
  en contrôlant l'âge (\texttt{Age}) (corrélation partielle). Représenter
  graphiquement la covariation entre \texttt{PEmax} et \texttt{Weight} en
  mettant en évidence les deux terciles les plus extrêmes pour la variable
  \texttt{Age}. 
\end{description}
\end{exo}
\vskip1em

\begin{exo}\label{exo:10.2}
{\footnotesize Identique à l'énoncé 5.2 (p.~\pageref{exo:5.2}), questions
  a–e.}

Les données disponibles dans le fichier \texttt{quetelet.csv} renseignent
sur la pression artérielle systolique (\texttt{PAS}), l'indice de Quetelet
(\texttt{QTT}), l'âge (\texttt{AGE}) et la consommation de tabac
(\texttt{TAB}=1 si fumeur, 0 sinon) pour un échantillon de 32 hommes de plus
de 40 ans. 
\begin{description}
\item[(a)] Indiquer la valeur du coefficient de corrélation linéaire entre
  la pression artérielle systolique et l'indice de Quetelet, avec un
  intervalle de confiance à 90~\%.
\item[(b)] Donner les estimations des paramètres de la droite de régression
  linéaire de la pression artérielle sur l'indice de Quetelet.
\item[(c)] Tester si la pente de la droite de régression est différente de 0
  (au seuil 5~\%).
\item[(d)] Représenter graphiquement les variations de pression artérielle
  en fonction de l'indice de Quetelet, en faisant apparaître distinctement
  les fumeurs et les non-fumeurs avec des symboles ou des couleurs
  différentes, et tracer la droite de régression dont les paramètres ont été
  estimés en (b). 
\item[(e)] Refaire l'analyse (b-c) en restreignant l'échantillon aux
  fumeurs.
\end{description}
\end{exo}
\vskip1em

\begin{exo}\label{exo:10.3}
{\footnotesize Identique à l'énoncé 5.3 (p.~\pageref{exo:5.3}), questions
  a–d.}

Dans l'étude Framingham, on dispose de donnée sur la pression artérielle
systolique (\texttt{sbp}) et l'indice de masse corporelle (\texttt{bmi}) de
2047 hommes et 2643 femmes.\autocite[p.~63]{dupont09} On s'intéresse à la
relation entre ces deux variables (après transformation logarithmique) chez
les hommes et chez les femmes séparément.
Les données sont disponibles dans le fichier \texttt{Framingham.csv}.
\begin{description}
\item[(a)] Représenter graphiquement les variations entre pression
  artérielle et IMC (\texttt{bmi}) chez les hommes et chez les femmes.
\item[(b)] Les coefficients de corrélation linéaire estimés chez les hommes
  et chez les femmes sont-ils significativement différents à 5~\% ?
\item[(c)] Estimer les paramètres du modèle de régression linéaire
  considérant la pression artérielle comme variable réponse et l'IMC comme
  variable explicative, pour ces deux sous-échantillons. Donner un
  intervalle de confiance à 95~\% pour l'estimé des pentes respectives.
\item[(d)] Tester l'égalité des deux coefficients de régression associés à
  la pente (au seuil 5~\%).
\end{description}
\end{exo}
\vskip1em

\begin{exo}\label{exo:10.4}
{\footnotesize Identique à l'énoncé 6.1 (p.~\pageref{exo:6.1}) et répondre
  aux questions a–d.} 

On étudie l'effet d'un traitement prophylactique d'un macrolide à faibles
doses (Traitement A) sur les épisodes infectieux chez des patients atteints
de mucoviscidose dans un essai randomisé multicentrique contre placebo
(B). Les résultats sont les suivants :
\vskip1em

\begin{tabular}{lccc}
\toprule
& \multicolumn{2}{c}{Infection} & \\
\cmidrule(r){2-3}
& Non & Oui & Total \\
\midrule
Traitement (A) & 157 & 52 & 209 \\
Placebo (B) & 119 & 103 & 222 \\
Total & 276 & 155 & 431 \\
\bottomrule
\end{tabular}
\vskip1em

\begin{description}
\item[(a)] À partir d'un test du $\chi^2$, que peut-on répondre à la
  question : le traitement permet-il de prévenir la survenue d'épisodes
  infectieux (au seuil $\alpha=0.05$) ? Vérifier que les effectifs
  théoriques sont bien tous supérieurs à 5.
\item[(b)] Conclut-on de la même manière à partir de l'intervalle de
  confiance de l'odds-ratio associé à l'effet traitement ?
\item[(c)] On souhaite vérifier s'il existe une disparité du point de vue
  des pourcentages d'épisodes infectieux en fonction du centre. Les données
  par centre sont indiquées dans le tableau ci-après. Conclure à partir d'un
  test du $\chi^2$.

  \begin{table}[!htb] \hskip40pt
  \begin{minipage}[b]{0.33\linewidth}
  \scalebox{0.65}{\begin{tabular}{|l|r|r|r|}
    \multicolumn{1}{c}{} & \multicolumn{2}{c}{Infection} &  \multicolumn{1}{c}{} \\
    \cline{2-4}
    \multicolumn{1}{c|}{} & Non & Oui & Total \\
    \hline
    Traitement (A) & 51 & 8 & 59 \\
    \hline
    Placebo (B) & 47 & 19 & 66 \\
    \hline
    Total & 98 & 27 & 125 \\
    \hline
    \multicolumn{4}{c}{Centre 1}
  \end{tabular}} 
  \end{minipage} \hspace{0.1cm}
  \begin{minipage}[b]{0.3\linewidth}
  \scalebox{0.65}{\begin{tabular}{|l|r|r|r|}
    \multicolumn{1}{c}{} & \multicolumn{2}{c}{Infection} &  \multicolumn{1}{c}{} \\
    \cline{2-4}
    \multicolumn{1}{c|}{} & Non & Oui & Total \\
    \hline
    Traitement (A) & 91 & 35 & 126 \\
    \hline
    Placebo (B) & 61 & 71 & 132 \\
    \hline
    Total & 152 & 106 & 258 \\
    \hline
    \multicolumn{4}{c}{Centre 2}
  \end{tabular}} 
  \end{minipage} \hspace{0.1cm}
  \begin{minipage}[b]{0.3\linewidth}
  \scalebox{0.65}{\begin{tabular}{|l|r|r|r|}
    \multicolumn{1}{c}{} & \multicolumn{2}{c}{Infection} &  \multicolumn{1}{c}{} \\
    \cline{2-4}
    \multicolumn{1}{c|}{} & Non & Oui & Total \\
    \hline
    Traitement (A) & 15 & 9 & 24 \\
    \hline
    Placebo (B) & 11 & 13 & 24 \\
    \hline
    Total & 26 & 22 & 48 \\
    \hline
    \multicolumn{4}{c}{Centre 3}
  \end{tabular}}
  \end{minipage}
  \end{table}
\item[(d)] À partir du tableau précédent, on cherche à vérifier si l'effet
  traitement est indépendent du centre ou non. On se propose de réaliser un
  test de comparaison entre les deux traitements ajustés sur le centre (test
  de Mantel-Haenszel). Indiquer le résultat du test ainsi que la valeur de
  l'odds-ratio ajusté.
\end{description}
\end{exo}
\vskip1em

\begin{exo}\label{exo:10.5}
{\footnotesize Identique à l'énoncé 6.3 (p.~\pageref{exo:6.3}), questions
  a–e.}

On dispose de données issues d'une étude cherchant à établir la validité
pronostique de la concentration en créatine kinase dans l'organisme sur la
prévention de la survenue d'un infarctus du myocarde.\autocite[p.~115]{rabe-hesketh04}

Les données sont disponibles dans le fichier \texttt{sck.dat} : la première
colonne correspond à la variable créatine kinase (\texttt{ck}), la deuxième
à la variable présence de la maladie (\texttt{pres}) et la dernière à la
variable absence de maladie (\texttt{abs}).
\begin{description}
\item[(a)] Quel est le nombre total de sujets ?
\item[(b)] Calculer les fréquences relatives malades/non-malades, et
  représenter leur évolution en fonction des valeurs de créatine kinase à
  l'aide d'un diagramme de dispersion (points + segments reliant les points).
\item[(c)] À partir d'un modèle de régression logistique dans lequel on
  cherche à prédire la probabilité d'être malade, calculer la valeur de
  \texttt{ck} à partir de laquelle ce modèle prédit que les personnes
  présentent la maladie en considérant une valeur seuil de 0.5
  (si $P(\text{malade})\ge 0.5$ alors \texttt{malade=1}).
\item[(d)] Représenter graphiquement les probabilités d'être malade prédites
  par ce modèle ainsi que les proportions empiriques en fonction des valeurs
  \texttt{ck}. 
\item[(e)] Établir la courbe ROC correspondante, et reporter la
  valeur de l'aire sous la courbe. 
\end{description}
\end{exo}
\vskip1em

\begin{exo}\label{exo:10.6}
{\footnotesize Identique à l'énoncé 6.5 (p.~\pageref{exo:6.5}), questions
  a–d.} 

Une enquête cas-témoin a porté sur la relation entre la consommation
d'alcool et de tabac et le cancer de l'oesophage chez l'homme (étude "Ille
et Villaine"). Le groupe des cas était composé de 200 patients atteints d'un
cancer de l'oesophage et diagnostiqué entre janvier 1972 et avril 1974. Au
total, 775 témoins de sexe masculin ont été sélectionnés à partir des listes
électorales. Le tableau suivant indique la répartition de l'ensemble des
sujets selon leur consommation journalière d'alcool, en considérant qu'une
consommation supérieure à 80 g est considérée comme un facteur de
risque.\autocite{breslow80} 
\vskip1em

\begin{tabular}{lccc}
\toprule
& \multicolumn{2}{c}{Consommation d'alcool (g/jour)} & \\
\cmidrule(r){2-3}
& $\ge 80$ & $<80$ & Total \\
\midrule
Cas & 96 & 104 & 200 \\
Témoins & 109 & 666 & 775 \\
Total & 205 & 770 & 975 \\
\bottomrule
\end{tabular}
\vskip1em

\begin{description}
\item[(a)] Quelle est la valeur de l'odds-ratio et son intervalle de
  confiance à 95~\% (méthode de Woolf) ? Est-ce une bonne estimation du
  risque relatif ? 
\item[(b)] Est-ce que la proportion de consommateurs à risque est la même
  chez les cas et chez les témoins (considérer $\alpha=0.05$) ?
\item[(c)] Construire le modèle de régression logistique permettant de
  tester l'association entre la consommation d'alcool et le statut des
  sujets. Le coefficient de régression est-il significatif ?
\item[(d)] Retrouvez la valeur de l'odds-ratio observé, calculé en (b), et
  son intervalle de confiance à partir des résultats de l'analyse de
  régression.
\end{description}
\end{exo}

%--------------------------------------------------------------- Devoir 09 ---
\chapter*{Devoir \no 9}
\addcontentsline{toc}{chapter}{Devoir \no 9}

Les exercices sont indépendants. Une seule réponse est correcte pour chaque
question. Lorsque vous ne savez pas répondre, cochez la case correspondante.

\section*{Exercice 1}
Dans une étude clinique portant sur le niveau d'intelligence d'enfants
souffrant d'une malformation cardiaque, les investigateurs se sont
intéressés à la variation de QI avant et après une opération
chirurgicale. Les enfants n'étaient pas tirés au sort pour la chirurgie
\citep[p.~399]{mickey04}. Les résultats, contenus dans le fichier
\texttt{chiriq.dat}, sont résumés dans le tableau ci-dessous :
\vskip1em

\begin{tabular}{lcccc}
\toprule
& \multicolumn{2}{c}{Sans chirurgie} & \multicolumn{2}{c}{Chirurgie} \\
\cmidrule{2-5}
& Avant & Après & Avant & Après \\
\midrule
Moyenne & 104.25 & 103.63 & 98.63 & 102.04 \\ 
(Ety) & (12.20) & (12.43) & (12.50) & (11.45) \\
\bottomrule
\end{tabular}
\vskip1em

Voici un aperçu des données importées sous \Stata :
\begin{verbatim}
. list in 1/5

     +---------------+
     |   y     x   g |
     |---------------|
  1. | 127   124   0 |
  2. |  82    95   0 |
  3. | 124   127   0 |
  4. | 111   125   0 |
  5. | 100    90   0 |
     +---------------+
\end{verbatim}
Les variables \texttt{x}, \texttt{y} et \texttt{g} désignent,
respectivement, le QI avant et après, et le groupe de traitement
(\texttt{g}=1 pour chirurgie). On fera dans tous les cas l'hypothèse que les
variances sont homogènes pour les quatre échantillons.
\begin{description}
\item[\bf 1.1] Quelle(s) option(s) faut-il rajouter à la commande suivante
  pour calculer les moyennes et écart-types des deux variables \texttt{x} et
  \texttt{y} par groupe de traitement ? \marginpar{1.1 $\square$}
\begin{verbatim}
. tabstat x y, 
\end{verbatim}
\begin{description}
\item[A.] \verb|over(g) stats(mean sd)|
\item[B.] \verb|by(g) stats(mean sd)|
\item[C.] Je ne sais pas.
\end{description}
\item[\bf 1.2] On souhaite vérifier à l'aide d'un test statistique que le QI
  des enfants en début d'étude était comparable entre les deux groupes (avec
  ou sans chirurgie). Quelle commande est la plus appropriée pour répondre à
  cette question ?  \marginpar{1.2 $\square$}
\begin{description}
\item[A.] \verb|by g, sort: ttest x == y|
\item[B.] \verb|ttest x == y if g == 0|
\item[C.] \verb|ttest x == y if g == 0, paired|
\item[D.] Je ne sais pas.
\end{description}
\item[\bf 1.3] On souhaite afficher les QI individuels en début et fin
  d'étude pour les deux groupes de patients dans deux graphiques
  séparés. Quelle commande peut-on utiliser ? \marginpar{1.3 $\square$}
\begin{description}
\item[A.] \verb|by g, scatter y x|
\item[B.] \verb|twoway scatter y x, by(g)|
\item[C.] \verb|twoway scatter y x, over(g)|
\item[D.] \verb+scatter x y if g == 0 || scatter x y if g == 1+
\item[E.] Je ne sais pas.
\end{description}
\item[\bf 1.4] Quelle est la valeur du coefficient de corrélation linéaire
  entre les mesures avant après dans le groupe ayant subi une intervention
  chirurgicale ? \marginpar{1.4 $\square$}
\begin{description}
\item[A.] \verb|correlate x y, if g == 1|
\item[B.] \verb|correlate x y if g == 1|
\item[C.] \verb|by g, correlate x y|
\item[D.] Je ne sais pas.
\end{description}
\item[\bf 1.5] On souhaite estimer la pente de régression de \texttt{x} sur
  \texttt{y}, de manière indépendante pour les deux groupes de patients. La
  commande suivante est-elle correctement spécifiée ? \marginpar{1.5 $\square$}
\begin{verbatim}
. regress y x g
\end{verbatim}
\begin{description}
\item[A.] Oui.
\item[B.] Non.
\item[C.] Je ne sais pas.
\end{description}
\item[\bf 1.6]  Voici les résultats de la régression de \texttt{x} sur
  \texttt{y} dans le groupe des patients n'ayant pas subi d'intervention
  chirurgicale (\texttt{g}=0). \marginpar{1.6 $\square$}
\begin{verbatim}
      Source |       SS       df       MS              Number of obs =      24
-------------+------------------------------           F(  1,    22) =   12.52
       Model |  1289.31276     1  1289.31276           Prob > F      =  0.0018
    Residual |  2266.31224    22  103.014193           R-squared     =  0.3626
-------------+------------------------------           Adj R-squared =  0.3336
       Total |    3555.625    23  154.592391           Root MSE      =   10.15

------------------------------------------------------------------------------
           y |      Coef.   Std. Err.      t    P>|t|     [95% Conf. Interval]
-------------+----------------------------------------------------------------
           x |   .6135932   .1734403     3.54   0.002     .2539001    .9732863
       _cons |   39.65791   18.19946     2.18   0.040     1.914545    77.40127
------------------------------------------------------------------------------
\end{verbatim}
Quel résultat la série de commandes suivante permet-elle d'obtenir ?
\begin{verbatim}
. egen sx = sd(x) if g == 0
. egen sy = sd(y) if g == 0
. matrix b =e(b)
. svmat b
. di (b1*sx/sy)^2
\end{verbatim}
\begin{description}
\item[A.] La covariance entre les variables \texttt{x} et \texttt{y}.
\item[B.] Le coefficient de corrélation linéaire entre les variables
  \texttt{x} et \texttt{y}.
\item[C.] Le coefficient de détermination pour le modèle reporté ci-dessus. 
\item[D.] Je ne sais pas.
\end{description}
\item[\bf 1.7] La commande suivante permet de renvoyer la valeurs des
  résidus du modèle de régression précédent. \marginpar{1.7 $\square$}
\begin{verbatim}
. predict, residaals
\end{verbatim}
\begin{description}
\item[A.] Vrai.
\item[B.] Faux.
\item[C.] Je ne sais pas.
\end{description}
\end{description}

\section*{Exercice 2}
Considérons les données sur les poids à la naissance décrites à
l'exercice~8.6. On s'intéresse à la relation entre le poids à la naissance,
traité comme variable numérique (\texttt{bwt}) ou qualitative à deux classes
(\texttt{low}=1 si poids $<2.5$ kg). Parmi l'ensemble des facteurs de
risque, on se concentrera sur les antécedents d'hypertension (\texttt{ht}=1
si oui) et le status fumeur de la mère (\texttt{smoke}=1 si oui).

Les données ont été pré-traitées de la même manière, et on dispose donc de
la liste de variables suivantes sous \Stata :
\begin{verbatim}
. list in 1/5

     +---------------------------------------------------------------+
     | low   age   lwt    race   smoke   ptl   ht    ui   ftv    bwt |
     |---------------------------------------------------------------|
  1. |  no    19   182   Black      no     0   no   yes     0   2523 |
  2. |  no    33   155   Other      no     0   no    no     3   2551 |
  3. |  no    20   105   White     yes     0   no    no     1   2557 |
  4. |  no    21   108   White     yes     0   no   yes     2   2594 |
  5. |  no    18   107   White     yes     0   no   yes     0   2600 |
     +---------------------------------------------------------------+
\end{verbatim}
\begin{description}
\item[\bf 2.1] On souhaite afficher les trois tableaux de contingence
  croisant les variables \texttt{low}, \texttt{ht} et \texttt{smoke}. On
  propose d'utiliser la commande : \marginpar{2.1 $\square$}
\begin{verbatim}
. tabulate low ht smoke
\end{verbatim}
Cette commande fournit-elle le résultat escompté ?
\begin{description}
\item[A.] Oui.
\item[B.] Non.
\item[C.] Je ne sais pas.
\end{description}
\item[\bf 2.2] Si l'on souhaitait modéliser la relation entre le poids des
  bébés (en grammes, variable \texttt{bwt}) et les antécedents
  d'hypertension, on pourrait utiliser la commande : \marginpar{2.2 $\square$}
\begin{description}
\item[A.] \verb|oneway bwt i.ht|
\item[B.] \verb|regress bwt ht|
\item[C.] Je ne sais pas.
\end{description}
\item[\bf 2.3] Les deux commandes suivantes fournissent des résultats
  comparables. \marginpar{2.3 $\square$} 
\begin{verbatim}
. regress bwt i.ht
. regress bwt ht
\end{verbatim}
\begin{description}
\item[A.] Vrai.
\item[B.] Faux.
\item[C.] Je ne sais pas.
\end{description}
\item[\bf 2.4] Quelle commande devrait-on utiliser si au lieu d'une
  régression linéaire considérant comme variable réponse le poids des bébés
  comme variable numérique (\texttt{bwt}) on souhaitait travailler avec le
  poids des bébés en deux classes (\texttt{low}), la variable explicative
  restant inchangée ? \marginpar{2.4 $\square$}
\begin{description}
\item[A.] \verb|logistic low ht|
\item[B.] \verb|logis low ht|
\item[C.] \verb|blogit low ht|
\item[D.] Je ne sais pas.
\end{description}
\item[\bf 2.5] Est-il possible à l'aide de la commande précédente d'obtenir
  directement une interprétation de l'effet du facteur d'intérêt
  (\texttt{ht}) en termes d'odds-ratio ? \marginpar{2.5 $\square$}
\begin{description}
\item[A.] Oui.
\item[B.] Non.
\item[C.] Je ne sais pas.
\end{description}
\end{description}

\section*{Exercice 3}
Dans l'essai NINDS, qui est un essai clinique randomisé en double aveugle
avec groupes parallèles : placebo contre rt-PA (activateur plasminogène
recombinant de tissue), on s'est intéressé, d'une part, à l'activité
clinique de rt-PA en termes d'amélioration clinique précoce par rapport au
placebo (phase 1), et, d'autre part, à la proportion de répondeurs avec
récupération intégrale trois mois après traitement entre le groupe rt-PA et
le placebo \cite{chow04}. Une randomisation stratifiée a été retenue, avec
une première stratification sur le centre puis par rapport à la durée entre
le début de la crise et le début du traitement. On distingue donc deux
variables essentielles : la phase de l'essai (1 et 2) et la durée (0–90 ou
91–180 min). Le critère de jugement (amélioration de l'état clinique du
patient) est défini comme une variable binaire (amélioration oui/non) basée
sur un score de sévérité du déficit neurologique.

Voici les données disponibles concernant le nombre de patients pour lesquels
a été observée une amélioration (/effectif total) selon le critère mentionné
ci-dessus : 
\vskip1em
\begin{tabular}{lcrr}
\toprule
  Phase & Durée & rt-PA & Placebo \\
\midrule
  1 & 0–90 & 36/71 & 31/68\\
  1 & 91–180 & 31/73 & 26/79\\
  2 & 0–90 & 51/86 & 30/77\\
  2 & 91–180 & 29/82 & 35/88\\
\bottomrule
\end{tabular}
\vskip1em

\begin{description}
\item[\bf 3.1] De quelle manière est-il le plus naturel et le plus simple de
  saisir ces données sours \Stata ? \marginpar{3.1 $\square$}
\begin{description}
\item[A.] Une liste de deux variables représentant le croisement des
  variables \texttt{phase} et \texttt{duree} et les effectifs associés pour
  chaque cellule du tableua de contingence.
\item[B.] Une liste de quatre variables représentant chaque combinaison
  \texttt{phase} $\times$ \texttt{duree} (1/0–90, 1/91–180, etc.), avec les
  proportions de répondeurs dans chacun des traitements.
\item[C.] 
\item[D.] Je ne sais pas.
\end{description}
\item[\bf 3.2] Considérons les patients traité par rt-PA présentant un signe
  d'amélioration. Quelle commande permet de retrouver le nombre total de ces
  patients ? \marginpar{3.2 $\square$}
\begin{description}
\item[A.] \verb||
\item[B.] \verb||
\item[C.] \verb||
\item[D.] Je ne sais pas.
\end{description}
\item[\bf 3.3] Avec le même échantillon qu'à l'exercice précédent, on
  souhaite combiner les effets du traitement entre les différentes strates
  définie par la phase et la durée. Pour cela, on propose de vérifier
  l'homogénéité des odds-ratio entre les strates. Quelle commande permet de
  répondre à cette question ?  \marginpar{3.3 $\square$}
\item[\bf 3.4] Quelle est la valeur de l'odds-ratio commun ? 
  \marginpar{3.4 $\square$} 
\item[\bf 3.5] À présent, on souhaite modéliser la relation entre la réponse
  au traitement (en 0/1, telle que définie dans l'énoncé) et le traitement
  (rt-PA ou placebo). Quelle commande peut-on utiliser pour effectuer la
  régression logistique ? \marginpar{3.5 $\square$}
\begin{description}
\item[A.] \verb|logistic|
\item[B.] \verb|logistic| avec l'option \texttt{or}
\item[C.] \verb|blogit|
\item[D.] Je ne sais pas.
\end{description}
\item[\bf 3.6] Voici les résultats d'une régression logistique dans laquelle
  en plus de l'effet traitement on prend en considération la phase (0–90=1,
  91–180=2). \marginpar{3.6 $\square$}
\begin{verbatim}
Logistic regression for grouped data              Number of obs   =        624
                                                  LR chi2(2)      =       4.14
                                                  Prob > chi2     =     0.1265
Log likelihood = -424.51098                       Pseudo R2       =     0.0048

------------------------------------------------------------------------------
    _outcome |      Coef.   Std. Err.      z    P>|z|     [95% Conf. Interval]
-------------+----------------------------------------------------------------
          tx |   .3271774   .1622558     2.02   0.044     .0091618     .645193
       phase |   .0350803    .162612     0.22   0.829    -.2836335     .353794
       _cons |  -.8238462   .3577782    -2.30   0.021    -1.525079   -.1226138
------------------------------------------------------------------------------
\end{verbatim}
Quelle commande 
\end{description}


%--------------------------------------------------------------- Chapter 11 ---
\chapter{Analyse de données de survie}

\begin{exo}\label{exo:11.1}
{\footnotesize Identique à l'énoncé 7.1 (p.~\pageref{exo:7.1}), questions
  a–i.}

Dans un essai contre placebo sur la cirrhose biliaire, la D-penicillamine
(DPCA) a été introduite dans le bras actif sur une cohorte de 312
patients. Au total, 154 patients ont été randomisés dans le bras actif
(variable traitement, \texttt{rx}, 1=Placebo, 2=DPCA). Un ensemble de
données telles que l'âge, des données biologiques et signes cliniques variés
incluant le niveau de bilirubine sérique (\texttt{bilirub}) sont disponibles
dans le fichier \texttt{pbc.txt}.\autocite{vittinghoff05} Le status du
patient est enregistré dans la variable \texttt{status} (0=vivant, 1=décédé)
et la durée de suivi (\texttt{years}) représente le temps écoulé en années
depuis la date de diagnostic.
\begin{description}
\item[(a)] Combien dénombre-t-on d'individus décédés? Quelle proportion de
  ces décès retrouve-t-on dans le bras actif ?  
\item[(b)] Afficher la distribution des durées de suivi des 312 patients, en
  faisant apparaître distinctement les individus décédés. Calculer le temps
  médian (en années) de suivi pour chacun des deux groupes de
  traitement. Combien y'a-t-il d'événements positifs au-delà de 10.5 années
  et quel est le sexe de ces patients ?
\item[(c)] Les 19 patients dont le numéro (\texttt{number}) figure parmi la
  liste suivante ont subi une transplantation durant la période de suivi.
\begin{verbatim}  
5 105 111 120 125 158 183 241 246 247 254 263 264 265 274 288 291
295 297 345 361 362 375 380 383
\end{verbatim}   
  Indiquer leur âge moyen, la distribution selon le sexe et la durée médiane
  de suivi en jours jusqu'à la transplantation.
\item[(d)] Afficher un tableau résumant la distribution des événements à
  risque en fonction du temps, avec la valeur de survie associée.
\item[(e)] Afficher la courbe de Kaplan-Meier avec un intervalle de
  confiance à 95~\%, sans considérer le type de traitement.
\item[(f)] Calculer la médiane de survie et son intervalle de confiance à
  95~\% pour chaque groupe de sujets et afficher les courbes de survie
  correspondantes.
\item[(g)] Effectuer un test du log-rank en considérant comme prédicteur le
  facteur \texttt{rx}. Comparer avec un test de Wilcoxon.
\item[(h)] Effectuer un test du log-rank sur le facteur d'intérêt
  (\texttt{rx}) en stratifiant sur l'âge. On considèrera trois groupe
  d'âge : 40 ans ou moins, entre 40 et 55 ans inclus, plus de 55 ans.
\item[(i)] Retrouver les résultats de l'exercice 1.g avec une régression de
  Cox. 
\end{description}
\end{exo}
\vskip1em

\begin{exo}\label{exo:11.2}
{\footnotesize Identique à l'énoncé 7.3 (p.~\pageref{exo:7.3}), questions
  a–d}.

Dans un essai randomisé, on a cherché à comparer deux traitements pour le
cancer de la prostate. Les patients prenaient chaque jour par voie orale
soit 1 mg de diethylstilbestrol (DES, bras actif) soit un placebo, et le
temps de survie est mesuré en mois.\autocite{collett94} La question
d'intérêt est de savoir si la survie diffère entre les deux groupes de
patients, et on négligera les autres variables présentes dans le fichier de
données \texttt{prostate.dat}. 
\begin{description}
\item[(a)] Calculer la médiane de survie pour l'ensemble des patients, et
  par groupe de traitement.
\item[(b)] Quelle est la différence entre les proportions de survie dans les
  deux groupes à 50 mois ?
\item[(c)] Afficher les courbes de survie pour les deux groupes de patients.
\item[(d)] Effectuer un test du log-rank pour tester l'hypothèse selon
  laquelle le traitement par DES a un effet positif sur la survie des
  patients. 
\end{description}
\end{exo}

%--------------------------------------------------------------- Devoir 10 ---
\chapter*{Devoir \no 10}
\addcontentsline{toc}{chapter}{Devoir \no 10}

Les exercices sont indépendants. Une seule réponse est correcte pour chaque
question. Lorsque vous ne savez pas répondre, cochez la case correspondante.

\section*{Exercice 1}
Le fichier \texttt{ovarian.csv} contient les données d'une étude sur des
patientes atteintes d'un cancer des ovaires en stade II ou IIIA. L'objectif
de l'étude était de déterminer si le stade était lié ou non à la progression
de la maladie \cite[p.~389]{chow04}. Les données sous \Stata sont reportées
ci-dessous :
\begin{verbatim}
. list in 1/5

     +--------------------------+
     |  time   censored   grade |
     |--------------------------|
  1. |   .92         no     low |
  2. |  2.93         no     low |
  3. |  5.76         no     low |
  4. |  6.41         no     low |
  5. | 10.16         no     low |
     +--------------------------+
\end{verbatim}
On dispose donc des données sur la durée (en mois) avant évolution du cancer
(\texttt{time}) et d'un indicateur sur la censure éventuelle de
l'observation (\texttt{censored}). Le stade du cancer est reporté dans la
variable \texttt{grade} (\texttt{low}=II, \texttt{high}=IIIA).
\begin{description}
\item[\bf 1.1]  \marginpar{1.1 $\square$}
\item[\bf 1.2]  \marginpar{1.2 $\square$}
\item[\bf 1.3]  \marginpar{1.3 $\square$}
\item[\bf 1.4]  \marginpar{1.4 $\square$}
\item[\bf 1.5]  \marginpar{1.5 $\square$}
\item[\bf 1.6]  \marginpar{1.6 $\square$}
\end{description}

\section*{Exercice 2}
Dans l'étude UIS, on a cherché à modéliser la durée d'abstinence effective
en sortie d'étude pour des usagers de drogue inclus dans un programme
thérapeutique de courte ou longue durée \cite{hosmer08}. Par tirage au sort,
les patients étaient inclus sur l'un des deux centres (A et B) retenus pour
réaliser l'étude. Les autres informations disponibles, pour chaque patient,
sont les suivantes : âge à l'inclusion, consommation d'héroïne ou de cocaïne
durant les trois mois précédents l'inclusion, nombre de traitements relatifs
à la consommation stupéfiants. Le temps, mesuré en jours, indique la durée
entre la fin du programme thérapeutique et la reprise d'une consommation
régulière de drogue, cette dernière servant également à identifier les
données censurées (1 si consommation drogue observée pendant la période de
suivi, 0 sinon). Les données sous \Stata sont indiquées ci-dessous.
\begin{verbatim}
. list in 1/5

     +-----------------------------------------------------------+
     | ID   age   ndrugtx   treat   site   time   censor   herco |
     |-----------------------------------------------------------|
  1. |  1    39         1       1      0    188        1       3 |
  2. |  2    33         8       1      0     26        1       3 |
  3. |  3    33         3       1      0    207        1       2 |
  4. |  4    32         1       0      0    144        1       3 |
  5. |  5    24         5       1      0    551        0       2 |
     +-----------------------------------------------------------+
\end{verbatim}
La variable \texttt{ID} est le numéro d'identification du patient,
\texttt{age} l'âge du patient en années arrondies à l'entier le plus proche,
\texttt{ndrugtx} le nombre de traitements déjà effectués, \texttt{treat} le
type de programme thérapeutique (0 pour le programme court, 1 pour le
programme long), \texttt{site} le centre (A=0, B=1), \texttt{time} le temps
avant d'observer un retour à une consommation régulière de drogue chez le
patient, \texttt{censor} la présence d'un tel événement et \texttt{herco} la
consommation de cocaïne/héroïne durant les trois mois précédent l'étude
codée 1 pour la consommation des deux substances, 2 pour l'une des deux
substances et 3 pour l'absence de consommation de ces substances. 

Au total, il y a 628 patients, et on dénombre 5 valeurs manquantes pour la
variable \texttt{age} et 17 valeurs manquantes pour la variable
\texttt{ndrugtx}. 
\begin{description}
\item[\bf 2.1] On souhaite dénombrer les événements (\texttt{censor}) dans
  chacun des deux sites, pour les personnes âgées de plus de 25 ans. La
  commande suivante est-elle correctement spécifiée ? \marginpar{2.1 $\square$}
\begin{description}
\item[A.] Oui.
\item[B.] Non.
\item[C.] Je ne sais pas.
\end{description}
\item[\bf 2.2] Pour tester l'égalité des courbes de survie associée à chacun
  des deux programmes thérapeutiques, on peut utiliser la commande :
  \marginpar{2.2 $\square$}
\begin{description}
\item[A.] \verb|sts test treat|
\item[B.] \verb|sts test treat, trend|
\item[C.] \verb|stci, by(treat)|
\item[D.] Je ne sais pas.
\end{description}
\item[\bf 2.3] Quelle commande a permis de produire la figure suivante
  (indépendemment du rapport hauteur/largeur etd e la couleur) ?
  \marginpar{2.3 $\square$} 
  \begin{center}
    \includegraphics{./figs/dev10_kaplan}
  \end{center}
\begin{description}
\item[A.] \verb|sts graph, by(treat) cens|
\item[B.] \verb|sts graph, over(reat) cens ci|
\item[C.] \verb|sts graph, by(treat) ci|
\item[D.] Je ne sais pas.
\end{description}
\item[\bf 2.4] \marginpar{2.4 $\square$}
\item[\bf 2.5] \marginpar{2.5 $\square$}
\item[\bf 2.6] \marginpar{2.6 $\square$}
\end{description}
