%---------------------------------------------------------------- Séance 08 --
\chapter*{Semaine 8\markboth{Corrigés de la semaine 8}{}}

\soln{\ref{exo:8.1}} \Stata dispose d'un éditeur de données qui se présente
sous la forme d'un tableur (à l'image d'Excel), mais on peut saisir
dirctement des données à l'aide de la commande \texttt{input}. Après avoir
donné le nom de la ou des variables (lorsqu'il y a plusieurs variables, on
sépare leur nom par un espace), on presse sur la touche \textsf{Entrée}
et on saisit les observations (idem, lorsqu'il y a plusieurs variables on
saisit les observations ou valeurs observées en les séparant par un
espace). Lorsque la saisie est terminé, on saisit le mot \texttt{end} et on
presse sur \textsf{Entrée} pour indiquer à \Stata que la saisie est
terminée. \label{para:edit}
\begin{verbatim}
. input x

             x
  1. 3.68
  2. 2.21
  3. 2.45
  4. 8.64
  5. 4.32
  6. 3.43
  7. 5.11
  8. 3.87
  9. end
\end{verbatim}
On peut vérifier la saisie en affichant les valeurs 3 ou 4 premières valeurs
de \texttt{x} :
\begin{verbatim}
. list in 1/4

     +------+
     |    x |
     |------|
  1. | 3.68 |
  2. | 2.21 |
  3. | 2.45 |
  4. | 8.64 |
     +------+
\end{verbatim}
Les commandes \texttt{describe} ou \texttt{summarize} fournissent toutes les
deux le nombre d'observations contenues dans la variable \texttt{x} (comme
il n'y a qu'une seule variable, il n'est même pas nécessaire de spécifier le
nom de la variable lorsque l'on utilise ces commandes).
\begin{verbatim}
. summarize

    Variable |       Obs        Mean    Std. Dev.       Min        Max
-------------+--------------------------------------------------------
           x |         8     4.21375    2.019526       2.21       8.64
\end{verbatim}
Pour corriger la valeur erronée (8.64), on utilisera \texttt{replace} en
sépcifiant la position de l'observation dans la variable (son rang
numérique) :
\begin{verbatim}
. replace x = 3.64 in 4
(1 real change made)
\end{verbatim}
On procèdera de même pour recoder la 7\ieme\ observation en valeur
manquante, sachant que les valeurs manquantes simples sont codées à l'aide
d'un point :
\begin{verbatim}
. replace x = . in 7   
(1 real change made, 1 to missing)
\end{verbatim}
D'où le résumé numérique final suivant :
\begin{verbatim}
. summarize

    Variable |       Obs        Mean    Std. Dev.       Min        Max
-------------+--------------------------------------------------------
           x |         7    3.371429    .7656246       2.21       4.32
\end{verbatim}
% 
%
%
\soln{\ref{exo:8.2}} On peut utiliser la même solution qu'à l'exercice
précédent pour saisir les données, c'est-à-dire utiliser la commande
\texttt{input} ou bien se servir de l'éditeur de données qui se présente sous
la forme d'un tableur :

\includegraphics{./figs/stata_editX}

On en profitera pour nommer la variable (\texttt{var1} par défaut pour la
première colonne du tableur). Le seuil de détection en logarithme vaut :
\begin{verbatim}
. display log10(50)
1.69897
\end{verbatim}
On peut donc compter le nombre d'observation ne remplissant pas la condition
$X>\log(50)$ à l'aide de la commande \texttt{count}.
\begin{verbatim}
. count if X <= log10(50)
    4
\end{verbatim}
D'où le calcul de la charge virale médiane en ne considérant que les données
au-dessus de la limite de détection :
\begin{verbatim}
. egen Xm = median(10^X) if X > log10(50)
. display round(Xm)
12980
\end{verbatim}
Pour calculer la médiane des $X_i$ remplissant la condition $X_i>\log(50)$,
on a utilisé directement la commande \texttt{egen} qui fournit un certain
nombre de fonctions de transformations et de calcul de base (voir l'aide en
ligne, \verb|help egen|).
%
%
%
\soln{\ref{exo:8.3}}
Pour lire les données contenues dans le fichier \texttt{anorexia.data} dont
un aperçu est fourni ci-dessous
\begin{verbatim}
Group Before After
g1 80.5  82.2
g1 84.9  85.6
g1 81.5  81.4
g1 82.6  81.9
g1 79.9  76.4
\end{verbatim}
on va utiliser la commande \texttt{infile} à partir d'un fichier \og
dictionnaire\fg\ qui décrit la structure des données. Ce fichier porte
généralement le même nom que le fichier source de données, et possède
l'extension \texttt{dct}. En voici le listing complet
(\texttt{anorexia.dct}) :
\begin{verbatim}
infile dictionary using anorexia.dat {
  _first(2)
  str2 Group "Type de therapie"
  double Before "Avant"
  double After "Apres"
}
\end{verbatim}
Dans celui-ci, on indique que les observations commencent à la 2\ieme\ ligne
(on ignore la ligne d'en-tête), et que l'on a trois variables,
\texttt{Group} (variable qualitative), \texttt{Before} et \texttt{After}
(variables numériques). Notons que l'on a associé des étiquettes de
description pour ces trois variables. Il suffit ensuite d'utiliser la
commande \texttt{infile} en fournissant le nom de ce fichier dictionnaire
(inutile de préciser l'extension du fichier).
\begin{verbatim}
. infile using anorexia

infile dictionary using anorexia.dat {
  _first(2)
  str2 Group "Type de therapie"
  double Before "Avant"
  double After "Apres"
}

(72 observations read)
\end{verbatim}
On peut ensuite vérifier que les données ont été importées correctement à
l'aide de \texttt{describe}. Cette commande fournit par ailleurs le nombre
d'observations disponibles dans le tableau de données ($N=72$).
\begin{verbatim}
. describe

Contains data
  obs:            72                          
 vars:             3                          
 size:         1,296                          
----------------------------------------------------------------------------------------
              storage  display     value
variable name   type   format      label      variable label
----------------------------------------------------------------------------------------
Group           str2   %9s                    Type de therapie
Before          double %10.0g                 Avant
After           double %10.0g                 Apres
----------------------------------------------------------------------------------------
Sorted by:  
     Note:  dataset has changed since last saved
\end{verbatim}

Pour obtenir les effectifs par type de thérapie, on utilise un simple tri à
plat à l'aide de la commande \texttt{tabulate}.
\begin{verbatim}
. tabulate Group

    Type de |
   therapie |      Freq.     Percent        Cum.
------------+-----------------------------------
         g1 |         29       40.28       40.28
         g2 |         26       36.11       76.39
         g3 |         17       23.61      100.00
------------+-----------------------------------
\end{verbatim}

La transformation d'unités pour les poids ne pose pas de problème
spécifique, mais il faut décider si l'on crée de nouvelles variables
(\texttt{generate}) ou si l'on remplace les valeurs existantes
(\texttt{replace}). Ici, on remplacera les valeurs existantes :
\begin{verbatim}
. replace Before = Before/2.2
(72 real changes made)
. replace After = After/2.2
(72 real changes made)
\end{verbatim}

Pour les scores de différences, on crée cette fois-ci une nouvelles variable
à l'aide de la commande \texttt{generate}.
\begin{verbatim}
. generate diff = After - Before
. summarize diff

    Variable |       Obs        Mean    Std. Dev.       Min        Max
-------------+--------------------------------------------------------
        diff |        72    1.256313    3.628908  -5.545455   9.772727
\end{verbatim}

La commande \texttt{summarize} peut être utilisée pour fournir un résumé
numérique pour chacun des groupes de traitement, par exemple 
\verb|by Group, sort: summarize diff|. Mais, pour calculer spécifiquement
certains indicateurs descriptifs, il est plus commode d'utiliser la comamnde
\texttt{tabstat} à laquelle on fournit la variable réponse et le facteur de
classification, ainsi que les statistiques recherchées via
\texttt{stats()}. \label{para:tabstat}
\begin{verbatim}
. tabstat diff, by(Group) stats(mean min max)

Summary for variables: diff
     by categories of: Group (Type de therapie)

 Group |      mean       min       max
-------+------------------------------
    g1 |  1.366771 -4.136364       9.5
    g2 | -.2045454 -5.545455  7.227273
    g3 |  3.302139 -2.409091  9.772727
-------+------------------------------
 Total |  1.256313 -5.545455  9.772727
--------------------------------------
\end{verbatim}
%
%
%
\soln{\ref{exo:8.4}} Pour la saisie des données, on procéder de la même
manière qu'à l'exercice~8.1, p.~\pageref{para:edit}, ou tout simplement
saisir les données dans un fichier au format texte. Supposons que les
données aient été entrées dans un fichier appelé \verb|saisie_x.txt| dont
voici un aperçu :
\begin{verbatim}
24.9 25.0 25.0 25.1 25.2 ...
\end{verbatim}
Les valeurs de $X$ sont simplement séparées par un espace. Dans ce cas, les
données peuvent être lues et importées dans \Stata avec la commande \texttt{infile}.
\begin{verbatim}
. infile x using "saisie_x.txt"             
(26 observations read)
\end{verbatim}
On n'oubliera pas d'ajouter l'option \texttt{clear} si des données existe
déjà dans l'espace de travail de \Stata.

La commande \texttt{tabstat} peut être utilisée pour calculer la moyenne et
la médiane des observations. On utilisera par contre la commande
\texttt{egen} avec la fonction \texttt{mode} pour calculer les valeurs
modales de $X$. Comme il y a plusieurs modes, on demandera à \Stata de
renvoyer la valeur la plus basse des deux modes. Notons que la commande
\texttt{egen} pourrait également être utilisée pour calculer la moyenne et
la médiane de $X$. 
\begin{verbatim}
. tabstat x, stats(mean median)

    variable |      mean       p50
-------------+--------------------
           x |  25.44615     25.45
----------------------------------
. egen xmode = mode(x), minmode
. display xmode
25.4
\end{verbatim}

Concernant la variance, on a directement accès à l'écart-type à partir de 
\verb|summarize x|, mais on peut également la calculer avec \texttt{egen}
comme le carré de l'écart-type estimé à partir de l'échantillon, puis
l'afficher sur la console des résultats :
\begin{verbatim}
. egen varx = sd(x)
. di varx^2
.0793846
\end{verbatim}

Pour le recodage en 4 classes d'intervalles pré-définis pour la variable
\texttt{x}, on peut toujours utiliser la fonction \texttt{cut} avec
\texttt{egen}. On peut ensuite vérifier que les intervalles de classe sont
bien respectés à l'aide de \texttt{table} qui est plus souple que la
commande \texttt{summarize} et permet de préciser la liste des valeurs
statistiques à afficher.
\begin{verbatim}
. egen xc = cut(x), at(24.8,25.2,25.5,25.8,26.1) label
. table xc, contents(min x max x)

----------------------------------
       xc |     min(x)      max(x)
----------+-----------------------
    24.8- |       24.9        25.1
    25.2- |       25.2        25.4
    25.5- |       25.5        25.8
    25.8- |       25.9          26
----------------------------------
\end{verbatim}
En revanche, le tableau d'effectifs peut s'obtenir directement avec
\texttt{tabulate}, par exemple :
\begin{verbatim}
. tabulate xc, plot

         xc |      Freq.
------------+------------+-----------------------------------------------------
      24.8- |          4 |****
      25.2- |          9 |*********
      25.5- |         11 |***********
      25.8- |          2 |**
------------+------------+-----------------------------------------------------
      Total |         26
\end{verbatim}
Enfin, pour la représentation sous forme d'histogramme, \Stata dispose de ses
propres algorithmes de calcul pour la détermination du nombre de classes à
construire, tout comme R. Tout est géré à partir de la commande
\texttt{histogram}. 
\begin{verbatim}
. histogram x, frequency
(bin=5, start=24.9, width=.22000008)
\end{verbatim}
Ne pas oublier l'option \texttt{frequency} si l'on souhaite afficher les
effectifs plutôt que la densité (qui est le choix par défaut).

\includegraphics{./figs/stata_histx}
%
%
%
\soln{\ref{exo:8.5}} Pour importer les données stockées dans un simple
fichier texte, on utilise la commande \verb|infile|.
\begin{verbatim}
. infile tailles using "elderly.dat", clear
\end{verbatim}
Ici, on notera que l'on ne précise pas la manière dont sont codées les
valeurs manquantes car le "." est le format utilisé par défaut par \Stata
(pour les variables numériques uniquement).

Il existe plusieurs commandes qui facilitent la détection et
l'identification des valeurs manquantes. Parmi celles disponibles par défaut
sur \Stata, on distingue \verb|codebook| qui fournit un résumé de la
variable, ainsi que les fonctions de base qui permettent de tabuler ou
compter les observations répondant à un certain critère.
\begin{verbatim}
. count if tailles == .
    5
\end{verbatim}

Pour obtenir la taille moyenne et son intervalle de confiance à 95~\%
associé, il suffit d'entrer la commande suivante :
\begin{verbatim}
. ci tailles

    Variable |        Obs        Mean    Std. Err.       [95% Conf. Interval]
-------------+---------------------------------------------------------------
     tailles |        346    159.8324    .3232401        159.1966    160.4681
\end{verbatim}

Enfin, pour afficher la distribution des tailles sous forme d'une courbe de
densité, on utilise la commande \texttt{histogram} avec l'option
\texttt{kdensity}. Le degré de lissage peut être contrôlé à l'aide de
l'option \texttt{kdenopts} ; par exemple, ajouter \verb|kdenopts(gauss width(1))| 
à la commande ci-dessous produirait une courbe \og moins lisse\fg\
(c'est-à-dire s'ajustant plus aux données).
\begin{verbatim}
. histogram tailles, kdensity
(bin=18, start=142, width=2)
\end{verbatim}

\includegraphics{./figs/stata_kdens}

%
%
%
\soln{\ref{exo:8.6}} Les données sur les poids à la naissance de Hosmer \&
Lemeshow (1989) peuvent être obtenues directement au format \Stata grâce à la
commande \verb|webuse lbw|, mais par souci de simplicité on utilisera les
mêmes données que celles traitées avec \R. Celles-ci ont été exportées au
format texte dans le fichier \texttt{birthwt.dat}, et elles peuvent être
importées ainsi : 
\begin{verbatim}
. infile low age lwt race smoke ptl ht ui ftv bwt using "birthwt.dat", clear
\end{verbatim}

On peut vérifier avec la commande \verb|- describe -| que les variables sont
bien toutes au format numérique (colonne \texttt{storage type}).
\begin{verbatim}
. describe

Contains data
  obs:           189                          
 vars:            10                          
 size:         7,560                          
-----------------------------------------------------------------------------------------
              storage  display     value
variable name   type   format      label      variable label
-----------------------------------------------------------------------------------------
low             float  %9.0g                  
age             float  %9.0g                  
lwt             float  %9.0g                  
race            float  %9.0g                  
smoke           float  %9.0g                  
ptl             float  %9.0g                  
ht              float  %9.0g                  
ui              float  %9.0g                  
ftv             float  %9.0g                  
bwt             float  %9.0g                  
-----------------------------------------------------------------------------------------
Sorted by:  
     Note:  dataset has changed since last saved
\end{verbatim}
Pour associer de nouvelles étiquettes aux variables \texttt{low},
\texttt{race}, \texttt{smoke}, \texttt{ui} et \texttt{ht}, il suffit de
définir des "labels" et de les associer aux variables en question (cela ne
change le format de représentation des variables en mémoire, qui restent
stockées sous forme de nombre).
\begin{verbatim}
. label define yesno 0 "no" 1 "yes"
. label define ethn 1 "White" 2 "Black" 3 "Other"
. label values low smoke ui ht yesno
. label values race ethn
\end{verbatim}
On peut vérifier que le tableau de données a bien été mis à jour
\begin{verbatim}
. list in 1/5

     +---------------------------------------------------------------+
     | low   age   lwt    race   smoke   ptl   ht    ui   ftv    bwt |
     |---------------------------------------------------------------|
  1. |  no    19   182   Black      no     0   no   yes     0   2523 |
  2. |  no    33   155   Other      no     0   no    no     3   2551 |
  3. |  no    20   105   White     yes     0   no    no     1   2557 |
  4. |  no    21   108   White     yes     0   no   yes     2   2594 |
  5. |  no    18   107   White     yes     0   no   yes     0   2600 |
     +---------------------------------------------------------------+
\end{verbatim}

La conversion du poids des mères en \emph{kg} ne pose pas de problème
particulier, et ici on remplacera directement les données disponibles à
l'aide de la commande \texttt{replace}
\begin{verbatim}
. replace lwt = lwt/2.2
(189 real changes made)
\end{verbatim}
Les indicateurs de tendance centrale et de dispersion relative sont obtenus
à partir de la commande \verb|summarize|.
\begin{verbatim}
. summarize lwt, detail

                             lwt
-------------------------------------------------------------
      Percentiles      Smallest
 1%     38.63636       36.36364
 5%     42.72727       38.63636
10%     44.54546       38.63636       Obs                 189
25%           50       40.45454       Sum of Wgt.         189

50%           55                      Mean           59.00673
                        Largest       Std. Dev.      13.89972
75%     63.63636       104.0909
90%     77.27273       106.8182       Variance       193.2022
95%     85.90909       109.5455       Skewness       1.390855
99%     109.5455       113.6364       Kurtosis       5.309181
\end{verbatim}
On a ajouté l'option \texttt{detail} car la commande \verb|summarize lwt| ne
fournit pas la médiane. On a cependant vu une alternative possible avec la
commande \texttt{tabstat} dans l'exercice~1.5
(p.~\pageref{para:tabstat}). Pour obtenir l'intervalle inter-quartile, on
peut le calculer ainsi :
\begin{verbatim}
. egen lwtiqr = iqr(lwt)
. display lwtiqr
13.636364
\end{verbatim}
Enfin, un histogramme du poids des mères est construit à l'aide de la
commande \verb|histogram|. Ici, sans autre option un histogramme de
densité sera construit.
\begin{verbatim}
. histogram lwt
(bin=13, start=36.363636, width=5.9440557)
\end{verbatim}

\includegraphics{./figs/stata_birthwt}

Concernant la proportion de mères ayant fumé durant la grossesse et
le calcul de l'intervalle de confiance à 95~\% associé, on peut utiliser la
commande \verb|prtest|, qui comme dans le cas de la commande \R
\texttt{prop.test} suppose de grands échantillons :
\begin{verbatim}
. tabulate smoke

      smoke |      Freq.     Percent        Cum.
------------+-----------------------------------
         no |        115       60.85       60.85
        yes |         74       39.15      100.00
------------+-----------------------------------
      Total |        189      100.00
. prtest smoke == .5

One-sample test of proportion                  smoke: Number of obs =      189
------------------------------------------------------------------------------
    Variable |       Mean   Std. Err.                     [95% Conf. Interval]
-------------+----------------------------------------------------------------
       smoke |   .3915344   .0355036                      .3219487    .4611201
------------------------------------------------------------------------------
    p = proportion(smoke)                                         z =  -2.9823
Ho: p = 0.5

     Ha: p < 0.5                 Ha: p != 0.5                   Ha: p > 0.5
 Pr(Z < z) = 0.0014         Pr(|Z| > |z|) = 0.0029          Pr(Z > z) = 0.9986
\end{verbatim}
Le test utilisé par \Stata diffère du $\chi^2$ de \R mais ces deux tests
fournissent essentiellement les mêmes résultats. Ici, la proportion estimée
vaut 0.392 avec un intervalle de confiance de $[0.322;0.461]$. Si l'on ne
souhaite pas utiliser l'approximation pour grands échantillons, on peut
utiliser \verb|bintest| à la place.

La commande \verb|graph bar| permet de construire assez facilement des
diagrammes en barres. 
\begin{verbatim}
. gen one = 1
. graph bar (sum) one, over(smoke) ytitle("Effectif")
\end{verbatim}

\includegraphics{./figs/stata_birthwt2}

Pour générer les terciles, on utilise la commande \verb|- xtile -| en
spécifiant le nombre de quantile désiré, \texttt{nq(3)}. 
\begin{verbatim}
. xtile ageter = age, nq(3)
. tabulate ageter

3 quantiles |
     of age |      Freq.     Percent        Cum.
------------+-----------------------------------
          1 |         69       36.51       36.51
          2 |         66       34.92       71.43
          3 |         54       28.57      100.00
------------+-----------------------------------
      Total |        189      100.00
\end{verbatim}
On notera que \Stata n'attribue pas automatiquement des labels aux catégories
générées, mais simplement un rang (ici, ${1,2,3}$). On peut toutefois
obtenir les bornes des intervalles des terciles de deux manières. La
première solution, 
\begin{verbatim}
. _pctile age, n(3)
. return list

scalars:
                 r(r1) =  20
                 r(r2) =  25
\end{verbatim}
renvoit les bornes inférieures (exclues) des deux derniers intervalles, d'où
les trois intervalles : (14,20], (20,25] et (25,45], la borne inférieure du
1\ier intervalle étant obtenue comme le \texttt{min} de la variable
\texttt{age}. La seconde méthode consiste à fournir un résumé plus détaillé
de la variable \texttt{ageter} :
\begin{verbatim}
. table ageter, contents(freq min age min age)

----------------------------------------------
3         |
quantiles |
of age    |      Freq.    min(age)    min(age)
----------+-----------------------------------
        1 |         69          14          14
        2 |         66          21          21
        3 |         54          26          26
----------------------------------------------
\end{verbatim}

On notera qu'une commande telle que \texttt{cut} permettrait de générer des
groupes approximativement équivalents en termes d'effectifs, mais cela ne
correspond pas tout à fait au résultat souhaité.
\begin{verbatim}
. egen ageter = cut(age), group(3) label
\end{verbatim}

Le croisement de cette nouvelle variable avec la variable indicatrice de
sous-poids donne les résultats suivants, exprimées en termes de proportions
(par colonne) :
\begin{verbatim}
. tabulate ageter low, col nofreq

         3 |
 quantiles |          low
    of age |        no        yes |     Total
-----------+----------------------+----------
         1 |     35.38      38.98 |     36.51 
         2 |     33.08      38.98 |     34.92 
         3 |     31.54      22.03 |     28.57 
-----------+----------------------+----------
     Total |    100.00     100.00 |    100.00
\end{verbatim}

Un tri à plat de la variable \texttt{race} est effectuée de la même manière
à partir de la commande \verb|tabulate|.
\begin{verbatim}
. tabulate race

       race |      Freq.     Percent        Cum.
------------+-----------------------------------
      White |         96       50.79       50.79
      Black |         26       13.76       64.55
      Other |         67       35.45      100.00
------------+-----------------------------------
      Total |        189      100.00
\end{verbatim}


Enfin, pour résumer la distribution des variables selon la variable
indicatrice \texttt{low}, on peut procéder en deux temps. D'abord, on résume
les variables quantitatives :
\begin{verbatim}
. by low, sort: summarize age lwt ptl ftv

-------------------------------------------------------------------------------------------------------------------
-> low = no

    Variable |       Obs        Mean    Std. Dev.       Min        Max
-------------+--------------------------------------------------------
         age |       130    23.66154    5.584522         14         45
         lwt |       130    60.59091    14.42001   38.63636   113.6364
         ptl |       130    .1307692    .4556019          0          3
         ftv |       130    .8384615    1.069694          0          6

-------------------------------------------------------------------------------------------------------------------
-> low = yes

    Variable |       Obs        Mean    Std. Dev.       Min        Max
-------------+--------------------------------------------------------
         age |        59    22.30508    4.511496         14         34
         lwt |        59    55.51618     12.0724   36.36364   90.90909
         ptl |        59    .3389831    .5448875          0          2
         ftv |        59    .6949153    1.038139          0          4
\end{verbatim}
Puis, on procède de même avec les variables qualitatives :
\begin{verbatim}
. by low, sort: tab1 race smoke ht ui
\end{verbatim}


%---------------------------------------------------------------- Séance 09 --
\chapter*{Semaine 9\markboth{Corrigés de la semaine 9}{}}

\soln{\ref{exo:9.1}} On rappelle que seuls les valeurs numériques des poids
à la naissance sont disponibles dans le fichier \texttt{sirds.dat}, et qu'il
nous faut contruire la variable de groupement (enfants décédés \emph{versus}
vivants). Une manière de procéder consiste à générer une variable prenant
deux valeurs puis à associer des étiquettes ("labels") aux valeurs
distinctes prises par cette variable. Par exemple,
\begin{verbatim}
. infile poids using "sirds.dat", clear
. gen status = 0
. replace status = 1 if _n>27
. label define labstatus 0 "décédé" 1 "vivant"
. label values status labstatus
\end{verbatim}

Il est possible d'obtenir un résumé numérique de la variable \texttt{poids}
pour chaque modalité de la variable \texttt{status} nouvellement créée à
l'aide de la commande \verb|summarize|.
\begin{verbatim}
. by status, sort: summarize poids

-> status = décédé

    Variable |       Obs        Mean    Std. Dev.       Min        Max
-------------+--------------------------------------------------------
       poids |        27    1.691741    .5176473       1.03       2.73

-----------------------------------------------------------------------------------------
-> status = vivant

    Variable |       Obs        Mean    Std. Dev.       Min        Max
-------------+--------------------------------------------------------
       poids |        23    2.307391    .6647229       1.13       3.64
\end{verbatim}
Une manière équivalente de procéder, si l'on souhaite uniquement obtenir la
moyenne et l'écart-type, consiste à utiliser \texttt{tabstat} ou à combiner
les commandes \texttt{tabulate} et \texttt{summarize}, par exemple :
\begin{verbatim}
. tabulate status, summarize(poids)
\end{verbatim}

Pour réaliser un histogramme de la distribution des poids à la naissance en
fonction du status clinique, on peut procéder ainsi :
\begin{verbatim}
. histogram poids, frequency by(status)
\end{verbatim}
\includegraphics{./figs/stata_sirds}

Enfin, le test de Student est réalisé en utilisant la commande
\texttt{ttest}, qui par défaut suppose l'homogénéité des variances vérifiée :
\begin{verbatim}
. ttest poids, by(status)

Two-sample t test with equal variances
------------------------------------------------------------------------------
   Group |     Obs        Mean    Std. Err.   Std. Dev.   [95% Conf. Interval]
---------+--------------------------------------------------------------------
  décédé |      27    1.691741    .0996213    .5176473    1.486966    1.896515
  vivant |      23    2.307391    .1386043    .6647229    2.019944    2.594839
---------+--------------------------------------------------------------------
combined |      50     1.97494    .0934493     .660786    1.787147    2.162733
---------+--------------------------------------------------------------------
    diff |           -.6156506    .1673084               -.9520467   -.2792545
------------------------------------------------------------------------------
    diff = mean(décédé) - mean(vivant)                            t =  -3.6797
Ho: diff = 0                                     degrees of freedom =       48

    Ha: diff < 0                 Ha: diff != 0                 Ha: diff > 0
 Pr(T < t) = 0.0003         Pr(|T| > |t|) = 0.0006          Pr(T > t) = 0.9997
\end{verbatim}
\Stata reporte les degrés de significativité correspondant aux deux types de
tests, uni- et bilatéral. Ici, pour le test bilatéral, le degré de
significativité estimé est de 0.0006 ; ce résultat est arrondi, mais il est
possible d'obtenir une valeur plus précise à l'aide de la commande suivante :
\begin{verbatim}
. display r(p)
.00059019
\end{verbatim}
La commande ci-dessus doit être entrée immédiatement après l'appel à la
commande \verb|ttest|.
%
%
%
\soln{\ref{exo:9.2}} Les données de l'étude sur le sommeil servant de base à
l'article de Student ne sont pas disponibles sous \Stata. On peut en revanche
les saisir manuellement à l'aide de la commande \verb|input|, comme on l'a
vu p.~\pageref{para:edit}. Dans le cas présent, on va donc créer deux
variables, \texttt{DHH} et \texttt{LHH}, correspondant aux temps de sommeil
enregistrées pour les molécules D. hyoscyamine hydrobromide et
L. hyoscyamine hydrobromide, respectivement. 
\begin{verbatim}
. clear all
. input DHH LHH

           DHH        LHH
  1. 0.7 1.9
  2. -1.6 0.8
  3. -0.2 1.1
  4. -1.2 0.1
  5. -0.1 -0.1
  6. 3.4 4.4
  7. 3.7 5.5
  8. 0.8 1.6
  9. 0.0 4.6
 10. 2.0 3.4
 11. end
\end{verbatim}
On peut vérifier que la saisie donne bien les résultats attendus en
affichant les 5 premières observations :
\begin{verbatim}
. list in 1/5

     +------------+
     |  DHH   LHH |
     |------------|
  1. |   .7   1.9 |
  2. | -1.6    .8 |
  3. |  -.2   1.1 |
  4. | -1.2    .1 |
  5. |  -.1   -.1 |
     +------------+
\end{verbatim}

Les moyennes par groupe de traitement sont obtenues à l'aide de
\texttt{tabstat}, qui sans autre option renvoit la moyenne des variables
listées dans la commande.
\begin{verbatim}
. tabstat DHH LHH, save

   stats |       DHH       LHH
---------+--------------------
    mean |       .75      2.33
------------------------------
\end{verbatim}
L'option \texttt{save} utilisée ci-dessus permet de stocker temporairement
les résultats renvoyés par la commande \texttt{tabstat}, ce qui permet de
les réutiliser pour calculer la différence de moyennes entre les deux
molécules. 
\begin{verbatim}
. return list

matrices:
          r(StatTotal) :  1 x 2
. matrix m = r(StatTotal)
. mat li m

m[1,2]
            DHH        LHH
mean  .75000001       2.33
. display m[1,2] - m[1,1]
1.58
\end{verbatim}
Comme on l'a vu avec R, la manipulation de la variable auxiliaire
\texttt{m}, dans laquelle on a stocké les deux moyennes de groupe, se fait
en appelant ses éléments par numéro de position (la moyenne pour le groupe
LHH se trouve en seconde position, donc peut être utilisée comme
\verb|m[1,2]|). 

On calculera les scores de différences par simple soustraction, et on
stockera les résultats dans une nouvelle variable comme indiqué ci-après.
\begin{verbatim}
. gen sdif = LHH - DHH
. tabstat sdif, stats(mean sd)

    variable |      mean        sd
-------------+--------------------
        sdif |      1.58  1.229995
----------------------------------
\end{verbatim}
De manière équivalente, on pourrait utiliser \verb|summarize sdif|.

Enfin, pour afficher la distribution des scores de différences sous forme
d'un histogramme en imposant des intervalles de classe de 0.5 heure, il
faudra ajouter les options \texttt{bin(8)} (8 intervalles au total) et
\texttt{start(0)} (en débutant à 0).
\begin{verbatim}
. histogram sdif, percent bin(8) start(0)
(bin=8, start=0, width=.57499999)
\end{verbatim}

\includegraphics{./figs/stata_sdif}

Pour réaliser un test $t$ pour données appariées, on utilise toujours la
commande \verb|ttest|, avec cette fois-ci une syntaxe légèrement
différente :
\begin{verbatim}
. ttest DHH == LHH

Paired t test
------------------------------------------------------------------------------
Variable |     Obs        Mean    Std. Err.   Std. Dev.   [95% Conf. Interval]
---------+--------------------------------------------------------------------
     DHH |      10         .75    .5657345     1.78901   -.5297804     2.02978
     LHH |      10        2.33    .6331666    2.002249    .8976776    3.762322
---------+--------------------------------------------------------------------
    diff |      10       -1.58    .3889587    1.229995   -2.459886   -.7001143
------------------------------------------------------------------------------
     mean(diff) = mean(DHH - LHH)                                 t =  -4.0621
 Ho: mean(diff) = 0                              degrees of freedom =        9

 Ha: mean(diff) < 0           Ha: mean(diff) != 0           Ha: mean(diff) > 0
 Pr(T < t) = 0.0014         Pr(|T| > |t|) = 0.0028          Pr(T > t) = 0.9986
\end{verbatim}

Enfin, on peut représenter les gains moyens de temps de sommeil pour
chaque molécule à l'aide d'un diagramme en barres.
\begin{verbatim}
. graph hbar DHH LHH, bargap(20)
\end{verbatim}
\includegraphics{./figs/stata_student}

Ici, on a choisi de représenter les données sous forme de barres
horizontales (\texttt{hbar} au lieu de \texttt{bar}), sachant que par défaut
\Stata calcule automatiquement les moyennes conditionnelles. Pour afficher la
médiane, on utiliserait la commande \verb|graph hbar (median) DHH LHH|.
% 
%
%
\soln{\ref{exo:9.3}} \Stata dispose de commandes appelées "immédiates"
permettant de calculer les statistiques de test associées à des données
saisies directement à l'invite de commande, c'est-à-dire sans passer par
l'importation d'un fichier de données ou la saisie d'un tableau
d'effectifs. Dans le cas présent, on peut répondre aux trois questions
posées à partir de la même commande, \texttt{tabi} (à ne pas confondre
avec la commande externe \texttt{chitesti} qui réalise des tests
d'ajustement à une loi donnée).  
\begin{verbatim}
. tabi 26 21\ 38 44, exact chi2 expected

+--------------------+
| Key                |
|--------------------|
|     frequency      |
| expected frequency |
+--------------------+

           |          col
       row |         1          2 |     Total
-----------+----------------------+----------
         1 |        26         21 |        47 
           |      23.3       23.7 |      47.0 
-----------+----------------------+----------
         2 |        38         44 |        82 
           |      40.7       41.3 |      82.0 
-----------+----------------------+----------
     Total |        64         65 |       129 
           |      64.0       65.0 |     129.0 

          Pearson chi2(1) =   0.9632   Pr = 0.326
           Fisher's exact =                 0.364
   1-sided Fisher's exact =                 0.212
\end{verbatim}
\emph{À noter} : \Stata ne fournit pas de correction de continuité de Yates pour ce
type de test parmi les commandes de base.  
%
%
%
\soln{\ref{exo:9.4}} Comme dans l'exercice précédent, on pourrait travailler
à l'aide de commandes "immédiates". Toutefois, il est parfois utile de
recréer directement le tableau d'effectif sous \Stata. Voici une façon de
procéder. \label{exo:3.5stata}

La première étape consiste à créer trois variables qui permettront de
stocker, pour chaque combinaison de traitement (placebo ou aspirine) et de
réponse (infarctus ou pas d'infarctus), les effectifs observés. Les deux
variables \texttt{traitement} et \texttt{infarctus} sont des variables
binaires (deux modalités, codées 0 et 1). 
\begin{verbatim}
. clear all
. input traitement infarctus N

     traitem~t  infarctus          N
  1. 0 1 28
  2. 0 0 656
  3. 1 1 18
  4. 1 0 658
  5. end
\end{verbatim}
Ensuite, on associe des labels aux modalités des variables nouvellement
créées. On peut ensuite vérifier que les données ont bien été enregistrées
dans le format attendu.
\begin{verbatim}
. label define tx 0 "Placebo" 1 "Aspirine"
. label values traitement tx
. label define ouinon 0 "Non" 1 "Oui"
. label values infarctus ouinon
. list

     +---------------------------+
     | traite~t   infarc~s     N |
     |---------------------------|
  1. |  Placebo        Oui    28 |
  2. |  Placebo        Non   656 |
  3. | Aspirine        Oui    18 |
  4. | Aspirine        Non   658 |
     +---------------------------+
\end{verbatim}

Avec ce type de représentation, la commande \texttt{summarize} présente peu
d'intérêt, et on préférera généralement utiliser \texttt{tabulate}, par
exemple 
\begin{verbatim}
. tabulate infarctus traitement [fweight=N]
\end{verbatim}
Notons l'usage de l'option \verb|[fweight=N]| qui permet de pondérer les
cellules du tableau de contingence par les effectifs donnés dans la variable
\texttt{N}. La commande précédente fournit également l'effectif total, que
l'on peut retrouver à partir d'une commande \texttt{egen} avec la fonction
\texttt{total}. 

Les diagrammes en barres sont produits avec la commande \verb|graph bar|.
\begin{verbatim}
. graph bar (asis) N, over(infarctus) asyvars over(traitement) legend(title("Infarctus"))
\end{verbatim}
L'option \texttt{asyvars} insérée entre les deux variables de classification
est indispensable si l'on souhaite que les barres aient des couleurs
différentes selon le type de variable considéré. 

\includegraphics{./figs/stata_infarctus}
% FIXME:
% display as percent
% catplot infarctus traitement [fw=N], percent recast(bar)

Pour représenter les proportions d'infarctus selon le type de traitement, on
peut utiliser une commande externe de \Stata (à installer de la manière
suivante : \verb|ssc install catplot|) qui est bien commode pour la
représentation graphique des données catégorielles. Son usage est le suivant :
\begin{verbatim}
. catplot infarctus traitement [fw=N] if infarctus==1, percent
\end{verbatim}

\includegraphics{./figs/stata_infarctus2}

On pourrait également utiliser une option comme \texttt{percent} après
\texttt{asyvars} avec la commande \texttt{graph bar} (utilisation de
pourcentages au lieu d'effectifs). 

On verra dans le chapitre consacré aux données épidémiologiques comment
estimer l'odds-ratio et son intervalle de confiance à partir de la commande
\texttt{cc} (ou \texttt{cci}). Dans le cas présent, on se contentera
d'utiliser \texttt{tabodds} qui fournit une estimation des odds et de
l'odds-ratio. Pour ce dernier, il faut préciser l'option \texttt{or} comme
indiqué ci-après ainsi que la catégorie de référence (\texttt{base}).
\begin{verbatim}
. tabodds infarctus traitement [fweight=N], or base(2)

---------------------------------------------------------------------------
  traitement |  Odds Ratio       chi2       P>chi2     [95% Conf. Interval]
-------------+-------------------------------------------------------------
     Placebo |    1.560298       2.13       0.1446      0.854078   2.850478
    Aspirine |    1.000000          .           .              .          .
---------------------------------------------------------------------------
Test of homogeneity (equal odds): chi2(1)  =     2.13
                                  Pr>chi2  =   0.1446

Score test for trend of odds:     chi2(1)  =     2.13
                                  Pr>chi2  =   0.1446
\end{verbatim}
%
%
%
\soln{\ref{exo:9.5}} Le chargement des données ne pose pas vraiment de
problème puisque les données sont déjà disponibles au format \Stata.
\begin{verbatim}
. use polymorphism.dta
. by genotype: summarize age

-----------------------------------------------------------------------------------------
-> genotype = 1.6/1.6

    Variable |       Obs        Mean    Std. Dev.       Min        Max
-------------+--------------------------------------------------------
         age |        14    64.64286    11.18108         43         79

-----------------------------------------------------------------------------------------
-> genotype = 1.6/0.7

    Variable |       Obs        Mean    Std. Dev.       Min        Max
-------------+--------------------------------------------------------
         age |        29    64.37931    13.25954         38         87

-----------------------------------------------------------------------------------------
-> genotype = 0.7/0.7

    Variable |       Obs        Mean    Std. Dev.       Min        Max
-------------+--------------------------------------------------------
         age |        16      50.375    10.63877         33         72
\end{verbatim}
Les intervalles de confiance reportés ci-dessus ne sont pas calculés à
partir de l'ANOVA réalisée ci-après, mais en considérant une approximation par
la loi normale. On peut en revanche obtenir un résumé descriptif plus
détaillé en ajoutant l'option \texttt{detail}: 
\begin{verbatim}
. by genotype: summarize age, detail
\end{verbatim}

Il est possible d'afficher la distribution des âges pour chacun des
génotypes à l'aide de boites à moustaches.
\begin{verbatim}
. graph box age, over(genotype)
\end{verbatim}

\includegraphics{./figs/stata_polymsm}

Si l'on souhaite représenter la distribution des âges à l'aide
d'histogrammes, on peut utiliser la commande \texttt{histogram} de la
manière suivante :
\begin{verbatim}
. histogram age, by(genotype, col(3))
\end{verbatim}
L'option \verb|by(genotype, col(3))| permet d'afficher les distributions
conditionnellement au génotype et d'aligner les histogrammes horizontalement
(c'est-à-dire sur 3 colonnes).

\includegraphics{./figs/stata_histby}

La commande \texttt{oneway} permet de réaliser une analyse de variance à un
effet fixe, comme suit :
\begin{verbatim}
. oneway age genotype

                        Analysis of Variance
    Source              SS         df      MS            F     Prob > F
------------------------------------------------------------------------
Between groups      2315.73355      2   1157.86678      7.86     0.0010
 Within groups      8245.79187     56   147.246283
------------------------------------------------------------------------
    Total           10561.5254     58   182.095266

Bartlett's test for equal variances:  chi2(2) =   1.0798  Prob>chi2 = 0.583
\end{verbatim}
Le tableau d'ANOVA produit par \Stata est sensiblement identique à celui
fourni par \R, à ceci près que \Stata reporte également les totaux pour les
sommes de carrés, carrés moyens et degrés de liberté associés. On notera
également que \Stata fournit le résultat d'un test de Bartlett concernant
l'hypothèse d'homogénéité des variances. Le test de Levene pour tester
l'homogénéité des variances peut être obtenu à l'aide d'une approche un peu
différente : la commande \verb|robvar| qui fournit un ensemble de tests
robustes pour l'égalité des variances.
\begin{verbatim}
. robvar age, by(genotype)

            |     Summary of Age at Diagnosis
   Genotype |        Mean   Std. Dev.       Freq.
------------+------------------------------------
    1.6/1.6 |   64.642857   11.181077          14
    1.6/0.7 |    64.37931   13.259535          29
    0.7/0.7 |      50.375   10.638766          16
------------+------------------------------------
      Total |   60.644068   13.494268          59

W0  =  0.83032671   df(2, 56)     Pr > F = 0.44120161

W50 =  0.60460508   df(2, 56)     Pr > F = 0.54981692

W10 =  0.79381598   df(2, 56)     Pr > F = 0.45713722
\end{verbatim}
Le test de Levene est reporté sous la statistique de test \texttt{W0}.

Enfin, notons que le résumé numérique produit à l'étape précédente (moyennes
conditionnelles) peut être reproduit partiellement (sans les intervalles de
confiance) en ajoutant l'option \texttt{tabulate} :
\begin{verbatim}
. oneway age genotype, tabulate
\end{verbatim}
Si l'on souhaite former les intervalles de confiance associés aux moyennes
de groupe à partir des résultats de l'ANOVA, c'est-à-dire en considérant une
estimation de la résiduelle basée sur la variance commune, on peut procéder
comme dans le cas \R, en utilisant le carré moyen de l'erreur
(\texttt{Within groups}) et le quantile de référence de la loi $t$ (0.975)
que l'on peut obtenir comme ceci sous \Stata (ici pour le premier génotype,
1.6/1.6) :
\begin{verbatim}
. display invttail(14-3, 0.025)
2.2009852
\end{verbatim}

Plus généralement, on pourrait générer les bornes inférieures et supérieures
des intervalles de confiance à 95~\% des trois moyennes de groupe à partir
des données aggrégées, comme le montre l'exemple suivant (on ne tient plus
compte de l'estimé de la variance commune dans ce cas) :
\begin{verbatim}
. collapse (mean) agem=age (sd) ages=age (count) n=age, by(genotype)
. generate agelci = agem - invttail(n-1, 0.025)*(ages/sqrt(n))
. generate ageuci = agem + invttail(n-1, 0.025)*(ages/sqrt(n))
\end{verbatim}
et une commande telle que \verb|twoway (bar agem genotype) (rcap ageuci agelci genotype)|
permettrait d'afficher les résultats sous forme graphique. Une solution
alternative et plus commode est d'utiliser la commande \texttt{serrbar}  qui
permet d'afficher une série de moyennes associées à leur erreurs
standard. Comme on souhaite utiliser les intervalles de confiance plutôt que
les erreurs standard, il suffit juste d'ajouter une petite modification à
l'usage standard. En utilisant les calculs précédents, on calcule la
demi-largeur de l'IC (sachant que celui-ci est symétrique autour de la moyenne)
\begin{verbatim}
. gen ageb = agem-agelci
. serrbar agem ageb genotype, xlabel(1 "1.6/1.6" 2 "1.6/0.7" 3 "0.7/0.7")
\end{verbatim}

\includegraphics{./figs/stata_sebar}


On notera que les différences de moyennes peuvent être obtenues en formant
des contrastes spécifiques à partir d'un modèle de régression qui donne des
résultats équivalents au modèle d'ANOVA, sous réserve de coder la variable
qualitative \texttt{genotype} sous forme d'une matrice d'indicatrices à
l'aide de l'opérateur \verb|i.*|. On ne présente pas les résultats de
l'appel à la commande \texttt{regress} puisque ce qui nous intéresse est
simplement d'exploiter les résultats qu'elle sauvegarde (cf. \verb|e()|).
\begin{verbatim}
. regress age i.genotype
\end{verbatim}

Pour la différence de moyennes entre génotype 0.7/0.7 et 1.6/0.7, par
exemple, on utiliserait la commande suivante :
\begin{verbatim}
. lincom 3.genotype - 2.genotype

 ( 1)  - 2.genotype + 3.genotype = 0

------------------------------------------------------------------------------
         age |      Coef.   Std. Err.      t    P>|t|     [95% Conf. Interval]
-------------+----------------------------------------------------------------
         (1) |  -14.00431   3.778935    -3.71   0.000    -21.57443   -6.434194
------------------------------------------------------------------------------
\end{verbatim}
On procédera de même pour les deux autres différences de moyennes, 0.7/0.7 -
1.6/1.6 et 1.6/0.7 - 1.6/1.6.

En fait, la même approche (par régression) nous permettrait d'obtenir les
intervalles de confiance à 95~\% pour chaque moyenne de groupe :
\begin{verbatim}
. lincom _cons + 1.genotype

 ( 1)  1b.genotype + _cons = 0

------------------------------------------------------------------------------
         age |      Coef.   Std. Err.      t    P>|t|     [95% Conf. Interval]
-------------+----------------------------------------------------------------
         (1) |   64.64286   3.243084    19.93   0.000     58.14618    71.13953
------------------------------------------------------------------------------
\end{verbatim}

\soln{\ref{exo:9.6}}
Les données peuvent être saisies manuellement dans \Stata (ou à l'aide de
l'éditeur de données, ou \texttt{Data Editor}), comme présentées dans le
tableau de l'énoncé, c'est-à-dire trois variables organisées en colonnes :

\includegraphics{./figs/stata_dataentry}

Pour utiliser la commande \texttt{oneway}, il est toutefois nécessaire de
réarranger ces trois variables de façon à avoir une variable réponse et une
variable décrivant le facteur de classification (à trois classes, A, B et
C). Pour cela, on utilise la commande \texttt{stack} qui permet de
concaténer (verticalement) des variables arrangées en colonnes.
\begin{verbatim}
. stack A B C, into(pb) clear
. rename _stack tx
. label define txlab 1 "A" 2 "B" 3 "C"
. label values tx txlab
. list in 1/5

     +-----------+
     | tx     pb |
     |-----------|
  1. |  A   19.8 |
  2. |  A   20.5 |
  3. |  A   23.7 |
  4. |  A   27.1 |
  5. |  A   29.6 |
     +-----------+
\end{verbatim}

Concernant le résumé descriptif des données, on pourra procéder comme suit :
\begin{verbatim}
. tabstat pb, stats(mean sd n) by(tx)

Summary for variables: pb
     by categories of: tx 

    tx |      mean        sd         N
-------+------------------------------
     A |      25.1  4.438468         6
     B |  20.76667  2.798333         6
     C |  18.28333  1.965113         6
-------+------------------------------
 Total |  21.38333  4.199335        18
--------------------------------------
. graph box pb, over(tx)
\end{verbatim}

\includegraphics{./figs/stata_pbtx}

L'ANOVA à un facteur est réalisée comme dans l'exercice précédent avec la
commande \texttt{oneway}. Notons que l'on a ajouté l'option
\texttt{bonferroni} pour tester automatiquement les paires de moyennes entre
elles. Le résultat du test F est significatif et le test de Bartlett
concernant l'homogénéité des variances suggère que cette hypothèse est
raisonnable au vu des données. Les tests multiples protégés permettent de
préciser 
\begin{verbatim}
. oneway pb tx, bonferroni

                        Analysis of Variance
    Source              SS         df      MS            F     Prob > F
------------------------------------------------------------------------
Between groups      142.823331      2   71.4116655      6.82     0.0078
 Within groups      156.961681     15   10.4641121
------------------------------------------------------------------------
    Total           299.785012     17   17.6344125

Bartlett's test for equal variances:  chi2(2) =   3.0035  Prob>chi2 = 0.223

                            Comparison of pb by tx
                                (Bonferroni)
Row Mean-|
Col Mean |          A          B
---------+----------------------
       B |   -4.33333
         |      0.105
         |
       C |   -6.81667   -2.48333
         |      0.007      0.610
\end{verbatim}

Si l'hypothèse de normalité doit être remise en cause, on peut alors
utiliser la commande \texttt{kwallis} pour réaliser une ANOVA sur les rangs
(approche non-paramétrique).
\begin{verbatim}
. kwallis pb, by(tx)

Kruskal-Wallis equality-of-populations rank test

  +---------------------+
  | tx | Obs | Rank Sum |
  |----+-----+----------|
  |  A |   6 |    82.00 |
  |  B |   6 |    59.00 |
  |  C |   6 |    30.00 |
  +---------------------+

chi-squared =     7.942 with 2 d.f.
probability =     0.0189

chi-squared with ties =     7.942 with 2 d.f.
probability =     0.0189
\end{verbatim}
Le résultat ci-dessus est cohérent avec le résultat de l'ANOVA
paramétrique. Pour la comparaison des paires de variables, on utilisera un
test de Wilcoxon (échantillons indépendants), en corrigeant la $p$-valeur
par le nombre de tests réalisés (ici, 3). On peut soit utiliser la commande
\texttt{ranksum}, par exemple \verb+ranksum pb if tx == 1 | tx == 2, by(tx)+, 
pour comparer les deux premiers traitements, ou alors la commande
\texttt{kwallis2} qui fournit les mêmes résultats que \texttt{kwallis} mais
également les tests post-hoc associés. Il est toutefois nécessaire
d'installer ce package (\texttt{findit kwallis2}, puis suivre la procédure
d'installation indiquée).
\begin{verbatim}
. kwallis2 pb, by(tx)

One-way analysis of variance by ranks (Kruskal-Wallis Test)

tx       Obs   RankSum  RankMean 
--------------------------------
  1        6     82.00     13.67
  2        6     59.00      9.83
  3        6     30.00      5.00

Chi-squared (uncorrected for ties) =     7.942 with    2 d.f. (p = 0.01886)
Chi-squared (corrected for ties)   =     7.942 with    2 d.f. (p = 0.01886)

Multiple comparisons between groups
-----------------------------------
(Adjusted p-value for significance is 0.008333)

Ho: pb(tx==1) = pb(tx==2)
    RankMeans difference =      3.83  Critical value =      7.38
    Prob = 0.106805 (NS)

Ho: pb(tx==1) = pb(tx==3)
    RankMeans difference =      8.67  Critical value =      7.38
    Prob = 0.002463 (S)

Ho: pb(tx==2) = pb(tx==3)
    RankMeans difference =      4.83  Critical value =      7.38
    Prob = 0.058424 (NS)
\end{verbatim}

\soln{\ref{exo:9.7}}
Il n'existe pas de solution très commode pour importer un fichier SPSS sous
\Stata directement, sauf lorsque l'on travaille sous Windows (voir la
commande \texttt{usespss}). Le fichier contenant les données,
\texttt{weights.sav}, a été exporté depuis R au format \Stata à l'aide des
commandes suivantes :
\begin{verbatim}
library(foreign)
weights <- read.spss("weights.sav", to.data.frame=TRUE)
write.dta(weights, file="weights.dta")
\end{verbatim}
Depuis \Stata, on peut donc l'importer très simplement à l'aide de la
commande \texttt{use} :
\begin{verbatim}
. use "weights.dta", clear
. list in 1/5

     +----------------------------------------------------------------+
     |   ID   WEIGHT   LENGTH   HEADC   GENDER   EDUCATIO      PARITY |
     |----------------------------------------------------------------|
  1. | L001     3.95     55.5    37.5   Female   tertiary   3 or more |
  2. | L003     4.63       57    38.5   Female   tertiary   Singleton |
  3. | L004     4.75       56    38.5     Male     year12   2 sibling |
  4. | L005     3.92       56      39     Male   tertiary   One sibli |
  5. | L006     4.56       55    39.5     Male     year10   2 sibling |
     +----------------------------------------------------------------+
\end{verbatim}

Le tableau d'effectifs et de fréquences relatives pour la variable
\texttt{PARITY} s'obtient à partir de la commande \texttt{tabulate} :
\begin{verbatim}
. tabulate PARITY

            PARITY |      Freq.     Percent        Cum.
-------------------+-----------------------------------
         Singleton |        180       32.73       32.73
       One sibling |        192       34.91       67.64
        2 siblings |        116       21.09       88.73
3 or more siblings |         62       11.27      100.00
-------------------+-----------------------------------
             Total |        550      100.00
\end{verbatim}
Les moyennes et écarts-type du poids selon la taille de la fratrie
s'obtiennent ainsi :
\begin{verbatim}
. tabstat WEIGHT, stats(mean sd) by(PARITY) format(%9.2f)

Summary for variables: WEIGHT
     by categories of: PARITY (PARITY)

          PARITY |      mean        sd
-----------------+--------------------
       Singleton |      4.26      0.62
     One sibling |      4.39      0.59
      2 siblings |      4.46      0.61
3 or more siblin |      4.43      0.54
-----------------+--------------------
           Total |      4.37      0.60
--------------------------------------
\end{verbatim}
L'option \verb|format(%9.2f)| permet de limiter l'affichage des nombres à
deux décimales.

L'ANOVA à un facteur se réalise comme dans les exercices précédents, à
l'aide de la commande \texttt{oneway} et en fournissant la variable réponse
et la variable qualitative décrivant les groupes à comparer.
\begin{verbatim}
. oneway WEIGHT PARITY

                        Analysis of Variance
    Source              SS         df      MS            F     Prob > F
------------------------------------------------------------------------
Between groups       3.4769599      3   1.15898663      3.24     0.0219
 Within groups      195.364879    546   .357811134
------------------------------------------------------------------------
    Total           198.841839    549   .362189142

Bartlett's test for equal variances:  chi2(3) =   1.9085  Prob>chi2 = 0.592
\end{verbatim}
Comme on l'a vu avec R, on peut calculer la part de variance expliquée à
partir des sommes de carré reportées ci-dessus. On peut également utiliser
la commande \texttt{anova} qui renvoit, outre un tableau d'analyse de
variance, le coefficient de détermination associé au modèle.
\begin{verbatim}
. anova WEIGHT PARITY

                           Number of obs =     550     R-squared     =  0.0175
                           Root MSE      = .598173     Adj R-squared =  0.0121

                  Source |  Partial SS    df       MS           F     Prob > F
              -----------+----------------------------------------------------
                   Model |   3.4769599     3  1.15898663       3.24     0.0219
                         |
                  PARITY |   3.4769599     3  1.15898663       3.24     0.0219
                         |
                Residual |  195.364879   546  .357811134   
              -----------+----------------------------------------------------
                   Total |  198.841839   549  .362189142
\end{verbatim}

Pour afficher les données sous forme d'histogrammes pour chaque groupe, il
faut utiliser la commande \texttt{histogram} avec l'option \texttt{by} pour
définir le facteur de classification. L'option \texttt{freq} permet quant à
elle d'afficher des effectifs plutôt que des proportions (ou densités).
\begin{verbatim}
. histogram WEIGHT, by(PARITY) freq
\end{verbatim}

\includegraphics{./figs/stata_histweight}

Pour afficher afficher un diagramme de dispersion, comme sous R, on peut
utiliser une commande externe, telle que \texttt{stripplot} (\texttt{ssc
  install stripplot}), par exemple
\begin{verbatim}
. stripplot WEIGHT, over(PARITY) stack height(.4) center vertical width(.3)
\end{verbatim}
ou plus simplement 
\begin{verbatim}
. twoway scatter WEIGHT PARITY, jitter(3) xlabel(1 "Singleton" 2 "One sibling" 3 "2 siblings" 4 "3 more")
\end{verbatim}

\includegraphics{./figs/stata_stripheight}

La commande \texttt{oneway} affiche par défaut le résultat d'un test de
Bartlett pour comparer les variances des groupes entre elles. Si l'on
souhaite utiliser un test de Levene, il faut utiliser la commande
\texttt{robvar} qui fournit le résultat du test de Levene (\texttt{W0}) :
\begin{verbatim}
. robvar WEIGHT, by(PARITY)

            |          Summary of WEIGHT
     PARITY |        Mean   Std. Dev.       Freq.
------------+------------------------------------
  Singleton |   4.2589445   .61950106         180
  One sibli |   4.3886979   .59258231         192
  2 sibling |   4.4600862   .60519991         116
  3 or more |   4.4341935   .53526321          62
------------+------------------------------------
      Total |   4.3664182   .60182152         550

W0  =  0.63850632   df(3, 546)     Pr > F = 0.59046456

W50 =  0.64415596   df(3, 546)     Pr > F = 0.58688851

W10 =  0.64584135   df(3, 546)     Pr > F = 0.58582455
\end{verbatim}

Il existe plusieurs façon de recoder des variables sous \Stata, mais dans le
cas présent le plus simple pour aggréger les deux dernières classes
(\texttt{2 siblings} et \texttt{3 or more}) consiste à générer une deuxième
variable, \texttt{PARITY2}, de la manière suivante :
\begin{verbatim}
. recode PARITY (1=1) (2=2) (3/4=3), gen(PARITY2)
\end{verbatim}
Pour vérifier que la conversion s'est déroulée correctement, on affichera un
simple tableau de contingence croisant les effectifs des deux variables :
\begin{verbatim}
. tabulate PARITY PARITY2

                   |    RECODE of PARITY (PARITY)
            PARITY |         1          2          3 |     Total
-------------------+---------------------------------+----------
         Singleton |       180          0          0 |       180 
       One sibling |         0        192          0 |       192 
        2 siblings |         0          0        116 |       116 
3 or more siblings |         0          0         62 |        62 
-------------------+---------------------------------+----------
             Total |       180        192        178 |       550
\end{verbatim}

Les résultats de l'ANOVA à un facteur considérant la variable nouvellement
créée sont reportés ci-dessous :
\begin{verbatim}
. oneway WEIGHT PARITY2

                        Analysis of Variance
    Source              SS         df      MS            F     Prob > F
------------------------------------------------------------------------
Between groups      3.44987154      2   1.72493577      4.83     0.0083
 Within groups      195.391968    547   .357206522
------------------------------------------------------------------------
    Total           198.841839    549   .362189142

Bartlett's test for equal variances:  chi2(2) =   0.7963  Prob>chi2 = 0.672
\end{verbatim}
Pour le test de la tendance linéaire, on présente deux manières (méthode des
contrastes et régression linéaire). Pour générer un contraste testant la
tendance linéaire, il faut explicitement demander à \Stata d'effectuer une
régression en considérant la variable \texttt{PARITY2} comme une variable
qualitative (passage par des variables indicatrices codant pour les deux
dernières modalités du facteur). On utilisera dans ce cas la commande de
post-estimation \texttt{contrast}. Comme le résultat de la régression sur la
variable qualitative ne nous intéresse pas vraiment, on demandera à \Stata de
ne pas afficher les résultats à l'aide de l'instruction \verb|quietly :|.
\begin{verbatim}
. quietly: regress WEIGHT i.PARITY2
. contrast p.PARITY2, noeffects

Contrasts of marginal linear predictions

Margins      : asbalanced

------------------------------------------------
             |         df           F        P>F
-------------+----------------------------------
     PARITY2 |
   (linear)  |          1        9.25     0.0025
(quadratic)  |          1        0.40     0.5288
      Joint  |          2        4.83     0.0083
             |
    Residual |        547
------------------------------------------------
\end{verbatim}
Le contraste d'intérêt apparaît sur la ligne intitulée \verb|(linear)|.

Pour l'approche par régression linéaire simple, la commande est plus simple :
\begin{verbatim}
. regress WEIGHT PARITY2

      Source |       SS       df       MS              Number of obs =     550
-------------+------------------------------           F(  1,   548) =    9.27
       Model |  3.30800821     1  3.30800821           Prob > F      =  0.0024
    Residual |  195.533831   548   .35681356           R-squared     =  0.0166
-------------+------------------------------           Adj R-squared =  0.0148
       Total |  198.841839   549  .362189142           Root MSE      =  .59734

------------------------------------------------------------------------------
      WEIGHT |      Coef.   Std. Err.      t    P>|t|     [95% Conf. Interval]
-------------+----------------------------------------------------------------
     PARITY2 |   .0961272   .0315707     3.04   0.002     .0341129    .1581415
       _cons |   4.174513   .0679786    61.41   0.000     4.040983    4.308044
------------------------------------------------------------------------------
\end{verbatim}
Le test de tendance correspond au test associé à la pente de la droite de
régression, ici le coefficient \texttt{PARITY2}.


%---------------------------------------------------------------- Séance 10 --
\chapter*{Semaine 10\markboth{Corrigés de la semaine 10}{}}

\soln{\ref{exo:10.1}}
Les données tabulées sont disponibles dans un fichier texte que l'on peut
importer à l'aide de la commande \texttt{insheet}. Toutefois, on peut
également importer les données avec \texttt{infile} qui a l'avantage de
fonctionner même dans le cas où les données ne sont pas issue d'un tableau
de type Excel.
\begin{verbatim}
. infile int (Sub Age Sex) Height Weight BMP FEV RV FRC TLC PEmax using "cystic.dat" in 2/26, clear
\end{verbatim}
Dans l'instruction ci-dessus, on a explicitement demandé à \Stata de coder
les trois premières variables sous forme d'entiers et non de nombres réels,
ce qui dans le présent ne change pas grand-chose mais permet d'économiser de
l'espace mémoire pour les gros jeux de données.
Comme on l'a fait avec R, on recodera d'emblée la variable \texttt{Sex} en
variable qualitative pour éviter toute confusion.
\begin{verbatim}
. label define labsex 0 "M" 1 "F"
. label values Sex labsex
. tabulate Sex

        Sex |      Freq.     Percent        Cum.
------------+-----------------------------------
          M |         14       56.00       56.00
          F |         11       44.00      100.00
------------+-----------------------------------
      Total |         25      100.00
\end{verbatim}

L'estimation du coefficient de corrélation de Bravais-Pearson se fait avec
la commande \texttt{correlate}, mais celle-ci ne fournit pas d'options pour
l'intervalle de confiance. On peut utiliser des commandes externes comme
\texttt{ci2} qui fonctionne sur le même principe que la commande interne
\texttt{ci} (pour les moyennes et proportions) dans le cas paramétrique ou
non-paramétrique (Spearman) ou bien \texttt{corrci}, uniquement pour le cas
paramétrique. \label{cmd:corrci}
\begin{verbatim}
. correlate PEmax Weight
(obs=25)

             |    PEmax   Weight
-------------+------------------
       PEmax |   1.0000
      Weight |   0.6363   1.0000
. corrci PEmax Weight

(obs=25)

                correlation and 95% limits
PEmax  Weight      0.636    0.322    0.824
\end{verbatim}

Pour tester ce coefficient de corrélation contre l'alternative $H_0:
\rho=0.3$, il faut
% FIXME:
% change to compute test against H_0: \rho=0.3
\begin{verbatim}
. pwcorr PEmax Weight, sig

             |    PEmax   Weight
-------------+------------------
       PEmax |   1.0000 
             |
             |
      Weight |   0.6363   1.0000 
             |   0.0006
             |
\end{verbatim}

Pour afficher l'ensemble des diagrammes de dispersion, on utilise la
commande \texttt{graph matrix} en sépcifiant la liste de variables que l'on
souhaite voir figurer dans le graphique. Ici, on omettra la variable
\texttt{Sex} : 
\begin{verbatim}
. graph matrix Age Height-PEmax
\end{verbatim}

\includegraphics{./figs/stata_splom}

Les corrélations pour chaque paire de variables s'obtiennent avec la
commande \texttt{pwcorr}, et on utilisera la même notation pour indiquer les
variables qui nous intéressent.
\begin{verbatim}
. pwcorr Age Height-PEmax

             |      Age   Height   Weight      BMP      FEV       RV      FRC      TLC      PEmax
-------------+-----------------------------------------------------------------------------------
         Age |   1.0000 
      Height |   0.9261   1.0000 
      Weight |   0.9065   0.9221   1.0000 
         BMP |   0.3778   0.4408   0.6703   1.0000 
         FEV |   0.2945   0.3167   0.4492   0.5455   1.0000 
          RV |  -0.5519  -0.5695  -0.6234  -0.5824  -0.6659   1.0000 
         FRC |  -0.6394  -0.6243  -0.6182  -0.4344  -0.6651   0.9106   1.0000 
         TLC |  -0.4734  -0.4595  -0.4214  -0.3633  -0.4425   0.5899   0.7056    1.0000 
       PEmax |   0.6135   0.5992   0.6363   0.2295   0.4534  -0.3156  -0.4172   -0.1805   1.0000
\end{verbatim}
Pour les corrélations de Spearman, on remplacera \texttt{pwcorr} par
\texttt{spearman}.
\begin{verbatim}
. spearman Age Height-PEmax, stats(rho)
(obs=25)

             |      Age   Height   Weight      BMP      FEV       RV      FRC      TLC    PEmax
-------------+---------------------------------------------------------------------------------
         Age |   1.0000 
      Height |   0.9335   1.0000 
      Weight |   0.9013   0.9619   1.0000 
         BMP |   0.5091   0.5734   0.7264   1.0000 
         FEV |   0.2975   0.4256   0.4637   0.5623   1.0000 
          RV |  -0.5815  -0.6221  -0.7005  -0.6917  -0.6830   1.0000 
         FRC |  -0.7185  -0.6642  -0.6706  -0.5547  -0.6044   0.8547   1.0000 
         TLC |  -0.4926  -0.4734  -0.4846  -0.4935  -0.4398   0.5895   0.6721   1.0000 
       PEmax |   0.5198   0.5920   0.4881   0.2224   0.3140  -0.3089  -0.3835  -0.1482   1.0000
\end{verbatim}

La corrélation partielle entre les variables \texttt{PEmax} et
\texttt{Weight} en tenant compte de \texttt{Age} est obtenue à l'aide de la
commande \texttt{pcorr} ; la variable d'intérêt doit être située en première
position dans la liste des variables et par défaut \Stata affiche l'ensemble
des coefficients de corrélation partielle pour les autres variables.
\begin{verbatim}
. pcorr PEmax Weight Age
(obs=25)

Partial and semipartial correlations of PEmax with

               Partial   Semipartial      Partial   Semipartial   Significance
   Variable |    Corr.         Corr.      Corr.^2       Corr.^2          Value
------------+-----------------------------------------------------------------
     Weight |   0.2405        0.1899       0.0578        0.0361         0.2577
        Age |   0.1126        0.0869       0.0127        0.0076         0.6002
\end{verbatim}
Pour réaliser un tercilage de la variable \texttt{Age}, on peut procéder comme à
l'exercice~\ref{exo:2.4}, c'est-à-dire en créant une variable dérivée, avec
\texttt{egen}, et en utilisant la fonction \texttt{cut}. Mais on peut
directement utiliser la commande \texttt{xtile} qui est plus simple d'emploi :
\begin{verbatim}
. xtile Age3 = Age, nq(3)
\end{verbatim}
Enfin, pour le diagramme de dispersion, voici une première solution :
\begin{verbatim}
. scatter PEmax Weight if Age3 != 2, mlab(Age3)
\end{verbatim}
On pourrait également (2\ieme\ solution) enchaîner deux appels à la commande
\texttt{scatter}, en restreignant à chaque fois l'échantillon aux seules
classe de \texttt{Age3} qui nous intéressent (1 et 3, en l'occurence).
\begin{verbatim}
. scatter PEmax Weight if Age3 == 1, msymbol(circle) || scatter PEmax Weight if Age3 == 3, msymbol(square) 
  legend(label(1 "1st tercile") label(2 "3rd tercile"))
\end{verbatim}

\includegraphics{./figs/stata_tercile}

\soln{\ref{exo:10.2}}
Le chargement des données ne pose pas de difficultés particulières car
celles-ci ont été exportées depuis Excel et sont au format CSV. On veillera
cependant à préciser le type de délimiteur de champ, ici un point-virgule.
\begin{verbatim}
. insheet using "quetelet.csv", delim(";") clear
(5 vars, 32 obs)
. describe

Contains data
  obs:            32                          
 vars:             5                          
 size:           320                          
-----------------------------------------------------------------------------------------
              storage  display     value
variable name   type   format      label      variable label
-----------------------------------------------------------------------------------------
id              byte   %8.0g                  ID
pas             int    %8.0g                  PAS
qtt             str5   %9s                    QTT
age             byte   %8.0g                  AGE
tab             byte   %8.0g                  TAB
-----------------------------------------------------------------------------------------
Sorted by:
\end{verbatim}
Après l'importation, on remarque que la variable \texttt{qtt} n'est pas
reconnue en tant que nombre mais a été codée sous forme de chaîne de
caractère. Ceci s'explique par le fait que la partie décimale est séparée de
la partie entière par une virgule et non un point. Il faut donc convertir cette
variable en nombre, ce que l'on peut réaliser avec la commande
\texttt{destring} et l'option \texttt{dpcomma}.
\begin{verbatim}
. destring qtt, dpcomma replace
qtt has all characters numeric; replaced as double
\end{verbatim}
Ensuite, on recode la variable \texttt{tab} en variable qualitative avec des
étiquettes plus informatives. 
\begin{verbatim}
. label define ltab 0 "NF" 1 "F"
. label values tab ltab
. list in 1/5

     +------------------------------+
     | id   pas     qtt   age   tab |
     |------------------------------|
  1. |  1   135   2.876    45    NF |
  2. |  2   122   3.251    41    NF |
  3. |  3   130     3.1    49    NF |
  4. |  4   148   3.768    52    NF |
  5. |  5   146   2.979    54     F |
     +------------------------------+
\end{verbatim}
Enfin, le résumé numérique des variable s'obtient avec la commande \texttt{summarize}.
\begin{verbatim}
. summarize pas-tab

    Variable |       Obs        Mean    Std. Dev.       Min        Max
-------------+--------------------------------------------------------
         pas |        32    144.5313    14.39755        120        180
         qtt |        32    3.441094    .4970781      2.368      4.637
         age |        32       53.25    6.956083         41         65
         tab |        32      .53125    .5070073          0          1
\end{verbatim}
Comme la variable \texttt{tab} est internellement codée en 0/1, la moyenne
correspond à la fréquence relative des fumeurs, soit 53~\%.

Le coefficient de corrélation liénaire entre les variables \texttt{pas} et
\texttt{qtt} peut être estimé, ainsi que son intervalle de confiance, avec
la commande \texttt{corrci} (p.~\pageref{cmd:corrci}). L'option
\texttt{level(90)} permet de spécifier le niveau de confiance.
\begin{verbatim}
. corrci pas qtt, level(90)

(obs=32)

          correlation and 90% limits
pas qtt      0.742    0.571    0.851
\end{verbatim}

La commande \texttt{regress} permet d'effectuer une régression linéaire
pour une variable réponse (placée en premier dans la liste des variables) et
une ou plusieurs variables explicatives. On l'utilise comme suit :
\begin{verbatim}
. regress pas qtt

      Source |       SS       df       MS              Number of obs =      32
-------------+------------------------------           F(  1,    30) =   36.75
       Model |  3537.94574     1  3537.94574           Prob > F      =  0.0000
    Residual |  2888.02301    30  96.2674337           R-squared     =  0.5506
-------------+------------------------------           Adj R-squared =  0.5356
       Total |  6425.96875    31  207.289315           Root MSE      =  9.8116

------------------------------------------------------------------------------
         pas |      Coef.   Std. Err.      t    P>|t|     [95% Conf. Interval]
-------------+----------------------------------------------------------------
         qtt |   21.49167   3.545147     6.06   0.000     14.25151    28.73182
       _cons |    70.5764   12.32187     5.73   0.000     45.41179    95.74101
------------------------------------------------------------------------------
\end{verbatim}
Par défaut, on obtient un tableau d'analyse de variance pour le modèle de
régression et un tableau des coefficients du modèle, ici l'ordonnée à
l'origine (70.58) et la pente de la droite de régression (21.49). Le test
de Student associé à la pente permet d'évaluer sa significativité au vu des
données. Si l'on souhaite manipuler les coefficients de régression, il est
possible de les extraire du tableau de résultats comme indiqué ci-dessous :
\begin{verbatim}
. matrix b = e(b)
. svmat b
. di "pente = " b1 ", ordonnée origine = " b2
pente = 21.491669, ordonnée origine = 70.576401
\end{verbatim}

En ce qui concerne l'affichage du diagramme de dispersion avec les deux
droites de régression superposées, on utilise une combinaison de
\texttt{lfit} (pour tracer la droite de régression) et \texttt{scatter}
(pour afficher les observations). L'inconvénient est qu'il est nécessaire de
construire manuellement la légende. On notera également que l'on impose que
les droites de régression soient présentées sur toute l'étendue de la
variable \texttt{qtt}, d'où l'option \texttt{range(2, 5)}. \label{para:twoway}
\begin{verbatim}
. twoway lfit pas qtt if tab == 0, range(2 5) lpattern(dot) || 
  scatter pas qtt if tab == 0, msymbol(square) || 
  lfit pas qtt if tab == 1, range(2 5) ||
  scatter pas qtt if tab == 1, msymbol(circle) 
  legend(label(1 "") label(2 "NF") label(3 "") label(4 "F"))
\end{verbatim}

\includegraphics{./figs/stata_lfit}

La régression de \texttt{qtt} sur \texttt{pas} en restreignant l'analyse aux
seules observations du groupe fumeur (\verb|tab == 1| ou \verb|tab == "F"|) 
ne pose pas de problème car on vient de le voir dans l'application
précdente, il suffit d'ajouter une option \texttt{if tab == 1} :
\begin{verbatim}
. regress pas qtt if tab == 1

      Source |       SS       df       MS              Number of obs =      17
-------------+------------------------------           F(  1,    15) =   19.40
       Model |  2088.16977     1  2088.16977           Prob > F      =  0.0005
    Residual |  1614.30082    15  107.620055           R-squared     =  0.5640
-------------+------------------------------           Adj R-squared =  0.5349
       Total |  3702.47059    16  231.404412           Root MSE      =  10.374

------------------------------------------------------------------------------
         pas |      Coef.   Std. Err.      t    P>|t|     [95% Conf. Interval]
-------------+----------------------------------------------------------------
         qtt |   20.11804   4.567193     4.40   0.001      10.3833    29.85278
       _cons |   79.25533   15.76837     5.03   0.000     45.64585    112.8648
------------------------------------------------------------------------------
\end{verbatim}

\soln{\ref{exo:10.3}}
Puisque les données sont disponibles au format CSV \og classique\fg\
(utilisant la virgule comme séparateur de champ), on peut les importer très
simplement à l'aide de la commande \texttt{insheet}. 
\begin{verbatim}
. insheet using "Framingham.csv"
(10 vars, 4699 obs)
. describe, simple
sex       sbp       dbp       scl       chdfate   followup  age       bmi   month     id
. list in 1/5

     +------------------------------------------------------------------------+
     | sex   sbp   dbp   scl   chdfate   followup   age    bmi   month     id |
     |------------------------------------------------------------------------|
  1. |   1   120    80   267         1         18    55     25       8   2642 |
  2. |   1   130    78   192         1         35    53   28.4      12   4627 |
  3. |   1   144    90   207         1        109    61   25.1       8   2568 |
  4. |   1    92    66   231         1        147    48   26.2      11   4192 |
  5. |   1   162    98   271         1        169    39   28.4      11   3977 |
     +------------------------------------------------------------------------+
\end{verbatim}
Pour faciliter l'interprétation et la lecture des tableaux de résultats, on
associera des étiquettes plus informatives à la variable qualitative \texttt{sex}.
\begin{verbatim}
. label define labsex 1 "M" 2 "F"
. label values sex labsex
. tabulate sex

        sex |      Freq.     Percent        Cum.
------------+-----------------------------------
          M |      2,049       43.61       43.61
          F |      2,650       56.39      100.00
------------+-----------------------------------
      Total |      4,699      100.00
\end{verbatim}
Pour afficher un résumé du nombre de données manquantes pour chacune des
variables de ce tableau de données, on peut utiliser la commande suivante :
\begin{verbatim}
. misstable summarize
                                                               Obs<.
                                                +------------------------------
               |                                | Unique
      Variable |     Obs=.     Obs>.     Obs<.  | values        Min         Max
  -------------+--------------------------------+------------------------------
           scl |        33               4,666  |    248        115         568
           bmi |         9               4,690  |    248       16.2        57.6
  -----------------------------------------------------------------------------
\end{verbatim}
On voit donc que les variables \texttt{scl} et \texttt{bmi} incluent 33 et 9
valeurs manquantes, respectivement. D'où le tableau corrigé des effectifs
par sexe :
\begin{verbatim}
. tabulate sex if bmi < .

        sex |      Freq.     Percent        Cum.
------------+-----------------------------------
          M |      2,047       43.65       43.65
          F |      2,643       56.35      100.00
------------+-----------------------------------
      Total |      4,690      100.00
\end{verbatim}

Sous \Stata, il n'est pas possible d'utiliser des marqueurs transparents dans
un diagramme de dispersion. Dans le cas où le nombre d'observations est
élevé, Il est donc préférable de modifier le type de symbole par défaut
(disque) et d'utiliser de petits cercles (\texttt{oh} ou \texttt{Oh}).
\begin{verbatim}
. scatter sbp bmi, by(sex) msymbol(Oh)
\end{verbatim}

\includegraphics{./figs/stata_overplotting}

Le coefficient de corrélation linéaire entre les variables \texttt{sbp} et
\texttt{bmi} selon le sexe peut être obtenu à l'aide de \texttt{correlate}
après stratification sur le facteur \texttt{sex}. On notera que l'option
\texttt{by} figure en premier et qu'il est nécessaire de trier les données
dans un premier temps, d'où l'ajout de la commande \texttt{sort}
immédiatement après la stratification.
\begin{verbatim}
. by sex, sort: correlate sbp bmi

----------------------------------------------------------------------------------------
-> sex = M
(obs=2047)

             |      sbp      bmi
-------------+------------------
         sbp |   1.0000
         bmi |   0.2364   1.0000


----------------------------------------------------------------------------------------
-> sex = F
(obs=2643)

             |      sbp      bmi
-------------+------------------
         sbp |   1.0000
         bmi |   0.3736   1.0000
\end{verbatim}

% FIXME:
% test correlation by sex

Pour le modèle de régression, on va d'abord transformer les variables
\texttt{bmi} (variable explicative) et \texttt{sbp} (variable réponse) à
l'aide d'une transformation logarithmique.
\begin{verbatim}
. gen logbmi = log(bmi)
(9 missing values generated)
. gen logsbp = log(sbp)
\end{verbatim}
On peut ensuite vérifier visuellement à l'aide d'un histogramme (ou d'un
QQ-plot) que cette transformation a bien permis de ramener les distributions
de ces deux variables proches de la normale. Pour afficher simultanément les
quatre histogrammes, on peut sauver chacune des figures au format \Stata
(\texttt{gph}) et ensuite les combiner à l'aide de la commande \texttt{graph combine}.
\begin{verbatim}
. histogram bmi, saving(gphbmi)
(bin=36, start=16.200001, width=1.1499999)
(file gphbmi.gph saved)
. histogram logbmi, saving(gphlogbmi)
(bin=36, start=2.7850113, width=.03523642)
(file gphlogbmi.gph saved)
. histogram sbp, saving(gphsbp)
(bin=36, start=80, width=5.2777778)
(file gphsbp.gph saved)
. histogram logsbp, saving(gphlogsbp)
(bin=36, start=4.3820267, width=.03378876)
(file gphlogsbp.gph saved)
. graph combine gphbmi.gph gphlogbmi.gph gphsbp.gph gphlogsbp.gph
\end{verbatim}

\includegraphics{./figs/stata_combine}

Le modèle de régression stratifié par sexe ne pose pas de problème majeur,
et contrairement à R il n'est pas nécessaire de calculer les intervalles de
confiance pour les pentes avec une commande séparée car ceux-ci sont
directement fournis dans le tableau de résultats renvoyés par \Stata.
\begin{verbatim}
. regress logsbp logbmi if sex == 1, noheader
------------------------------------------------------------------------------
      logsbp |      Coef.   Std. Err.      t    P>|t|     [95% Conf. Interval]
-------------+----------------------------------------------------------------
      logbmi |    .272646   .0232152    11.74   0.000     .2271182    .3181739
       _cons |   3.988043   .0754584    52.85   0.000      3.84006    4.136026
------------------------------------------------------------------------------

. regress logsbp logbmi if sex == 2, noheader
------------------------------------------------------------------------------
      logsbp |      Coef.   Std. Err.      t    P>|t|     [95% Conf. Interval]
-------------+----------------------------------------------------------------
      logbmi |   .3985947   .0185464    21.49   0.000     .3622278    .4349616
       _cons |   3.593017   .0597887    60.10   0.000     3.475779    3.710254
------------------------------------------------------------------------------
\end{verbatim}

\soln{\ref{exo:10.4}}
Dans un premier temps, il est nécessaire de construire le tableau
d'effectifs donné dans l'énoncé. On s'inspirera de la méthode manuelle vue à
l'exercice~\ref{exo:3.5stata} (p.~\pageref{exo:3.5stata}). Par contre, on va
saisir directement les labels des variables et non des codes
numériques. Pour cela, il est nécessaire d'indiquer à \Stata quel est le
format de ces labels à l'aide de l'instruction \texttt{str} que l'on suffixe
par le nombre de caractères que l'on souhaite utiliser pour le codage des
modalités des variables.
\begin{verbatim}
. clear all
. input str1 traitement str3 infection N

     traitem~t  infection          N
  1. "A" "Non" 157
  2. "B" "Non" 119
  3. "A" "Oui" 52
  4. "B" "Oui" 103
  5. end
. list

     +---------------------------+
     | traite~t   infect~n     N |
     |---------------------------|
  1. |        A        Oui   157 |
  2. |        B        Oui   119 |
  3. |        A        Non    52 |
  4. |        B        Non   103 |
     +---------------------------+
\end{verbatim}


Voici donc le tableau de l'énoncé, avec le test du $\chi^2$ associé :
\begin{verbatim}
. tabulate traitement infection [fweight=N], chi

           |       infection
traitement |       Non        Oui |     Total
-----------+----------------------+----------
         A |       157         52 |       209 
         B |       119        103 |       222 
-----------+----------------------+----------
     Total |       276        155 |       431 

          Pearson chi2(1) =  21.6401   Pr = 0.000
\end{verbatim}
Si l'on souhaite une valeur plus précise pour le degré de significativité du
test, on peut utiliser les informations renvoyées (de manière invisible) par
la commande précédente :
\begin{verbatim}
. return list

scalars:
                  r(N) =  431
                  r(r) =  2
                  r(c) =  2
               r(chi2) =  21.64010539408598
                  r(p) =  3.28902266933e-06
\end{verbatim}
D'où pour le "petit $p$", 
\begin{verbatim}
. display %10.9f r(p)
0.000003289
\end{verbatim}
La commande ci-dessus demande à \Stata d'afficher le résultat sous forme d'un
nombre avec 9 décimales.

Concernant les effectifs théoriques, on utilise exactement la même commande
que précédemment mais en ajoutant l'option \texttt{expected}.
\begin{verbatim}
. tabulate traitement infection [fweight=N], chi expected nofreq

           |       infection
traitement |       Non        Oui |     Total
-----------+----------------------+----------
         A |     133.8       75.2 |     209.0 
         B |     142.2       79.8 |     222.0 
-----------+----------------------+----------
     Total |     276.0      155.0 |     431.0 

          Pearson chi2(1) =  21.6401   Pr = 0.000
\end{verbatim}
Notons que l'on a supprimé l'affichage des effectifs observés grâce à
l'option \texttt{nofreq}.

% FIXME:
% tabodds infection traitement [fw=N], or
Pour calculer la valeur de l'odds-ratio, on utilisera la commande
\texttt{cc}. Toutefois, les commandes "epitab" nécessitent que les variables
soient codées sous forme binaire (0 = non-exposé/non-malade, 1 =
exposé/malade). La saisie des données effectuée à l'étape précédente n'étant
pas compatible avec ce format, il est nécessaire de recoder les données
comme dans l'exercice~\ref{exo:3.5stata}. Par exemple, en utilisant la
commande \texttt{input} :
\begin{verbatim}
. input traitement infection N

     traitem~t  infection          N
  1. 1 0 157
  2. 0 0 119
  3. 1 1 52
  4. 0 1 103
  5. end
. label define tx 0 "Placebo" 1 "Traitement"
. label values traitement tx
. label define ouinon 0 "Non" 1 "Oui"
. label values infection ouinon
\end{verbatim}
Ensuite, il est possible de procéder à l'estimation de l'odds-ratio
\begin{verbatim}
. cc infection traitement [fweight=N], woolf exact

                 | traitement             |             Proportion
                 |   Exposed   Unexposed  |      Total     Exposed
-----------------+------------------------+------------------------
           Cases |        52         103  |        155       0.3355
        Controls |       157         119  |        276       0.5688
-----------------+------------------------+------------------------
           Total |       209         222  |        431       0.4849
                 |                        |
                 |      Point estimate    |    [95% Conf. Interval]
                 |------------------------+------------------------
      Odds ratio |         .3826603       |    .2540087    .5764721 (Woolf)
 Prev. frac. ex. |         .6173397       |    .4235279    .7459913 (Woolf)
 Prev. frac. pop |         .3511679       |
                 +-------------------------------------------------
                                  1-sided Fisher's exact P = 0.0000
                                  2-sided Fisher's exact P = 0.0000
\end{verbatim}
On remarquera que cette fois-ci on a précisé l'option \texttt{exact} pour
obtenir un test de Fisher au lieu de l'approximation par la loi du $\chi^2$
pour le test d'hypothèse sur le tableau.

Si l'on tient compte des données par centre, il est nécessaire de
reconstruire les tableaux d'effectifs, en ne considérant que les marges
colonnes des tableaux d'effectifs donnés dans l'énoncé. La saisie des
données avec \texttt{input} ne pose pas de difficultés particulières.
\begin{verbatim}
. input infection centre N

     infection     centre          N
  1. 0 1 98
  2. 1 1 27
  3. 0 2 152
  4. 1 2 106
  5. 0 3 26
  6. 1 3 22
  7. end
. tabulate infection centre [fweight=N], chi

           |              centre
 infection |         1          2          3 |     Total
-----------+---------------------------------+----------
         0 |        98        152         26 |       276 
         1 |        27        106         22 |       155 
-----------+---------------------------------+----------
     Total |       125        258         48 |       431 

          Pearson chi2(2) =  16.1673   Pr = 0.000
\end{verbatim}
Une fois de plus on adopte le format rapide de saisie des données aggrégées
: 2 variables (épisodes infectieux oui/non, \no\ de centre) et les effectifs
associés au croisement de chacun des niveaux de ces variables. L'option
\texttt{fweight} permet ensuite d'appliquer un test du $\chi^2$ via la
commande \texttt{tabulate} en pondérant le tableau à deux entrées par les
effectifs \texttt{N}. 

Pour réaliser un test de Mantel-Haenszel, il est nécessaire de revenir à
l'ensemble des données (3 tableaux $2\times 2$) et d'utiliser la commande
\texttt{cc} en précisant le facteur de stratification grâce à l'option
\texttt{by}. Voici une manière de procéder, en considérant trois variables
\texttt{tx} (traitement A (1) ou B (0)), \texttt{inf} (infection non (0)/oui (1)) et
\texttt{cen} (centre, 1 à 3). Attention à bien coder les classes des deux
variables de classification en 0 et 1.
\begin{verbatim}
. input tx inf cen N

            tx        inf        cen          N
  1. 1 0 1 51
  2. 1 1 1 8
  3. 0 0 1 47
  4. 0 1 1 19
--%<----
 13. end
\end{verbatim}
On peut vérifier la structure des données à l'aide de la commande
\texttt{table}, en suivant exactement le même principe pour les options
(variables de classification, facteur de pondération et variable de
stratification). On en profitera pour ajouter des étiquettes plus
informatives aux modalités des variables de classification.
\begin{verbatim}
. label define txlab 0 "A" 1 "B"
. label define inflab 0 "Non" 1 "Oui"
. label values tx txlab
. label value inf inflab
. table tx inf [fw=N], by(cen)

----------------------
cen and   |    inf    
tx        |  Non   Oui
----------+-----------
1         |
        A |    8    51
        B |   19    47
----------+-----------
2         |
        A |   35    91
        B |   71    61
----------+-----------
3         |
        A |    9    15
        B |   13    11
----------------------
\end{verbatim}
Concernant le test de Mantel-Haenszel, on obtient les résultats suivants :
\begin{verbatim}
. cc tx inf [freq=N], by(cen)

             cen |       OR       [95% Conf. Interval]   M-H Weight
-----------------+-------------------------------------------------
               1 |   .3880289      .1346123    1.04317        7.752 (exact)
               2 |   .3304442      .1899288   .5726148     25.04264 (exact)
               3 |   .5076923      .1370954   1.857073       4.0625 (exact)
-----------------+-------------------------------------------------
           Crude |   .3826603      .2483246   .5877322              (exact)
    M-H combined |   .3620925      .2379264   .5510569              
-------------------------------------------------------------------
Test of homogeneity (M-H)      chi2(2) =     0.47  Pr>chi2 = 0.7898

                   Test that combined OR = 1:
                                Mantel-Haenszel chi2(1) =     23.01
                                                Pr>chi2 =    0.0000
\end{verbatim}
%
%
%
\soln{\ref{exo:10.5}}
Pour le chargement des données qui se présentent sous la forme de données
tabulées dans un fichier texte sans ligne d'en-tête, on utilisera la
commande \texttt{infile} en spécificiant le nom des variables. On notera que
\Stata indique le nombre de lignes ("observations") lues, ce qui permet de
d'assurer de l'intégrité des données si l'on connaît le nombre
d'observations à l'avance.
\begin{verbatim}
. infile ck pres abs using sck.dat
(13 observations read)
\end{verbatim}

Le nombre total de sujets correspond à la somme des valeurs dans les
variables \texttt{pres} et \texttt{abs}. Le plus simple est donc de faire la
somme de l'ensemble de ces valeurs pour avoir l'effectif total :
\begin{verbatim}
. generate tot = pres+abs
. egen ntot = sum(tot)
. display ntot
360
\end{verbatim}

%% Concernant la représentation des fréquences relatives des variables
%% \texttt{pres} et \texttt(abs}, il est nécessaire de \og construire\fg\ deux
%% nouvelles variables. Voici une solution possible :
%% \begin{verbatim}
%% . quietly: summarize pres
%% . scalar npres = r(sum)
%% . gen ppres = pres / npres
%% \end{verbatim}

Il existe plusieurs commandes pour construire un modèle de régression
logistique sous \Stata. Dans le cas des données dites "groupées", on peut
utiliser la commande \texttt{blogit} qui nécessite de connaître les
effectifs pour chacune des deux classes de la variable binaire à prédire, ou
plus exactement les effectifs de la classe "positive" et les effectifs
totaux. Bien qu'on présente son usage ici, on verra qu'il est préférable
d'utiliser un autre type de commande par la suite.
\footnote{\url{http://www.stata.com/support/faqs/statistics/logistic-regression-with-grouped-data/}}
\begin{verbatim}
. blogit pres tot ck

Logistic regression for grouped data              Number of obs   =        360
                                                  LR chi2(1)      =     283.15
                                                  Prob > chi2     =     0.0000
Log likelihood = -93.886407                       Pseudo R2       =     0.6013

------------------------------------------------------------------------------
    _outcome |      Coef.   Std. Err.      z    P>|z|     [95% Conf. Interval]
-------------+----------------------------------------------------------------
          ck |   .0351044   .0040812     8.60   0.000     .0271053    .0431035
       _cons |  -2.326272   .2993611    -7.77   0.000    -2.913009   -1.739535
------------------------------------------------------------------------------
\end{verbatim}
La lecture de ce type de sortie ne présente pas de problème particulier : on
tiendra compte du fait que le coefficient de régression associé à la
variable d'étude \texttt{ck} est exprimé sur une échelle log odds ; pour
obtenir la valeur de l'odds-ratio associé, il est donc nécessaire de prendre
l'exponentielle de la valeur retournée (.0351044), par exemple
\begin{verbatim}
. display exp(_b[ck])
1.0357278
\end{verbatim}

Pour afficher sur un même graphique les proportions empiriques (malades) et
les valeurs prédites par le modèle (sous forme d'une ligne brisée), il est
nécessaire de calculer ces deux quantités.
\begin{verbatim}
. gen prop = pres/tot
. predict pred, p
. label variable prop "observed"
. label variable pred "predicted"
. graph twoway (line pred ck) (scatter prop ck), ytitle("Probabilité")
\end{verbatim}

\includegraphics{./figs/stata_sck}

La commande \texttt{predict} comprend le nom dela variable (\texttt{pred}
dans laquelle on souhaite stocker les résultats, et le type de prédictions
que l'on souhaite réaliser (\texttt{p}, pour probabilités). Un diagramme de
dispersion composé de deux séries d'instructions délimitées par des
parenthèses est ensuite généré à l'aide de la commande \texttt{twoway}.

La commande \texttt{blogit} permet de travailler avec des données
groupées. Il existe une autre manière de construire des modèles de
régression logistique, pour données individuelles ou groupées. En
particulier, les comamndes \texttt{logit} et \texttt{logistic} fournissent
des résultats additionnels. Dans la suite, on va transformer le jeu de
données initial (données groupées) en données individuelles, puis ré-estimer
les paramètres du même modèle de régression logistique. Les commandes
suivantes permettent de générer autant de lignes qu'il y a d'unités
statistiques (nombre d'observations tel que calculé et stocké dans
\texttt{ntot}) ainsi qu'une variable binaire, \texttt{infct}, codant pour
l'observation de l'événement (malade/non-malade).
\begin{verbatim}
. expand tot
(347 observations created)
. bysort ck: gen infct = _n <= pres
. logit infct ck, nolog

Logistic regression                               Number of obs   =        360
                                                  LR chi2(1)      =     283.15
                                                  Prob > chi2     =     0.0000
Log likelihood = -93.886407                       Pseudo R2       =     0.6013

------------------------------------------------------------------------------
       infct |      Coef.   Std. Err.      z    P>|z|     [95% Conf. Interval]
-------------+----------------------------------------------------------------
          ck |   .0351044   .0040812     8.60   0.000     .0271053    .0431035
       _cons |  -2.326272   .2993611    -7.77   0.000    -2.913009   -1.739535
------------------------------------------------------------------------------
\end{verbatim}

Parmi les commandes de post-estimation disponible, la commande
\texttt{classification} fournit automatiquement un tableau de synthèse des
unités correctement ou incorrectement classées selon le critère choisi (par
défaut, $P(\texttt{malade})>0.5$).
\begin{verbatim}
. estat classification

Logistic model for infct

              -------- True --------
Classified |         D            ~D  |      Total
-----------+--------------------------+-----------
     +     |       215            16  |        231
     -     |        15           114  |        129
-----------+--------------------------+-----------
   Total   |       230           130  |        360

Classified + if predicted Pr(D) >= .5
True D defined as infct != 0
--------------------------------------------------
Sensitivity                     Pr( +| D)   93.48%
Specificity                     Pr( -|~D)   87.69%
Positive predictive value       Pr( D| +)   93.07%
Negative predictive value       Pr(~D| -)   88.37%
--------------------------------------------------
False + rate for true ~D        Pr( +|~D)   12.31%
False - rate for true D         Pr( -| D)    6.52%
False + rate for classified +   Pr(~D| +)    6.93%
False - rate for classified -   Pr( D| -)   11.63%
--------------------------------------------------
Correctly classified                        91.39%
--------------------------------------------------
\end{verbatim}

% \begin{verbatim}
% . estat classification, cutoff(.15)

% Logistic model for infct

%               -------- True --------
% Classified |         D            ~D  |      Total
% -----------+--------------------------+-----------
%      +     |       228            42  |        270
%      -     |         2            88  |         90
% -----------+--------------------------+-----------
%    Total   |       230           130  |        360

% Classified + if predicted Pr(D) >= .15
% True D defined as infct != 0
% --------------------------------------------------
% Sensitivity                     Pr( +| D)   99.13%
% Specificity                     Pr( -|~D)   67.69%
% Positive predictive value       Pr( D| +)   84.44%
% Negative predictive value       Pr(~D| -)   97.78%
% --------------------------------------------------
% False + rate for true ~D        Pr( +|~D)   32.31%
% False - rate for true D         Pr( -| D)    0.87%
% False + rate for classified +   Pr(~D| +)   15.56%
% False - rate for classified -   Pr( D| -)    2.22%
% --------------------------------------------------
% Correctly classified                        87.78%
% --------------------------------------------------
% \end{verbatim}

%% FIXME:
%% Write me!

ROC curve
\begin{verbatim}
. lroc

Logistic model for infct

number of observations =      360
area under ROC curve   =   0.9593
\end{verbatim}

\includegraphics{./figs/stata_sck2}

Sensibilité/spécificité
\begin{verbatim}
. lsens
\end{verbatim}

\includegraphics{./figs/stata_sck3}

% 
%
%
\soln{\ref{exo:10.6}}
Les données ont été sauvegardées dans un format compact (3 colonnes
indiquant la présence ou non d'un cancer, le niveau de consommation
d'alcool, et les effectifs associés).
\begin{verbatim}
. insheet using "cc_oesophage.csv", clear
. label define yesno 0 "No" 1 "Yes" 
. label values cancer yesno 
. label define dose 0 "< 80g" 1 ">= 80g"
. label values alcohol dose
. list

     +-----------------------------+
     | cancer   alcohol   patients |
     |-----------------------------|
  1. |     No     < 80g        666 |
  2. |    Yes     < 80g        104 |
  3. |     No    >= 80g        109 |
  4. |    Yes    >= 80g         96 |
     +-----------------------------+
\end{verbatim}
Les commandes précédentes permettent de charger le fichier de données et
d'associer aux modalités des variables (\texttt{cancer} et \texttt{alcohol})
des noms plus informatifs. Pour obtenir le nombre total de patients, on peut
utiliser la commande suivante :
\begin{verbatim}
. egen ntot = sum(patients)
. display ntot
975
\end{verbatim}

La proportion d'individus à risque, c'est-à-dire ayant une consommation
journalière d'alcool $\ge 80$ g s'obtient à partir d'un simple tableau
d'effectifs croisant les variables \texttt{cancer} et \texttt{alcohol} (il
faut indiquer comment remplir les cellules en ajoutant une option
\texttt{weight}), et en demandant les profils lignes, c'est-à-dire les les
fréquences relatives par ligne.
\begin{verbatim}
. tabulate cancer alcohol [fweight=patients], row

+----------------+
| Key            |
|----------------|
|   frequency    |
| row percentage |
+----------------+

           |        alcohol
    cancer |     < 80g     >= 80g |     Total
-----------+----------------------+----------
        No |       666        109 |       775 
           |     85.94      14.06 |    100.00 
-----------+----------------------+----------
       Yes |       104         96 |       200 
           |     52.00      48.00 |    100.00 
-----------+----------------------+----------
     Total |       770        205 |       975 
           |     78.97      21.03 |    100.00
\end{verbatim}

Le calcul de l'odds-ratio peut se faire à l'aide de l'une des commandes
\Stata dites "immédiates", \texttt{cc}, en gardant à l'esprit qu'il faut bien
tenir compte de la colonne \texttt{patients}, comme précédemment :
\begin{verbatim}
. cc cancer alcohol [fweight=patients], woolf

                 | alcohol                |             Proportion
                 |   Exposed   Unexposed  |      Total     Exposed
-----------------+------------------------+------------------------
           Cases |        96         104  |        200       0.4800
        Controls |       109         666  |        775       0.1406
-----------------+------------------------+------------------------
           Total |       205         770  |        975       0.2103
                 |                        |
                 |      Point estimate    |    [95% Conf. Interval]
                 |------------------------+------------------------
      Odds ratio |         5.640085       |    4.000589    7.951467 (Woolf)
 Attr. frac. ex. |         .8226977       |    .7500368    .8742371 (Woolf)
 Attr. frac. pop |         .3948949       |
                 +-------------------------------------------------
                               chi2(1) =   110.26  Pr>chi2 = 0.0000
\end{verbatim}

Si l'on dispose du tableau d'effectifs, c'est-à-dire la répartition des 975
sujets dans les quatre cellules du tableau croisant l'exposition à l'alcool
et le statut cas-témoin, il est également possible d'utiliser le calculateur
pour odds-ratio dans les études cas-témoins accessible par le menu
\textsf{Statistics} $\rhd$ \textsf{Epidemiology and related} $\rhd$
\textsf{Tables for epidemiologists}.

\includegraphics{./figs/stata_ORcalculator}

Pour tester l'hypothèse que la proportion de personnes avec une consommation
journalière d'alcool $\ge 80$ g est identique chez les cas ($p_1$) et les
témoins ($p_1$), on peut procéder comme suit :
\begin{verbatim}
. prtesti 96 0.4800 109 0.1406

Two-sample test of proportions                     x: Number of obs =       96
                                                   y: Number of obs =      109
------------------------------------------------------------------------------
    Variable |       Mean   Std. Err.      z    P>|z|     [95% Conf. Interval]
-------------+----------------------------------------------------------------
           x |        .48   .0509902                      .3800611    .5799389
           y |      .1406   .0332949                      .0753433    .2058567
-------------+----------------------------------------------------------------
        diff |      .3394   .0608978                      .2200424    .4587576
             |  under Ho:   .0641131     5.29   0.000
------------------------------------------------------------------------------
        diff = prop(x) - prop(y)                                  z =   5.2938
    Ho: diff = 0

    Ha: diff < 0                 Ha: diff != 0                 Ha: diff > 0
 Pr(Z < z) = 1.0000         Pr(|Z| < |z|) = 0.0000          Pr(Z > z) = 0.0000
\end{verbatim}

Une autre façon de tester cette hypothèse consiste à remarquer que
l'hypothèse précédente, $H_0:\, \pi_0=\pi_1$, n'est vraie que si
l'odds-ratio vaut 1. D'où l'idée d'exploiter directement le test du $\chi^2$
pour l'odds-ratio donné par la commande \texttt{cc} (ici, $\chi^2=110.26$,
$p<0.001$). 

Le modèle de régression logistique, comme les autres modèles de régression
sous \Stata, se formule ainsi : 

\begin{verbatim}
. logistic alcohol cancer [freq=patients]

Logistic regression                               Number of obs   =        975
                                                  LR chi2(1)      =      96.43
                                                  Prob > chi2     =     0.0000
Log likelihood =  -453.2224                       Pseudo R2       =     0.0962

------------------------------------------------------------------------------
     alcohol | Odds Ratio   Std. Err.      z    P>|z|     [95% Conf. Interval]
-------------+----------------------------------------------------------------
      cancer |   5.640085   .9883492     9.87   0.000     4.000589    7.951467
       _cons |   .1636637   .0169104   -17.52   0.000     .1336604    .2004018
------------------------------------------------------------------------------
\end{verbatim}

Par défaut, \Stata présente les résultats (coefficients du modèle) sous forme
d'odds-ratio, avec leurs intervalles de confiance associés. Si l'on souhaite
obtenir directement les coefficients de régression (sur l'échelle du log
odds), il faut utiliser la commande \texttt{logit} après avoir utilisé
\texttt{logistic}. On pourrait également utiliser directement une commande de
type \verb|logistic alcohol cancer [freq=patients]|.
\begin{verbatim}
. logit

Logistic regression                               Number of obs   =        975
                                                  LR chi2(1)      =      96.43
                                                  Prob > chi2     =     0.0000
Log likelihood =  -453.2224                       Pseudo R2       =     0.0962

------------------------------------------------------------------------------
     alcohol |      Coef.   Std. Err.      z    P>|z|     [95% Conf. Interval]
-------------+----------------------------------------------------------------
      cancer |   1.729899   .1752366     9.87   0.000     1.386442    2.073356
       _cons |  -1.809942   .1033238   -17.52   0.000    -2.012453   -1.607431
------------------------------------------------------------------------------
\end{verbatim}

%---------------------------------------------------------------- Séance 11 --
\chapter*{Semaine 11\markboth{Corrigés de la semaine 11}{}}

\soln{\ref{exo:11.2}}
Le fichier de données est un fichier texte avec des tabulations comme
séparateur de champ. On peut l'importer sous \Stata en utilisant la commande
\texttt{insheet}. Pour afficher le nom des variables après importation, il
suffit d'utiliser \texttt{describe} avec l'option \texttt{simple}.
\begin{verbatim}
. insheet using "pbc.txt", tab
(28 vars, 312 obs)
. describe, simple
number    rx        asictes   spiders   bilirub   albumin   alkphos   trigli
prothrom  age       sample    logalbu   _st       _t
status    sex       hepatom   edema     cholest   copper    sgot      platel
histol    years     logbili   logprot   _d        _t0
\end{verbatim}
Après recodage des étiquettes des variables \texttt{rx} et \texttt{sex}, 
\begin{verbatim}
. label define trt 1 "Placebo" 2 "DPCA"
. label define sexe 0 "M" 1 "F"
. label values rx trt
. label values sex sexe
\end{verbatim}
on peut vérifier la proportion de patients décédés (\texttt{status}, 0 =
vivant et 1 = décédé) et leur répartition selon le groupe de traitement à
l'aide de tris simple et croisé. Pour le tri croisé, on ajoutera l'option
\texttt{row} pour obtenir les fréquences relatives par status.
\begin{verbatim}
. tabulate status

     status |      Freq.     Percent        Cum.
------------+-----------------------------------
          0 |        187       59.94       59.94
          1 |        125       40.06      100.00
------------+-----------------------------------
      Total |        312      100.00

. tabulate status rx, row

+----------------+
| Key            |
|----------------|
|   frequency    |
| row percentage |
+----------------+

           |          rx
    status |   Placebo       DPCA |     Total
-----------+----------------------+----------
         0 |        93         94 |       187 
           |     49.73      50.27 |    100.00 
-----------+----------------------+----------
         1 |        65         60 |       125 
           |     52.00      48.00 |    100.00 
-----------+----------------------+----------
     Total |       158        154 |       312 
           |     50.64      49.36 |    100.00
\end{verbatim}

Pour afficher la distribution des temps de suivi, on utilisera un simple
diagramme de dispersion comme on l'a vu dans le cas de \R. Pour faire
appraître distinctement les observations selon le status (0 ou 1), on
pourrait très bien superposer deux séries de points sur le même graphique
(cf. exercice~5.2, p.~\pageref{para:twoway}). Voici une autre manière de
procéder :
\begin{verbatim}
. separate number, by(status)

              storage  display     value
variable name   type   format      label      variable label
-----------------------------------------------------------------------------------------
number0         int    %8.0g                  number, status == 0
number1         int    %8.0g                  number, status == 1

. twoway scatter number0 number1 years, msymbol(S O)
\end{verbatim}
La première commande permet en fait de séparer les numéros de patients en
fonction du status afin d'afficher les deux séries d'observation en fonction
de la durée de suivi en années.

\includegraphics{./figs/stata_followup}

La médiane de la durée de suivi par groupe de traitement peut s'obtenir avec
la commande \texttt{tabstat} en opérant par groupe grâce à l'option
\texttt{by}. 
\begin{verbatim}
. tabstat years, by(rx) stats(median) nototal

Summary for variables: years
     by categories of: rx 

     rx |       p50
--------+----------
Placebo |    5.1882
   DPCA |   4.95825
-------------------
\end{verbatim}

Le nombre de décès enregistrés au-delà de 10.5 années de suivi s'obtient
avec un simple tri à plat par la commande \texttt{tabulate} :
\begin{verbatim}
. tabulate status if years > 10.49

     status |      Freq.     Percent        Cum.
------------+-----------------------------------
          0 |         23       85.19       85.19
          1 |          4       14.81      100.00
------------+-----------------------------------
      Total |         27      100.00
\end{verbatim}
de même que le sexe des patients décédé après cette période :
\begin{verbatim}
. tabulate sex if years > 10.49 & status == 1

        sex |      Freq.     Percent        Cum.
------------+-----------------------------------
          M |          2       50.00       50.00
          F |          2       50.00      100.00
------------+-----------------------------------
      Total |          4      100.00
\end{verbatim}

Concernant l'analyse des patients transplantés, on peut également procéder
comme on l'a fait avec R, c'est-à-dire restreindre le tableau de données à
ces seuls patients. Comme \Stata ne permet de travailler qu'avec un seul
tableau de données à la fois, il est toutefois nécessaire de sauvegarder
temporairement les données actuelles avant de créer un nouveau tableau.
\begin{verbatim}
. preserve
. egen idx = anymatch(number), values(5 105 111 120 125 158 183 241 246 247 254 263 264
  265 274 288 291 295 297 345 361 362 375 380 383)
. keep if idx
(293 observations deleted)
. gen days = years*365
. tabstat age sex days, stats(mean median sum)

   stats |       age       sex      days
---------+------------------------------
    mean |  41.17568  .8421053  1507.177
     p50 |   40.9008         1  1434.012
     sum |  782.3379        16  28636.37
----------------------------------------
\end{verbatim}
La première commande permet de construire une liste des individus que l'on
souhaite utiliser pour filtrer le tableau de données d'origine (sur la base
des numéros de sujet contenus dans la variable \texttt{number}). Ensuite, on
applique le calcul des statistiques descriptives à l'aide d'une commande
\texttt{tabstat}. Une fois les calculs terminés, on peut restorer les
données d'origine de la manière suivante :
\begin{verbatim}
. restore
\end{verbatim}

Comme R, \Stata utilise ses propres conventions pour le codage des données de
survie. Les commandes essentielles sont ainsi : \texttt{stset} pour définir
la façon dont les événements sont enregistrés et le temps d'observation,
\texttt{sts} pour calculer un tableau de survie à partir de l'estimateur de
Kaplan-Meier. Voici comment appliquer ces commandes pour construire le
tableau et la courbe de survie, sans considération du facteur traitement.
\begin{verbatim}
. stset years, failure(status)

     failure event:  status != 0 & status < .
obs. time interval:  (0, years]
 exit on or before:  failure

------------------------------------------------------------------------------
      312  total obs.
        0  exclusions
------------------------------------------------------------------------------
      312  obs. remaining, representing
      125  failures in single record/single failure data
 1713.854  total analysis time at risk, at risk from t =         0
                             earliest observed entry t =         0
                                  last observed exit t =   12.4736

. sts list
\end{verbatim}
La deuxième commande affiche le tableau demandé. Pour la courbe de survie,
on utilisera :
\begin{verbatim}
. sts graph, ci censored(single)
\end{verbatim}
L'option \texttt{ci} permet d'afficher l'intervalle de confiance à 95~\%
pour l'estimateur de KM.

\includegraphics{./figs/stata_stsgraph}

% FIXME:
% finish me: (f)-(i)

\soln{\ref{exo:11.2}}
Le format texte du fichier de données \texttt{prostate.dat} est identique à
celui du fichier \texttt{pbc.txt} de l'exercice précédent, à ceci près que
les champs sont séparés par un simple espace. On utilisera donc la commande
\texttt{insheet} :
\begin{verbatim}
. insheet using "prostate.dat", delimiter(" ")
(7 vars, 38 obs)
. list in 1/5

     +--------------------------------------------------------+
     | treatm~t   time   status   age   haem   size   gleason |
     |--------------------------------------------------------|
  1. |        1     65        0    67   13.4     34         8 |
  2. |        2     61        0    60   14.6      4        10 |
  3. |        2     60        0    77   15.6      3         8 |
  4. |        1     58        0    64   16.2      6         9 |
  5. |        2     51        0    65   14.1     21         9 |
     +--------------------------------------------------------+
\end{verbatim}
Le nombre de patients vivants à la date de point est obtenu à partir de
\texttt{tabulate} :
\begin{verbatim}
. tabulate status

     Status |      Freq.     Percent        Cum.
------------+-----------------------------------
          0 |         32       84.21       84.21
          1 |          6       15.79      100.00
------------+-----------------------------------
      Total |         38      100.00
\end{verbatim}
soit 32 personnes encore en vie à la fin de la durée de suivi.

Comme dans l'exercice~7.1, il est nécessaire d'indiquer à \Stata quelle
variable sert à identifier les événements (\texttt{status}) et le temps
(\texttt{time}). On utilisera la commande \texttt{stset} de la manière
suivante :
\begin{verbatim}
. stset time, failure(status)

     failure event:  status != 0 & status < .
obs. time interval:  (0, time]
 exit on or before:  failure

------------------------------------------------------------------------------
       38  total obs.
        0  exclusions
------------------------------------------------------------------------------
       38  obs. remaining, representing
        6  failures in single record/single failure data
     1890  total analysis time at risk, at risk from t =         0
                             earliest observed entry t =         0
                                  last observed exit t =        70
\end{verbatim}

La médiane de survie selon le traitement est disponible à partir de la
commande \texttt{stci}, en précisant le percentile d'intérêt (ici,
\texttt{p(50)}). 
\begin{verbatim}
. stci, by(treatment) p(50)

         failure _d:  status
   analysis time _t:  time

             |    no. of 
treatment    |  subjects         50%     Std. Err.     [95% Conf. Interval]
-------------+-------------------------------------------------------------
           1 |        18          69      .3043484           42          .
           2 |        20           .             .            .          .
-------------+-------------------------------------------------------------
       total |        38           .             .           69          .
\end{verbatim}

% FIXME:
% différence proportion survie à 50 mois (b)

Pour afficher les courbes de survie pour chacun des bras de traitement, on
utilisera la commande \texttt{sts graph} en spécifiant le facteur de
classification à l'aide de l'option \texttt{by}. L'affichage des données
censurées se fait à l'aide de l'option \texttt{censored}.
\begin{verbatim}
. sts graph, by(treatment) censored(s)
\end{verbatim}

\includegraphics{./figs/stata_ststwoway}

Le test du log-rank s'effectue avec la commande \texttt{sts test}. Notons
que l'on n'a besoin de spécifier que le facteur traitement.
\begin{verbatim}
. sts test treatment

         failure _d:  status
   analysis time _t:  time


Log-rank test for equality of survivor functions

          |   Events         Events
treatment |  observed       expected
----------+-------------------------
1         |         5           2.47
2         |         1           3.53
----------+-------------------------
Total     |         6           6.00

                chi2(1) =       4.42
                Pr>chi2 =     0.0355
\end{verbatim}

