%---------------------------------------------------------------- Séance 08 --
\chapter*{Semaine 8\markboth{Corrigés de la semaine 8}{}}

\soln{\ref{exo:8.1}}
Pour saisir les données sous \SAS, on effectue dans un premier temps une
étape \texttt{DATA} dans laquelle les données individuelles sont saisie
manuellement à l'aide de la commande \texttt{input} pour indiquer le nom de
la variable, suivi de \texttt{cards} qui place les données directement dans
le programme \SAS.
\begin{verbatim}
DATA Exercice1_1;
INPUT x;
CARDS;
3.68
2.21
2.45
8.64
4.32
3.43
5.11
3.87
;
RUN;
\end{verbatim}

Le nombre d'observations ainsi que les valeurs des observations extrêmes de
la distribution empirique sont obtenues à partir de la commande \texttt{PROC
  SUMMARY}, en précisant les options correspondantes (\texttt{n},
\texttt{min} et \texttt{max} et \texttt{range}).
\begin{verbatim}
PROC SUMMARY DATA=Exercice1_1 PRINT n min max range; VAR x; RUN;
\end{verbatim}

Comme le jeu de données est limité en termes d'effectifs, le plus simple
consiste à modifier l'étape \texttt{DATA} précédente en effectuant les
modifications demandées : (a) remplacement de la valeur 8.64 par 3.64, et
(b) recodage de la valeur 5.11 en point (".") qui est le symbole par défaut
utilisé par \SAS pour représenter les données manquantes.
\begin{verbatim}
DATA Exercice1_1;
INPUT x;
CARDS;
3.68
2.21
2.45
3.64    /* (a) */
4.32
3.43
.       /* (b) */
3.87
;
RUN;
\end{verbatim}
%
%
%
\soln{\ref{exo:8.2}}
La saisie des données s'effectue de la même manière qu'à l'exercice
précédent, c'est-à-dire à partir d'une étape \texttt{DATA} :
\begin{verbatim}
DATA Exercice1_2;
INPUT X;
detect=1;
IF X <= log10(50) THEN detect=0;
CARDS;
3.64
2.27
1.43
1.77
4.62
3.04
1.01
2.14
3.02
5.62
5.51
5.51
1.01
1.05
4.19
2.63
4.34
4.85
4.02
5.92
;
RUN;
\end{verbatim}
On notera une petite différence par rapport à l'exercice~\ref{exo:1.1}, à
savoir que l'on définit également une variable \texttt{detect} qui prend la
valeur 0 si $X\le \log_{10}(50)$ et 1 autrement. Ceci est effectué avant
l'instruction \texttt{cards} qui signifie le début de la sauvegarde des
données numériques en mémoire.

Le nombre de patients avec une charge virale indétectable s'obtient à partir
d'un simple tri à plat de la variable \texttt{detect} grâce à la commande
\texttt{PROC FREQ}.
\begin{verbatim}
PROC FREQ DATA=exercice1_2; TABLES detect; RUN;
\end{verbatim}

Pour calculer la charge virale médiane pour les seules observations valides
(c'est-à-dire ayant une valeur au-dessus du seuil limite de détection), il
est nécessaire de sélectionner les unités statistiques remplissant les
conditions de validité et d'utiliser la commande \texttt{PROC SUMMARY}.
\begin{verbatim}
DATA detect; SET exercice1_2;
Y=exp(X*log(10));
IF detect=1; RUN;
PROC SUMMARY DATA=detect PRINT median; VAR Y; RUN;
\end{verbatim}
%
%
%
\soln{\ref{exo:8.3}}
Pour lire les données contenues dans le fichier \texttt{anorexia.data} dont
un aperçu est fourni ci-dessous
\begin{verbatim}
Group Before After
g1 80.5  82.2
g1 84.9  85.6
g1 81.5  81.4
g1 82.6  81.9
g1 79.9  76.4
\end{verbatim}
on va utiliser la commande \texttt{infile} dans l'étape \texttt{DATA}, en
précisant que la lecture des données doit commencer à la 2\ieme\ ligne
(\texttt{firstobs=2}) et que les données sont séparées par des espaces (en
nombre variable).
\begin{verbatim}
DATA anorexia;
INFILE "C:\data\anorexia.dat" firstobs=2 dlm="09"X;
INPUT  groupe $ 1-2  before 3-7   after 8-13;
RUN;
\end{verbatim}
On peut très bien associer des étiquettes plus informatives aux modalités
prises par la variable qualitative \texttt{Group} à l'aide d'une étape
\texttt{FORMAT} :
\begin{verbatim}
PROC FORMAT;
VALUE Therapie
  1='Thérapie comportementale'
  2='Thérapie familiale'
  3='Thérapie contrôle'
;
RUN;
\end{verbatim}

Pour obtenir les effectifs par type de thérapie, on utilise un simple tri à
plat à l'aide de la commande \texttt{PROC FREQ}. L'affichage des résultats
peut être personnalisé en utilisant le renommage des groupes réalisé à
l'étape précédente.
\begin{verbatim}
PROC FREQ DATA=anorexia; TABLES groupe; FORMAT groupe therapie.; RUN;
\end{verbatim}

La transformation d'unités pour les poids ne pose pas de problème
spécifique, mais il faut décider si l'on crée de nouvelles variables ou si
l'on remplace les valeurs existantes. Ici, on créera deux nouvelles
variables, \verb|before_kg| et \verb|after_kg| :
\begin{verbatim}
DATA anorexia; SET anorexia;
before_kg=before/2.2;
after_kg=after/2.2;
RUN;
\end{verbatim}

Pour les scores de différences, on crée également une nouvelles variable
à l'aide d'une étape \texttt{DATA} :
\begin{verbatim}
DATA anorexia; SET anorexia;
diff=after_kg-before_kg;
RUN;
\end{verbatim}
et l'on résumé la distribution des scores de différence (moyenne et étendue)
par groupe de traitement à l'aide de \texttt{PROC SUMMARY}. On notera que
puisque l'on fait intervenir une variable de groupement, il est nécessaire
dans un premier temps de trier les données par classe (variable
\texttt{groupe}). 
\begin{verbatim}
PROC SORT DATA=anorexia; BY groupe; RUN;

PROC SUMMARY DATA=anorexia PRINT n mean min max range; VAR diff; BY groupe; FORMAT groupe therapie.; RUN;
\end{verbatim}
%
%
%
\soln{\ref{exo:8.4}}
La saisie des données brutes se fera comme dans les exercices~8.1 et 8.2 à
l'aide d'une étape \texttt{DATA} (commande \texttt{cards}).
\begin{verbatim}
DATA X;
INPUT X;
CARDS;
24.9
25.0
25.0
25.1
25.2
25.2
25.3
25.3
25.3
25.4
25.4
25.4
25.4
25.5
25.5
25.5
25.5
25.6
25.6
25.6
25.7
25.7
25.8
25.8
25.9
26.0
;
RUN;
\end{verbatim}

Les indicateurs de tendance centrale (moyenne, médiane et mode) peuvent être
calculés et affichés à partir de la commande \texttt{PROC SUMMARY}.
\begin{verbatim}
PROC SUMMARY DATA=X PRINT mean median mode; VAR X; RUN;
\end{verbatim}

Quant à la variance, on changera simplement les options de calcul dans
\texttt{PROC SUMMARY} :
\begin{verbatim}
PROC SUMMARY DATA=X PRINT var; VAR X; RUN;
\end{verbatim}

Pour le recodage en 4 classes d'intervalles pré-définis pour la variable
\texttt{X}, on crée une variable auxiliaire \verb|X_classes| à laquelle on
affecte les valeurs 1, 2, 3 et 4 selon les valeurs prises par \texttt{X}.
\begin{verbatim}
DATA X; SET X;
/* 24.9-25.1, 25.2-25.4, 25.5-25.7, 25.8-26.0 */

X_classes=1;
IF X ge 25.2 THEN X_classes=2;
IF X ge 25.5 THEN X_classes=3;
IF X ge 25.8 THEN X_classes=4;
RUN;
\end{verbatim}
Le tableau d'effectifs peut s'obtenir directement avec une commande
\texttt{tables} dans \texttt{PROC FREQ} :
\begin{verbatim}
PROC FREQ DATA=X; TABLES X_classes; RUN;
\end{verbatim}
Pour faciliter la lecture des résultats, on peut tout à fait associer des
étiquettes plus informatives aux quatre classes avec \texttt{PROC FORMAT}
et associer celles-ci au tableau de résultat renvoyé par \texttt{tables}.
\begin{verbatim}
PROC FORMAT;
    VALUE classes   24.9-25.1="Intervalle 1"
                    25.2-25.4="Intervalle 2"
                    25.5-25.7="Intervalle 3"
                    25.8-26.0="Intervalle 4";
RUN;

PROC FREQ DATA=X; TABLES X; FORMAT X classes.; RUN;
\end{verbatim}

Enfin, pour la représentation sous forme d'histogramme, \SAS dispose de ses
propres algorithmes de calcul pour la détermination du nombre de classes à
construire, tout comme R. Tout est géré à partir de la commande
\texttt{PROC GCHART}.
\begin{verbatim}
PROC GCHART DATA=X; Hbar X; RUN;  
\end{verbatim}

On pourrait également utiliser un diagramme en barres orientées
verticalement en précisant l'option \texttt{Vbar}.
\begin{verbatim}
PROC GCHART DATA=X; Vbar X; RUN;
\end{verbatim}
%
%
%
\soln{\ref{exo:8.5}}
Pour importer les données stockées dans un simple fichier texte, on utilise
la commande \verb|infile| dans une étape \texttt{DATA}.
\begin{verbatim}
DATA elderly;
INFILE  "C:\data\elderly.dat" dlm="09"X ;
INPUT  taille @@ ;
RUN;
\end{verbatim}
Ici, on notera que l'on ne précise pas la manière dont sont codées les
valeurs manquantes car le "." est le format utilisé par défaut par \SAS. une
autre solution consisterait à introduire une commande \texttt{datalines}
après la commande \texttt{input} et à copier/coller les données du fichier
texte. 

Le nombre total d'observations manquantes peut s'obtenir à partir de
\texttt{PROC SUMMARY} en spécifiant l'option \texttt{nmiss}.
\begin{verbatim}
PROC SUMMARY DATA=elderly PRINT nmiss; VAR taille; RUN;
\end{verbatim}

Pour obtenir la taille moyenne et son intervalle de confiance à 95~\%
associé, il suffit d'entrer la commande suivante :
\begin{verbatim}
PROC SUMMARY DATA=elderly PRINT clm uclm lclm; VAR taille; RUN;
\end{verbatim}

Enfin, pour afficher la distribution des tailles sous forme d'une courbe de
densité, on utilise la commande \texttt{PROC UNIVARIATE} et
\texttt{HISTOGRAM} avec l'option \texttt{/ kernel}. Le degré de lissage peut
être contrôlé à l'aide de l'option \texttt{C}.
\begin{verbatim}
PROC UNIVARIATE DATA=elderly; VAR taille; HISTOGRAM taille / kernel; RUN;
\end{verbatim}
%
%
%
\soln{\ref{exo:8.6}}
Les données sur les poids à la naissance de Hosmer \& Lemeshow (1989) ont
été exportées au format texte dans le fichier \texttt{birthwt.dat}, et elles
peuvent être importées ainsi :
\begin{verbatim}
PROC IMPORT OUT= WORK.BIRTHWT
            DATAFILE= "C:\data\birthwt.dat"
            DBMS=DLM REPLACE;
     DELIMITER='20'x;
     GETNAMES=NO;
     DATAROW=1;
RUN;

DATA birthwt1; SET birthwt;
  low = var1;
  age = var2;
  lwt = var3 ;
  race = var4 ;
  smoke = var5;
  ptl = var6 ;
  ht = var7 ;
  ui = var8 ;
  ftw = var9 ;
  bwt = var10 ;
  DROP var1-var10;
RUN;
\end{verbatim}
La première étape (\texttt{PROC IMPORT}) consiste à définir les options pour
l'importation des données : les noms de variables ne figurent pas dans le
fichier de données (\texttt{GETNAMES=NO}) et la lecture des données doit
commencer dès la 1\iere\ ligne du fichier (\texttt{DATAROW=1}), sachant que
les données individuelles sont séparées par des espaces. La seconde étape
(\texttt{DATA}) consiste à assigner des noms de variables au tableau de
données \texttt{WORK.BIRTHWT} généré par \SAS. On notera qu'il est
nécessaire de supprimer les anciens noms de variable après renommage, d'où
l'usage de la commande \texttt{DROP}.

Pour associer de nouvelles étiquettes aux variables \texttt{low},
\texttt{race}, \texttt{smoke}, \texttt{ui} et \texttt{ht}, il suffit de
définir des "labels" et de les associer aux variables en question (cela ne
change le format de représentation des variables en mémoire, qui restent
stockées sous forme de nombre).
\begin{verbatim}
PROC FORMAT;
      VALUE  low 1="Poids inférieur à 2.5 Kg"
                 0="Poids supérieur à 2.5 Kg";
      VALUE  ethnicite 1="White"
                       2="Black"
                       3="Other";
      VALUE  tabac 1="consommation tabac durant grossesse"
                   0="Pas de consommation tabac durant grossesse";
      VALUE  Hypert 1="Antecedent d hypertension"
                    0="Pas d antecedent d hypertension";
      VALUE  uterine 1="Manisfestation d irritabilite uterine"
                     0="Pas de manisfestation d irritabilite uterine";
RUN;
\end{verbatim}
On pourra vérifier que ces modifications ont bien été prises en compte en
utilisant une comamnde telle que \texttt{PROC FREQ} pour afficher les
tableaux d'effectifs associés à ces variables qualitatives.
\begin{verbatim}
PROC FREQ DATA=birthwt1; 
  TABLES low race smoke ui ht;
  FORMAT low low. race ethnicite. smoke tabac. ht hypert. ui uterine.;
RUN;
\end{verbatim}

La conversion du poids des mères en \emph{kg} ne pose pas de problème
particulier, et ici on remplacera directement les données disponibles dans
une étape \texttt{DATA}.
\begin{verbatim}
DATA birthwt2; SET birthwt1;
  lwt=lwt/2.2; 
RUN;
\end{verbatim}

Les indicateurs de tendance centrale et de dispersion relative sont obtenus
à partir de la commande \texttt{PROC SUMMARY}, en spécifiant les options
adéqautes. 
\begin{verbatim}
PROC SUMMARY DATA=birthwt2 PRINT mean median qrange; VAR lwt; RUN;
\end{verbatim}

Enfin, un histogramme du poids des mères est construit à l'aide de la
commande \texttt{PROC GCHART}. Ici, sans autre option un histogramme de
densité sera construit.
\begin{verbatim}
PROC GCHART DATA=birthwt2; Hbar lwt; RUN;

PROC UNIVARIATE DATA=birthwt2;
  VAR lwt;
  HISTOGRAM lwt / kernel;
RUN;
\end{verbatim}

Concernant la proportion de mères ayant fumé durant la grossesse et
le calcul de l'intervalle de confiance à 95~\% associé, on peut utiliser la
commande \texttt{PROC SUMMARY}, qui comme dans le cas de la commande \R
\texttt{prop.test} suppose de grands échantillons :
\begin{verbatim}
PROC SUMMARY DATA=birthwt2 PRINT clm uclm lclm; VAR smoke; RUN;
\end{verbatim}

Les diagrammes en barres peuvent être affichés, selon une orientation
verticale ou horizontale, à l'aide de \texttt{PROC GCHART}, avec une syntaxe
identique au cas des histogrammes pour des variables numériques.
\begin{verbatim}
PROC GCHART DATA=birthwt2; Vbar smoke; RUN;
PROC GCHART DATA=birthwt2; Hbar smoke; RUN;
\end{verbatim}

Pour générer les terciles, on classera dans un premier temps les
observations par ordre croissant, puis on découpera les valeurs en trois
sous-effectifs égaux (de taille $189/3=63$).
\begin{verbatim}
/* PROC UNIVARIATE DATA=birthwt2; VAR lwt; RUN; */
PROC RANK DATA=birthwt2 OUT=ranking; VAR age; RANKS ordre; RUN;

DATA birthwt3; SET ranking;
age_classe=1;
IF ordre gt 189/3 THEN age_classe=2; 
IF ordre gt 2*(189/3) THEN age_classe=3;
RUN;
\end{verbatim}
Le croisement de cette nouvelle variable avec la variable indicatrice de
sous-poids donne les résultats suivants, exprimées en termes de proportions :
\begin{verbatim}
PROC FREQ DATA=birthwt3; TABLE age_classe*low; RUN;
\end{verbatim}

Un tri à plat de la variable \texttt{race} est effectuée de la même manière
à partir de la commande \texttt{PROC FREQ}..
\begin{verbatim}
PROC FREQ DATA=birthwt3; TABLES race; FORMAT race ethnicite.; RUN;
\end{verbatim}

Enfin, pour résumer la distribution des variables selon la variable
indicatrice \texttt{low}, on reprendra la commande \texttt{PROC FREQ} en
sépcifiant la liste des variables d'intérêt, 
\verb|(race smoke ui ht age_classe)|, à croiser avec \verb|low| via
l'opérateur \texttt{*}.
\begin{verbatim}
PROC FREQ DATA=birthwt3; 
  TABLES (race smoke ui ht age_classe)*low;
  FORMAT low low. race ethnicite. smoke tabac. ht hypert. ui uterine.;
RUN;
\end{verbatim}

%---------------------------------------------------------------- Séance 09 --
\chapter*{Semaine 9\markboth{Corrigés de la semaine 9}{}}

\soln{\ref{exo:9.1}} On rappelle que seuls les valeurs numériques des poids
à la naissance sont disponibles dans le fichier \texttt{sirds.dat}, et qu'il
nous faut contruire la variable de groupement (enfants décédés \emph{versus}
vivants). Une manière de procéder consiste à effectuer deux étapes
\texttt{DATA} en incluant une variable qualitative indiquant le status, en
plus des données brutes insérer par copier/coller.
\begin{verbatim}
DATA DCD;
INPUT poids @@;
deces=1;
DATALINES;
1.050 1.175 1.230 1.310 1.500 1.600 1.720 1.750 1.770 2.275
2.500 1.030 1.100 1.185 1.225 1.262 1.295 1.300 1.550 1.820
1.890 1.940 2.200 2.270 2.440 2.560 2.730
;
RUN;

DATA VIV;
INPUT poids @@;
deces=0;
CARDS;
1.130 1.575 1.680 1.760 1.930 2.015 2.090 2.600 2.700 2.950
3.160 3.400 3.640 2.830 1.410 1.715 1.720 2.040 2.200 2.400
2.550 2.570 3.005
;
RUN;

DATA DRIA; SET DCD VIV; RUN;
\end{verbatim}

Le test de Student est réalisé en utilisant la commande \texttt{PROC TTEST},
qui par défaut fournit les résultats sous l'hypothèse d'homoskédasticité ou
d'hétéroskédasticité, ainsi que les résumés descriptifs numériques et
graphiques pour la distribution du poids selon le status clinique. La
variable de classification est introduite après l'instruction \texttt{CLASS}
et la variable réponse après l'instruction \texttt{VAR}. Par défaut, le test
reporté est bilatéral.
\begin{verbatim}
PROC TTEST DATA=DRIA PLOTS=all;
  CLASS deces;
  VAR poids;
RUN;
\end{verbatim}

\includegraphics{./figs/sas_ttest}
%
%
%
\soln{\ref{exo:9.2}}
Les données de l'étude sur le sommeil servant de base à
l'article de Student peuvent être importées sous \SAS comme à l'exercice
précédent, c'est-à-dire en combinant les résultats de deux étapes
\texttt{DATA}. On en profitera lors de la dernière étape pour créer une
variable auxiliaire pour les scores de différence.
\begin{verbatim}
DATA DHH;
INPUT GMSD @@;
CARDS;
0.7 -1.6 -0.2 -1.2 -0.1 3.4 3.7 0.8 0.0 2.0
;
RUN;

DATA LHH;
INPUT GMSL @@;
CARDS;
1.9 0.8 1.1 0.1 -0.1 4.4 5.5 1.6 4.6 3.4
;
RUN;

DATA HH; MERGE DHH LHH; diff_GMS=GMSL-GMSD; RUN;
\end{verbatim}

Un résumé numérique pour l'ensemble des variables numériques (\texttt{GMSD},
\texttt{GMSL}, et \verb|diff_GMS|) peut être obtenu avec \texttt{PROC SUMMARY} 
de la manière suivante :
\begin{verbatim}
PROC SUMMARY DATA=HH PRINT n mean var lclm uclm; VAR GMSD GMSL diff_GMS; RUN;
\end{verbatim}

Les gains moyens de temps de sommeil pour chaque molécule peuvent être
représentés à l'aide d'un diagramme en barres grâce à \texttt{PROC GCHART}.
\begin{verbatim}
PROC GCHART DATA=hh; Hbar diff_gms / midpoints=(0 0.5 1 1.5 2 2.5 3 3.5 4 4.5 5 5.5 6); RUN;
\end{verbatim}

Enfin, pour réaliser un test $t$ pour données appariées, on utilisera toujours la
commande \texttt{PROC TTEST}, mais en spécifiant l'option \texttt{PAIRED},
comme indiqué ci-après.
\begin{verbatim}
PROC TTEST DATA=hh;
  PAIRED GMSD*GMSL;
RUN;
\end{verbatim}

\includegraphics{./figs/sas_ttestpaired}
%
%
%
\soln{\ref{exo:9.3}}
Dans les cas des données dites "groupées", ou plus généralement d'un tableau
de contingence quelconque, on se contente généralement de saisir le tableau
d'effectif en faisant apparaître distinctement les variables de
classification et les effectifs associés. Sous \SAS, on peut procéder ainsi :
\begin{verbatim}
DATA symptom;
INPUT regime amelioration effectif;
CARDS;
1 1 26
0 1 21
1 0 38
0 0 44
;
RUN;

PROC FORMAT; VALUE ouinon 1="Oui" 0="Non"; RUN;
\end{verbatim}

Ensuite, on peut répondre aux trois questions à partir d'une même commande,
\texttt{PROC FREQ}, en se rappelant qu'il est nécessaire de renseigner
l'option \texttt{WEIGHT} pour indiquer une pondération par les effectifs.
\begin{verbatim}
PROC FREQ DATA=symptom ORDER=data;
  TABLES amelioration * regime / chisq;
  WEIGHT effectif;
  FORMAT regime amelioration ouinon.;
RUN;
\end{verbatim}

\includegraphics{./figs/sas_regime}
%
%
%
\soln{\ref{exo:9.4}}
Comme dans l'exercice précédent, le plus simple pour travailler avec ce type
de tableau à deux entrées est de créer une struture de donnes où l'on fait
correspondre les effectifs pour chacun des croisement des modalités des deux
facteurs d'étude (infarctus et traitement) dans une étape \texttt{DATA}.
\begin{verbatim}
DATA myocarde;
INPUT Infractus traitement nombre;
CARDS;
1 1 28
1 2 18
2 1 656
2 2 658
;
RUN;

PROC FORMAT; 
  VALUE ouinon 1="Oui" 2 = "Non";
  VALUE treat 1=" Placebo" 2 = "Aspirine";
RUN;
\end{verbatim}

Ensuite, on peut répondre aux trois questions à partir d'une même commande,
\texttt{PROC FREQ} et l'option \texttt{/ all}, en se rappelant qu'il est
nécessaire de renseigner l'option \texttt{WEIGHT} pour indiquer une
pondération par les effectifs.
\begin{verbatim}
PROC FREQ DATA=myocarde;
  TABLES traitement*Infractus / all;
  WEIGHT nombre;
  FORMAT Infractus ouinon. traitement treat.;
RUN;
\end{verbatim}
%
%
%
\soln{\ref{exo:9.5}}
Les données au format \Stata peuvent être importées sous \SAS à l'aide de
\texttt{PROC IMPORT} en précisant le type de source de données, ici
\texttt{DBMS=STATA}.
\begin{verbatim}
PROC IMPORT OUT= WORK.polymorphism
            DATAFILE= "C:\data\polymorphism.dta"
            DBMS=STATA REPLACE;
RUN;
\end{verbatim}


\begin{verbatim}
PROC GLM DATA=polymorphism; 
  CLASS genotype; 
  MODEL age=genotype; 
RUN;
\end{verbatim}


\begin{verbatim}
PROC SORT DATA=polymorphism; BY genotype; RUN;

PROC GCHART DATA=polymorphism; Vbar age; BY genotype; RUN;
\end{verbatim}


\includegraphics{./figs/sas_boxplot}

\begin{verbatim}
PROC SORT DATA=polymorphism; BY genotype; RUN;

PROC SUMMARY DATA=polymorphism PRINT n mean stddev ucl lcl; VAR age; BY genotype; RUN;
\end{verbatim}

\begin{verbatim}
PROC GLM DATA=polymorphism; 
  CLASS genotype; 
  MODEL age=genotype;
  MEANS genotype / BON CLDIFF;
RUN;
\end{verbatim}

\begin{verbatim}
PROC GLM DATA=polymorphism PLOT=MEANPLOT(CLBAND); 
  CLASS genotype; 
  MODEL age=genotype;
  MEANS genotype / BON CLM;
RUN;
\end{verbatim}